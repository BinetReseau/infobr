%$Id: config_nux.tex 145 2005-03-25 08:26:35Z myk $

\bghdr{images/fond_ubuntu}



%\begin{center}
%\includegraphics{images/logo_Linux}
%\end{center}

\subsection{Configuration sous Ubuntu/Kubuntu}

Cette section décrit la configuration de ta connexion Internet sous Ubuntu GNU/Linux (ou une de ses variantes). Pour les autres distributions, tu peux adapter les instructions ci-dessous ou consulter la version 
en ligne de commande, page \pageref{linux_cmdline}.

\subsubsection{Configuration IP}
Tu as besoin de conna\^itre ton adresse IP, ton masque de sous-r\'eseau et ta  passerelle. Toutes les informations se trouvent page \pageref{calcul_ip}. Bien s\^ur, pour  l'ensemble des manipulations d\'ecrites ci-dessous tu auras besoin de ton  mot de passe super-utilisateur (\emph{root}) !

\label{Ubuntu:IP}
Il existe deux mani\`eres de configurer tes param\`etres r\'eseaux: l'une utilise les outils graphiques de l'environnement que tu as choisi (Gnome ou KDE), 
l'autre utilise simplement la ligne de commande. Bien sûr, les outils graphiques ne sont qu'un interm\'ediaire modifiant les fichiers dont on te parle 
plus bas. Ils te permettent parfois d'enregistrer une configuration r\'eseau, ce qui facilite la gestion si tu rentres souvent chez toi. Pour obtenir le 
m\^eme r\'esultat en ligne de commande il faut utiliser un script.
\begin{description}
\item[\'Etape 1 : configuration de la connexion au r\'eseau] \
 
\begin{itemize}
\item Va dans \menu{Syst\`eme}, \menu{Pr\'ef\'erences} puis \menu{Connexions r\'eseau}.
\item Dans l'onglet \menu{Filaire}, clique sur \menu{Ajouter}.
\item Compl\`ete le champ \menu{Nom de la connexion}  par ce que tu veux ; "Casert de Polytechnique" par exemple.
\item Puis va dans l'onglet \menu{Param\`etres IPv4}.
\item S\'electionne la m\'ethode \menu{Manuel}.
\item Clique sur \menu{Ajouter}, puis remplis les champs \menu{Adresse}, \menu{Masque de r\'eseau} et  \menu{Passerelle} par les donn\'ees qui te sont propres. 
\item Compl\`ete le  champ \menu{Serveurs DNS} par \server{129.104.201.53, 129.104.201.51} et le champ \menu{Domaines de recherche} par \server{eleves.polytechnique.fr, polytechnique.fr}. 
\item Coche enfin l'option \menu{Disponible pour tous les utilisateurs}, clique sur \menu{Appliquer} et enfin rentre ton mot de passe super-utilisateur (root).
\end{itemize}

\item[\'Etape 2 : configuration du proxy (= serveur mandataire)] \
\begin{itemize}
\item Va  dans \menu{Param\`etres Syst\`eme}, \menu{R\'eseau} puis \menu{Serveur Mandataire}.
\item S\'electionne  la \menu{M\'ethode} \menu{Automatique}.
\item Compl\`ete le champ  \menu{URL de configuration} par \urllink{http://config/proxy.pac}. 
\item Clique sur \menu{Appliquer à tout le syst\`eme...} et rentre ton mot de passe super-utilisateur si on te le demande.
\end{itemize}

\item[\'Etape 3 (\'eventuellement)] \
\begin{itemize}
\item Clique  sur l'ic\^one de l'applet R\'eseau dans la zone de notification, en forme  de fl\`eches t\^ete-b\^eche ou d'ondes. S\'electionne le r\'eseau que tu as  configur\'e dans la 1\`ere \'etape,
et te voilà connect\'e à Internet !
\item Une fois ta configuration r\'eseau termin\'ee, tu peux la tester en \emph{pinguant} \fkz (dans une console), o\`u tu devrais voir quelque chose comme :
\end{itemize}

\cmdline{\$ ping frankiz\\
PING frankiz.eleves.polytechnique.fr (129.104.201.51) 56(84) bytes of data.\\
64 bytes from Frankiz.eleves.polytechnique.fr ...}

\end{description}



\subsubsection{Configuration du gestionnaire de paquets}
\label{ubuntu_mirror}

Il faut d\'esormais configurer le gestionnaire de paquets pour qu'il utilise les miroirs du BR et non les miroirs à l'ext\'erieur du campus, qui sont plus lents. \
Va  dans \menu{Applications}, \menu{Logith\`eque Ubuntu} puis menu \menu{\'edition}, \menu{Sources de logiciels...}. 
Entre ton  mot de passe super-utilisateur puis s\'electionne l'onglet \menu{Autres  logiciels}. 
D\'ecoche les cases comprenant une adresse du type \urllink{http://archive.canonical.com/ubuntu version}, o\`u \textit{version} correspond à la version d'Ubuntu install\'ee sur ton ordinateur. 
À l'impression de l'InfoBR, la version actuelle est \textbf{oneiric} et la pr\'ec\'edente est \textbf{natty}. \
Clique sur \menu{Ajouter}, puis entre dans le champ \menu{Ligne APT} :
\cmdline{deb ftp://miroir/linux/ubuntu version main restricted universe multiverse}
Tu auras bien s\^ur remplac\'e \textit{version} par ta version d'Ubuntu (\textit{oneiric}/\textit{natty}/\textit{maverick}/...). \\
Clique ensuite sur \menu{Ajouter une source de mise à jour}. Fais de même pour les lignes suivantes :
\cmdline{deb ftp://miroir/linux/ubuntu version-updates main restricted universe  multiverse \\
deb ftp://miroir/linux/ubuntu version-security main restricted universe  multiverse}
Tu peux aussi utiliser le d\'ep\^ot suivant mais attention il contient des logiciels non support\'es par Canonical, l'\'equipe de d\'eveloppement d'Ubuntu (en particulier il peut arriver que certains logiciels contiennent des erreurs) :
\cmdline{deb ftp://miroir/linux/ubuntu version-backports main restricted universe multiverse}

Clique enfin sur \menu{Fermer} puis r\'eponds \menu{Actualiser} à la fenêtre de dialogue qui appara\^it. \\

Note : il n'est pas n\'ecessaire de configurer Synaptic dans ses Pr\'ef\'erences pour y sp\'ecifier un proxy quelconque.

\subsubsection{Configuration antivirus}

{C'est pas non plus comme si y'en avait besoin \dots}

Pour configurer ton navigateur web, si ce n'est déjà fait, reporte-toi page \pageref{browser}.

%\subsubsection{Configuration du pare-feu}
%
%La solution la plus simple pour se faire un \emph{firewall} sous linux est d'utiliser les \emph{iptables}. Pour ceci la premi\`ere \'etape est
%d'installer le paquet \app{iptables} pour ta distribution. Pour savoir comment configurer ton \emph{firewall} pour le r\'eseau de l'X, consulte le Wikix.

\clearpage
