%$Id: config_nux.tex 145 2005-03-25 08:26:35Z myk $

\bghdr{images/fond_ubuntu}



%\begin{center}
%\includegraphics{images/logo_Linux}
%\end{center}

\subsection{Configuration sous Ubuntu/Kubuntu}

Cette section décrit la configuration de ta connexion Internet sous Ubuntu GNU/Linux (ou une de ses variantes). Pour les autres distributions, tu peux adapter les instructions ci-dessous ou consulter la version 
en ligne de commande, page \pageref{linux_cmdline}.

\subsubsection{Configuration IP}
Tu as besoin de connaître ton adresse IP, ton masque de sous-réseau et ta  passerelle. Toutes les informations se trouvent page \pageref{calcul_ip}. Bien sûr, pour  l'ensemble des manipulations décrites ci-dessous tu auras besoin de ton  mot de passe super-utilisateur (\emph{root}) !

\label{Ubuntu:IP}
Il existe deux manières de configurer tes paramètres réseaux: l'une utilise les outils graphiques de l'environnement que tu as choisi (Gnome ou KDE), 
l'autre utilise simplement la ligne de commande. Bien sûr, les outils graphiques ne sont qu'un intermédiaire modifiant les fichiers dont on te parle 
plus bas. Ils te permettent parfois d'enregistrer une configuration réseau, ce qui facilite la gestion si tu rentres souvent chez toi. Pour obtenir le 
même résultat en ligne de commande il faut utiliser un script.
\begin{description}
\item[Étape 1 : configuration de la connexion au réseau] \
 
\begin{itemize}
\item Va dans \menu{Système}, \menu{Préférences} puis \menu{Connexions réseau}.
\item Dans l'onglet \menu{Filaire}, clique sur \menu{Ajouter}.
\item Complète le champ \menu{Nom de la connexion}  par ce que tu veux ; "Casert de Polytechnique" par exemple.
\item Puis va dans l'onglet \menu{Paramètres IPv4}.
\item Sélectionne la méthode \menu{Manuel}.
\item Clique sur \menu{Ajouter}, puis remplis les champs \menu{Adresse}, \menu{Masque de réseau} et  \menu{Passerelle} par les données qui te sont propres. 
\item Complète le  champ \menu{Serveurs DNS} par \server{129.104.201.53, 129.104.201.51} et le champ \menu{Domaines de recherche} par \server{eleves.polytechnique.fr, polytechnique.fr}. 
\item Coche enfin l'option \menu{Disponible pour tous les utilisateurs}, clique sur \menu{Appliquer} et enfin rentre ton mot de passe super-utilisateur (\emph{root}).
\end{itemize}

\item[Étape 2 : configuration du \emph{proxy} (serveur mandataire)] \
\begin{itemize}
\item Va  dans \menu{Paramètres Système}, \menu{Réseau} puis \menu{Serveur Mandataire}.
\item Sélectionne  la \menu{Méthode} \menu{Automatique}.
\item Complète le champ  \menu{URL de configuration} par \urllink{http://config/proxy.pac}. 
\item Clique sur \menu{Appliquer à tout le système...} et rentre ton mot de passe super-utilisateur si on te le demande.
\end{itemize}

\item[Étape 3 (éventuellement)] \
\begin{itemize}
\item Clique  sur l'icône de l'applet Réseau dans la zone de notification, en forme  de flèches tête-bêche ou d'ondes. Sélectionne le réseau que tu as  configuré dans la première étape,
et te voilà connecté à Internet !
\item Une fois ta configuration réseau terminée, tu peux la tester en \emph{pinguant} \fkz (dans une console), où tu devrais voir quelque chose comme :
\end{itemize}

\cmdline{\$ ping frankiz\\
PING frankiz.eleves.polytechnique.fr (129.104.201.51) 56(84) bytes of data.\\
64 bytes from Frankiz.eleves.polytechnique.fr ...}

\end{description}



\subsubsection{Configuration du gestionnaire de paquets}
\label{ubuntu_mirror}

Il faut désormais configurer le gestionnaire de paquets pour qu'il utilise les miroirs du BR et non les miroirs à l'extérieur du campus, qui sont plus lents. \
Va  dans \menu{Applications}, \menu{Logithèque Ubuntu} puis menu \menu{édition}, \menu{Sources de logiciels...}. 
Entre ton  mot de passe super-utilisateur puis sélectionne l'onglet \menu{Autres  logiciels}. 
Décoche les cases comprenant une adresse du type \urllink{http://archive.canonical.com/ubuntu version}, où \urllink{version} correspond à la version d'Ubuntu installée sur ton ordinateur. 
À l'impression de l'InfoBR, la version actuelle est \urllink{oneiric} et la précédente est \urllink{natty}. \
Clique sur \menu{Ajouter}, puis entre dans le champ \menu{Ligne APT} :
\cmdline{deb ftp://miroir/linux/ubuntu \urllink{version} main restricted universe multiverse}
Tu auras bien sûr remplacé \urllink{version} par ta version d'Ubuntu (\textit{oneiric}/\textit{natty}/\textit{maverick}/...). \\
Clique ensuite sur \menu{Ajouter une source de mise à jour}. Fais de même pour les lignes suivantes :
\cmdline{deb ftp://miroir/linux/ubuntu \urllink{version}-updates main restricted universe  multiverse \\
deb ftp://miroir/linux/ubuntu \urllink{version}-security main restricted universe  multiverse}
Tu peux aussi utiliser le dépôt suivant mais attention il contient des logiciels non supportés par Canonical, l'équipe de développement d'Ubuntu (en particulier il peut arriver que certains logiciels contiennent des erreurs) :
\cmdline{deb ftp://miroir/linux/ubuntu \urllink{version}-backports main restricted universe multiverse}

Clique enfin sur \menu{Fermer} puis réponds \menu{Actualiser} à la fenêtre de dialogue qui apparaît. \\

Note : il n'est pas nécessaire de configurer Synaptic dans ses Préférences pour y spécifier un \emph{proxy} quelconque.

Pour configurer ton navigateur web, si ce n'est déjà fait, reporte-toi page \pageref{browser}.

%\subsubsection{Configuration du pare-feu}
%
%La solution la plus simple pour se faire un \emph{firewall} sous linux est d'utiliser les \emph{iptables}. Pour ceci la première étape est
%d'installer le paquet \app{iptables} pour ta distribution. Pour savoir comment configurer ton \emph{firewall} pour le réseau de l'X, consulte le Wikix.

\clearpage
