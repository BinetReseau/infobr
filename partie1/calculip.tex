\subsection{Comment calculer ton adresse IP ?}
%% Calculer l'adresse \emph{IP} de son casert

\label{calcul_ip}

Une adresse IP est une suite de quatre nombres compris entre 0 et
255 séparés par des points ; en gros, elle identifie de manière
unique toute machine connectée au réseau mondial. \emph{Exemple :}
l'adresse IP de \fkz est \server{129.104.201.51}.

Les adresses IP de l'X sont toutes de la forme \server{129.104.AAA.BBB}.
Les pages suivantes t'indiquent comment calculer \server{AAA} et \server{BBB} pour que ton
ordinateur ait une adresse unique et correcte. Une erreur sur cette adresse t'empêche tout simplement d'accéder au réseau, donc assure-toi que ton calcul est juste !

Calcule ci-dessous ton adresse IP, celle de ta passerelle (\emph{gateway}), ton masque de sous-réseau
(\emph{netmask}) et enfin ton adresse de diffusion (\emph{broadcast}).
Revérifie bien que tu ne t'es pas trompé,  ça t'évitera de te prendre la tête pendant la suite de la configuration !
Note-les soigneusement au bas de la page \pageref{tableauIp} pour référence ultérieure.

\subsubsection{En Maunoury, Foch, Fayolle et Joffre}
Les identifiants \server{AAA} et \server{BBB} se calculent à partir de ton numéro de casert. \server{BBB} correspond aux deux derniers chiffres. Par exemple, pour le casert $12.10.61$, \server{BBB} $= 61$. \server{AAA} et les autres adresses nécessaires à la configuration sont indiquées dans le tableau ci-dessous, et dépendent des quatre premiers chiffres de ton numéro de casert, notés $xx\ yy$ :
\\
\begin{center}
\begin{tabular}{|>{\ungaramond}c|>{\ungaramond}c|c|c|c|}
\hline \multirow{2}{*}{$xx\ yy$} & \multirow{2}{*}{AAA} & \multirow{2}{*}{\bf Passerelle} & \bf Adresse de  & \bf Masque de  \\ 
 & & & \bf{diffusion} & \bf sous-réseau \\
\hline 09 10 & 220 & \multirow{4}{*}{\server{129.104.223.254}} & \multirow{4}{*}{\server{129.104.223.255}} & \multirow{16}{*}{\server{255.255.252.0}} \\ 
\cline{1-2} 09 20 & 221 &  &  &  \\ 
\cline{1-2} 09 30 & 222 &  &  &  \\ 
\cline{1-2} 09 40 & 223 &  &  &  \\ 
\cline{1-4} 10 10 & 212 & \multirow{4}{*}{\server{129.104.215.254}} & \multirow{4}{*}{\server{129.104.215.255}} & \\ 
\cline{1-2} 10 20 & 213 &  &  &  \\ 
\cline{1-2} 10 30 & 214 &  &  &  \\ 
\cline{1-2} 10 40 & 215 &  &  &  \\ 
\cline{1-4} 11 10 & 232 & \multirow{4}{*}{\server{129.104.235.254}} & \multirow{4}{*}{\server{129.104.235.255}} & \\ 
\cline{1-2} 11 20 & 233 &  &  &  \\ 
\cline{1-2} 11 30 & 234 &  &  &  \\ 
\cline{1-2} 11 40 & 235 &  &  &  \\ 
\cline{1-4} 12 10 & 216 & \multirow{4}{*}{\server{129.104.219.254}} & \multirow{4}{*}{\server{129.104.219.255}} & \\ 
\cline{1-2} 12 20 & 217 &  &  &  \\ 
\cline{1-2} 12 30 & 218 &  &  &  \\ 
\cline{1-2} 12 40 & 219 &  &  &  \\ 
\hline
\end{tabular} 
\end{center}

\exemple{l'adresse IP associée au casert $12.10.61$ est \server{129.104.216.61}, sa passerelle est \server{129.104.219.254}, son adresse de
diffusion est \server{129.104.219.255} et son masque de sous-réseau est \server{255.255.252.0}.}

\subsubsection{Résidences Schaeffer et Lemonnier}

Ta prise réseau possède un numéro à six chiffres de la forme $xx\ yy\ zz$. On prend $xx$ pour calculer ton sous-réseau, l'adresse de ta passerelle (\emph{gateway}): \server{129.104.AAA.CCC} et l'adresse de diffusion (\emph{broadcast}): \server{129.104.AAA.EEE}. Le masque de sous-réseau (\emph{netmask}) est toujours
\server{255.255.255.128}. Ensuite, tu peux déterminer la partie \server{BBB} de ton IP avec $zz$ et $xx$ :


\begin{center}
\begin{tabular}{|>{\ungaramond}c|>{\ungaramond}c|>{\ungaramond}c|>{\ungaramond}c|>{\ungaramond}c|}
\hline \rule[-2ex]{0pt}{5ex}$xx$ & $AAA$ & $CCC$ & $EEE$ & $BBB$\\ 
\hline 70 & 224 & 254 & 255 & $128+zz$ \\
71 & 224 & 126 & 127 & $zz$ \\
72 & 228 & 254 & 255 & $128+zz$ \\
73 & 225 & 126 & 127 & $zz$ \\
74 & 225 & 254 & 255 & $128+zz$ \\
75 & 226 & 126 & 127 & $zz$ \\
76 & 227 & 126 & 127 & $zz$ \\
77 & 227 & 254 & 255 & $128+zz$ \\
78 & 228 & 126 & 127 & $zz$ \\
79 & 229 & 126 & 127 & $zz$ \\
80 & 226 & 254 & 255 & $128+zz$ \\ \hline
\end{tabular} 
\end{center}

\exemple{l'adresse IP associée à la prise $70 30 30$ est \server{129.104.224.158} ($158 = 128 + 30$) ; sa passerelle est \server{129.104.224.254}, son
adresse de diffusion est \server{129.104.224.255} et son masque de sous-réseau est \server{255.255.255.128}.}

\subsubsection{Au B.E.M.}

\newlength{\ecart}
\settowidth{\ecart}{Masque de sous-reseau}
\addtolength{\ecart}{2em}
\noindent \begin{tabular}{p{\ecart}<{\dotfill}@{}l}
  Sous-réseau (\server{AAA}) & {\ungaramond 203} pour le bâtiment A ; {\ungaramond 204} pour le bâtiment D\\
  IP (\server{BBB})            & {\ungaramond 50} + les deux derniers chiffres du numéro de ta chambre \\
  Passerelle                   & \server{129.104.AAA.13} \\
  Masque de sous-réseau     & \server{255.255.255.0} \\
    Adresse de diffusion       & \server{129.104.AAA.255} \\
\end{tabular}

\subsubsection{Au P.E.M.}

 \noindent \begin{tabular}{p{\ecart}<{\dotfill}@{}l}
  Sous-réseau (\server{AAA})           & {\ungaramond 205} \\
  IP (\server{BBB}), rez-de-chaussée & {\ungaramond 15} + les deux derniers chiffres du numéro  de ta chambre \\
  IP (\server{BBB}), premier étage   & {\ungaramond 70} + les deux derniers chiffres du numéro de ta chambre \\
  Passerelle                             & \server{129.104.205.13} \\
  Masque de sous-réseau                & \server{255.255.255.0} \\
  Adresse de diffusion                   & \server{129.104.205.255} \\
\end{tabular}

% \subsubsection{IP des serveurs DNS}
%
% Le BR offre quatre serveurs DNS redondants, qui ont les IP's suivantes :
% \begin{itemize}
%   \item Serveur principal : $129.104.201.53$
%   \item Serveurs secondaires : $129.104.201.51$, $129.104.201.52$ et $129.104.201.54$
% \end{itemize}

\pagebreak
