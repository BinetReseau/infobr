\subsection{\emph{Wi-Fi}}
\label{wifi} 
Le DSI propose actuellement un réseau \emph{Wi-Fi}, qui couvre le grand hall, les amphis, les salles de PC, le bataclan (bâtiment qui va de la Kès au bâtiment des
binets/langues), le bâtiment des binets/langues.

Pour te connecter au \emph{Wi-Fi} avec Windows, Mac, Linux ou un iPhone, tu trouveras les instructions sur la page \urllink{http://wifi}.

Avec Windows, tu dois t\'el\'echager un logiciel appel\'e SecureW2 qui est fourni par la DSI sur son site, \urllink{http://www.dsi.polytechnique.fr/fr/telecommunications/wifi/}.

Avec MacOS ou iOS (iPhone, iPad), il faut t\'el\'echarger un fichier \file{wifi.mobileconfig} dont le lien se trouve sur le wikiBR (l\`a encore, suis les intructions de \urllink{http://wifi}).

Avec Linux ou Android, les noms des param\`etres d\'ependent du syst\`eme utilis\'e, mais voici un tableau r\'ecapitulatif :
\begin{center}
\begin{tabular}{|r|l|}
 SSID & Polytechnique \\
 Nom d'utilisateur/Mot de passe & Identifiants DSI (salle info) \\
 S\'ecurit\'e & WPA1 Entreprise \\
 Gestion des cl\'es & WPA-EAP \\
 Pairwise & TKIP \\
 Authentification & Tunneled TLS (TTLS) ou EAP-FAST \\
 Authentification interne & PAP \\
 Proxy HTTP pour tous les protocoles & 129.104.247.2 (port 8080) \\
 Serveurs DNS & 129.104.201.53, 129.104.201.51
\end{tabular}
\end{center}




%Deux réseaux ont été déployés :

%\begin{description}
%  \item[keriadenn] : c'est le réseau public, qui te permet uniquement d'accéder au portail wifi (\url{http://wifi/}, accessible également depuis le réseau normal). Tu trouveras à cette adresse toutes les informations de configuration nécessaires pour te connecter au second réseau, \server{kastell}.

%  \item[kastell] : réseau protégé et caché qui permet, après authentification, de te connecter au réseau et à Internet comme si tu étais dans ton casert !
%\end{description}
