\subsection{\emph{Wi-Fi}}
La DSI propose actuellement un réseau \emph{Wi-Fi}, qui couvre le grand hall, les amphis, les salles de PC,
le bataclan (bâtiment qui va de la Kès au bâtiment des binets/langues), le bâtiment des binets/langues.

%Pour te connecter au \emph{Wi-Fi} Polytechnique avec Windows, Mac, Linux ou un iPhone, tu trouveras les instructions sur la page :
%\begin{center}
%\urllink{http://www.dsi.polytechnique.fr/fr/telecommunications/wifi/wifi-73063.kjsp}
%\end{center}

%Avec Windows, \textbf{avant Windows 8}, tu dois téléchager un logiciel appelé \app{SecureW2} qui est fourni par la DSI sur son site, \urllink{http://www.dsi.polytechnique.fr/fr/telecommunications/wifi/}. Ce n'est plus nécessaire depuis Windows 8 mais des paramètres supplémentaires portant sur l'authenfication 802.1x sont à modifier, la procédure est décrite sur la page susmentionnée.
%
%Avec Mac OS X Lion ou iOS (iPhone, iPad), il faut télécharger un fichier 
%\newline \file{Ecole-Polytechnique.mobileconfig} dont le lien se trouve sur \urllink{http://www.dsi.polytechnique.fr/fr/telecommunications/wifi/wifi-73063.kjsp}. Pour des versions plus anciennes de Mac OS, consulte \urllink{http://br.binets.fr/Configuration\_du\_WiFi\_sous\_Mac}. Attention, avec iOS7, il arrive que le fichier .mobileconfig fourni par la DSI ne marche pas, tu peux essayer avec celui là: \urllink{br.binets.fr/files/WifiPoly.mobileconfig}.
%
%Avec Linux ou Android, les noms des paramètres dépendent du système utilisé, mais voici un tableau récapitulatif :
%\begin{center}
%\begin{tabular}{r|l}
% SSID & Polytechnique \\
% Nom d'utilisateur/Mot de passe & Identifiants DSI (salle info) \\
% Sécurité & WPA1 Entreprise \\
% Gestion des clés & WPA-EAP \\
% Pairwise & TKIP \\
% Authentification & Tunneled TLS (TTLS) ou EAP-FAST \\
% Authentification interne & PAP \\
% Proxy HTTP pour tous les protocoles & 129.104.247.2 (port 8080) \\
% Serveurs DNS & 129.104.201.53, 129.104.201.51
%\end{tabular}
%\end{center}

Pour te connecter au \emph{Wi-Fi} Polytechnique avec Windows, Mac, Linux ou un iPhone, tu trouveras les instructions sur la page :
\begin{center}
\urllink{https://portail.polytechnique.edu/dsi/wifi}
\end{center}

%Avec Windows, \textbf{avant Windows 8}, tu dois téléchager un logiciel appelé \app{SecureW2} qui est fourni par la DSI sur son site, \urllink{http://www.dsi.polytechnique.fr/fr/telecommunications/wifi/}. Ce n'est plus nécessaire depuis Windows 8.\\

%Avec Mac OS X Lion ou iOS (iPhone, iPad), il faut télécharger un fichier intitulé
%\newline \file{Ecole-Polytechnique.mobileconfig} dont le lien se trouve sur :
%\begin{center}
%\urllink{http://www.dsi.polytechnique.fr/fr/telecommunications/wifi/wifi-73063.kjsp}
%\end{center}
%
%Pour des versions plus anciennes de Mac OS, consulte :\\ \urllink{http://br.binets.fr/Configuration\_du\_WiFi\_sous\_Mac}. Attention, avec iOS7, il arrive que le fichier .mobileconfig fourni par la DSI ne marche pas, tu peux essayer avec celui là : \urllink{br.binets.fr/files/WifiPoly.mobileconfig}.\\
%
%Avec Linux ou Android, les noms des paramètres dépendent du système utilisé, mais voici un tableau récapitulatif :
%\begin{center}
%\begin{tabular}{r|l}
% SSID & Polytechnique \\
% Nom d'utilisateur/Mot de passe & Identifiants DSI (salle info) \\
% Sécurité & WPA1 Entreprise \\
% Gestion des clés & WPA-EAP \\
% Pairwise & TKIP \\
% Authentification & Tunneled TLS (TTLS) ou EAP-FAST \\
% Authentification interne & PAP \\
% Proxy HTTP pour tous les protocoles & 129.104.247.2 (port 8080) \\
% Serveurs DNS & 129.104.201.53, 129.104.201.51
%\end{tabular}
%\end{center}




%Deux réseaux ont été déployés :

%\begin{description}
%  \item[keriadenn] : c'est le réseau public, qui te permet uniquement d'accéder au portail wifi (\url{http://wifi/}, accessible également depuis le réseau normal). Tu trouveras à cette adresse toutes les informations de configuration nécessaires pour te connecter au second réseau, \server{kastell}.

%  \item[kastell] : réseau protégé et caché qui permet, après authentification, de te connecter au réseau et à Internet comme si tu étais dans ton casert !
%\end{description}
