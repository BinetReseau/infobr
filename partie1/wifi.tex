\subsection{\emph{Wi-Fi}}
La DSI propose actuellement un r\'eseau \emph{Wi-Fi}, qui couvre le grand hall, les amphis, les salles de PC, le bataclan (b\^atiment qui va de la K\`es au b\^atiment des
binets/langues), le b\^atiment des binets/langues.

Pour te connecter au \emph{Wi-Fi} avec Windows, Mac, Linux ou un iPhone, tu trouveras les instructions sur la page \urllink{http://wifi}.

Avec Windows, tu dois t\'el\'echager un logiciel appel\'e \app{SecureW2} qui est fourni par la DSI sur son site, \urllink{http://www.dsi.polytechnique.fr/fr/telecommunications/wifi/}.

Avec Mac OS X Lion ou iOS (iPhone, iPad), il faut t\'el\'echarger un fichier 
\newline \file{Ecole-Polytechnique.mobileconfig} dont le lien se trouve sur \urllink{http://wifi}. Pour des versions plus anciennes de Mac OS, consulte \urllink{http://br.binets.fr/Configuration\_du\_WiFi\_sous\_Mac}.

Avec Linux ou Android, les noms des param\`etres d\'ependent du syst\`eme utilis\'e, mais voici un tableau r\'ecapitulatif :
\begin{center}
\begin{tabular}{r|l}
 SSID & Polytechnique \\
 Nom d'utilisateur/Mot de passe & Identifiants DSI (salle info) \\
 S\'ecurit\'e & WPA1 Entreprise \\
 Gestion des cl\'es & WPA-EAP \\
 Pairwise & TKIP \\
 Authentification & Tunneled TLS (TTLS) ou EAP-FAST \\
 Authentification interne & PAP \\
 Proxy HTTP pour tous les protocoles & 129.104.247.2 (port 8080) \\
 Serveurs DNS & 129.104.201.53, 129.104.201.51
\end{tabular}
\end{center}




%Deux r\'eseaux ont \'et\'e d\'eploy\'es :

%\begin{description}
%  \item[keriadenn] : c'est le r\'eseau public, qui te permet uniquement d'acc\'eder au portail wifi (\url{http://wifi/}, accessible \'egalement depuis le r\'eseau normal). Tu trouveras à cette adresse toutes les informations de configuration n\'ecessaires pour te connecter au second r\'eseau, \server{kastell}.

%  \item[kastell] : r\'eseau prot\'eg\'e et cach\'e qui permet, apr\`es authentification, de te connecter au r\'eseau et à Internet comme si tu \'etais dans ton casert !
%\end{description}
