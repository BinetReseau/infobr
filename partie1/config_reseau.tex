
\subsection{Connexion au réseau}

Cette configuration réseau s'applique uniquement aux bâtiments Foch, Fayolle, Joffre, Maunoury, 70, 71, 73 et 74. Si tu as déjà configuré ton ordi pour te connecter à internet depuis un autre bâtiment, annule cette configuration.\footnote{Pour le bâtiment 76, il devrait être possible d'accéder à l'InfoBR en Ligne depuis votre prise, pour les autres bâtiments, vous devez utiliser un autre moyen pour accéder à l'InfoBR en ligne. Dans les deux cas remplacez casert X par Logement non X}

Branche ton ordinateur au réseau au moyen d'un câble Ethernet, ouvre un navigateur et va à l'adresse \urllink{https://portail.polytechnique.edu/dsi/internet}\footnote{En fait n'importe quel site devrait te rediriger vers cette adresse, sauf au bat. 76}. Tu accéderas alors à l'InfoBR en ligne. Clique sur le lien Casert X à gauche (cf figure), et suis les instructions correspondant à ton système d'exploitation.

Attends, il y a deux InfoBR~? En fait pas exactement. La version en ligne, écrite conjointement par le BR et la DSI de l'École,  contient uniquement les informations de configuration permettant d'accéder à Internet (et te permet de télécharger les fichiers nécessaires, ce que ne permet pas cette version papier~!), alors que celui que tu tiens entre tes mains te présente les différents services internes que nous proposons, et t'aide lorsque tu n'as pas accès au réseau et à l'InfoBR en ligne.

\imagepos{images/portail_dsi_mis_en_valeur}{0.8}{Le portail de l'InfoBR en ligne}{!h}
