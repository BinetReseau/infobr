% Commandes speciales definies ici


% pour les images de fond sur les headers
\newcommand{\bghdr}[1]{
  \leftwatermark{
    \raisebox{-1.5cm}{
      \hspace{-2.5cm}
      \includegraphics{#1}
    }
  }
  \rightwatermark{
    \raisebox{-1.5cm}{
      \hspace{14.1cm}
      \includegraphics{#1}
    }
  }
}


% pour inclure des images
\newcommand{\image}[3]{ % arguments : path, largeur (entre 0 et 1), l�gende
  \begin{figure*}
    \begin{center}
      \includegraphics[width=#2\textwidth]{#1}
      \caption{#3}
    \end{center}
  \end{figure*}
}
\newcommand{\imagepos}[4]{ % arguments : path, largeur (entre 0 et 1), l�gende, positionnement
  \begin{figure*}[#4]
    \begin{center}
      \includegraphics[width=#2\textwidth]{#1}
      \caption{#3}
    \end{center}
  \end{figure*}
}
\newcommand{\imageref}[5]{ % arguments : path, largeur (entre 0 et 1), l�gende, positionnement, label
  \begin{figure*}[#4]
    \begin{center}
      \includegraphics[width=#2\textwidth]{#1}
      \caption{#3}
      \label{#5}
    \end{center}
  \end{figure*}
}
\newcommand{\flimage}[3]{ % arguments : path, largeur (entre 0 et 1), position
  \begin{wrapfigure}{#3}{0pt}
    \includegraphics[width=#2\textwidth]{#1}
  \end{wrapfigure}
}

% texttt : ligne de commande, serveurs
% textsf : tout le reste : url, email, dossiers

% pour les serveurs
\definecolor{DarkBlue}{cmyk}{0.95,0.8,0.0,0.0}
\newcommand{\server}[1]{\texttt{\color{DarkBlue}#1}\xspace}
% pour les lignes de commande
\definecolor{LightRed}{cmyk}{0.,0.1,0.15,0.0}
\definecolor{BrightGreen}{cmyk}{0.8,0.,1.0,0.0}
\newcommand{\cmdline}[1]{
  \vspace{4pt}
  \noindent
��\colorbox{LightRed}{
    \parbox[c]{.9\textwidth}{
    \NoAutoSpaceBeforeFDP
    \texttt{\footnotesize{\color{BrightGreen}#1}}
    \AutoSpaceBeforeFDP
    }
  } \hfill
  \vspace{4pt}
}
\newcommand{\cmdlineshort}[1]{
  \noindent
��\colorbox{LightRed}{
    \parbox{.2\textwidth}{
    \NoAutoSpaceBeforeFDP
    \texttt{\normalsize{\color{BrightGreen}#1}}
    \AutoSpaceBeforeFDP
    }
  }
  \vspace{4pt}
}

% pour les URLs
\definecolor{Blue}{cmyk}{0.6,0.0,0.05,0.0}
\newcommand{\url}[1]{\NoAutoSpaceBeforeFDP\textsf{\color{Blue}#1}\AutoSpaceBeforeFDP\xspace}
% pour les mails
\definecolor{Blue2}{cmyk}{0.6,0.2,0.0,0.0}
\newcommand{\mail}[1]{\textsf{\color{Blue2}#1}\xspace}
% pour les newsgroups
\definecolor{DarkGreen}{cmyk}{0.8,0.1,0.9,0.4}
\newcommand{\ngname}[1]{\textsf{\color{DarkGreen}#1}\xspace}

% pour les applications
\definecolor{GreenApp}{cmyk}{0.7,0.0,1.0,0.8}
\newcommand{\app}[1]{\textbf{\color{GreenApp}#1}\xspace}
% pour les menus et les elements de menu
\definecolor{GrayMenu}{cmyk}{0.7,0.5,0.6,0.4}
\newcommand{\menu}[1]{\textsf{\color{GrayMenu}`#1'}\xspace}
% pour les repertoires
\definecolor{Orange}{cmyk}{0.,0.2,1.0,0.2}
\newcommand{\rep}[1]{\textsf{\color{Orange}#1}\xspace}
% pour les fichiers
\definecolor{DarkOrange}{cmyk}{0.,0.4,1.0,0.4}
\newcommand{\file}[1]{\textsf{\color{DarkOrange}#1}\xspace}
% pour les distributions Linux
\newcommand{\distrib}[1]{{\color{green}#1}}
% pour les liens (sous frankiz)
\newcommand{\lien}[1]{\textsf{\color{Blue}`#1'}\xspace}
% pour les pseudos
\definecolor{GreenPseudo}{cmyk}{0.8,0.1,1.0,0.1}
\newcommand{\mbr}[5]{\noindent#3 \bsc{\color{GreenPseudo}#1} (#4) \textbf{#2} : #5}

% divers
\newcommand{\fkz}{\server{frankiz}}
\newcommand{\xshare}{la rubrique \menu{T�l�charger} de \fkz}

% Les logos win, nux et mac
\newcommand{\nuxs}{\includegraphics{images/logo_Linux_s}}
\newcommand{\wins}{\includegraphics{images/logo_Windows_s}}
\newcommand{\macs}{\includegraphics{images/logo_Mac_s}}

\newcommand{\exemple}[1]{ \fcolorbox{black}{LightRed} {
  \begin{minipage}{0.9\textwidth}
    \emph{Exemple :} #1
  \end{minipage}
}}

% pour les traductions
\newcommand{\trad}[1]{ \textit{#1} }
