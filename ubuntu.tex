\paragraph{\distrib{Ubuntu/Kubuntu}}
\label{Ubuntu:IP}

Lors de l'installation d'une nouvelle Ubuntu, le programme d'installation t'a normalement demand\'e de rentrer toutes ces informations de mani\`ere
interactive. Si tu ne l'as pas fait \`a ce moment l\`a tu peux modifier tes param�tres r�seaux via l'ic\^one pr\'esente dans ta barre des t\^aches symbolisant
le r\'eseau. Un simple clic droit dessus affiche un menu te permettant d'ouvrir la fen�tre de configuration. \emph{Attention} ici le r\'eseau est \`a configurer en
\emph{statique} pas en automatique avec DHCP !! Rentre ensuite ton adresse IP, ta passerelle, ton adresse de broadcast et ton masque de sous-r\'eseau.

Ensuite il faut configurer la r\'esolution DNS, sous Kubuntu va sur l'onglet \textit{Syst\`eme de noms de domaines} ou dans l'outils de configuration du r\'eseau
(param\`etres du Syst\`eme -> r\'eseau). Les adresses IP des serveurs DNS sont les suivantes: \server{129.104.201.53} et \server{129.104.201.51}, le domaine r\'eseau
est \textit{eleves.polytechnique.fr}, tu peux aussi ajouter \textit{polytechnique.fr} pour acc\'eder plus facilement aux machines des salles info. Choisis un nom de machine
(en g\'en\'eral ton pseudo) que tu reporteras dans qRezix pour qu'il soit enregistr\'e.
\imagepos{images/kubuntu_config_reseau0}{0.65}{Configuration du r\'eseau sous Kubuntu}{!ht}
\imagepos{images/kubuntu_config_reseau1}{0.55}{Configuration de la r\'esolution DNS sous Kubuntu}{!ht}
% Lors de l'installation d'une nouvelle Ubuntu, le programme d'installation t'a normalement demand\'e de rentrer toutes ces informations de mani\`ere
% interactive. Si tu ne l'as pas fait \`a ce moment l\`a, tu peux les modifier comme ci-dessous avec ton \'editeur de texte pr\'ef\'er\'e (le tout avec les droits
% administrateur \'evidemment!), ou si tu pr\'ef\`eres les outils graphiques avec \app{KNetworkManager} sous Kubuntu (KDE) (l'ic\^one en forme de c\^able r\'eseau dans
% la barre des t\^aches) et avec \app{l'applet r\'eseau} sous Ubuntu (GNOME) (l'ic\^one en haut \`a droite symbolisant un r\'eseau -> propri\'et\'es).
% \newline
% \newline
% \textit{Les \'etapes qui suivent peuvent \^etres enti\`erement r\'ealis\'ees avec les outils graphiques mentionn\'es plus haut.}
% \newline
% \begin{itemize}
% \item Le fichier \file{/etc/hostname} contient ton nom de machine. Il doit contenir uniquement:
% 
% \cmdline[0.85]{tonPseudo.eleves.polytechnique.fr}
% 
% \item Le fichier \file{/etc/resolv.conf} d\'ecrit comment associer le nom d'une machine \`a une adresse IP.
% Il doit contenir:
% 
% \cmdline[0.85]{
% domain eleves.polytechnique.fr\\
% search eleves.polytechnique.fr polytechnique.fr\\
% nameserver 129.104.201.53\\
% nameserver 129.104.201.51
% }
% 
% \item Le fichier \file{/etc/network/interfaces} contient entre autres ton IP,
% ton sous-r\'eseau et la passerelle pour en sortir. Ce fichier doit
% ressembler (avec \'eventuellement une config wifi \`a la suite\ldots,
% voir la page~\pageref{wifi}) \`a :
% 
% \cmdline[0.85]{
% \# The loopback network interface\\
% auto lo\\
% iface lo inet loopback\\
% \\
% \# The primary network interface\\
% auto eth0\\
% iface eth0 inet static\\
%         address   129.104.AAA.BBB\\
%         netmask   255.255.FFF.DDD\\
%         broadcast 129.104.GGG.EEE\\
%         gateway   129.104.GGG.CCC
% }
% 
% \end{itemize}
% 
% Ensuite il faut red\'emarrer ta configuration r\'eseau (les outils graphiques devraient le faire tout seuls apr\`es validation) :
% 
% \cmdline{\$ sudo /etc/init.d/networking restart}

% Voil\`a, ta configuration r\'eseau est termin\'ee! Tu peux la tester en \emph{pinguant} \fkz, o\`u tu devrais voir quelque chose comme (ceci reste utile m\^eme si tu
% as fait ta configuration avec les outils graphiques) :

Une fois ta configuration r\'eseau termin\'ee, tu peux la tester en \emph{pinguant} \fkz (dans une console), o\`u tu devrais voir quelque chose comme :

\cmdline{\$ ping frankiz\\
PING frankiz.eleves.polytechnique.fr (129.104.201.51) 56(84) bytes of data.\\
64 bytes from Frankiz.eleves.polytechnique.fr ...
}

\label{ubuntu_mirror} {\bf Configuration du gestionnaire de paquets
:} Il faut d\'esormais configurer le gestionnaire de paquets pour
qu'il utilise les miroirs du BR et non les miroirs \`a l'ext\'erieur du
campus qui sont plus lents.

Le fichier \file{/etc/apt/sources.list} liste les miroirs utilis\'es par le gestionnaire de paquets. Il faut commenter la premi\`ere ligne (qui
correspond au CD d'installation) ainsi que toutes les lignes non comment\'ees du fichier (qui correspondent aux miroirs ext\'erieurs au campus) de la
fa\c{c}on suivante:

\cmdline{deb cdrom:[...]/ [version] main restricted}

devient

\cmdline{\#deb cdrom:[...]/ [version] main restricted}

Il faut ensuite ajouter les lignes suivantes, qui correspondent aux miroirs du BR, au \emph{d\'ebut} du fichier:

\cmdline{
deb ftp://miroir/ubuntu [version] main restricted universe multiverse\\
deb ftp://miroir/ubuntu [version]-updates main restricted universe multiverse\\
deb ftp://miroir/ubuntu [version]-security main restricted universe multiverse
}

o\`u \textbf{[version]} correspond \`a la version d'Ubuntu install\'ee. La version actuelle est \textbf{hardy} et la pr\'ec\'edente est \textbf{gutsy}.

Tu peux aussi utiliser le d\'ep\^ot suivant mais attention il contient des logiciels \emph{non support\'es par l'\'equipe de d\'eveloppement d'Ubuntu} (en particulier il peut arriver que certains logiciels soient bogu\'es !):

\cmdline{
deb ftp://miroir/ubuntu [version]-backports main restricted universe multiverse
}


Le BLL (Binet Logiciels Libres) dispose par ailleurs d'un miroir non-officiel qui contient \app{QRezix} ainsi que des paquets tr\`es importants (codecs
vid\'eo, java,\dots) ou pas (bureau 3d, google earth,\dots), non inclus dans la distribution de base pour diverses raisons, en particulier l\'egales ou
\'ethiques. Pour en profiter, il faut rajouter \`a la suite des lignes pr\'ec\'edentes:

\cmdline{deb ftp://miroir/bll [version] main}


On finit par v\'erifier que tout fonctionne en mettant \`a jour la liste
des paquets disponibles:

\cmdline{\$ sudo aptitude update}

S'il n'y a pas de message d'erreur c'est que tout fonctionne bien.

\textbf{NB:} on peut aussi faire cette configuration depuis \app{synaptic} (ubuntu) ou \app{adept} (kubuntu), il suffit d'aller dans le menu faisant r\'ef\'erence aux d\'ep\^ots.