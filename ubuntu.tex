\paragraph{Sous Ubuntu/Kubuntu}
\label{Ubuntu:IP}
Il existe deux mani\`eres de configurer tes param\`etres r\'eseaux: l'une utilise les outils graphiques de l'environnement que tu as choisi (Gnome ou KDE), 
l'autre utilise simplement la ligne de commande. Bien sûr, les outils graphiques ne sont qu'un interm\'ediaire modifiant les fichiers dont on te parle 
plus bas. Ils te permettent parfois d'enregistrer une configuration r\'eseau, ce qui facilite la gestion si tu rentres souvent chez toi. Pour obtenir le 
m\^eme r\'esultat en ligne de commande il faut utiliser un script.
\begin{description}
\item[\'Etape 1 : configuration de la connexion au r\'eseau] \
 
\begin{itemize}
\item Va dans \menu{Syst\`eme}, \menu{Pr\'ef\'erences} puis \menu{Connexions r\'eseau}.
\item Dans l'onglet \menu{Filaire}, clique sur \menu{Ajouter}.
\item Compl\`ete le champ \menu{Nom de la connexion}  par ce que tu veux ; "Casert de Polytechnique" par exemple.
\item Puis va dans l'onglet \menu{Param\`etres IPv4}.
\item S\'electionne la m\'ethode \menu{Manuel}.
\item Clique sur \menu{Ajouter}, puis remplis les champs \menu{Adresse}, \menu{Masque de r\'eseau} et  \menu{Passerelle} par les donn\'ees qui te sont propres. 
\item Compl\`ete le  champ \menu{Serveurs DNS} par \server{129.104.201.53, 129.104.201.51} et le champ \menu{Domaines de recherche} par \server{eleves.polytechnique.fr, polytechnique.fr}. 
\item Coche enfin l'option \menu{Disponible pour tous les utilisateurs}, clique sur \menu{Appliquer} et enfin rentre ton mot de passe super-utilisateur (root).
\end{itemize}

\item[\'Etape 2 : configuration du proxy (= serveur mandataire)] \
\begin{itemize}
\item Va  dans \menu{Syst\`eme}, \menu{Pr\'ef\'erences} puis \menu{Serveur mandataire}.
\item S\'electionne  les options \menu{Configuration manuelle du serveur mandataire} et \menu{Utiliser  le m\^eme serveur mandataire pour tous les protocoles}.
\item Compl\`ete le champ  \menu{Serveur mandataire HTTP} par \server{kuzh} et le champ \menu{Port} par \server{8080}. 
\item Clique sur \menu{Appliquer à l'ensemble du syst\`eme...} puis rentre ton mot de passe super-utilisateur deux fois.
\end{itemize}

\item[\'Etape 3 (\'eventuellement)] \
\begin{itemize}
\item Clique  sur l'ic\^one de l'applet R\'eseau dans la zone de notification, en forme  de fl\`eches t\^ete-b\^eche ou d'ondes. S\'electionne le r\'eseau que tu as  configur\'e dans la 1\`ere \'etape, et te voilà connect\'e à Internet !
\item Tu choisiras par ailleurs un nom de machine (en g\'en\'eral ton pseudo) que tu reporteras dans qRezix pour qu'il soit enregistr\'e.
\item Une fois ta configuration r\'eseau termin\'ee, tu peux la tester en \emph{pinguant} \fkz (dans une console), o\`u tu devrais voir quelque chose comme :
\end{itemize}

\cmdline{\$ ping frankiz\\
PING frankiz.eleves.polytechnique.fr (129.104.201.51) 56(84) bytes of data.\\
64 bytes from Frankiz.eleves.polytechnique.fr ...}

\end{description}