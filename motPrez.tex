\thispagestyle{empty}

\section*{Le mot du Prez'}

\vspace{2em}

Salut cher TOS,
\vspace{1em}

Te voil\`a donc arriv\'e sur le Platal. Tu te sens peut-être un peu perdu, mais ne t'inqui\`ete pas: le BR a pens\'e \`a toi. Gr\^ace \`a cet infoBR, tu vas pouvoir subvenir \`a au moins un de tes besoins primitifs: te connecter \`a Internet. Tu trouveras au fil de ces pages des explications simples et claires --- enfin on l'esp\`ere --- pour configurer ton ordinateur, ainsi que tes mails.

Mais ce n'est pas tout: cet infoBR d\'ecrit aussi les nombreux services mis en place par le BR pour les X et les binets. \'Evite donc de le convertir trop rapidement en liti\`ere pour chat, tu seras bien content de le retrouver le jour o\`u tu d\'ecideras de cr\'eer un FTP pour partager tes photos de vacances, ou quand tu auras besoin te de connecter en urgence à un ordi des salles info pour r\'ecup\'erer ton projet avant la deadline.

Pour \'eviter de flamber toutes tes cotiz K\`es avant m\^eme que tu ne sois arriv\'e, nous avons limit\'e l'infoBR \`a l'essentiel. Si tu es curieux et que tu as envie de te plonger plus avant dans les myst\`eres du r\'eseau de l'X, n'h\'esite pas \`a aller chercher des infos sur \urllink{https://br.binets.fr/}.

\vspace{1em}
En cas de probl\`eme, une proc\'edure typique de r\'esolution est expliqu\'ee en quatri\`eme de couverture. Si cela ne suffit pas, tu peux nous envoyer un mail \`a \mail{support@eleves.polytechnique.fr}. Un BRman charitable prendra le temps d'\'etudier ton cas.

Bref, \`a toi de bosser maintenant: rendez-vous page \pageref{ip} pour te connecter au plus vite!

\vspace{3em}
\hfill JayTe, Prez', pour le BR 2012


\vfill


Guide d'utilisation rapide :
\begin{itemize}
\item Pour te connecter \`a Internet, calcule ton IP page \pageref{ip} puis consulte, selon ton syst\`eme d'exploitation, la section Windows page \pageref{windows},
la section Ubuntu page \pageref{ubuntu} ou la section Mac page \pageref{mac}.
\item Pour te renseigner sur le \textbf{WikiX}, sur \textbf{IRC} ou sur les autres services du BR et de Polytechnique.org, rendez-vous page \pageref{services}
\item Tu ne comprends rien, \c{c}a ne marche pas, tu as des questions : beaucoup de r\'eponses sont donn\'ees page \pageref{faq} et suivantes.

\end{itemize}

\begin{center}
\label{tableauIp}
\'Ecris ci-dessous les diff\'erents nombres que tu auras calcul\'es, ils te serviront plus d'une fois.
\end{center}

\begin{center}
  \begin{tabular}{|rp{5cm}|}
  \hline
  \rule[-8pt]{0pt}{24pt} \textbf{Mon adresse IP :} \ungaramond 129.104. & \\ \hline
  \rule[-8pt]{0pt}{24pt} \textbf{Ma passerelle :} \ungaramond 129.104. & \\ \hline
  \rule[-8pt]{0pt}{24pt} \textbf{Mon adresse de diffusion :} \ungaramond 129.104. & \\ \hline
  \rule[-8pt]{0pt}{24pt} \textbf{Mon masque de sous-r\'eseau :} \ungaramond 255.255. & \\ \hline
  \end{tabular}
  \label{tableau:mon_IP}
\end{center}
