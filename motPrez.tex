\thispagestyle{empty}

\section*{Le mot du Prez'}

\vspace{1em}

Salut cher TOS,
\vspace{0.5em}

Te voilà donc arrivé sur le Platal. Tu te sens peut-être un peu perdu, mais ne t'inquiète pas: le BR a pensé à toi. Grâce à cet infoBR, tu vas pouvoir subvenir à au moins un de tes besoins primitifs: te connecter à Internet. Tu trouveras au fil de ces pages des explications simples et claires --- enfin on l'espère --- pour configurer ton ordinateur, ainsi que tes mails.

Mais ce n'est pas tout: cet infoBR décrit aussi les nombreux services mis en place par le BR pour les X et les binets. Évite donc de le convertir trop rapidement en litière pour chat, tu seras bien content de le retrouver le jour où tu décideras de créer un FTP pour partager tes photos de vacances, ou quand tu auras besoin te de connecter en urgence à un ordi des salles info pour récupérer ton projet avant la deadline.

Pour éviter de flamber toutes tes cotiz Kès avant même que tu ne sois arrivé, nous avons limité l'infoBR à l'essentiel. Si tu es curieux et que tu as envie de te plonger plus avant dans les mystères du réseau de l'X, n'hésite pas à aller chercher des infos sur \urllink{https://br.binets.fr/}.

\vspace{1em}
En cas de problème, une procédure typique de résolution est expliquée en quatrième de couverture. Si cela ne suffit pas, tu peux nous envoyer un mail à \mail{support@eleves.polytechnique.fr}. Un BRman charitable prendra le temps d'étudier ton cas.

Bref, à toi de bosser maintenant: rendez-vous page \pageref{ip} pour te connecter au plus vite!

\vspace{2.5em}
\hfill JayTe, Prez', pour le BR 2012


\vfill


Guide d'utilisation rapide :
\begin{itemize}
\item Pour te connecter à Internet, calcule ton IP page \pageref{ip} puis consulte, selon ton système d'exploitation, la section Windows page \pageref{windows}, la section Ubuntu page \pageref{ubuntu} ou la section Mac page \pageref{mac}. Mais avant de te ruer sur le net, lis la section sur \fkz page \pageref{services}! Tu apprendras plein de choses sur le site des élèves.
\item Pour te renseigner sur le \textbf{WikiX}, sur \textbf{IRC} ou sur les autres services du BR et de Polytechnique.org, rendez-vous page \pageref{services}
\item Tu ne comprends rien, a ne marche pas, tu as des questions : beaucoup de réponses sont données page \pageref{faq} et suivantes.

\end{itemize}

\begin{center}
\label{tableauIp}
Écris ci-dessous les différents nombres que tu auras calculés, ils te serviront plus d'une fois.
\end{center}

\begin{center}
  \begin{tabular}{|rp{5cm}|}
  \hline
  \rule[-8pt]{0pt}{24pt} \textbf{Mon adresse IP :} \ungaramond 129.104. & \\ \hline
  \rule[-8pt]{0pt}{24pt} \textbf{Ma passerelle :} \ungaramond 129.104. & \\ \hline
  \rule[-8pt]{0pt}{24pt} \textbf{Mon adresse de diffusion :} \ungaramond 129.104. & \\ \hline
  \rule[-8pt]{0pt}{24pt} \textbf{Mon masque de sous-réseau :} \ungaramond 255.255. & \\ \hline
  \end{tabular}
  \label{tableau:mon_IP}
\end{center}
