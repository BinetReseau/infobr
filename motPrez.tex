\begin{center} {
	\LARGE Le mot du prez
} \end{center}

Salut \`a tous !

Voil\`a la nouvelle version de l'InfoBR !
Ce petit fascicule t'aidera \`a d\'ecouvrir comment fonctionne le r\'eseau \`a l'X
et \`a configurer le mieux possible ton ordinateur.
Il a \'et\'e r\'edig\'e dans le but de rendre ais\'e l'acc\`es au r\'eseau de l'\'ecole,
et aux services du binet r\'eseau, et d'\^etre compl\'ementaire des r\'eponses d\'ej\`a \`a votre disposition en ligne.
FTP, qRezix, firewall et cross-posts seront bient\^ot tes meilleurs amis !

Le r\'eseau \'el\`eve de l'\'ecole est g\'er\'e par un groupe de personnes qui, b\'en\'evolement,
investissent beaucoup de temps afin que tu puisses acc\'eder \`a Internet
et aux services que nous proposons aux X du campus
--- d'ailleurs, il songer penser \`a passer aux journ\'ees de 35 heures.
Alors prends soin du mat\'eriel r\'eseau, respecte la charte d'utilisation de la DSI
(Direction des Syst\`emes d'Information, les gourous informatiques de l'X), et celle du BR ;
ainsi tout se passera dans les meilleures conditions.

Nous vous sugg\'erons enfin de passer sous Linux pour d\'ecouvrir ce syst\`eme d'exploitation.
Si tu aimes un peu l'informatique, ce n'est vraiment pas difficile, alors n'h\'esites pas...
Viens tout simplement � une install-party organis�e par le BLL (Binet Logiciel Libre).

Profites bien de ces ann\'ees \`a l'Ecole, et des services qui te sont propos�s !

\begin{flushright}
\bsc{alakazam}, Prez, pour le Binet R\'eseau 2k4
\end{flushright}