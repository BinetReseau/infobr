\begin{center} {
	\LARGE Le mot du prez
} \end{center}

Salut \`a tous !

Voil\`a la nouvelle version de l'InfoBR !
Ce petit fascicule t'aidera \`a d\'ecouvrir comment fonctionne le r\'eseau \`a l'X
et \`a configurer le mieux possible ton ordinateur.
Il a \'et\'e r\'edig\'e dans le but de rendre ais\'e l'acc\`es au r\'eseau de l'\'ecole,
et aux services du binet r\'eseau, et d'\^etre compl\'ementaire des r\'eponses d\'ej\`a \`a votre disposition, en ligne.
FTP, qRezix, firewall et cross-posts seront bient\^ot tes meilleurs amis !

Le r\'eseau \'el\`eve de l'\'ecole est g\'er\'e par un groupe de personnes qui, b\'en\'evolement,
investissent beaucoup de temps afin que tu puisses acc\'eder \`a Internet
et aux services que nous proposons aux X du campus
--- d'ailleurs, il songer penser \`a passer aux journ\'ees de 35 heures.
Alors prends soin du mat\'eriel r\'eseau, respecte la charte d'utilisation de la DSI
(Direction des Syst\`emes d'Information, les gourous informatiques de l'X), et celle du BR ;
ainsi tout se passera dans les meilleures conditions.
Cependant s'il s'av\'erait que tu posais des probl\`emes,
nous n'aurions aucun scrupule \`a prendre les mesures qui s'imposent
--- et qui pourraient te priver de certains services, voire de ton acc\`es au r\'eseau pour quelque temps.

Si tu utilises Windows, nous te sugg\'erons de faire tout particuli\`erement attention aux virus et aux spywares.
Nous te conseillons d'utiliser notre domaine Windows qui assurera automatiquement la s\'ecurit\'e de ton PC,
tout particuli\`erement si tu n'as pas beaucoup l'habitude de l'informatique, et il n'y a pas de honte \`a cela.
Sinon, installe l'antivirus que nous te proposons et mets-le \`a jour r\'eguli\`erement.
Il est obligatoire d'installer un anti-virus \`a jour.
Dans tous les cas, ne clique pas sur les pi\`eces jointes de mails sans \^etre s\^ur que la personne qui te l'envoie
est une personne de confiance et que tu lui as bien demand\'e le fichier en question.
Le reste du temps, cela a de fortes chances d'\^etre un virus.

Nous vous sugg\'erons enfin de passer sous Linux pour d\'ecouvrir ce syst\`eme d'exploitation.
Si tu aimes un peu l'informatique, ce n'est vraiment pas difficile, alors n'h\'esites pas...
Au cours d'une des install-parties qu'organise le BLL (Binet Logiciel Libre),
tu trouveras des gens compl\`etement disponibles pour r\'epondre \`a toutes tes questions,
et t'assister pour un bon d\'epart.
Pendant ces soir\'ees, des linuxiens exp\'eriment\'es viennent t'aider
\`a installer un Linux qui tourne bien sur ton ordinateur,
avec lequel tu pourras faire tout ce que tu voudras,
avec une s\'ecurit\'e, une stabilit\'e, et une r\'eactivit\'e spectaculaires.

Enfin, on te rassure, si on est un peu stressants c'est surtout pour prot\'eger le r\'eseau
des ordinateurs mal configur\'es qui permettent aux virus de se propager ;
si tu fais ce qui t'es conseill\'e dans cet infoBR, tout se passera tr\`es bien pour toi.
Profites bien de ces ann\'ees \`a l'Ecole !

\begin{flushright}
\bsc{alakazam}, Prez, pour le Binet R\'eseau 2k4
\end{flushright}