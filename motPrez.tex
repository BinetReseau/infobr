

\begin{center}
    { \Huge Le mot du Prez }
\end{center}

{   Salut \`a toi,}
\newline
\newline
{   Tu tiens entre tes mains la nouvelle version de l'InfoBR. Il s'agira de ta r\'ef\'erence pour tout ce qui concerne la configuration de ton ordinateur pour la connexion au r\'eseau, mais aussi pour l'acc\`es aux diff\'erents services du BR :
frankiz, les newsgroups, qRezix, les install-parties, ou encore la t\'el\'evision.}
\newline
\newline
Le r\'eseau est g\'er\'e par la DSI (Direction des Syst\`emes d'Information), qui poss\`ede l'infrastructure mat\'erielle et g\`ere la connexion \`a Internet,
et par le Binet R\'eseau (BR pour les intimes), pour ce qui est des services aux \'el\`eves. Tous deux investissent beaucoup de temps pour le maintenir au meilleur niveau.
Fais donc bien attention \`a respecter leurs chartes respectives et \`a ne pas abuser du r\'eseau, et tout se passera bien ; pour tout probl\`eme, adresse toi au
BR-man le plus proche.
\newline
\newline
Si tu souhaite nous rejoindre, nous serons tr\`es heureux de t'accueillir dans le binet. Rassures-toi, ce n'est pas r\'eserv\'e aux geeks : de nombreux postes ne recqui\`erent pas de connaissances tr\`es pouss\'ees, et pour le reste, nous pratiquons beaucoup l'autoformation.
\newline
\newline
Le Binet R\'eseau te souhaite de passer de bonnes ann\'ees \`a l'X.
\newline

\begin{flushright}
    \bsc{Xelnor}, Prez, pour le Binet R\'eseau 2k6
\end{flushright}
