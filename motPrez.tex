\vspace*{\stretch{5}}


\begin{center}
    { \Huge Le mot du Prez }
\end{center}
\vspace*{\stretch{1}}

Salut \`a toi
\vspace*{\stretch{1}}

Tu viens d'ouvrir l'InfoBR 2011. C'est l'ouvrage vers lequel tu te tourneras d\`es que tu voudras configurer ton ordinateur pour acc\'eder au r\'eseau de l'X, \`a Internet ou aux services du BR comme \fkz (le site des \'el\`eves). Tu y trouveras aussi la configuration de logiciels particuliers : mails, mises \`a jour de ton syst\`eme, ssh \dots
\vspace*{\stretch{1}}

C'est la DSI (Direction des Syst\`emes d'Information) de l'\'Ecole qui fournit l'acc\`es Internet. C'est elle qui g\`ere le r\'eseau et qui poss\`ede l'infrastructure mat\'erielle. C'est \'egalement elle qui se charge d'appliquer les sanctions en cas de non respect des chartes d'utilisation du r\'eseau. Le BR t'offres des services sur le r\'eseau, t'aide \`a configurer ton ordinateur et fait le lien entre les \'el\`eves et la DSI.
\vspace*{\stretch{1}}

\textbf{Le BR recrute.} \`A n'importe quelle \'epoque de l'ann\'ee, tu peux venir nous voir et nous t'accueillerons \`a bras ouverts. Il y a toujours des choses \`a faire pour am\'eliorer les services ou en d\'evelopper de nouveaux. Pas besoin de connaissances particuli\`eres en informatique, ta motivation suffira largement.
\vspace*{\stretch{1}}

\begin{flushright}
    Kangz, Prez, pour le Binet R\'eseau 2010
\end{flushright}

