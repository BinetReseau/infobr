\thispagestyle{empty}

\section*{Le mot du Prez'}

Bonjour cher TOS,

Tu tiens entre tes mains l'infoBR, ouvrage mythique qui te sauvera la vie dans le d\'edale que constituent les menus de configuration de ton ordinateur et le r\'eseau internet de l'\'Ecole. Comme il est d'usage de dire dans ces situations :  \og Don't panic \fg .

Tu trouveras entre ces pages des explications simples et claires qui te permettront de r\'egler l'acc\`es \`a internet de ton casert, ainsi que de nombreuses explications sur divers services mis en place par le Binet R\'eseau par et pour les \'el\`eves.

Je te conseille de ne pas le br\^uler trop rapidement, l'infoBR pourra t'\^etre \`a nouveau utile par la suite, lorsque que tu voudras finalement configurer ton client mail, ou lorsque suite \`a la mort pr\'ematur\'ee de ton ordinateur, tu en auras achet\'e un nouveau.

Nous n'avons n\'eanmoins pas r\'eussi \`a obtenir la capacit\'e de synth\`ese du fameux Guide bien connu, et tous les conseils que tu peux trouver n'ont pas su rentrer dans les quelques pages que voici, aussi n'h\'esite pas \`a te rendre sur \urllink{https://br.binets.fr/} si tu cherches de plus amples informations.

En cas de probl\`eme, une proc\'edure typique de r\'esolution se trouve en quatri\`eme de couverture. Si \c{c}a ne te suffit pas, tu peux \'egalement contacter le BR \`a l'adresse \newline \mail{support@eleves.polytechnique.fr}, une \^ame charitable prendra le temps d'\'etudier ton probl\`eme et de t'aider \`a le r\'esoudre.

Tu peux maintenant retirer le cran de s\^uret\'e de ta souris et t'attaquer \`a la configuration de ton ordinateur, car contrairement \`a ce que l'on pourrait croire, il ne suffit pas toujours de rentrer le code 42.
\\
\\
Le Prez'

\vfill


Tu pourras faire de cet InfoBR les usages suivants :
\begin{itemize}
\item Pour te renseigner sur le \textbf{WikiX}, sur \textbf{IRC} ou sur les autres services du BR ou de Polytechnique.org, rendez-vous page \pageref{services}
\item Pour te connecter au plus vite \`a Internet, calcule ton IP page \pageref{ip} puis consulte, selon ton syst\`eme d'exploitation, la section Windows page \pageref{windows},
la section Ubuntu page \pageref{ubuntu} ou la section Mac page \pageref{mac}.
\item Tu ne comprends rien, \c{c}a ne marche pas, tu as des questions : beaucoup de r\'eponses sont donn\'ees page \pageref{faq} et suivantes.

\end{itemize}

\begin{center}
\label{tableauIp}
\'Ecris ci-dessous les diff\'erents nombres que tu auras calcul\'es, ils te serviront plus d'une fois.
\end{center}

\begin{center}
  \begin{tabular}{|rp{5cm}|}
  \hline
  \rule[-8pt]{0pt}{24pt} \textbf{Mon adresse IP :} \ungaramond 129.104. & \\ \hline
  \rule[-8pt]{0pt}{24pt} \textbf{Ma passerelle :} \ungaramond 129.104. & \\ \hline
  \rule[-8pt]{0pt}{24pt} \textbf{Mon adresse de diffusion :} \ungaramond 129.104. & \\ \hline
  \rule[-8pt]{0pt}{24pt} \textbf{Mon masque de sous-r\'eseau :} \ungaramond 255.255. & \\ \hline
  \end{tabular}
  \label{tableau:mon_IP}
\end{center}
