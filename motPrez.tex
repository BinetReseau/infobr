\begin{center}
    { \LARGE Le mot du prez }
\end{center}

Salut \`a tous !

Voici la nouvelle version de l'InfoBR ! Ce petit fascicule t'aidera \`a d\'ecouvrir le fonctionnement du r\'eseau de l'X et \`a configurer le mieux possible ton ordinateur. Il a \'et\'e r\'edig\'e pour rendre ais\'e l'acc\`es au r\'eseau de l'\'ecole et aux services du binet r\'eseau, et \^etre compl\'ementaire aux r\'eponses d\'ej\`a \`a votre disposition en ligne. FTP, qRezix, firewall et cross-posts seront bient\^ot tes meilleurs amis !

Le r\'eseau \'el\`eve de l'\'ecole est g\'er\'e par un groupe de personnes qui, b\'en\'evolement, investissent beaucoup de temps afin que tu puisses acc\'eder \`a Internet et aux services que nous proposons aux X du campus --- d'ailleurs, on songe \`a passer aux journ\'ees de 35 heures, pour avoir le temps de tout faire. Alors respectes la charte d'utilisation de la DSI (Direction des Syst\`emes d'Information, les gourous informatiques de l'X) ainsi que celle du BR, le tout afin de profiter au mieux du r\'eseau.

Nous vous proposons aussi de d\'ecouvrir Linux. Si tu aimes un peu l'informatique, ce n'est vraiment pas difficile, alors n'h\'esites pas... Viens tout simplement \`a une install-party organis\'ee par le BLL (Binet Logiciel Libre).

Profite bien de ces ann\'ees \`a l'Ecole, et des services qui te
sont propos\'es !
\begin{flushright}
    \bsc{alakazam}, Prez, pour le Binet R\'eseau 2k4
\end{flushright}
