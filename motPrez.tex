\thispagestyle{empty}

\section*{Le mot du Prez'}

\vspace{1em}

Salut cher 2015,
\vspace{0.5em}

Tu ne mesures pas la joie que nous avons de t'accueillir sur le Plâtal ! Tous les binets s'activent pour faire en sorte que ton arrivée se passe le mieux possible. Ici, au Binet Réseau (ou BR), on t'a concocté un tout nouvel InfoBR avec les dernières instructions pour te connecter à Internet et les descriptions des derniers services que nous avons développés !

Mais tout d'abord, \textbf{qu'est-ce que le Binet Réseau} ? C'est l'association d'informatique de l'X. Son action certes discrète mais néanmoins indispensable profite aux X, aux binets, mais aussi aux PEI, doctorants et autres extra-terrestres. On développe et administre des services internes (Frankiz, Chocapix, Phoenix) hébergés sur nos propres serveurs dont on assure la maintenance, on aide tous ceux qui ont des problèmes informatiques, on héberge les sites des binets et on donne régulièrement des formations.

Pour éviter de flamber toutes tes cotiz Kès avant même que tu ne sois arrivé, nous avons limité l’InfoBR à l’essentiel. Si tu es curieux et que tu as déjà envie de te plonger plus avant dans les mystères du réseau de l’X, n’hésite pas à aller chercher des infos sur \urllink{https://wikibr.binets.fr/}

\vspace{1em}
\textbf{Le BR recrute} ! Si l'informatique t'intéresse, si tu veux apprendre à administrer des serveurs ou à développer des services webs utiles à toute la promo, n'hésite pas à nous contacter (par un des innombrables moyens que tu trouveras dans cet InfoBR) et/ou à venir à l'amphi de présentation du BR. Cette année, nous inaugurons le \emph{mentoring}. L'idée ? Même si tu n'y connais pas grand chose, nous allons te faire progresser énormément, tant que tu es motivé !

\vspace{1em}
Je ne vais pas faire plus long, je te sens impatient~! Tu peux aller page \pageref{config} pour configurer ton ordinateur et \textbf{avoir enfin accès à internet}.

\vspace{1em}
En cas de problème, une \textbf{procédure typique de résolution} est expliquée en quatrième de couverture. Si cela ne suffit pas, tu peux nous envoyer un mail à \mail{support@eleves.polytechnique.fr}. Cependant, essaye d’abord de bien tout re-vérifier auparavant, et commence par demander au geek à côté de chez toi. Il ne mord pas.

Et même s’il est de bonne volonté, le BR-man moyen trouve ça un peu abusif d’être dérangé si ton réseau ne marche pas parce que tu as écrit \texttt{polytechnqiue} au lieu de \texttt{polytechnique}.
\vspace{1.5em}

\hfill Varal7 / Victor Quach, Prez', pour le BR 2014


\vfill


Guide d'utilisation rapide :
\begin{itemize}
\item Pour te connecter à Internet, branche ton ordinateur, ouvre un navigateur et va sur \urllink{http://infobr.eleves.polytechnique.fr} ;
\item Pour te renseigner sur le \textbf{WikiX}, sur \textbf{IRC} ou sur les autres services du BR et de Polytechnique.org, rendez-vous page \pageref{services} ;
\item Ta connexion internet ne marche pas, vas page \pageref{diagnostic} ;
\item Tu ne comprends rien, tu as des questions : beaucoup de réponses sont données page \pageref{faq} et suivantes.

\end{itemize}

