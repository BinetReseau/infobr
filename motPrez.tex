\thispagestyle{empty}

\section*{Le mot du Prez'}

\vspace{1em}

Salut cher 2014,
\vspace{0.5em}

Tu tiens entre tes mains l'infoBR, ce qui signifie "Information du Binet Réseau". Dedans tu trouveras tout d'abord des explications sur la manière d'accéder à internet, une présentation des différents outils internes au plâtal ainsi que des pistes pour se sortir d'un problème informatique. Ne le brûle pas tout de suite, il pourra t'être utile plus tard !

Mais tout d'abord qu'est-ce que le Binet Réseau (ou BR) ? C'est l'association d'informatique de l'X. On développe et on administre des services internes (Frankiz, Chocapix), hébergés sur nos propres serveurs dont on assure le maintient, on aide tous ceux qui ont des problèmes informatiques, on héberge les sites des binets et on donne régulièrement des formations... En outre on a développé l'année dernière une toute nouvelle version du site des bars d'étages (Chocapix) que tu vas bientôt découvrir, et en partenariat avec la DSI on a simplifié la manière d'accéder au réseau.

Les différents services internes que l'on propose sont détaillés page \pageref{services}. 

Pour éviter de flamber toutes tes cotiz Kès avant même que tu ne sois arrivé, nous avons limité l'infoBR à l'essentiel. Si tu es curieux et que tu as envie de te plonger plus avant dans les mystères du réseau de l'X, n'hésite pas à aller chercher des infos sur \urllink{https://br.binets.fr/}.

\vspace{1em}
En cas de problème, une procédure typique de résolution est expliquée en quatrième de couverture. Si cela ne suffit pas, tu peux nous envoyer un mail à \mail{support@eleves.polytechnique.fr}. Un BRman charitable prendra le temps de s'occuper de toi.

Je ne vais pas faire plus long, je te sens impatient ! Tu peux aller page \pageref{config} pour configurer ton ordinateur et avoir enfin accès à internet.

\vspace{2.5em}
\hfill NTag / Basile Bruneau, Prez', pour le BR 2013


\vfill


Guide d'utilisation rapide :
\begin{itemize}
\item Pour te connecter à Internet, branche ton ordinateur, ouvre un navigateur et va sur \urllink{http://infobr.eleves.polytechnique.fr} ;
\item Pour te renseigner sur le \textbf{WikiX}, sur \textbf{IRC} ou sur les autres services du BR et de Polytechnique.org, rendez-vous page \pageref{services} ;
\item Tu ne comprends rien, ça ne marche pas, tu as des questions : beaucoup de réponses sont données page \pageref{faq} et suivantes.

\end{itemize}
