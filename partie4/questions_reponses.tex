\subsection{Questions-réponses}

Les questions les plus courantes sont répertoriées ici pour te faire gagner du temps !

\begin{description}

\item[J'ai une question sur l'informatique] je cherche sur le WikiBR, puis je demande \`a un BRman.

\item[Je veux voir mon pseudo quand j'ai voté à la QDJ] je définis mon pseudo sur ma fiche \linebreak trombino sur la page \menu{Administration / Compte / Profil} sur \fkz.

\item[J'ai un deuxième ordinateur, qu'est-ce que je fais ?] je demande une deuxième adresse IP en envoyant un mail à \mail{root@eleves.polytechnique.fr} en préfixant le sujet par [IP] 

\item[Je n'ai plus de réseau] je vais voir la 4\ieme de couverture.

\item[Je viens de changer de casert/section, il faudrait mettre à jour ma fiche TOL] j'envoie \linebreak un mail à \mail{tol@frankiz} avec les modifications à effectuer ainsi que la raison de ces
modifications.

\item[Mon client mail dit que \guillemotleft~l'autorité de certification est inconnue~\guillemotright ] je vais télécharger le certificat de sécurité sur \urllink{https://poly/} et je l'installe.

\item[Je ne reçois pas mes mails] je vérifie ma redirection sur \urllink{http://poly}.

\item[Si la boite mail poly est saturée] je me connecte sur :\\
\urllink{https://webmail.enseignement.polytechnique.fr/imp/login.php}, 
login : nom de famille --
password : mdp enex ; et je supprime mes mails.


\item[Je n'arrive pas à me connecter à \server{poly}] j'essaye \server{poly.polytechnique.fr}.

\item[Mon ordinateur n'a pas de nom sur le réseau] je lui en donne un gr\^ace \`a \urllink{http://dnsapp/} .

\item[Je cherche des informations sur l'Ecole] je regarde sur \urllink{http://intranet}.

\item[Je cherche à joindre une personne de l'administration] l'annuaire de l'École est sur \linebreak{} \urllink{http://annuaire/}.

\item[Je cherche le numéro de portable d'un X] \urllink{http://www.polytechnique.org}.

\item[J'aimerais être un geek moi aussi !] j'apprends par c\oe ur \urllink{9gag.com/}.

\end{description}
