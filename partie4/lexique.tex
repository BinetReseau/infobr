\subsection{Lexique}

Une grande proportion des entrées de ce lexique est à prendre au second degré. Quelques définitions sérieuses se sont glissées ci-après, sauras-tu les retrouver ?

\begin{description}
  \item[Adresse IP] Adresse de ton ordinateur sur le réseau, composée de quatre nombres compris entre 0 et 255  (\server{129.104.xxx.xxx}). Elle identifie ta machine auprès des autres utilisateurs du réseau.
  \item[Annonce] Version non-violente du \emph{mail} promo.
  \item[Arm*lle P*c*lli] Recordwoman du nombre de \emph{mails} promo adressés avec la mauvaise pièce jointe.
  \item[Avaya] Fournisseur de coupures réseau.
  \item[Binet BD] Dortoir du BR.
  \item[Blah] Complémentaire de \og chombier \fg.
  \item[BR] Binet Réseau : le binet qui s'occupe d'administrer le réseau des élèves, de développer et de maintenir le site \urllink{http://www.frankiz.net/}.
  \item[Chombier] Complémentaire de \og blah \fg.
  \item[Client] (voir serveur) : programme qui permet de se connecter à un serveur. Par exemple un client \emph{mail}
	(comme \app{Thunderbird}) ou un client FTP (comme \app{FileZilla}).
  \item[Crosspost] Paradis perdu.
  \item[DNS] \emph{Domain Name Server} : associe un nom de machine à une adresse IP, par exemple \server{frankiz} à  $129.104.201.51$.
  \item[DSI] (Direction des Systèmes d'Information) ce sont eux qui gèrent tout le matériel informatique de l'\'Ecole, ton téléphone, ton accès internet, tes \emph{mails}\ldots\ un conseil : ne joue pas au plus malin avec eux.
  \item[Faërix] Cantine du BR.
  \item[Firewall] Logiciel de protection de ton ordinateur contre les infiltrations de vers ou de pirates informatiques.
  \item[FTP] \emph{File Transfer Protocol} : Protocole réseau qui permet de s'échanger des fichiers en toute simplicit\'e.
  \item[Fr*nço*se H*mberd*t] Recordwoman du nombre de \emph{mails} adressés à la mauvaise promo.
  \item[Frankiz] Le portail \emph{web} des élèves, édité par le BR.
  \item[Fruit] Personne ayant un haut potentiel de fabrication semi-volontaire de merdasse. Permet aussi de voir des films et des séries.
  \item[Geek] Personne d'un contact social facile, très à l'aise avec les femmes, sourtout derrière une webcam. (Note de l'InfoBR-man : la copine du Prez est somptueuse.)
  \item[Git] Gestionnaire de versions arborescent. Permet de bosser à plusieurs sur le même code source de façon efficace. Fait aussi le café.
  \item[INF311 et INF321] Si tu lis ces lignes, tu es bon pour INF321. La répartition entre les deux cours est totalement arbitraire.
  \item[InfoGeek et Inf'Autiste] Respectivement INF422 : \og Composants d'un système informatique \fg, et INF423 : \og Fondements de l'informatique : logique, modèles, calculs \fg.
Deux cours courts que le BR t'encourage vivement à choisir pour ta 2A.
  \item[Iooss] Descendant du \emph{Fëandil}. Demande à un BR-man de traduire ce qu'il te raconte lorsqu'il vient réparer ton PC, ton site, ton serveur, ton robot, ton ordi bar ou
ta machine à café.
  \item[Java] Langage de programmation utilisé dans le cadre des cours d'informatique. Ses qualités sont sujettes à débat.
  \item[JTX] Quand l'intersection avec le BR est vide, ils sont dans la mouise.
  \item[\emph{Mail} promo] Cauchemar du BR dès qu'ils comportent une pièce jointe. Voir \og Arm*lle P*c*lli \fg et \og Fr*nço*se H*mberd*t \fg.
  \item[Newsgroups] Moyen de communication sûr, efficace et pratique, que la désaffection des X a malheureusement fait fermer.
  \item[SCP] \'Equivalent de SSH pour les téléchargements : permet de récupérer des fichiers depuis une machine distante.
  \item[Serveur] (Voir client) Programme qui permet d'accueillir des services. Comme par exemple le partage de fichiers, le voisinage réseau, un site \emph{web}\ldots\
  \item[Serveur] (Ce n'est pas le même qu'avant ;-)) Machine qui accueille des serveurs (là c'est celui d'au-dessus).
  \item[Serveur mandataire (\emph{proxy})] Machine qui autorise
(et du même coup restreint) les communications avec l'extérieur. Le \emph{proxy} protège tous les ordinateurs du réseau des attaques.
  \item[SSH] Connexion permettant de travailler sur une machine distante. Télécharge \app{PuTTY} !
  \item[Troll] Débat polémique sans fin permettant de déployer la mauvaise foi des deux parties.
  \item[VPN] Permet de jouer à des jeux en ligne. L'usage en étant interdit, le BR ne peut vous le recommander.
  \item[WikiBR] \urllink{https://br.binets.fr}, Le site du BR ; c'est là que tu peux trouver une description des services proposés par le BR, de ses projets, ainsi que
  d'informations plus approfondies concernant ta configuration.
  \item[Wiki*] Mot hawaïen pour \og rapide \fg, a donné son nom à la célèbre Wikipedia, encyclopédie éditable par tous.
  Par extension, d\'esigne tout site construit sur ce mod\`ele, par exemple le WikiX et le WikiBR (enfin, le BR ne te permet pas de tout \'editer sur ses pages).
  \item[Windows] Moyen le plus efficace de télécharger Linux.
\end{description}
