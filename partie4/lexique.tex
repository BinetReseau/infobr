\subsection{Lexique}


\begin{description}
  \item[Adresse IP] Adresse de ton ordinateur sur le r\'eseau, compos\'ee de quatre nombres compris entre 0 et 255  (\server{129.104.xxx.xxx}). Elle identifie ta machine aupr\`es des autres utilisateurs du r\'eseau.
  \item[Annonce] Version non-violente du \emph{mail} promo.
  
  \item[BR] Binet R\'eseau : le binet qui s'occupe d'administrer le r\'eseau des \'el\`eves, de d\'evelopper et de maintenir le site \urllink{http://www.frankiz.net/}.

  \item[Client] (voir serveur) : programme qui permet de se connecter \`a un serveur. Par exemple un client \emph{mail}
	(comme \app{Thunderbird}) ou un client FTP (comme \app{FileZilla}).
  \item[Crosspost] Paradis perdu. Voir page 11.
  \item[DNS] \emph{Domain Name Server} : associe un nom de machine \`a une adresse IP, par exemple \server{frankiz} \`a  $129.104.201.51$.
  \item[DSI] (Direction des Syst\`emes d'Information) ce sont eux qui g\`erent tout le mat\'eriel informatique de l'\'Ecole, ton t\'el\'ephone, ton acc\`es internet, tes \emph{mails}\ldots\ un conseil : ne joue pas au plus malin avec eux.

  \item[Firewall] Logiciel de protection de ton ordinateur contre les infiltrations de vers ou de pirates informatiques.
  \item[FTP] \emph{File Transfer Protocol} : Protocole r\'eseau qui permet de s'\'echanger des fichiers en toute simplicit\'e.

  \item[Frankiz] Le portail \emph{web} des \'el\`eves, \'edit\'e par le BR.
  \item[Fruit] Personne ayant un haut potentiel de fabrication semi-volontaire de merdasse. Permet aussi de voir des films et des s\'eries.

  \item[Git] Gestionnaire de versions arborescent. Permet de bosser \`a plusieurs sur le m\^eme code source de fa\c{c}on efficace. Fait aussi le caf\'e.
  
  \item[InfoGeek et Inf'Autiste] Respectivement INF422 : \og Composants d'un syst\`eme informatique \fg, et INF423 : \og Fondements de l'informatique : logique, mod\`eles, calculs \fg.
Deux cours courts que le BR t'encourage vivement \`a choisir pour ta 2A.

  \item[Iooss] Personnage mythique du BR. Certains racontent que son \^ame hante toujours le r\'eseau de l'\'Ecole...
  
   \item[Java] Langage de programmation utilis\'e dans le cadre des cours d'informatique. Ses qualit\'es sont sujettes \`a d\'ebat.
   
   
  \item[SCP] \'Equivalent de SSH pour les t\'el\'echargements : permet de r\'ecup\'erer des fichiers depuis une machine distante.
  
  \item[Serveur] (Voir client) Programme qui permet d'accueillir des services. Comme par exemple le partage de fichiers, le voisinage r\'eseau, un site \emph{web}\ldots\
  \item[Serveur] (Ce n'est pas le m\^eme qu'avant ;-)) Machine qui accueille des serveurs (l\`a c'est celui d'au-dessus).
  
  \item[Serveur mandataire (\emph{proxy})] Machine qui autorise
(et du m\^eme coup restreint) les communications avec l'ext\'erieur. Le \emph{proxy} prot\`ege tous les ordinateurs du r\'eseau des attaques.

  \item[SSH] Connexion permettant de travailler sur une machine distante. T\'el\'echarge \app{PuTTY} !
  
  \item[Troll] D\'ebat pol\'emique sans fin permettant de d\'eployer la mauvaise foi des deux parties.
  \item[VPN] Permet de jouer \`a des jeux en ligne. L'usage en \'etant interdit, le BR ne peut vous le recommander.
  
  \item[WikiBR] \urllink{https://br.binets.fr}, Le site du BR ; c'est l\`a que tu peux trouver une description des services propos\'es par le BR, de ses projets, ainsi que
  d'informations plus approfondies concernant ta configuration.
  
  \item[Wiki*] Mot hawa\"ien pour \og rapide \fg, a donn\'e son nom \`a la c\'el\`ebre Wikip\'edia, encyclop\'edie \'editable par tous.
  Par extension, d\'esigne tout site construit sur ce mod\`ele, par exemple le WikiX et le WikiBR (enfin, le BR ne te permet pas de tout \'editer sur ses pages).
  \item[Windows] Moyen le plus efficace de t\'el\'echarger Linux.
\end{description}
