\subsection{Lexique}


\begin{description}
  \item[Adresse IP] Adresse de ton ordinateur sur le réseau, composée de quatre nombres compris entre 0 et 255  (\server{129.104.xxx.xxx}). Elle identifie ta machine auprès des autres utilisateurs du réseau.
  
  \item[Annonce] Version non-violente du \emph{mail} promo.
  
  \item[BR] Binet Réseau : le binet qui s'occupe d'administrer le réseau des élèves, de développer et de maintenir le site \urllink{http://www.frankiz.net/}.

  \item[Client] (voir serveur) : programme qui permet de se connecter à un serveur. Par exemple un client \emph{mail}
	(comme \app{Thunderbird}) ou un client FTP (comme \app{FileZilla}).
%   \item[Crosspost] Paradis perdu. Voir page 11.

  \item[Chocapix] Site des bars d'étages, qui permet (en théorie) d'aider au bon fonctionnement de ceux-ci. Fonctionne bien mieux avec un lecteur de codes-barres. Accessible à \urllink{http://chocapix/}.
  
  \item[DNS] \emph{Domain Name Server} : associe un nom de machine à une adresse IP, par exemple \server{frankiz} à  $129.104.201.51$.
  
  \item[DSI] (Direction des Systèmes d'Information) ce sont eux qui gèrent tout le matériel informatique de l'École, ton téléphone, ton accès internet, tes \emph{mails}\ldots\ un conseil : ne joue pas au plus malin avec eux.

  \item[Firewall] Logiciel de protection de ton ordinateur contre les infiltrations de vers ou de pirates informatiques.
  
  \item[FTP] \emph{File Transfer Protocol} : Protocole réseau qui permet de s'échanger des fichiers en toute simplicité. profitablement remplacé par Phœnix, le cloud promo.

  \item[Frankiz] Le portail \emph{web} des élèves, édité par le BR.
  
  \item[Fruit] Personne ayant un haut potentiel de fabrication semi-volontaire de merdasse. Permet aussi de voir des films et des séries.

  \item[Git] Gestionnaire de versions arborescent. Permet de bosser à plusieurs sur le même code source de façon efficace. Fait aussi le café.
  
 % \item[InfoGeek et Inf'Autiste] Respectivement INF422 : \og Composants d'un système informatique \fg, et INF423 : \og Fondements de l'informatique : logique, modèles, calculs \fg.
%Deux cours courts que le BR t'encourage vivement à choisir pour ta 2A.

%   \item[Iooss] Personnage mythique du BR. Certains racontent que son âme hante toujours le réseau de l'École...
  
   \item[Java] Langage de programmation utilisé dans le cadre des cours d'informatique. Ses qualités sont sujettes à débat.
   
   \item[Mail Promo] Voir Spam Promo
   
  \item[Phoenix] Cloud promo, voir page \pageref{phoenix}, permet de partager des fichiers, et de stocker quelques fichiers persos accessibles partout.
  
  \item[SCP] Équivalent de SSH pour les téléchargements : permet de récupérer des fichiers depuis une machine distante.
  
  \item[Serveur] (Voir client) Programme qui permet d'accueillir des services. Comme par exemple le partage de fichiers, le voisinage réseau, un site \emph{web}\ldots\
  \item[Serveur] (Ce n'est pas le même qu'avant ;-)) Machine qui accueille des serveurs (là c'est celui d'au-dessus).
  
  \item[Serveur mandataire (\emph{proxy})] Machine qui autorise
(et du même coup restreint) les communications avec l'extérieur. Le \emph{proxy} protège tous les ordinateurs du réseau des attaques.

  \item[Spam Promo] Moyen de communication violent. À l'X, il est reservé aux informations importantes pour qu'elles ne soient pas noyées par du bruit. Voir page \pageref{mails}. Les sanctions à un abus peuvent être brutales, vous aurez été prévenus.
  
  \item[SSH] Connexion permettant de travailler sur une machine distante. Télécharge \app{PuTTY} !
  
  \item[Troll] Débat polémique sans fin permettant de déployer la mauvaise foi des deux parties.
  \item[VPN] Permet de jouer à des jeux en ligne. L'usage en étant interdit, le BR ne peut vous le recommander.
  
  %\item[WikiBR] \urllink{https://br.binets.fr}, Le site du BR ; c'est là que tu peux trouver une description des services proposés par le BR, de ses projets, ainsi que
  %d'informations plus approfondies concernant ta configuration.
  
  \item[Wiki*] Mot hawa\"ien pour \og rapide \fg, a donné son nom à la célèbre Wikipédia, encyclopédie éditable par tous.
  Par extension, désigne tout site construit sur ce modèle, par exemple le WikiX.% et le WikiBR (enfin, le BR ne te permet pas de tout éditer sur ses pages).
  
  \item[Xelnor] Divinité du BR.
  
  \item[thunder, IooNag, Aniem] Divinité mineures du BR.
  
  \item[Windows] Moyen le plus efficace de télécharger Linux.
  
  \item[Internet Explorer] Seul moyen de télécharger Firefox.
  
\end{description}
