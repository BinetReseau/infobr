\subsection{Lexique}

\begin{description}
  \item[Adresse IP] Adresse de ton ordinateur sur le réseau, composée de quatre nombres compris entre 0 et 255  (\server{129.104.xxx.xxx}). Elle identifie ta machine auprès des autres utilisateurs du réseau.
  \item[Avaya] Fournisseur de coupures réseau.
  \item[BR] Binet Réseau : le binet qui s'occupe d'administrer le réseau des élèves, de développer et de maintenir le site \urllink{http://www.frankiz.net/}.
  \item[Client] (voir serveur) : programme qui permet de se connecter à un serveur. Par exemple un client \emph{mail}
	(comme \app{Thunderbird}) ou un client FTP (comme \app{FileZilla}).
  \item[Crosspost] Paradis perdu.
  \item[DNS] \emph{Domain Name Server} : associe un nom de machine à une adresse IP, par exemple \server{frankiz} à  $129.104.201.51$.
  \item[DSI] (Direction des Systèmes d'Information) ce sont eux qui gèrent tout le matériel informatique de l'\'Ecole, ton téléphone, ton accès internet, tes \emph{mails}\ldots\ un conseil : ne joue pas au plus malin avec eux.
  \item[Firewall] Logiciel de protection de ton ordinateur contre les infiltrations de vers ou de pirates informatiques.
  \item[FTP] \emph{File Transfer Protocol} : Protocole réseau qui permet de s'échanger des fichiers en toute simplicit\'e.
%  \item[Résolution DNS] trouver l'IP associée à un nom DNS pour pouvoir se connecter à la machine concernée.
  \item[Serveur] (Voir client) Programme qui permet d'accueillir des services. Comme par exemple le partage de fichiers, le voisinage réseau, un site \emph{web}\ldots\
  \item[Serveur] (Ce n'est pas le même qu'avant ;-)) Machine qui accueille des serveurs (là c'est celui d'au-dessus).
    \item[Serveur mandataire (\emph{proxy})] Machine qui autorise (et du même coup restreint) les communications avec l'extérieur. Le \emph{proxy} protège tous les ordinateurs du réseau des attaques.
  \item[SSH] Connexion permettant de travailler sur une machine distante.
  \item[Troll] Débat polémique sans fin permettant de déployer la mauvaise foi des deux parties.
  \item[WikiBR] \urllink{https://br.binets.fr}, Le site du BR ; c'est là que tu peux trouver une description des services proposés par le BR, de ses projets, ainsi que
  d'informations plus approfondies concernant ta configuration.
  \item[Wiki*] Mot hawaïen pour \og rapide \fg, a donné son nom à la célèbre Wikipedia, encyclopédie éditable par tous.
  Par extension, d\'esigne tout site construit sur ce mod\`ele, par exemple le WikiX et le WikiBR (enfin, le BR ne te permet pas de tout \'editer sur ses pages).
\end{description}
