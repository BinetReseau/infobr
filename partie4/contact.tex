\subsection{Comment contacter le BR}
\label{contact}
Lorsque tu veux contacter le BR, il ne faut \emph{surtout pas} contacter un BRman particulier, utilise une des adresses ci-dessous (en rajoutant à chaque fois \server{.polytechnique.fr}). (Liste non exhaustive, d'autres alias existent pour des cas plus spécifiques).

\begin{description}

\item[\mail{web@eleves}] Pour contacter les personnes qui valident les annonces et activités sur \fkz, créent des groupes et des comptes.

\item[\mail{dev@eleves}] Si tu as trouvé un \emph{bug} sur \fkz ou si tu as une idée pour l'améliorer (minimodule, thème\dots).

\item[\mail{root@eleves}] Pour tout problème concernant le réseau ou les machines du BR.

\item[\mail{hebergement@eleves}] Pour obtenir un hébergement via le BR.

%\item[\mail{msdnaa@eleves}] Lorsque tu as un problème en rapport aux demandes de licence Windows (par contre les demandes se font sur \fkz).

\item[\mail{infobr@eleves}] De manière évidente lorsque tu veux faire des remarques sur l'InfoBR.

\item[\mail{bureau@eleves}] Surtout pour discuter de problématiques de binets et d'associations (ou si tu te demandes pourquoi on fait la distinction).

%\item[\mail{irc@eleves}] Tu t'es fait bannir de \#x~?

\item[\mail{support@eleves}] À utiliser seulement après avoir fait un \emph{diagnostic} (\ref{diagnostic}) si tu as un problème avec ton ordinateur. Agrémente ton mail d'informations les plus précises possible qui te permettront d'avoir une vraie réponse~: système d'exploitation (Windows, MacOS, Linux, etc.), adresse IP, configuration matérielle, et autres selon ton problème.

\item[\mail{matos@eleves}] Pour emprunter du matériel informatique au BR.

\item[\mail{br@eleves}] Lorsque tu veux contacter le BR tout entier. Il vaut mieux avoir une \emph{très bonne raison}~;).

\end{description}
