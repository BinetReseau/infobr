

%\subsection{Membres \'eminents du binet R\'eseau}

%Chaque BR-man signale quels syst\`emes d'exploitation il conna\^it.

%% Vide pour cette année

%\vspace{\stretch{1}}

%\emph{\`A force de rester enferm\'e dans leurs caserts sombres, les BR-men sont devenus trop sensibles �  la lumière pour supporter un flash d'appareil photo. Ainsi, seule une photo de dos a pu être prise.}

%\imagepos{images/br2k7.jpg}{0.9}{L'\'equipe du BR 2k7}{h}

%\vspace{\stretch{1}}



\subsubsection*{Description rapide des postes}

\begin{description}

  \item[Prez]{(\mail{prez@eleves}) Poste fictif, qui permet toutefois d'avoir
des relations bien plac\'ees.}

  \item[Trez]{(\mail{trez@eleves}) Garde-fou du Prez, elle est là pour contrôler que l'algorithme d'utilisation des sous est bien optimisé, sans aucune fuite de mémoire budgétaire.}

%  \item[relex]{Assistant du prez pour les relations avec les \emph{gens}.}

  \item[root]{(\mail{root@eleves}) Les \emph{roots} sont les administrateurs du r\'eseau. Ce sont eux qui s'\'evertuent \`a maintenir en \'etat de marche les serveurs, \`a rajouter de nouveaux services et \`a rep\'erer les boulets qui font de la merde sur le r\'eseau. S'il s'agit de g\'erer un compte de binet, utilise plut\^ot \mail{binets@eleves}.}

  \item[admin@windows] {(\mail{windows@eleves}) Administrateurs du domaine Windows. En cas de probl\`eme avec Windows, ce sont les mieux plac\'es pour t'aider ; c'est bien s\^ur  plus facile si tu es sur le domaine ! Ils gèrent le contrat
MSDNAA avec Microsoft, et ils s'occupent de distribuer les licences
aux élèves qui les demandent.}
  \item[support@windows] {(\mail{support@eleves}) SOS d\'epannage Windows, j'\'ecoute ! Pr\^ets \`a tout pour sauver une jeune demoiselle (ou un jeune \emph{gens} \`a la rigueur) en d\'etresse avec son Windows\dots }

  \item[support@mac] {(\mail{support@eleves}) C'est un poste naturellement tranquille. Qui a besoin d'\^etre d\'epann\'e sur Mac? Ah, c'est vrai : celui qui a install\'e Windows dessus en suivant les conseils de l'InfoBR... }

%  \item[devel]{(\mail{qrezix@eleves}) Joyeux programmeurs qui sont l\`a pour am\'eliorer les logiciels du BR. Leurs efforts se concentrent principalement sur le d\'eveloppement de Frankiz 3, mais ils s'occupent \'egalement de \app{qRezix} et ses \emph{plug-ins}.}

%  \item[news] {(\mail{news@eleves}) Mainteneurs du serveur de \emph{news}, ils surveillent aussi ce que tu postes et que tu respectes les r\`egles de base comme les \emph{crossposts} (marteau-th\'erapie) \mbox{;-)}}

  \item[web] {(\mail{web@eleves}) Webmestres de \fkz, ils valident les annonces et les activit\'es et surveillent le contenu du site de ton binet ou de ton site perso.}

%  \item[X-share] {(\mail{xshare@eleves}) Personne sympathique qui cherche \`a longueur de temps de nouveaux logiciels gratuits ou mieux, libres \`a proposer aux \'el\`eves dans \xshare.}

  \item[InfoBR]{Agent de transfert du savoir des roots aux habitants du plat\^al.}

  \item[support@tv]{Charg\'es de maintenir la diffusion de la t\'el\'e sur le r\'eseau. Changeurs de cha\^ine, dieux du \emph{multicast}, ils sont les amis des \emph{switches}\dots\ ou pas.}

%  \item[QDJ Master] {(\mail{qdj@eleves}) Chaque jour un nouveau dilemne sur \fkz\dots\ n'h\'esitez pas \`a faire vos propositions \`a \urllink{qdj@eleves}.}

  \item[IRC-Op]{Responsable des relations avec RezoSup. Viendez sur IRC (\urllink{http://ircserver/}) !}

%  \item[tol] {(\mail{tol@eleves}) V\'erificateur de photos, il surveille le Trombi-On-Line.}

  \item[BR-woman]{Preuve vivante que le BR n'est pas un milieu enti\`erement masculin.}

\end{description}

