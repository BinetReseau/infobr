
\subsection{Membres éminents du binet Réseau}

%% il faudrait faire une photo chombière comme les 2k7... (marteo)

%Chaque BR-man signale quels systèmes d'exploitation il connaît.

%% Vide pour cette année

%\vspace{\stretch{1}}

%\emph{ÀA force de rester enfermé dans leurs caserts sombres, les BR-men sont devenus trop sensibles �  la lumière pour supporter un flash d'appareil photo. Ainsi, seule une photo de dos a pu être prise.}

%\imagepos{images/br2k7.jpg}{0.9}{L'équipe du BR 2k7}{h}

%\vspace{\stretch{1}}

% \subsubsection*{Description rapide des postes}

Pour une liste plus complète des postes de chacun, et pour trouver un BR-man à contacter pour un problème bien particulier, va consulter la liste des membres sur le TOL !

\begin{description}
  \item[Le Prez]{Poste fictif, qui permet toutefois d'avoir des relations bien placées. Il est chargé des relations avec la DSI. Actuellement Victor ''Varal7'' Quach.}

  \item[Le Vice-Prez]{Chef des \emph{roots} et Grand Manitou du Réseau de l'X. Le plus beau, le plus grand, le plus geek. Actuellement Guillaume ''symphorien'' Girol.}

  \item[Le Vice$^2$-Prez et IRC-Op]{Responsable des relations avec RezoSup. Viendez sur IRC (\urllink{http://irc/}) ! \linebreak Actuellement Léo ''ekleog'' Gaspard.}

  \item[Trez]{Garde-fou du Prez, il est là pour contrôler que l'algorithme d'utilisation des sous est bien optimisé, sans aucune fuite de mémoire budgétaire. Actuellement Guillaume ''wilhelm'' Vizier.}

  \item[Secrétaire]{Poste fictif, il suffit de savoir taper au clavier, et encore. Actuellement Guillaume ''Sobek'', anciennement ''censure'', Didier.}

  \item[Respo matos]{Responsable du local et des serveurs qui y tournent. C'est en particulier elle qui s'occupe des problème de climatisation en été. Actuellement Pierre ''OKLR'' Auclair.}

  \item[root]{Les \emph{roots} sont les administrateurs du réseau. Ce sont eux qui s'évertuent à maintenir en état de marche les serveurs, à rajouter de
  nouveaux services et à repérer les boulets qui font de la merde sur le réseau. S'il s'agit de gérer un compte de binet, utilise plutôt \mail{binets@eleves}. Rejoins-les si tu veux participer à l'installation des nouveaux serveurs qu'on a commandés, apprendre Linux et les dernières technologies de virtualisation !}

  \item[dev]{Joyeux programmeurs qui développent les services du BR, et qui se sont attaqués à la réécriture de Frankiz 4. Rejoins-les si tu veux maitriser les dernières technos web (AngularJS, Django).}

  \item[web]{Webmestres de \fkz, ils valident les annonces et les activités et surveillent le contenu du site de ton binet ou de ton site perso.}

  \item[InfoBR]{Agent de transfert du savoir des roots aux habitants du plâtal.}
\end{description}
