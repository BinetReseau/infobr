
\subsection{Membres éminents du binet Réseau}

%% il faudrait faire une photo chombière comme les 2k7... (marteo)

%Chaque BR-man signale quels systèmes d'exploitation il connaît.

%% Vide pour cette année

%\vspace{\stretch{1}}

%\emph{ÀA force de rester enfermé dans leurs caserts sombres, les BR-men sont devenus trop sensibles �  la lumière pour supporter un flash d'appareil photo. Ainsi, seule une photo de dos a pu être prise.}

%\imagepos{images/br2k7.jpg}{0.9}{L'équipe du BR 2k7}{h}

%\vspace{\stretch{1}}

% \subsubsection*{Description rapide des postes}

Pour une liste plus complète des postes de chacun, et pour trouver un BR-man à contacter pour un problème bien particulier, va consulter la liste des membres sur le TOL !

\begin{description}

  \item[Le Prez]{Poste fictif, qui permet toutefois d'avoir des relations bien placées. Il est chargé des relations avec la DSI. Actuellement Basile ''NTag'' Bruneau.}
  
  \item[Le Vice-Prez]{Chef des \emph{roots} et Grand Manitou du Réseau de l'X. Le plus beau, le plus fort, le plus geek. Actuellement Guillaume ''Nadrieril" Boisseau.}

  \item[Trez]{Garde-fou du Prez, il est là pour contrôler que l'algorithme d'utilisation des sous est bien optimisé, sans aucune fuite de mémoire budgétaire. Actuellement Guillaume ''GHdMvlFdlF'' Hétier.}

  \item[Secrétaire]{Poste fictif, il suffit de savoir taper au clavier, et encore. Actuellement Maxence ''gbo'' Guillaud.}

 \item[Respo matos]{Responsable du local et des serveurs qui y tournent. C'est en particulier elle qui s'occupe des problème de climatisation en été. Actuellement Juliette "P12" Buet.}

  \item[root]{Les \emph{roots} sont les administrateurs du réseau. Ce sont eux qui s'évertuent à maintenir en état de marche les serveurs, à rajouter de 
  nouveaux services et à repérer les boulets qui font de la merde sur le réseau. S'il s'agit de gérer un compte de binet, utilise plutôt \mail{binets@eleves}. Rejoins-les si tu veux participer à l'installation des nouveaux serveurs qu'on a commandés, apprendre Linux et les dernières technologies de virtualisation (Docker) !}

%  \item[admin@windows] {Administrateurs du domaine Windows. En cas de problème avec Windows, ce sont les mieux placés pour t'aider ;
%  c'est bien sûr  plus facile si tu es sur le domaine ! Actuellement Alexandros "Pulsar" Hollender.}
%  
%  \item[support@windows] {SOS dépannage Windows, j'écoute ! Prêts à tout pour sauver une jeune demoiselle (ou un jeune \emph{gens} à la rigueur) en détresse avec son Windows\dots }
%
%  \item[support@mac] {C'est un poste naturellement tranquille, mais les problèmes sur Mac sont plus fréquents qu'on ne le croit... Mehdi ''Arantes'' Kouhen}
  
  %\item[support@linux] {Tu veux l'installer, l'essayer ? Tu as malencontreusement tapé \texttt{rm -rf /} dans ta console ? Ils sont là pour t'aider.}
  
  %\item[support@latex] {Pour le plaisir de voir ton commandant de compagnie s'extasier sur la mise en page de ton rapport de stage, \LaTeX~est fait pour toi !}

  \item[dev]{Joyeux programmeurs qui sont qui ont subi leurs vacances pour que Chocapix soit prêt à temps, et qui vont maintenant se lancer sur Frankiz 4 ! Rejoins-les si tu veux maitriser les dernières technos web (AngularJS, Django).}

%  \item[news] {(\mail{news@eleves}) Mainteneurs du serveur de \emph{news}, ils surveillent aussi ce que tu postes et que tu respectes les règles de base comme les \emph{crossposts} (marteau-thérapie) \mbox{;-)}}

  \item[web] {Webmestres de \fkz, ils valident les annonces et les activités et surveillent le contenu du site de ton binet ou de ton site perso.}

%  \item[X-share] {(\mail{xshare@eleves}) Personne sympathique qui cherche à longueur de temps de nouveaux logiciels gratuits ou mieux, libres à proposer aux élèves dans \xshare.}

  %\item[InfoBR]{Agent de transfert du savoir des roots aux habitants du plâtal.}



 % \item[IRC-Op]{Responsable des relations avec RezoSup. Viendez sur IRC (\urllink{http://irc/}) ! \linebreak Actuellement Quentin ''hiq'' Hibon.}

%  \item[tol] {(\mail{tol@eleves}) Vérificateur de photos, il surveille le Trombi-On-Line.}



\end{description}

