\section{Premiers pas\ldots}

\subsection{Comment calculer ton adresse IP ?}
%% Calculer l'adresse \emph{IP} de son casert

\label{calcul_ip}

Une adresse IP est une suite de quatre nombres compris entre 0 et
255 s\'epar\'es par des points ; en gros, elle identifie de mani\`ere
unique toute machine connect\'ee au r\'eseau mondial. \emph{Exemple :}
l'adresse IP de \server{frankiz} est \server{129.104.201.51}.

Les adresses IP de l'X sont toutes de la forme \server{129.104.AAA.BBB}.
Les pages suivantes t'indiquent comment calculer \server{AAA} et \server{BBB} pour que ton
ordinateur ait une adresse unique et correcte. Une erreur sur cette adresse t'emp\^eche tout simplement d'acc\'eder au r\'eseau, donc assure-toi que ton calcul est juste !


\subsubsection{En Foch, Fayolle, Joffre et Maunoury}
Tu trouveras sur ta prise r\'eseau un identifiant compos\'e d'une lettre et de trois chiffres.
On note les deux premiers caract\`eres $xx$ et les deux derniers $zz$.
\begin{itemize}
\item $zz$ sert \`a trouver les identifiants \server{BBB} par la r\`egle : \server{BBB} = $120 + zz$.

\item $xx$ sert \`a trouver ton sous-r\'eseau (\server{AAA}), ta passerelle (\emph{gateway}),
ton adresse de diffusion (\emph{broadcast}) et ton masque de sous-réseau (\emph{netmask}), selon le tableau suivant.
\end{itemize}


\begin{center}
\begin{tabular}{|>{\ungaramond}c|>{\ungaramond}c|c|c|c|}
\hline \multirow{2}{*}{$xx$} & \multirow{2}{*}{AAA} & \multirow{2}{*}{\bf Passerelle} & \bf Adresse de  & \bf Masque de  \\ 
 & & & \bf{diffusion} & \bf sous-réseau \\
\hline A0 & 208 & \multirow{4}{*}{\server{129.104.211.254}} & \multirow{4}{*}{\server{129.104.211.255}} & \multirow{16}{*}{\server{255.255.252.0}} \\ 
\cline{1-2} A1 & 209 &  &  &  \\ 
\cline{1-2} A2 & 210 &  &  &  \\ 
\cline{1-2} A3 & 211 &  &  &  \\ 
\cline{1-4}  B0 & 216  & \multirow{4}{*}{\server{129.104.219.254}} & \multirow{4}{*}{\server{129.104.219.255}} & \\ 
\cline{1-2} B1 & 217 &  &  &  \\ 
\cline{1-2} B2 & 218 &  &  &  \\ 
\cline{1-2} B3 & 219 &  &  &  \\ 
\cline{1-4}  C0 & 212  & \multirow{4}{*}{\server{129.104.215.254}} & \multirow{4}{*}{\server{129.104.215.255}} & \\ 
\cline{1-2} C1 & 213 &  &  &  \\ 
\cline{1-2} C2 & 214 &  &  &  \\ 
\cline{1-2} C3 & 215 &  &  &  \\ 
\cline{1-4}  D0 & 232  & \multirow{4}{*}{\server{129.104.235.254}} & \multirow{4}{*}{\server{129.104.235.255}} & \\ 
\cline{1-2} D1 & 233 &  &  &  \\ 
\cline{1-2} D2 & 234 &  &  &  \\ 
\cline{1-2} D3 & 235 &  &  &  \\ 
\hline
\end{tabular} 
\end{center}

\exemple{l'adresse IP associ\'ee \`a la prise $A145$ est \server{129.104.209.165} ($165 = 120 + 45$) ; sa passerelle est \server{129.104.211.254}, son adresse de
diffusion est \server{129.104.211.255} et son masque de sous-r\'eseau est \server{255.255.252.0}.}


\subsubsection{Dans les nouveaux caserts}

Ta prise r\'eseau poss\`ede un num\'ero \`a 6 chiffres de la forme $xx\ yy\ zz$. On prend $xx$ pour calculer ton sous-r\'eseau, l'adresse de ta passerelle (\emph{gateway}): \server{129.104.AAA.CCC} et l'adresse de diffusion (\emph{broadcast}): \server{129.104.AAA.EEE}. Le masque de sous-r\'eseau (\emph{netmask}) est toujours
\server{255.255.255.128}. Ensuite, tu peux d\'eterminer la partie \server{BBB} de ton IP avec $zz$ et $xx$ :


\begin{center}
\begin{tabular}{|>{\ungaramond}c|>{\ungaramond}c|>{\ungaramond}c|>{\ungaramond}c|>{\ungaramond}c|}
\hline \rule[-2ex]{0pt}{5ex}$xx$ & $AAA$ & $CCC$ & $EEE$ & $BBB$\\ 
\hline 70 & 224 & 254 & 255 & $128+zz$ \\
71 & 224 & 126 & 127 & $zz$ \\
72 & 228 & 254 & 255 & $128+zz$ \\
73 & 225 & 126 & 127 & $zz$ \\
74 & 225 & 254 & 255 & $128+zz$ \\
75 & 226 & 126 & 127 & $zz$ \\
76 & 227 & 126 & 127 & $zz$ \\
77 & 227 & 254 & 255 & $128+zz$ \\
78 & 228 & 126 & 127 & $zz$ \\
79 & 229 & 126 & 127 & $zz$ \\
80 & 226 & 254 & 255 & $128+zz$ \\ \hline
\end{tabular} 
\end{center}

\exemple{l'adresse IP associ\'ee \`a la prise $70 30 30$ est \server{129.104.224.158} ($158 = 128 + 30$) ; sa passerelle est \server{129.104.224.254}, son
adresse de diffusion est \server{129.104.224.255} et son masque de sous-r\'eseau est \server{255.255.255.128}.}

\subsubsection{Au BEM}

\newlength{\ecart}
\settowidth{\ecart}{Masque de sous-reseau}
\addtolength{\ecart}{2em}
\noindent \begin{tabular}{p{\ecart}<{\dotfill}@{}l}
  Sous-r\'eseau (\server{AAA}) & {\ungaramond 203} pour le bâtiment A ; {\ungaramond 204} pour le bâtiment D\\
  IP (\server{BBB})            & {\ungaramond 50} + les deux derniers chiffres du num\'ero de ta chambre \\
  Passerelle                   & \server{129.104.AAA.13} \\
  Masque de sous-r\'eseau     & \server{255.255.255.0} \\
    Adresse de diffusion       & \server{129.104.AAA.255} \\
\end{tabular}

\subsubsection{Le PEM}

 \noindent \begin{tabular}{p{\ecart}<{\dotfill}@{}l}
  Sous-r\'eseau (\server{AAA})           & {\ungaramond 205} \\
  IP (\server{BBB}), rez-de-chauss\'ee & {\ungaramond 15} + les deux derniers chiffres du num\'ero  de ta chambre \\
  IP (\server{BBB}), premier \'etage   & {\ungaramond 70} + les deux derniers chiffres du num\'ero de ta chambre \\
  Passerelle                             & \server{129.104.205.13} \\
  Masque de sous-r\'eseau                & \server{255.255.255.0} \\
  Adresse de diffusion                   & \server{129.104.205.255} \\
\end{tabular}

\subsubsection{Tes informations r\'eseau}
Maintenant, note ici ton adresse IP, celle de ta passerelle (\emph{gateway}) et de ton masque de sous-r\'eseau
(\emph{netmask}) et enfin ton l'adresse de diffusion (\emph{broadcast}). Rev\'erifie bien que tu ne t'es pas tromp\'e,  \c{c}a t'\'evitera de te prendre la t\^ete pendant la suite de la configuration !

\begin{center}
  \begin{tabular}{|rp{5cm}|}
  \hline
  \rule[-8pt]{0pt}{24pt} \textbf{Mon adresse IP :} \ungaramond 129.104. & \\ \hline
  \rule[-8pt]{0pt}{24pt} \textbf{Ma passerelle :} \ungaramond 129.104. & \\ \hline
  \rule[-8pt]{0pt}{24pt} \textbf{Mon adresse de diffusion :} \ungaramond 129.104. & \\ \hline
  \rule[-8pt]{0pt}{24pt} \textbf{Mon masque de sous-r\'eseau :} \ungaramond 255.255. & \\ \hline
  \end{tabular}
  \label{tableau:mon_IP}
\end{center}

% \subsubsection{IP des serveurs DNS}
%
% Le BR offre quatre serveurs DNS redondants, qui ont les IP's suivantes :
% \begin{itemize}
%   \item Serveur principal : $129.104.201.53$
%   \item Serveurs secondaires : $129.104.201.51$, $129.104.201.52$ et $129.104.201.54$
% \end{itemize}
