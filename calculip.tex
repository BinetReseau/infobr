\subsection{Comment calculer ton IP ?}
%% Calculer l'adresse \emph{IP} de son casert

\label{calcul_ip}

Une adresse IP est une suite de quatre nombres compris entre 0 et
255 s�par�s par des points ; en gros, elle identifie de mani�re
unique toute machine connect�e au r�seau mondial. \emph{Exemple :}
l'adresse IP de \server{frankiz} est \server{129.104.201.51}.

Les IP de l'X sont toutes de la forme \server{129.104.AAA.BBB}.
Les pages suivantes t'indiquent comment calculer \server{AAA} et \server{BBB} pour que ton
ordinateur ait une adresse unique et correcte.

Au cas o� deux personnes ont (par erreur ou pas) la m�me adresse,
cela implique des conflits r�seau qui font que les deux perdent
l'acc�s tant que cela n'est pas corrig�.

\subsubsection{Pour Foch, Fayolle et Maunoury}

Tu trouveras sur ta prise r�seau un identifiant compos� d'une lettre et de trois chiffres.
On note les deux premiers caract�res $xx$ et les deux derniers $zz$.

\begin{itemize}
\item $zz$ sert � trouver les identifiants \server{BBB} par la r�gle : \server{BBB} = $120 + zz$.

\item $xx$ sert � trouver ton sous-r�seau (\server{AAA}), ta passerelle,
ton adresse de \textit{broadcast} et ton masque de sous-r�seau,
selon le tableau suivant.
\end{itemize}

\vskip 12pt

\def\vsep{\vskip0.3em \relax}
\hfil \hbox{
\vrule
\valign{&\hrule#&\vsep\vfil\hbox{\quad#\quad}\vfil\vsep\cr
&  \multispan3 \vsep\vfil\hbox{\quad $xx$\quad}\vfil\vsep &&  A0&&  A1&&  A2&&  A3&&  C0&&  C1&&  C2&&  C3&&  D0&&  D1&&  D2&&  D3&\cr
\noalign{\vrule}
& \multispan3 \vsep\vfil\hbox{\quad $AAA$\quad}\vfil\vsep && 208&& 209&& 210&& 211&& 212&& 213&& 214&& 215&& 232&& 233&& 234&& 235&\cr
\noalign{\vrule}
& \multispan3 \vsep\vfil\hbox{\quad \bf Passerelle\quad}\vfil\vsep && \multispan7 \vsep\vfil\hbox{\quad 129.104.211.254\quad}\vfil\vsep&
               & \multispan7 \vsep\vfil\hbox{\quad 129.104.215.254\quad}\vfil\vsep&
               & \multispan7 \vsep\vfil\hbox{\quad 129.104.235.254\quad}\vfil\vsep&\cr
\noalign{\vrule}
&\bf Adresse &\omit& \bf de \textit{broadcast}&& \multispan7 \vsep\vfil\hbox{\quad 129.104.211.255\quad}\vfil\vsep&
               & \multispan7 \vsep\vfil\hbox{\quad 129.104.215.255\quad}\vfil\vsep&
               & \multispan7 \vsep\vfil\hbox{\quad 129.104.235.255\quad}\vfil\vsep&\cr
\noalign{\vrule}
&  \bf Masque de &\omit& \bf sous-r�seau&& \multispan{23} \vsep\vfil\hbox{\quad 255.255.252.0\quad}\vfil\vsep&\cr
}\vrule} \hfil

\vskip 6pt

\exemple{l'IP associ�e � la prise $A145$ est
\server{129.104.209.165} ($165 = 120 + 45$) ; sa passerelle est
\server{129.104.211.254}, son adresse de \textit{broadcast} est
\server{129.104.211.255} et son masque de sous-r�seau est
\server{255.255.252.0}.}

\subsubsection{Pour les nouveaux caserts}

Ta prise r�seau poss�de un num�ro � 6 chiffres de la forme $xx\ yy\
zz$. On prend $xx$ pour calculer ton sous-r�seau, l'adresse de ta
passerelle (\server{129.104.AAA.CCC}) et l'adresse de
\textit{broadcast} (\server{129.104.AAA.EEE}). Le masque de
sous-r�seau est toujours \server{255.255.255.128}. Ensuite, tu peux
d�terminer la partie \server{BBB} de ton IP avec $zz$ et $xx$ :

\vskip 12pt
\hfil \vbox{
\offinterlineskip
\hrule
\halign{&\vrule#&\strut\quad\hfil#\hfil\quad\cr
height2pt&\omit&&\omit&&\omit&&\omit&&\omit&\cr
&  $xx$&& $AAA$&& $CCC$&& $EEE$&&      $BBB$&\cr
height2pt&\omit&&\omit&&\omit&&\omit&&\omit&\cr
\noalign{\hrule}
&    70&&   224&&   254&&   255&& $128 + zz$&\cr
&    71&&   224&&   126&&   127&&       $zz$&\cr
&    72&&   228&&   254&&   255&& $128 + zz$&\cr
&    73&&   225&&   126&&   127&&       $zz$&\cr
&    74&&   225&&   254&&   255&& $128 + zz$&\cr
&    75&&   226&&   126&&   127&&       $zz$&\cr
&    76&&   227&&   126&&   127&&       $zz$&\cr
&    77&&   227&&   254&&   255&& $128 + zz$&\cr
&    78&&   228&&   126&&   127&&       $zz$&\cr
&    79&&   229&&   126&&   127&&       $zz$&\cr
&    80&&   226&&   254&&   255&& $128 + zz$&\cr
}
\hrule
} \hfil

\vskip 12pt

\exemple{l'IP associ�e � la prise $70 30 30$ est
\server{129.104.224.158} ($158 = 128 + 30$) ; sa passerelle est
\server{129.104.224.254}, son adresse de \textit{broadcast} est
\server{129.104.224.255} et son masque de sous-r�seau est
\server{255.255.255.128}.}

\subsubsection{Le BEM}
\newbox\point
\setbox\point=\hbox{.\hskip .15em}
\def\ecart{\hskip 1.5em plus 1fill\relax}
\noindent\begin{tabular}{l@{\xleaders\copy\point\ecart}l}
  Sous-r�seau (\server{AAA}) & 203 pour le b�timent A ; 204 au b�timent D \\
  IP (\server{BBB})            & 50 + les deux derniers chiffres du num�ro de ta chambre \\
  Passerelle                   & \server{129.104.AAA.13} \\
  Masque de sous-r�seau      & \server{255.255.255.0} \\
  Broadcast                    & \server{129.104.AAA.255} \\
\end{tabular}

\subsubsection{Le PEM}

\noindent\begin{tabular}{l@{\xleaders\copy\point\ecart}l}
  Sous-r�seau (\server{AAA})           & 205 \\
  IP (\server{BBB}), rez-de-chauss�e & 15 + les deux derniers chiffres du num�ro  de ta chambre \\
  IP (\server{BBB}), premier �tage   & 70 + les deux derniers chiffres du num�ro de ta chambre \\
  Passerelle                             & \server{129.104.205.13} \\
  Masque de sous-r�seau                & \server{255.255.255.0} \\
  Broadcast                              & \server{129.104.205.255} \\
\end{tabular}

\subsubsection{Tes informations r�seau}
Maintenant, note ici ton IP et les adresses IP de ta passerelle
(\trad{gateway}), de ton \trad{broadcast} et ton masque de
sous-r�seau (\trad{netmask}). Re-v�rifie bien que tu ne t'es pas
tromp�,  \c ca t'�vitera de te prendre la t�te pendant la suite de
la configuration !

\begin{center}
  \begin{tabular}{|rp{5cm}|}
  \hline
  \vrule height16pt depth8pt width0pt \textbf{Mon IP :} 129.104. & \\ \hline
  \vrule height16pt depth8pt width0pt \textbf{Ma passerelle :} 129.104. & \\ \hline
  \vrule height16pt depth8pt width0pt \textbf{Mon broadcast :} 129.104. & \\ \hline
  \vrule height16pt depth8pt width0pt \textbf{Mon masque de sous-r�seau :} 255.255. & \\ \hline
  \end{tabular}
  \label{tableau:mon_IP}
\end{center}

% \subsubsection{IP des serveurs DNS}
%
% Le BR offre quatre serveurs DNS redondants, qui ont les IP's suivantes :
% \begin{itemize}
%   \item Serveur principal : $129.104.201.53$
%   \item Serveurs secondaires : $129.104.201.51$, $129.104.201.52$ et $129.104.201.54$
% \end{itemize}
