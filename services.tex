== Commandes groupées ==
Le Binet Réseau organise tous les ans une commande groupée d'ordinateurs à la rentrée
précédant le tronc commun, début Mai environ. Cette commande est ouverte à tous les élèves,
et les réductions, qui ne sont pas nécessairement meilleures que les tarifs étudiants
classiques, sont compensées par le fait que les ordinateurs soient livrés sur le plateau.
Le Binet organise parfois, lorsqu'il y a une demande importante, des commandes groupées
de matériel informatique à d'autres moments de l'année : début 2006, il y ainsi eu une
commande d'onduleurs d'organisée, parce que le courant dans les nouveaux kaserts était
peu fiable. Ce sont cependant des événements ponctuels car lourds à organiser.
Renseignez-vous sur les brs (news) !

== Polytechnique.org ==
Pour présenter l'association polytechnique.org, rien de mieux que de citer leur site web :
  Nous avons créé une association afin de promouvoir l'image des Polytechniciens sur Internet.
  Il est important de noter que nous n'avons aucun mandat spécial pour le faire, ni non plus
  de contre-indication d'ailleurs. Cependant, les vues officielles de l'école peuvent être
  trouvées sur www.polytechnique.fr et www.polytechnique.edu. Le domaine polytechnique.org sert
  exclusivement à parler des X, élèves et anciens élèves, sur Internet par Internet.

  L'autre but est d'offrir le maximum de services de communication par Internet aux inscrits
  volontaires à notre site. Il s'agit là de favoriser la vie des promotions, des associations
  polytechniciennes (groupes X, binets, ...) et de la communauté en général.

L'association propose de nombreux services aux X, qu'ils soient ou non sur le plateau.
En général, ils sont peu connus, et pourtant souvent très utiles. Inscrivez-vous, et n'hésitez
pas à vous y connecter pour exploiter ces services, dont voici les principaux :
  des redirections mails nombreuses (adresses supplémentaires)
  des services de news comme le binet réseau, mais ouverts aux anciens, et aux non platâliens
  des contacts aisés vers les anciens, les camarades de promotion
  une newsletter, pour publier des informations de groupes X, des informations qui toucheront tous les polytechniciens
  des annonces d'événements
  des services d'hébergement pour les groupes et binets, des noms de domaine
  des listes de diffusion de mails (br2004@polytechnique.org, par exemple)

== miroirs == 
Le BR facilite pour les utilisateurs de macs la possibilité d'utiliser les logiciels faits pour le monde linux, la suite KDE comme le logiciel scilab ou subversion pour les projets communs de code. Pour cela, nous mettons à disposition des miroirs qui se trouve derrière le firewall de l'école, ce qui permet aisément et très rapidement de récupérer les paquets. Le BR propose les miroirs suivants:
	- Cygwin (Windows)
	- Debian
	- Fink (MacOS)
	- Gentoo
	- Knoppix
	- Mandriva
	- Ubuntu
	
La configuration, spécifique à chaque distribution et mise à jour régulièrement est expliquée sur le wiki du binet réseau : 
http://gwennoz/wiki/Miroir_Fink

== FedeRez ==
FedeRez est un projet étudiant qui vise à regrouper des associations de
grandes écoles et d'universités dévolues à l'informatique, aux réseaux
et aux télécommunications. FedeRez compte ainsi une quinzaine
d'associations étudiantes, nombre de celles-ci gèrent les réseaux des
campus de leur école. L'objectif de FedeRez est l'entraide, le partage
d'expérience et de connaissances, ainsi que le développement de projets
communs.
Chaque année, l'association facilite l'organisation de commandes groupées pour les associations la composant, et, depuis deux ans, organise une journée de rencontres et de conférences sur des thématiques informatiques, ToIP (téléphonie sur IP) ou THD (très haut débit) par exemple.