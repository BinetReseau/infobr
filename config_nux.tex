%$Id: config_nux.tex 145 2005-03-25 08:26:35Z myk $

\clearpage
\pagebreak

\def\distrib#1{{\bf #1}}
\def\file#1{{\it #1}}
\def\app#1{{\sl #1}}
\def\menu#1{{\sl #1}}
\def\fkz{\server{frankiz}}

% \bghdr{fond-nux}

\subsection{Configuration sous Linux}

\subsubsection{Configuration IP}
Quelle que soit ta distribution, il faut que tu d\'ecides d'un nom de machine --- ton pseudo en g\'en\'eral.
Tu as \'egalement besoin de conna�tre les informations suivantes :
\begin{description}
  \item [ton adresse IP] not\'ee dor\'enavant \server{129.104.AAA.BBB}
  \item [l'IP de ta passerelle (\emph{gateway})] not\'ee \server{129.104.GGG.CCC}
        (attention, pour Foch, Fayolle et Maunoury ce n'est pas n\'ecessairement vrai que $GGG = AAA$).
  \item [ton masque de sous-r\'eseau] \server{255.255.FFF.DDD}
  \item [ton broadcast] qui est de la forme \server{129.104.GGG.EEE} avec $EEE = 255 - DDD$.
\end{description}

Toutes les informations n\'ecessaires se trouvent page \pageref{calcul_ip}.

Bien s\^ur, l'ensemble des manipulations doit se faire en tant que root.

\paragraph{\distrib{Sous Mandriva}}

Il existe dans cette distribution une interface graphique pour configurer le r\'eseau, bien qu'il soit possible,
comme sur tout syst\`eme linux, de le faire \`a la main \`a partir des fichiers de configuration dans \file{/etc/}.

Lance \app{DrakConf} en tant que root, puis l'outil de gestion du r\'eseau.

Tu as autant de \texttt{eth0}, \texttt{eth1}, \ldots que de cartes r\'eseau d\'etect\'ees.
S\'electionne la carte sur laquelle est branch\'e ton c\^able r\'eseau, et lance l'Assistant.
Utilise la d\'etection automatique, puis s\'electionne \menu{LAN connection} et continue.
Tu dois maintenant fournir les param\`etres n\'ecessaires \`a ta connexion :

\begin{description}
	\item[Adresse IP] entre celle que tu as calcul\'ee : \server{129.104.AAA.BBB} (cf. page \pageref{calcul_ip})
	\item[Masque de sous-r\'eseau] \server{255.255.FFF.DDD} (cf. page \pageref{calcul_ip})
	\item[DHCP] laisse cette case d\'ecoch\'ee
\end{description}

Ensuite, tu dois encore entrer :
\begin{description}
	\item[Nom d'h\^ote] \url{'ton\_pseudo'.eleves.polytechnique.fr}
	\item[Serveur DNS] \server{129.104.201.53}
	\item[Passerelle] tu l'as aussi calcul\'ee : \server{129.104.GGG.CCC} (cf. page \pageref{calcul_ip})
\end{description}
 
Voil\`a, c'est bon ! Tu peux passer au paragraphe 2 !

\paragraph{\distrib{Sous Gentoo}}

Si tu n'as jamais fait de configuration r\'eseau sur ta Gentoo, il faudra certainement cr\'eer les fichiers qui suivent.

Le fichier \file{/etc/hostname} contient ton nom de machine.
Tu peux \'editer \file{/etc/hostname} avec ton \'editeur pr\'ef\'er\'e
(\app{vi} ou \app{vim}, \app{pico}, voire \app{emacs} si tu aimes),
mais la commande ci-dessous est suffisante :

\cmdline{echo 'ton\_pseudo' > /etc/hostname}

Le fichier \file{/etc/resolv.conf} d\'ecrit comment r\'esoudre les noms DNS.
La premi\`ere ligne donne le domaine sur lequel ta machine est, 
ensuite viennent les suffixes \`a utiliser par d\'efaut,
et les lignes suivantes indiquent les serveurs de noms,
ceux qui associent le nom \server{frankiz} ou \server{ton\_pseudo}
aux IP \server{129.104.201.51} ou \server{'ton\_ip'}.
Le fichier contient donc :

\noindent \cmdline{
domain eleves.polytechnique.fr\\
search eleves.polytechnique.fr polytechnique.fr\\
nameserver 129.104.201.53\\
namaserver 129.104.201.52}

Enfin, le fichier \file{/etc/conf.d/net} contient la description de l'interface r\'eseau et les informations de routage.
Il indique ton IP, ton sous-r\'eseau, ton masque de sous-r\'eseau et la passerelle pour sortir de ton sous-r\'eseau :

\noindent \cmdline{iface\_eth0="129.104.AAA.BBB broadcast 129.104.HHH.EEE $\backslash$\\
                            netmask 255.255.FFF.DDD"\\
                            gateway="eth0/129.104.GGG.CCC"}

Tu fais le grand test en rechargeant ta configuration r\'eseau par :

\cmdline{/etc/init.d/net.eth0 restart}

puis en pingant \fkz\ par exemple. Tu dois obtenir quelque chose comme :

\cmdline{root:~\# ping frankiz\\
PING frankiz (129.104.201.51) 56(84) bytes of data.\\
64 bytes from frankiz (129.104.201.51): ...}

Pour pouvoir utiliser \app{emerge} \`a travers le proxy de l'\'ecole,
il faut d\'efinir les variables d'environnement ci-dessous dans le fichier \file{/etc/make.conf} :

\cmdline{http\_proxy=http://kuzh.polytechnique.fr:8080\\
GENTOO\_MIRRORS="ftp://miroir/gentoo http://gentoo.osuosl.org/"\\
SYNC="rsync://rsync/gentoo-portage"}

Tu peux \'evidemment ajouter d'autres miroirs (s\'epar\'es par des espaces) dans ta liste
mais \url{ftp://miroir} \'etant interne, il sera toujours beaucoup plus rapide que les autres.
On te conseille ici \url{http://gentoo.osuosl.org/} qui est un miroir tr\`es complet mais tr\`es lent.

\input ubuntu.tex

\subsubsection{Configuration antivirus \footnotesize{(elle est dr\^ole celle-l\`a hein ?)}
}

\subsubsection{Configuration firewall}

La solution la plus simple pour se faire un firewall sous linux est d'utiliser les iptables.
Pour ceci la premi\`ere \'etape est d'installer le paquet \app{iptables} pour ta distribution.
Pour savoir comment configurer ton firewall pour le r\'eseau de l'X, consulte la FAQ de Frankiz.

\subsubsection{Configuration navigateur web}

%\image{partie1-config_reseau/nux_proxy_firefox}{0.66}{Configuration du proxy sous Firefox}

%\flimage{partie1-config_reseau/nux_firefox_icon}{0.12}{l}
Le BR te conseille d'utiliser \app{Firefox}, ou \app{Konqueror} (le navigateur fourni par d\'efaut avec KDE). Dans tous les cas, la seule configuration \`a mettre est celle du proxy.
Il suffit d'aller dans \menu{Edit}, \menu{Preferences} et dans l'onglet \menu{General}
cliquer sur \menu{Connection Settings} ; ensuite tu coches la case \menu{D\'etection automatique du proxy pour ce r\'eseau},
et c'est bon.

\subsubsection{Configuration mail}

%\flimage{partie1-config_reseau/nux_kmail_icon}{0.12}{l}
Le client mail le plus utilis\'e est \app{Kmail}, mais il en existe bien s\^ur d'autres comme \app{Thunderbird}.

Va dans \menu{Configuration}, \menu{Configurer Kmail}.
Choisis la rubrique \menu{R\'eseau}.
Commence par cr\'eer un nouveau compte dans l'onglet \menu{R\'eception des messages}
en cliquant sur le bouton \menu{Ajouter\ldots} et choisis le type POP3.

%\image{partie1-config_reseau/nux_config_kmail_pop}{0.66}{Configuration de la r\'eception des messages sous Kmail}

Utilise les param\`etres suivants pour configurer l'onglet \menu{G\'en\'eral} :
\begin{description}
  \item[Nom] le nom du compte, par exemple : Mails Poly
  \item[Utilisateur] rentre le login \server{poly} que t'a fourni la DSI \`a ton arriv\'ee sur le plateau
  \item[Mot de passe] idem
  \item[Serveur] \server{poly.polytechnique.fr}
  \item[Port] 995
\end{description}

Ensuite, va dans l'onglet \menu{Extras} et coche la case \menu{Utiliser SSL pour s\'ecuriser les t\'el\'echargements}.

%\image{partie1-config_reseau/nux_config_kmail_smtp}{0.66}{Configuration de l'envoi des messages sous Kmail}

Maintenant, dans l'onglet \menu{Envoi des messages} clique sur le bouton \menu{Ajouter\ldots}.
Utilise les param\`etres suivants pour le configurer :
\begin{description}
  \item[Nom] le m\^eme nom de compte que pr\'ec\'edemment
  \item[Serveur] \server{poly.polytechnique.fr}
  \item[Port] 25
\end{description}
Sinon, laisse toutes les cases d\'ecoch\'ees.

\subsubsection{Configuration news}

%\flimage{partie1-config_reseau/nux_knode_icon}{0.12}{l}
Le client news le plus utilis\'e est \app{Knode}.
Parmi les autres clients news, citons \app{Thunderbird} ou \app{slrn}.

%\image{partie1-config_reseau/nux_config_knode}{0.7}{Configuration de Knode}

Sous \app{Knode}, c'est dans le menu \menu{Configuration}, puis \menu{Configurer Knode}.
Va dans la rubrique \menu{Comptes, Forums de discussion} et cr\'ee un compte en cliquant sur \menu{Nouveau\ldots}.

Remplis l'onglet \menu{Serveur} avec les informations suivantes :
\begin{description}
  \item[Nom] ce que tu veux pour d\'ecrire ce compte, par exemple 'News Frankiz'
  \item[Serveur] \server{frankiz.polytechnique.fr}
  \item[Port] 119
\end{description}

Ensuite occupe-toi de l'onglet \menu{Identit\'e} :
\begin{description}
  \item[Nom] mets ton pseudo dans ce champ
  \item[Organisation] X, �cole Polytechnique, comme tu le sens
  \item[Adresse \'electronique] ton adresse mail, pour que les gens puissent te r\'epondre par mail.
\end{description}

Enfin, pour que \app{Knode} puisse envoyer des mails, il faut aller dans la rubrique \menu{Comptes},
sous-rubrique \menu{Courrier \'electronique}, et choisir comme serveur d'envoi de mails \server{poly.polytechnique.fr},
port 25 --- c'est exactement la m\^eme configuration SMTP que \app{Kmail}.

Si tu veux mettre une signature \`a la fin des messages que tu posteras,
il te suffit de la mettre dans l'onglet \menu{Identit\'e}.
Sur la plupart des clients la signature est interpr\'et\'ee comme ext\'erieure au message
et n'est en particulier pas incluse dans le texte cit\'e lorsque tu r\'eponds \`a un message.
Pour d\'efinir une signature \`a la main, il suffit de mettre \verb*+-- +\ (c'est \`a dire -{}-<espace>)
sur une ligne, et tout ce qui suivra cette ligne composera ta signature.

Il ne te reste plus qu'\`a t'inscrire \`a des newsgroups
(reporte-toi \`a la page \pageref{newsgroups} pour plus d'infos) et \`a poster !

Pour te connecter aux serveurs de news de Polytechnique.org, qui ont un acc\`es s\'ecuris\'e,
avec \app{Knode}, il y a une petite subtilit\'e car il ne g\`ere pas le SSL.
Il faut installer \app{stunnel} qui permet de d\'efinir une redirection SSL de port.
Dans \file{/etc/stunnel.conf} ou \file{/etc/stunnel/stunnel.conf} selon ta distribution,
mets les lignes suivantes (les trois premi\`eres y sont en principe d\'ej\`a) :

\cmdline{\# location of pid file\\
pid = /etc/stunnel/stunnel.pid\\
\\
\# user to run as\\
setuid = stunnel\\
setgid = stunnel\\
\\
\# Use it for client mode\\
client = yes\\
\\
\# sample service-level configuration\\
\\
{[}nntps{]}\\
accept  = 1119\\
connect = ssl.polytechnique.org:563\\
TIMEOUTclose = 0\\
}

Il ne te reste plus qu'\`a lancer \app{stunnel} par :

\cmdline{/etc/init.d/stunnel start}

Et tu peux ainsi lire les news de Polytechnique.org en mettant \server{localhost} comme serveur
et \server{1119} comme port.
Il faut aussi que tu coches \menu{Le serveur exige une identification}
et que tu rentres ton nom d'utilisateur \`a Polytechnique.org et ton mot de passe,
que tu peux d\'efinir sur \url{http://www.polytechnique.org/acces\_smtp.php}.

\vfill
