%$Id: config_nux.tex 145 2005-03-25 08:26:35Z myk $

\clearpage
\pagebreak

\bghdr{images/fond-nux}

%\begin{center}
%\includegraphics{images/logo_Linux}
%\end{center}

\subsection{Configuration sous Linux}

\subsubsection{Configuration IP}
Quelle que soit ta distribution, il faut que tu d\'ecides d'un nom de
machine --- ton pseudo en g\'en\'eral. Tu as \'egalement besoin de
conna\^itre les informations suivantes :
\begin{description}
  \item [ton adresse IP] not\'ee dor\'enavant \server{129.104.AAA.BBB} ;
  \item [l'IP de ta passerelle (\emph{gateway})] not\'ee \server{129.104.GGG.CCC} ;
  \item [ton masque de sous-r\'eseau] \server{255.255.FFF.DDD} ;
  \item [ton broadcast] qui est de la forme \server{129.104.GGG.EEE} avec $EEE = 255 - DDD$. \\
        \textcolor{red}{(attention, $GGG = AAA$ n'est valable que dans les nouveaux caserts, le PEM et le BEM. Pour Foch, Fayolle et Maunoury ce n'est donc pas n\'ecessairement vrai que $GGG =
        AAA$.)}.
\end{description}

Toutes les informations n\'ecessaires se trouvent page
\pageref{calcul_ip}.

Bien s\^ur, l'ensemble des manipulations doit se faire en tant que \emph{root}.


\paragraph{\distrib{Sous Gentoo}}

Tu dois d'abord configurer ta Gentoo pour utiliser le r\'eseau de l'X, donc configure le r�seau en statique avec ton adresse IP calcul\'ee pr\'ec\'edemment puis
configure la r\'esolution DNS, tu dois avoir ceci dans le fichier \file{/etc/resolv.conf} :

\noindent \cmdline{
domain eleves.polytechnique.fr\\
search eleves.polytechnique.fr polytechnique.fr\\
nameserver 129.104.201.53\\
namaserver 129.104.201.52}

\label{gentoo_mirror} Pour pouvoir utiliser \app{emerge} \`a travers le \emph{proxy} de l'\'ecole, il faut d\'efinir les variables d'environnement
ci-dessous dans le fichier \file{/etc/make.conf} :

\cmdline{http\_proxy=http://kuzh.polytechnique.fr:8080\\
ftp\_proxy=http://kuzh.polytechnique.fr:8080\\
no\_proxy=.eleves.polytechnique.fr\\
GENTOO\_MIRRORS="ftp://miroir.eleves.polytechnique.fr/gentoo http://gentoo.osuosl.org/"\\
SYNC="rsync://rsync/gentoo-portage"}

Tu peux \'evidemment ajouter d'autres miroirs (s\'epar\'es par des espaces) dans ta liste mais
\urllink{ftp://miroir} \'etant interne, il sera toujours beaucoup plus rapide que les autres. On te
conseille ici \urllink{http://gentoo.osuosl.org/} qui est un miroir tr\`es complet mais tr\`es lent.

\paragraph{Sous Ubuntu/Kubuntu}
\label{Ubuntu:IP}
Il existe deux manières de configurer tes paramètres réseaux: l'une utilise les outils graphiques de l'environnement que tu as choisi (Gnome ou KDE), l'autre utilise simplement la ligne de commande. Bien sûr, les outils graphiques ne sont qu'un intermédiaire modifiant les fichiers dont on te parle plus bas. Ils te permettent parfois d'enregistrer une configuration réseau, ce qui facilite la gestion si tu rentres souvent chez toi.

\begin{description}
\item[Configuration à l'aide des outils graphiques]
Tu peux modifier tes paramètres réseau \emph{via} l'icône présente dans ta barre des tâches symbolisant le réseau: c'est un raccourci vers \app{KNetworkManager} sous Kubuntu (KDE; l'icône représente un c\^able réseau) et vers \app{l'applet réseau} sous Ubuntu (Gnome; l'icône en haut à droite symbolisant un réseau, menu \menu{Propriétés}).

Un simple clic droit dessus affiche un menu te permettant d'ouvrir la fenêtre de configuration. \emph{Attention}, ici le réseau est à configurer en \emph{statique}; pas en automatique avec DHCP ! Rentre ensuite ton adresse IP, ta passerelle, ton adresse de diffusion et ton masque de sous-réseau.

Ensuite il faut configurer la résolution DNS, sous Kubuntu va sur l'onglet \menu{Système de noms de domaines} ou dans l'outil de configuration du réseau
(\menu{Paramètres du système}, \menu{Réseau}). Les adresses IP des serveurs DNS sont les suivantes: \server{129.104.201.53} et \server{129.104\-.201.51}, le domaine réseau
est \server{eleves.polytechnique.fr}, tu peux aussi ajouter \server{polytechnique.fr} pour accéder plus facilement aux machines des salles info. Choisis un nom de machine
(en général ton pseudo) que tu reporteras dans \app{qRezix} pour qu'il soit enregistré.


%\imagepos{images/kubuntu_config_reseau0}{0.65}{Configuration du réseau sous Kubuntu}{!ht}
%\imagepos{images/kubuntu_config_reseau1}{0.55}{Configuration de la résolution DNS sous Kubuntu}{!ht}

\noindent
  \begin{figure*}[!h]
    \begin{center}  
      \subfloat[Configuration IP]{ 
      \includegraphics[width=0.65\textwidth]{images/kubuntu_config_reseau0}} \\
      \subfloat[Configuration DNS]{ 
	\includegraphics[width=0.65 \textwidth]{images/kubuntu_config_reseau1}}
         	 \caption{Configuration réseau}
    \end{center}
  \end{figure*}
  
%%%%%%%%%%%%%%%%%%%%%%%%%%%%%%%%%%%%%%%%%%%%%%%%%%%%%%%%%%%%%%%%%%%%%%%%%%%%%%%%%%%%%%%%%%%%%%%%%%%%%%%%%%%%%%%%%%%%%%%%%%%%%%
%%%%%%%%%%%%%%%%%%%%%%%%%%%%%%%%%%%%%%%%%%%%%%%%%%%%%%%%%%%%%%%%%%%%%%%%%%%%%%%%%%%%%%%%%%%%%%%%%%%%%%%%%%%%%%%%%%%%%%%%%%%%%%
%%%%%%%%%%%%%%%%%%%%%%%%%%%%%%%%%%%%%%%%%%%%%%%%%%%%%%%%%%%%%%%%%%%%%%%%%%%%%%%%%%%%%%%%%%%%%%%%%%%%%%%%%%%%%%%%%%%%%%%%%%%%%%

\item[Configuration en ligne de commande]
 Tu peux également modifier directement les fichiers de configuration réseau comme indiqué ci-dessous avec ton éditeur de texte préféré (\app{vim}, \app{emacs}, \dots), le tout avec les droits
 administrateur évidemment! Encore une fois,  les étapes qui suivent peuvent \^etres entièrement réalisées avec les outils graphiques mentionnés plus haut.

 \begin{itemize}
 \item Le fichier \file{/etc/hostname} contient ton nom de machine. Il doit contenir uniquement: 
 \cmdline[0.85]{tonPseudo.eleves.polytechnique.fr}
  
 \item Le fichier \file{/etc/resolv.conf} décrit comment associer le nom d'une machine à une adresse IP.
 Il doit contenir: 
 \cmdline[0.85]{
 search eleves.polytechnique.fr polytechnique.fr\\
 nameserver 129.104.201.53\\
 nameserver 129.104.201.51
 }
 
 \item Le fichier \file{/etc/network/interfaces} contient entre autres ton adresse IP,
 ton sous-réseau et la passerelle pour en sortir. Ce fichier doit
 ressembler (avec éventuellement une configuration wifi à la suite\ldots,
 voir page~\pageref{wifi}) à :
  \cmdline[0.85]{
 \# The loopback network interface\\
 auto lo\\
 iface lo inet loopback\\
 \\
 \# The primary network interface\\
 auto eth0\\
 iface eth0 inet static\\
         address   129.104.AAA.BBB\\
         netmask   255.255.FFF.DDD\\
         broadcast 129.104.GGG.EEE\\
         gateway   129.104.GGG.CCC
 }
 
 \end{itemize}
 
 Ensuite il faut redémarrer ta configuration réseau (les outils graphiques devraient le faire tout seuls après validation) :
 
 \cmdline{\$ sudo /etc/init.d/networking restart}
 
 \end{description}
%%%%%%%%%%%%%%%%%%%%%%%%%%%%%%%%%%%%%%%%%%%%%%%%%%%%%%%%%%%%%%%%%%%%%%%%%%%%%%%%%%%%%%%%%%%%%%%%%%%%%%%%%%%%%%%%%%%%%%%%%%%%%%
%%%%%%%%%%%%%%%%%%%%%%%%%%%%%%%%%%%%%%%%%%%%%%%%%%%%%%%%%%%%%%%%%%%%%%%%%%%%%%%%%%%%%%%%%%%%%%%%%%%%%%%%%%%%%%%%%%%%%%%%%%%%%
%%%%%%%%%%%%%%%%%%%%%%%%%%%%%%%%%%%%%%%%%%%%%%%%%%%%%%%%%%%%%%%%%%%%%%%%%%%%%%%%%%%%%%%%%%%%%%%%%%%%%%%%%%%%%%%%%%%%%%%%%%%%%%

Une fois ta configuration réseau terminée, tu peux la tester en \emph{pinguant} \fkz (dans une console), où tu devrais voir quelque chose comme :

\cmdline{\$ ping frankiz\\
PING frankiz.eleves.polytechnique.fr (129.104.201.51) 56(84) bytes of data.\\
64 bytes from Frankiz.eleves.polytechnique.fr ...}

 \subparagraph{Configuration du gestionnaire de paquets} \label{ubuntu_mirror} Il faut désormais configurer le gestionnaire de paquets pour
qu'il utilise les miroirs du BR et non les miroirs à l'extérieur du
campus, qui sont plus lents.

Le fichier \file{/etc/apt/sources.list} liste les miroirs utilisés par le gestionnaire de paquets. Il faut commenter la première ligne (qui
correspond au CD d'installation) ainsi que toutes les lignes non commentées du fichier (qui correspondent aux miroirs extérieurs au campus) de la
façon suivante:
\cmdline{deb cdrom:[...]/ version main restricted}
devient
\cmdline{\#deb cdrom:[...]/ version main restricted}

Il faut ensuite ajouter les lignes suivantes, qui correspondent aux miroirs du BR, au \emph{début} du fichier:
\cmdline{
deb ftp://miroir/ubuntu version main restricted universe multiverse\\
deb ftp://miroir/ubuntu version-updates main restricted universe multiverse\\
deb ftp://miroir/ubuntu version-security main restricted universe multiverse
}
où \textbf{version} correspond à la version d'Ubuntu installée sur ton ordinateur. La version actuelle est \textbf{intrepid} et la précédente est \textbf{hardy}.

Tu peux aussi utiliser le dépôt suivant mais attention il contient des logiciels \emph{non supportés par l'équipe de développement d'Ubuntu} (en particulier il peut arriver que certains logiciels contiennent des erreurs):
\cmdline{deb ftp://miroir/ubuntu version-backports main restricted universe multiverse}

Le BLL (Binet Logiciels Libres) dispose par ailleurs d'un miroir non-officiel qui contient \app{qRezix} ainsi que des paquets très importants (codecs
vidéo, java,\dots) ou pas (\app{bureau 3D}, \app{Google Earth}, \dots), non inclus dans la distribution de base pour diverses raisons, en particulier légales ou
éthiques. Pour en profiter, il faut rajouter à la suite des lignes précédentes:
\cmdline{deb ftp://miroir/bll version main}


On finit par vérifier que tout fonctionne en mettant à jour la liste
des paquets disponibles:
\cmdline{\$ sudo aptitude update}

S'il n'y a pas de message d'erreur c'est que tout fonctionne bien.

\textbf{NB:} on peut aussi faire cette configuration depuis \app{synaptic} (ubuntu) ou \app{adept} (kubuntu), il suffit d'aller dans le menu faisant référence aux dépôts.


\subsubsection{Configuration antivirus}
\footnotesize{(elle est dr\^ole celle-l\`a hein ?)}

\subsubsection{Configuration du pare-feu}

La solution la plus simple pour se faire un \emph{firewall} sous linux est d'utiliser les \emph{iptables}. Pour ceci la premi\`ere \'etape est
d'installer le paquet \app{iptables} pour ta distribution. Pour savoir comment configurer ton \emph{firewall} pour le r\'eseau de l'X, consulte le Wikix.

\subsubsection{Configuration navigateur web}

\flimage{images/nux_firefox_icon}{0.12}{l} Le BR te conseille d'utiliser \app{Firefox} ou
\app{Konqueror} (le navigateur fourni par d\'efaut avec KDE). Dans tous les cas, la seule
configuration \`a effectuer est celle du \emph{proxy}. Pour \app{Firefox}, il suffit d'aller dans
\menu{Edit}, \menu{Preferences} et dans l'onglet \menu{Avanc\'e}, puis de cliquer sur l'onglet
\menu{R\'eseau}, et enfin \menu{Param\`etres de connexion} ; ensuite coche la case \menu{Adresse de
configuration automatique du proxy}, et rentre \urllink{http://frankiz/proxy.pac/}.

Sous \app{Konqueror}, cela se trouve dans le menu \menu{Configuration}, \menu{Configurer Konqueror},
dans l'onglet \menu{Serveur mandataire}. \emph{Attention}: si tu ne configures pas le proxy dans Konqueror,
les logiciels KDE (KGet, Adept, ...) n'utiliserons pas le proxy !!

\imagepos{images/nux_proxy_firefox}{0.35}{Configuration du proxy sous Firefox}{ht}

% \vfill
\pagebreak

\subsubsection{Configuration mail}

\flimage{images/nux_kmail_icon}{0.12}{l} Le client mail le plus
utilis\'e est \app{Kmail}, mais il en existe bien s\^ur d'autres comme
\app{Thunderbird}. La configuration est semblable, quel que soit le
client que tu utilises.

Va dans \menu{Configuration}, \menu{Configurer Kmail}. Choisis la
rubrique \menu{Comptes}. Commence par cr\'eer un nouveau compte dans
l'onglet \menu{R\'eception des messages} en cliquant sur le bouton
\menu{Ajouter\ldots} et choisis le type POP3.

\imagepos{images/nux_config_kmail_pop}{0.45}{Configuration de la r\'eception des messages sous Kmail}{pht}

Utilise les param\`etres suivants pour configurer l'onglet
\menu{G\'en\'eral} :
\begin{description}
  \item[Nom] le nom du compte, par exemple : Mails Poly
  \item[Utilisateur] rentre le login \server{poly} que t'a fourni la DSI \`a ton arriv\'ee sur le plateau
  \item[Mot de passe] et l\`a le mot de passe \server{poly}
  \item[Serveur] \server{poly.polytechnique.fr}
  \item[Port] 995
\end{description}

Ensuite, va dans l'onglet \menu{Extras} et coche la case
\menu{Utiliser SSL pour s\'ecuriser les t\'el\'echargements}.

\imagepos{images/nux_config_kmail_smtp}{0.45}{Configuration de l'envoi des messages sous Kmail}{pht}

Maintenant, dans l'onglet \menu{Envoi des messages} clique sur le
bouton \menu{Ajouter\ldots}. Utilise les param\`etres suivants pour le
configurer :
\begin{description}
  \item[Nom] le m\^eme nom de compte que pr\'ec\'edemment
  \item[Serveur] \server{poly.polytechnique.fr}
  \item[Port] 25
\end{description}
Sinon, laisse toutes les cases d\'ecoch\'ees.

\par Tu peux aussi configurer l'acc\`es \`a \app{l'annuaire LDAP} de l'\'ecole, sorte de carnet d'adresses en ligne qui contient les adresses mail de tout le monde sur le campus. Pour ce faire, commence par ouvrir \menu{Outils}, \menu{Carnet d'adresses}, puis va dans \menu{Configuration}, \menu{Configurer kAdressBook}, \menu{Consultation LDAP}. Clique ensuite sur \menu{Ajouter un h\^ote}, et configure comme suit:
\begin{description}
  \item[H\^ote] \server{ldap.eleves.polytechnique.fr}
  \item[Port] 389
  \item[Version de LDAP] 3
  \item[DN] dc=ldap,dc=eleves,dc=polytechnique,dc=fr
  \item[S\'ecurit\'e] Non
  \item[Identification] Anonyme
\end{description}
Une fois revenu dans \menu{Configuration LDAP}, coche la case \server{ldap.eleves.polytechnique.fr}. Tu as maintenant acc\`es au LDAP lors de la
r\'edaction de messages, avec tout au plus un red\'emarrage de \app{Kmail}. \imagepos{images/nux_config_ldap}{0.55}{Configuration de l'annuaire LDAP sous
Kmail}{pht}

\subsubsection{Configuration news}
\flimage{images/nux_knode_icon}{0.12}{l} Le client news le plus
utilis\'e est \app{Knode}. Parmi les autres clients news, citons
\app{Thunderbird} ou \app{slrn}. Ici aussi, la configuration est
presque ind\'ependante du logiciel choisi.

Sous \app{Knode}, c'est dans le menu \menu{Configuration}, puis \menu{Configurer Knode}. Va dans la rubrique \menu{Comptes, Forums de discussion} et
cr\'ee un compte en cliquant sur \menu{Ajouter\ldots}.

\imagepos{images/nux_config_knode}{0.45}{Configuration de Knode}{ht}

Remplis l'onglet \menu{Serveur} avec les informations suivantes :
\begin{description}
  \item[Nom] ce que tu veux pour d\'ecrire ce compte, par exemple 'News Frankiz'
  \item[Serveur] \server{frankiz.polytechnique.fr}
  \item[Port] 119
\end{description}
Ensuite occupe-toi de l'onglet \menu{Identit\'e} :
\begin{description}
  \item[Nom] mets ton pseudo dans ce champ
  \item[Organisation] X, \'ecole Polytechnique, comme tu le sens
  \item[Adresse \'electronique] ton adresse mail, pour que les gens puissent te r\'epondre par mail.
\end{description}

Enfin, pour que \app{Knode} puisse envoyer des mails, il faut aller
dans la rubrique \menu{Comptes}, sous-rubrique \menu{Serveur de
courrier (SMTP)}, et choisir comme serveur d'envoi de mails
\server{poly.polytechnique.fr}, port 25 --- c'est exactement la m\^eme
configuration SMTP que \app{Kmail}.

Si tu veux mettre une signature \`a la fin des messages que tu
posteras, il te suffit de la mettre dans l'onglet \menu{Identit\'e}.
Sur la plupart des clients la signature est interpr\'et\'ee comme
ext\'erieure au message et n'est en particulier pas incluse dans le
texte cit\'e lorsque tu r\'eponds \`a un message. Pour d\'efinir une
signature \`a la main, il suffit de mettre \verb*+-- +\ (c'est \`a dire
-{}-<espace>) sur une ligne, et tout ce qui suivra cette ligne
composera ta signature.

Il ne te reste plus qu'\`a t'inscrire \`a des \emph{newsgroups} (reporte-toi \`a la page \pageref{newsgroups} pour plus d'infos) et \`a poster !

Pour te connecter aux serveurs de \emph{news} de Polytechnique.org, qui ont un acc\`es s\'ecuris\'e, avec \app{Knode}, il y a une petite subtilit\'e car il
ne g\`ere pas le SSL. Il faut installer \app{stunnel} qui permet de d\'efinir une redirection SSL de port. Dans \file{/etc/stunnel.conf} ou
\file{/etc/stunnel/stunnel.conf} selon ta distribution, mets les lignes suivantes (les trois premi\`eres y sont en principe d\'ej\`a) :

\cmdline{\# location of pid file\\
pid = /etc/stunnel/stunnel.pid\\
\\
\# user to run as\\
setuid = stunnel\\
setgid = stunnel\\
\\
\# Use it for client mode\\
client = yes\\
\\
\# sample service-level configuration\\
\\
{[}nntps{]}\\
accept  = 1119\\
connect = ssl.polytechnique.org:563\\
TIMEOUTclose = 0
}

Il ne te reste plus qu'\`a lancer \app{stunnel} par :

\cmdline{/etc/init.d/stunnel start}

Et tu peux ainsi lire les news de Polytechnique.org en mettant \server{localhost} comme serveur et
\server{1119} comme port. Il faut aussi que tu coches \menu{Le serveur exige une identification} et
que tu rentres ton nom d'utilisateur \`a Polytechnique.org et ton mot de passe, que tu peux d\'efinir
sur \urllink{http://www.polytechnique.org/acces\_smtp.php}.
