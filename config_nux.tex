%$Id: config_nux.tex 145 2005-03-25 08:26:35Z myk $

\clearpage
\pagebreak

\def\distrib#1{{\bf #1}}
\def\file#1{{\it #1}}
\def\app#1{{\sl #1}}
\def\menu#1{{\sl #1}}
\def\fkz{\server{frankiz}}

% \bghdr{fond-nux}

\subsection{Configuration sous Linux}

\subsubsection{Configuration IP}
Quelle que soit ta distribution, il faut que tu d\'ecides d'un nom de machine --- ton pseudo en g\'en\'eral.
Tu as \'egalement besoin de conna�tre les informations suivantes :
\begin{description}
  \item [ton adresse IP] not\'ee dor\'enavant \server{129.104.AAA.BBB}
  \item [l'IP de ta passerelle (\emph{gateway})] not\'ee \server{129.104.GGG.CCC}
        (attention, pour Foch, Fayolle et Maunoury ce n'est pas n\'ecessairement vrai que $GGG = AAA$).
  \item [ton masque de sous-r\'eseau] \server{255.255.FFF.DDD}
  \item [ton broadcast] qui est de la forme \server{129.104.GGG.EEE} avec $EEE = 255 - DDD$.
\end{description}

Toutes les informations n\'ecessaires se trouvent page \pageref{calcul_ip}.

Bien s\^ur, l'ensemble des manipulations doit se faire en tant que root.

\paragraph{\distrib{Sous Mandrake}}

Il existe dans cette distribution une interface graphique pour configurer le r\'eseau, bien qu'il soit possible,
comme sur tout syst\`eme linux, de le faire \`a la main \`a partir des fichiers de configuration dans \file{/etc/}.

Lance \app{DrakConf} en tant que root, puis l'outil de gestion du r\'eseau.

Tu as autant de \texttt{eth0}, \texttt{eth1}, \ldots que de cartes r\'eseau d\'etect\'ees.
S\'electionne la carte sur laquelle est branch\'e ton c\^able r\'eseau, et lance l'Assistant.
Utilise la d\'etection automatique, puis s\'electionne \menu{LAN connection} et continue.
Tu dois maintenant fournir les param\`etres n\'ecessaires \`a ta connexion :

\begin{description}
	\item[Adresse IP] entre celle que tu as calcul\'ee : \server{129.104.AAA.BBB} (cf. page \pageref{calcul_ip})
	\item[Masque de sous-r\'eseau] \server{255.255.FFF.DDD} (cf. page \pageref{calcul_ip})
	\item[DHCP] laisse cette case d\'ecoch\'ee
\end{description}

Ensuite, tu dois encore entrer :
\begin{description}
	\item[Nom d'h\^ote] \url{'ton\_pseudo'.eleves.polytechnique.fr}
	\item[Serveur DNS] \server{129.104.201.53}
	\item[Passerelle] tu l'as aussi calcul\'ee : \server{129.104.GGG.CCC} (cf. page \pageref{calcul_ip})
\end{description}
 
Voil\`a, c'est bon ! Tu peux passer au paragraphe 2 !

\paragraph{\distrib{Sous Gentoo}}

Si tu n'as jamais fait de configuration r\'eseau sur ta Gentoo, il faudra certainement cr\'eer les fichiers qui suivent.

Le fichier \file{/etc/hostname} contient ton nom de machine.
Tu peux \'editer \file{/etc/hostname} avec ton \'editeur pr\'ef\'er\'e
(\app{vi} ou \app{vim}, \app{pico}, voire \app{emacs} si tu aimes),
mais la commande ci-dessous est suffisante :

\cmdline{echo 'ton\_pseudo' > /etc/hostname}

Le fichier \file{/etc/resolv.conf} d\'ecrit comment r\'esoudre les noms DNS.
La premi\`ere ligne donne le domaine sur lequel ta machine est, 
ensuite viennent les suffixes \`a utiliser par d\'efaut,
et les lignes suivantes indiquent les serveurs de noms,
ceux qui associent le nom \server{frankiz} ou \server{ton\_pseudo}
aux IP \server{129.104.201.51} ou \server{'ton\_ip'}.
Le fichier contient donc :

\noindent \cmdline{
domain eleves.polytechnique.fr\\
search eleves.polytechnique.fr polytechnique.fr\\
nameserver 129.104.201.53\\
namaserver 129.104.201.52}

Enfin, le fichier \file{/etc/conf.d/net} contient la description de l'interface r\'eseau et les informations de routage.
Il indique ton IP, ton sous-r\'eseau, ton masque de sous-r\'eseau et la passerelle pour sortir de ton sous-r\'eseau :

\noindent \cmdline{iface\_eth0="129.104.AAA.BBB broadcast 129.104.HHH.EEE $\backslash$\\
                            netmask 255.255.FFF.DDD"\\
                            gateway="eth0/129.104.GGG.CCC"}

Tu fais le grand test en rechargeant ta configuration r\'eseau par :

\cmdline{/etc/init.d/net.eth0 restart}

puis en pingant \fkz\ par exemple. Tu dois obtenir quelque chose comme :

\cmdline{root:~\# ping frankiz\\
PING frankiz (129.104.201.51) 56(84) bytes of data.\\
64 bytes from frankiz (129.104.201.51): ...}

Pour pouvoir utiliser \app{emerge} \`a travers le proxy de l'\'ecole,
il faut d\'efinir les variables d'environnement ci-dessous dans le fichier \file{/etc/make.conf} :

\cmdline{http\_proxy=http://kuzh.polytechnique.fr:8080\\
GENTOO\_MIRRORS="ftp://miroir/gentoo http://gentoo.osuosl.org/"\\
SYNC="rsync://rsync/gentoo-portage"}

Tu peux \'evidemment ajouter d'autres miroirs (s\'epar\'es par des espaces) dans ta liste
mais \url{ftp://miroir} \'etant interne, il sera toujours beaucoup plus rapide que les autres.
On te conseille ici \url{http://gentoo.osuosl.org/} qui est un miroir tr\`es complet mais tr\`es lent.

\input ubuntu.tex

\paragraph{\distrib{Sous Debian}}

Lors de l'installation d'une nouvelle Debian, le programme d'installation
te propose logiquement de rentrer ces informations de mani\`ere interactive ;
sinon, il est toujours possible de les modifier ainsi :

Le fichier \file{/etc/hostname} contient ton nom de machine.
De m\^eme que sous Gentoo, tu peux l'\'editer avec ton \'editeur pr\'ef\'er\'e ou utiliser la commande :

\cmdline{echo 'ton\_pseudo' > /etc/hostname}

Comme sous Gentoo, le fichier \file{/etc/resolv.conf} d\'ecrit comment r\'esoudre les noms DNS, il doit contenir :

\cmdline{search eleves.polytechnique.fr polytechnique.fr\\
nameserver 129.104.201.53\\
nameserver 129.104.201.51}

Enfin, le fichier \file{/etc/network/interfaces} contient la description de la boucle locale
(si tu veux faire du FTP vers toi-m\^eme par exemple :-) ), de l'interface r\'eseau et les informations de routage.
Il indique ton IP, ton sous-r\'eseau, et la passerelle pour sortir de ton sous-r\'eseau :

\cmdline{auto lo\\
iface lo inet loopback\\
\\
auto eth0\\
iface eth0 inet static\\
address 129.104.AAA.BBB\\
netmask 255.255.255.DDD\\
network 129.104.AAA.DDD\\
broadcast 129.104.AAA.EEE\\
gateway 129.104.GGG.CCC}

Toujours comme sous Gentoo, recharge ta configuration r\'eseau par :

\cmdline{/etc/init.d/networking restart}

puis essaye de pinger \fkz\ par exemple. Tu dois obtenir quelque chose comme:

\cmdline{root:~\# ping frankiz\\
PING frankiz (129.104.201.51) 56(84) bytes of data.\\
64 bytes from frankiz (129.104.201.51): ...}

Une derni\`ere chose sp\'ecifique \`a Debian, pour utiliser \app{apt-get} et \app{apt-cache}
\`a travers le proxy de l'\'ecole, il faut rajouter la ligne ci-dessous dans \file{/etc/apt/apt.conf} :

\cmdline{Acquire::http::Proxy "http://kuzh.polytechnique.fr:8080";}

Attention, le \file{apt.conf} ne g\`ere pas le protocole ftp via le proxy http.
Pour utiliser des \url{ftp://} dans ton \file{/etc/apt/sources.list}, il faut rajouter :

\cmdline{ftp\_proxy = http://kuzh:8080}

dans les param\`etres de wget (car apt utilise wget), dans les fichiers \file{/etc/wgetrc} ou \file{/root/.wgetrc}.
Tu peux aussi y mettre \`a ta guise:

\cmdline{http\_proxy = http://kuzh:8080}

Le BR maintient par ailleurs un miroir debian : tu peux l'utiliser avec le fichier \file{/etc/sources.list} suivant,
o� \texttt{[flavour]} est la saveur de ta debian, \`a choisir parmi
\texttt{stable}, \texttt{testing}, \texttt{unstable} ou \texttt{experimental}
(la ligne \texttt{debian-security} n'est n\'ecessaire que pour \texttt{stable} ou \texttt{testing}).

\cmdline{
deb ftp://miroir/debian [flavour] main contrib non-free\\
deb ftp://miroir/debian-non-US [flavour]/non-US main $\backslash$ \\ contrib non-free\\
deb ftp://miroir/debian-security [flavour]/updates main $\backslash$ \\ contrib non-free\\
deb ftp://miroir/debian-marillat [flavour] main}

\subsubsection{Configuration antivirus \footnotesize{(elle est dr\^ole celle-l\`a hein ?)}
}

\subsubsection{Configuration firewall}

La solution la plus simple pour se faire un firewall sous linux est d'utiliser les iptables.
Pour ceci la premi\`ere \'etape est d'installer le paquet \app{iptables} pour ta distribution.
Ensuite, la configuration peut para�tre d\'elicate mais en fait ce ne sont que des r\`egles simples,
il suffit de comprendre le principe.

Configurer les iptables c'est d\'efinir une s\'erie de r\`egles.
Chaque r\`egle a une forme simple, elle d\'efinit une action \`a effectuer dans des conditions donn\'ees.
Ces conditions portent sur l'origine du paquet (l'unit\'e \'el\'ementaire de ce qui circule sur le r\'eseau),
le format du paquet, le sens de circulation, \ldots

Un exemple de r\`egle serait :

\cmdline{-A INPUT -i eth0 -s 129.104.201.51 -p tcp -m tcp -{}-dport 80 -j ACCEPT}

Cette r\`egle accepte les paquets que l'on re\c{c}oit de \fkz\ (\server{129.104.201.51})
sur le port 80 qui utilisent le protocole TCP. Tout se lit clairement :
\begin{itemize}
	\item[\texttt{-A}] d\'efinit la cha�ne \`a filtrer. La cha�ne indique d'o� provient la requ\^ete : \texttt{INPUT} pour un paquet entrant, \texttt{OUTPUT} pour un paquet sortant sont les principales.
	\item[\texttt{-i}] d\'efinit l'interface r\'eseau, en gros la carte r\'eseau (\texttt{eth0} le plus souvent, mais il y a aussi \texttt{lo} pour la boucle locale - localhost).
	\item[\texttt{-s}] IP source du paquet \`a filtrer.
	\item[\texttt{-d}] IP destination du paquet \`a filtrer.
	\item[\texttt{-p}] protocole du paquet (tcp, udp, igmp, icmp sont les plus courants).
	\item[\texttt{-m}] d\'efinit un module particulier \`a utiliser pour filtrer plus pr\'ecisemment. Le plus souvent on utilise les modules tcp, udp, multiport ou state.
	\item[\texttt{-{}-dport}] port de destination du paquet (avec -m udp ou -m tcp).
	\item[\texttt{-{}-sport}] port d'origine du paquet (avec -m udp ou -m tcp).
	\item[\texttt{-{}-state}] d\'efinit l'\'etat de la connexion (n\'ecessite le module state). Les principales valeurs de test pour ce param\`etre sont \texttt{NEW} (nouvelle connexion), \texttt{ESTABLISHED} (connexion d\'ej\`a \'etablie), \texttt{RELATED} (connexion d\'ependante d'une autre).
	\item[\texttt{-j}] d\'efinit l'action \`a entreprendre (\texttt{ACCEPT} pour accepter le paquet, \texttt{REJECT} pour refuser le paquet et en informer l'emmeteur, \texttt{DROP} pour faire comme si le paquet n'existait pas, \texttt{LOG} pour logguer la r\'eception du paquet.
\end{itemize}

Tout \c{c}a c'est tr\`es bien mais tu n'as pas envie de te casser la t\^ete
\`a te faire un firewall tout seul --- on te comprend :-).
Voici un petit firewall basique pour le r\'eseau de l'X :

\noindent \cmdline{
\# Comportement par d\'efaut pour les diff\'erentes chaines\\
*filter\\
:INPUT DROP [0:0]\\
:FORWARD DROP [0:0]\\
:OUTPUT ACCEPT [0:0]\\
\\
\# Pour que tout se passe bien avec localhost\\
-A INPUT -i lo -j ACCEPT\\
\# Pour que les connexions sur les ports > 1024 soit autoris\'ees\\
\# pour les programmes qui les utilisent comme connexions\\
\# secondaires (Exemple : serveur ftp)\\
-A INPUT -p tcp -m tcp -{}-dport 1024:65535 -m state $\backslash$ \\
 -{}-state RELATED,ESTABLISHED -j ACCEPT\\
-A INPUT -p udp -m udp -{}-dport 1024:65535 -m state $\backslash$ \\
 -{}-state RELATED,ESTABLISHED -j ACCEPT\\
\\
\# Pour ouvrir les ports des utilitaires standards :\\
\# 20,21:ftp, 22:ssh, 80:http, 119:news, 5050:qRezix, 5055:Xftp\\
-A INPUT -p tcp -m multiport $\backslash$ \\
 -{}-dports 20,21,22,80,119,5050,5055 -j ACCEPT\\
% \# Pour profiter de la t\'el\'e sur le r\'eseau :\\
% -A INPUT -p upd -m udp -{}-dport 1234 -j ACCEPT\\
\\
\# Pour que les demandes de connexion vers l'ext\'erieur (impossible) ne ralentissent pas\\
-A OUTPUT -d ! 129.104.0.0/255.255.0.0 -j REJECT\\
\\
\# Pour pouvoir \^etre 'pingable'\\
-A INPUT -p icmp -j ACCEPT\\
\\
\# Fin du fichier\\
COMMIT
}

Pour activer ton firewall, tu n'as plus qu'\`a faire :

\cmdline{iptables-restore < le\_fichier\_ci\_dessus}

en rempla\c{c}ant \texttt{le\_fichier\_ci\_dessus} par le nom sous lequel tu as sauvegard\'e
ce fichier de configuration de firewall.


\subsubsection{Configuration navigateur web}

%\image{partie1-config_reseau/nux_proxy_firefox}{0.66}{Configuration du proxy sous Firefox}

%\flimage{partie1-config_reseau/nux_firefox_icon}{0.12}{l}
Le BR te conseille d'utiliser \app{Firefox}, qui est maintenant tout \`a fait op\'erationnel.
Il est beaucoup plus stable que \app{Konqueror} --- le navigateur fourni par d\'efaut avec KDE ---
surtout sur les distributions r\'eput\'ees instables.
Dans tous les cas, la seule configuration \`a mettre est celle du proxy.
Il suffit d'aller dans \menu{Edit}, \menu{Preferences} et dans l'onglet \menu{General}
cliquer sur \menu{Connection Settings} ; ensuite tu coches la case \menu{D\'etection automatique du proxy pour ce r\'eseau},
et c'est bon.

\subsubsection{Configuration mail}

%\flimage{partie1-config_reseau/nux_kmail_icon}{0.12}{l}
Le client mail le plus utilis\'e est \app{Kmail}, mais il en existe bien s\^ur d'autres comme \app{Thunderbird}.

Va dans \menu{Configuration}, \menu{Configurer Kmail}.
Choisis la rubrique \menu{R\'eseau}.
Commence par cr\'eer un nouveau compte dans l'onglet \menu{R\'eception des messages}
en cliquant sur le bouton \menu{Ajouter\ldots} et choisis le type POP3.

%\image{partie1-config_reseau/nux_config_kmail_pop}{0.66}{Configuration de la r\'eception des messages sous Kmail}

Utilise les param\`etres suivants pour configurer l'onglet \menu{G\'en\'eral} :
\begin{description}
  \item[Nom] le nom du compte, par exemple : Mails Poly
  \item[Utilisateur] rentre le login \server{poly} que t'a fourni la DSI \`a ton arriv\'ee sur le plateau
  \item[Mot de passe] idem
  \item[Serveur] \server{poly.polytechnique.fr}
  \item[Port] 995
\end{description}

Ensuite, va dans l'onglet \menu{Extras} et coche la case \menu{Utiliser SSL pour s\'ecuriser les t\'el\'echargements}.

%\image{partie1-config_reseau/nux_config_kmail_smtp}{0.66}{Configuration de l'envoi des messages sous Kmail}

Maintenant, dans l'onglet \menu{Envoi des messages} clique sur le bouton \menu{Ajouter\ldots}.
Utilise les param\`etres suivants pour le configurer :
\begin{description}
  \item[Nom] le m\^eme nom de compte que pr\'ec\'edemment
  \item[Serveur] \server{poly.polytechnique.fr}
  \item[Port] 25
\end{description}
Sinon, laisse toutes les cases d\'ecoch\'ees.

\subsubsection{Configuration news}

%\flimage{partie1-config_reseau/nux_knode_icon}{0.12}{l}
Le client news le plus utilis\'e est \app{Knode}.
Parmi les autres clients news, citons \app{Thunderbird} ou \app{slrn}.

%\image{partie1-config_reseau/nux_config_knode}{0.7}{Configuration de Knode}

Sous \app{Knode}, c'est dans le menu \menu{Configuration}, puis \menu{Configurer Knode}.
Va dans la rubrique \menu{Comptes, Forums de discussion} et cr\'ee un compte en cliquant sur \menu{Nouveau\ldots}.

Remplis l'onglet \menu{Serveur} avec les informations suivantes :
\begin{description}
  \item[Nom] ce que tu veux pour d\'ecrire ce compte, par exemple 'News Frankiz'
  \item[Serveur] \server{frankiz.polytechnique.fr}
  \item[Port] 119
\end{description}

Ensuite occupe-toi de l'onglet \menu{Identit\'e} :
\begin{description}
  \item[Nom] mets ton pseudo dans ce champ
  \item[Organisation] X, �cole Polytechnique, comme tu le sens
  \item[Adresse \'electronique] ton adresse mail, pour que les gens puissent te r\'epondre par mail.
\end{description}

Enfin, pour que \app{Knode} puisse envoyer des mails, il faut aller dans la rubrique \menu{Comptes},
sous-rubrique \menu{Courrier \'electronique}, et choisir comme serveur d'envoi de mails \server{poly.polytechnique.fr},
port 25 --- c'est exactement la m\^eme configuration SMTP que \app{Kmail}.

Si tu veux mettre une signature \`a la fin des messages que tu posteras,
il te suffit de la mettre dans l'onglet \menu{Identit\'e}.
Sur la plupart des clients la signature est interpr\'et\'ee comme ext\'erieure au message
et n'est en particulier pas incluse dans le texte cit\'e lorsque tu r\'eponds \`a un message.
Pour d\'efinir une signature \`a la main, il suffit de mettre \verb*+-- +\ (c'est \`a dire -{}-<espace>)
sur une ligne, et tout ce qui suivra cette ligne composera ta signature.

Il ne te reste plus qu'\`a t'inscrire \`a des newsgroups
(reporte-toi \`a la page \pageref{newsgroups} pour plus d'infos) et \`a poster !

Pour te connecter aux serveurs de news de Polytechnique.org, qui ont un acc\`es s\'ecuris\'e,
avec \app{Knode}, il y a une petite subtilit\'e car il ne g\`ere pas le SSL.
Il faut installer \app{stunnel} qui permet de d\'efinir une redirection SSL de port.
Dans \file{/etc/stunnel.conf} ou \file{/etc/stunnel/stunnel.conf} selon ta distribution,
mets les lignes suivantes (les trois premi\`eres y sont en principe d\'ej\`a) :

\cmdline{\# location of pid file\\
pid = /etc/stunnel/stunnel.pid\\
\\
\# user to run as\\
setuid = stunnel\\
setgid = stunnel\\
\\
\# Use it for client mode\\
client = yes\\
\\
\# sample service-level configuration\\
\\
{[}nntps{]}\\
accept  = 1119\\
connect = ssl.polytechnique.org:563\\
TIMEOUTclose = 0\\
}

Il ne te reste plus qu'\`a lancer \app{stunnel} par :

\cmdline{/etc/init.d/stunnel start}

Et tu peux ainsi lire les news de Polytechnique.org en mettant \server{localhost} comme serveur
et \server{1119} comme port.
Il faut aussi que tu coches \menu{Le serveur exige une identification}
et que tu rentres ton nom d'utilisateur \`a Polytechnique.org et ton mot de passe,
que tu peux d\'efinir sur \url{http://www.polytechnique.org/acces\_smtp.php}.

\vfill
