\vfill{}

\addtolength{\parskip}{.8em}

\section*{Avertissement}

Tu tiens entre tes mains l'InfoBR, document pr�cieux qui te permettra de te connecter facilement --- enfin, on l'esp�re --- au r�seau. Nous te conseillons gentiment d'�viter de le paumer sous un meuble ; il pourra te resservir le jour o� ton ordi te cr�vera mis�rablement entre les mains. Surtout si on te r�pond que la solution se trouve � telle page :p

Si tu rencontres un probl�me, la proc�dure � suivre pour le r�soudre est expliqu�e en quatri�me de couverture. Bien s�r, si tu as un vrai probl�me, tu peux appeler un des membres du binet. La liste est � l'int�rieur. Nous sommes l� pour te rendre service.

Mais essaye d'abord de bien tout re-v�rifier avant de le faire, et si possible, adresse toi d'abord � quelqu'un de ton �tage qui s'y conna�t. En effet le BR-man moyen, bien que de bonne volont�, appr�cie moyennement d'�tre d�rang� si ton r�seau ne marche pas parce que tu as �crit \texttt{polytechnoque} au lieu de \texttt{polytechnique}.

Ceci dit, en avant pour la configuration !

\addtolength{\parskip}{-.8em}

\setlength{\parindent}{2em}

\vfill{}