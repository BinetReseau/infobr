%note pour les générations suivantes : ceci est la version 2012 de l'infoBR par les infoBRmen 2011 : Arcos, gerard, Thunder ainsi que Chantal pour les couv. 


%La structure générale du document date de 2009 au moins, nous n'y avons pas touché. Elle est très pratique pour ne pas avoir un .tex énorme dans lequel on ne retrouve rien. Si suite à une erreur quelconque dans un des fichiers (typiquement : une balise de fin de liste oubliée), la compilation mène à une erreur et ne termine pas, corriger l'erreur dans le fichier concerné et relancer la compilation. Le fichier peut alors refuser de compiler (affiche une erreur à la ligne \begin{document}. Balancer le fichier .aux à la poubelle règle le problème.


%Le gros changement dans cette version est la baisse drastique du nombre de pages, qui passent de 42 pages pour la version 2011 à 28 pages pour cette version 2012. La principale raison de ce changement est le coût d'impression. On ne peut pas demander à la promo de payer des milliers d'euros pour des pages complètement inutiles ou qui ne sont lues que par 3 personnes. L'autre est qu'un document aussi long décourageait toute lecture. Nous avons donc viré pas mal de trucs, en essayant au maximum de mettre les info concernées sur le wikiBR et d'inciter les gens à y aller. 


% il peut rester des restes de cet élagage, nous avons commenté certains appels à des fichiers  ou morceaux de texte plut�t que de tout effacer.

%Note du BR 2012: nous avons réintroduit la section sur Frankiz, pour essayer de maximiser le nombre de gens qui sont au courant des diverses possibilités du site.
%La section est longue, mais le nombre de pages devant être multiple de 4 elle ne change pas le nombre de pages total (28).
%Nous avons aussi tout passé en niveaux de gris pour baisser encore les coûts d'impression

% -------------------- Header --------------------
\documentclass[12pt, a4paper, twoside]{article}
\usepackage{preambuleInfoBR}
\usepackage[utf8]{inputenc}

\title{InfoBR 2013}
\author{Binet Réseau 2k13}
\date{ }
\usepackage{garamond}



\begin{document}

\garamond
\setlength{\parskip}{0pt plus 2ex}
 



% ------------------------------ Exergue ------------------------------
% on peut l'enlever, selon le nombre de pages à atteindre

%\thispagestyle{empty}
\vspace*{\stretch{1}}

\begin{center}
\begin{Huge}
InfoBR 2010
\end{Huge}
\end{center}

\vspace*{\stretch{1}}

\begin{flushright}
\begin{large} 
 {\fontfamily{pzc} \selectfont
\`A mes illustres pr\'ed\'ecesseurs:\\
\smallskip
mYk, bernardofpc, questor, eracil, j-philippe, Lad \\ }
\end{large}
\vspace{1cm}
{Palaiseau, le \today \\
\medskip
 Geogeo, InfoBR-man 2k9  }
\end{flushright} 

%\vspace*{\stretch{2}}


% parfois il faut faire des choses moches. n\'ecessit\'e fait loi
\newpage
\thispagestyle{empty}
\vspace*{1cm}
\newpage


 
%\newpage

% -------------------- Le mot du prez: ----- 'motPrez.tex' --------------------
%\thispagestyle{empty}
%\setcounter{page}{1}
%\markright{Mot du prez}


%\input{motPrez2010}


%note : cette page a été supprimée dans la version 2012 et le mot du prez a fusionné avec la partie avertissement (cf la section suivante).

\newpage



% -------------------- Le mot du redacteur: ----- 'avertissement.tex' --------------------

\markright{Introduction}

\thispagestyle{empty}

\section*{Le mot du Prez'}

Bonjour cher TOS,

Tu tiens entre tes mains l'infoBR, ouvrage mythique qui te sauvera la vie dans le d\'edale que constituent les menus de configuration de ton ordinateur et le r\'eseau internet de l'\'Ecole. Comme il est d'usage de dire dans ces situations :  \og Don't panic \fg .

Tu trouveras entre ces pages des explications simples et claires qui te permettront de r\'egler l'acc\`es \`a internet de ton casert, ainsi que de nombreuses explications sur divers services mis en place par le Binet R\'eseau par et pour les \'el\`eves.

Je te conseille de ne pas le br\^uler trop rapidement, l'infoBR pourra t'\^etre \`a nouveau utile par la suite, lorsque que tu voudras finalement configurer ton client mail, ou lorsque suite \`a la mort pr\'ematur\'ee de ton ordinateur, tu en auras achet\'e un nouveau.

Nous n'avons n\'eanmoins pas r\'eussi \`a obtenir la capacit\'e de synth\`ese du fameux Guide bien connu, et tous les conseils que tu peux trouver n'ont pas su rentrer dans les quelques pages que voici, aussi n'h\'esite pas \`a te rendre sur \urllink{https://br.binets.fr/} si tu cherches de plus amples informations.

En cas de probl\`eme, une proc\'edure typique de r\'esolution se trouve en quatri\`eme de couverture. Si \c{c}a ne te suffit pas, tu peux \'egalement contacter le BR \`a l'adresse \newline \mail{support@eleves.polytechnique.fr}, une \^ame charitable prendra le temps d'\'etudier ton probl\`eme et de t'aider \`a le r\'esoudre.

Tu peux maintenant retirer le cran de s\^uret\'e de ta souris et t'attaquer \`a la configuration de ton ordinateur, car contrairement \`a ce que l'on pourrait croire, il ne suffit pas toujours de rentrer le code 42.
\\
\\
Le Prez'

\vfill


Tu pourras faire de cet InfoBR les usages suivants :
\begin{itemize}
\item Pour te renseigner sur le \textbf{WikiX}, sur \textbf{IRC} ou sur les autres services du BR ou de Polytechnique.org, rendez-vous page \pageref{services}
\item Pour te connecter au plus vite \`a Internet, calcule ton IP page \pageref{ip} puis consulte, selon ton syst\`eme d'exploitation, la section Windows page \pageref{windows},
la section Ubuntu page \pageref{ubuntu} ou la section Mac page \pageref{mac}.
\item Tu ne comprends rien, \c{c}a ne marche pas, tu as des questions : beaucoup de r\'eponses sont donn\'ees page \pageref{faq} et suivantes.

\end{itemize}

\begin{center}
\label{tableauIp}
\'Ecris ci-dessous les diff\'erents nombres que tu auras calcul\'es, ils te serviront plus d'une fois.
\end{center}

\begin{center}
  \begin{tabular}{|rp{5cm}|}
  \hline
  \rule[-8pt]{0pt}{24pt} \textbf{Mon adresse IP :} \ungaramond 129.104. & \\ \hline
  \rule[-8pt]{0pt}{24pt} \textbf{Ma passerelle :} \ungaramond 129.104. & \\ \hline
  \rule[-8pt]{0pt}{24pt} \textbf{Mon adresse de diffusion :} \ungaramond 129.104. & \\ \hline
  \rule[-8pt]{0pt}{24pt} \textbf{Mon masque de sous-r\'eseau :} \ungaramond 255.255. & \\ \hline
  \end{tabular}
  \label{tableau:mon_IP}
\end{center}



\newpage

\renewcommand{\contentsname}{Sommaire}
\bghdr{images/fond-infobr} \markright{Table des matières}

\tableofcontents

\newpage

% -------------------- Premiers pas : --------------------
\section{Premiers pas\ldots} \markright{Pour bien commencer}

% -------------------- Les differentes configs --------------------

\markright{Configuration sous Windows} \label{windows}
\bghdr{images/fond-win}

%\begin{center}
%\includegraphics{images/logo_Windows}
%\end{center}


\subsection{Configuration sous Microsoft Windows}

%Cette section décrit comment configurer un ordinateur tournant sous Windows XP, Windows Vista, Windows 7 ou Windows 8. Si tu possèdes une autre version de Windows,
%nous t'invitons à  regarder directement la section sur les licences MSDNAA en page \pageref{msdnaa}\dots ou alors à  te débrouiller ! ;-)
%
%\subsubsection{Configuration IP}
%
%\begin{itemize}
%
%\item \textbf{Sous Windows XP :} va dans le \menu{Menu Démarrer}, \menu{Panneau de configuration} et double-clique sur \menu{Connexions réseau} puis sur \menu{Connexion au réseau local}. Clique enfin sur \menu{Propriétés}.
%
%\item \textbf{Sous Windows Vista :} va dans le \menu{Menu Démarrer}, \menu{Panneau de configuration}, \menu{Réseau et Internet}, \menu{Centre réseau et partage}. Là, dans le menu à  gauche, clique sur \menu{Gérer les connexions réseau}, puis clique droit sur \menu{Connexion au réseau local} et enfin  \menu{Propriétés}~\footnote{ÀA ce stade, ainsi qu'à  plusieurs autres étapes de ce tutoriel, Windows Vista doit normalement t'afficher un message te demandant de confirmer l'action que tu viens d'effectuer. Donc tu confirmes, et cela à  chaque fois !}.
%
%\item \textbf{Sous Windows 7 et Windows 8 :} va dans le \menu{Menu Démarrer} (Windows 7) ou le \menu{Menu Paramètres} (Windows 8), \menu{Panneau de configuration}, \menu{Réseau et Internet}, \menu{Centre réseau et partage}, \menu{Modifier les paramètres de la carte}. Puis clique droit sur \menu{Connexion au réseau local} et enfin  \menu{Propriétés}.
%
%\end{itemize}
%
%
%
%%\flimage{images/win_connexion_icone}{0.15}{l} Va dans le \menu{Menu
%%Démarrer}, \menu{Panneau de configuration} et double-clique sur
%%\menu{Connexions réseau} puis sur \menu{Connexion au réseau local}.
%%Clique enfin sur \menu{Propriétés}.\\
%
%%Dans cette fenêtre, coche les trois cases \menu{Client pour les
%%réseaux Microsoft}, \menu{Partage de fichiers} et \menu{Protocole
%%Internet (TCP/IP)}:
%
%\imagepos{images/win_config_connexion2}{0.5}{Configurer la connexion au réseau local}{!h}
%
%
%
%%\imageref{images/win_config_ip}{0.5}{Configuration IP --- Propriétés de protocole Internet (TCP/IP)}{!ht}{config:win:IP1}
%%%%%\imageref{images/win_config_ip2}{0.71}{Configuration de la connexion
%%au réseau local et propriétés du TCP/IP}{!ht}{config:win:IP1}
%
%Sélectionne ensuite la ligne \menu{Protocole Internet Version 4 (TCP/IPv4)}~\footnote{\menu{Protocole Internet (TCP/IP)} pour certaines versions de Windows XP.},
%puis clique sur le bouton \menu{Propriétés} qui vient de se
%dégriser. Tu tombes alors sur l'écran de configuration de ta
%connexion vers l'extérieur.
%
%\noindent
%  \begin{figure*}[!h]
%    \begin{center}  
%      \subfloat[Configuration IP --- Propriétés de protocole Internet (TCP/IP)]{ 
%      \includegraphics[width=0.48\textwidth]{images/win_config_ip} \label{config:win:IP1}}
%      \hspace{\stretch{1}}
%      \subfloat[Configuration DNS]{ 
%         \includegraphics[width=0.48 \textwidth]{images/win_config_dns2} \label{config:win:IP2}}
%             \caption{Configuration réseau}
%    \end{center}
%  \end{figure*}
%
%
%
%%% \newpage
%Coche alors les cases \menu{Utiliser l'adresse IP suivante} et \menu{Utiliser l'adresse de serveur DNS suivante} et remplis les cinq champs d'adresse IP. Tu trouveras toutes les valeurs d'adresse IP nécessaires pour la configuration en page~\pageref{tableau:mon_IP} ; aide-toi de la capture d'écran~\ref{config:win:IP1} pour les placer. Si une partie d'adresse IP est blanche sur cette capture, c'est qu'elle t'est personnelle et que tu dois la calculer !
%Clique ensuite sur le bouton \menu{Avancé}, puis sur l'onglet
%\menu{DNS} en haut.
%Il n'y a plus qu'à  remplir les différents champs comme sur la
%capture d'écran suivante, avec le bouton \menu{Ajouter} et les
%flèches pour réordonner les éléments.
%
%
%%\subsubsection{Le domaine Windows}
%
%%\paragraph{Qu'est ce que c'est ?}
%%Le domaine Windows est un système d'automatisation de la
%%configuration de plusieurs ordinateurs sous Windows situés sur le
%%même réseau. En fait, c'est un outil d'administration, conçu par
%%exemple pour des entreprises où un service informatique doit gérer
%%de nombreuses machines; il permet d'appliquer des modifications de
%%configuration à  toutes les machines du domaine directement depuis un
%%serveur. Le BR possède un serveur dédié au domaine Windows,
%%\server{enez}.
%
%%Le domaine met à  jour automatiquement Windows et l'antivirus à  partir d'\server{enez} (très rapide car tu n'as pas besoin de récupérer des fichiers
%%en dehors de l'école!). Il configure le \emph{firewall} (pare-feu: système de protection contre les éventuelles attaques par le réseau) Windows, mais
%%il est toujours possible de le désactiver si tu préfères un autre \emph{firewall}. En bref, il permet de simplifier à  l'extrême %la mise à  jour
%%continuelle de l'ordinateur.%
%
%
%%\paragraph{Alors, domaine ou pas domaine ?} Soit tu choisis de te
%%mettre sur le domaine Windows, et tu vas alors au paragraphe
%%\guillemotleft~Inscription sur le domaine Windows~\guillemotright.%
%
%%% \newpage
%%\textbf{Avantages :}
%%\begin{itemize}
%%\item Windows est mis à  jour automatiquement ; tu as toujours les
%%dernières corrections de sécurité et un pare-feu correctement
%%configuré. Donc tu es mieux protégé contre les intrusions.
%%\item Surtout, tu n'as plus à  t'en occuper, presque tout est automatique.
%%\end{itemize}%
%
%%\textbf{Inconvénients :}
%%\begin{itemize}
%%  \item Tu délègues une partie des droits d'administration de ta machine au BR
%%        (tout ce qui concerne la sécurité du réseau en particulier).
%%        Cependant, si tu ne sais pas le faire, c'est plutôt un avantage
%%        de laisser le BR s'en occuper à  ta place.
%%  \item Cela ne marche qu'avec Windows 2000, Windows XP Pro ou Windows Vista Business.
%%        On te rappelle que tu peux facilement, gratuitement et légalement passer à 
%%        Windows XP Pro ou bien à  Windows Vista Business (section sur les licences
%%        MSDNAA en page \pageref{msdnaa}).
%%\end{itemize}
%
%%Bien s\^{u}r, tu peux sortir du domaine à  tout instant, et effectuer manuellement les réglages nécessaires à  la sécurité de ton ordinateur.
%
%%Soit tu choisis de configurer toi-même ton ordinateur, et tu peux passer
%%directement à  la section \guillemotleft Installation de l'antivirus
%%\guillemotright. Tu trouveras les informations nécessaires à  la configuration
%%manuelle du pare-feu et du proxy pour \app{Windows Update} en annexe à  la
%%fin de cette section, en page \pageref{horsdomaine}.
%
%%\textbf{Avantage :} Tu es le seul à  t'occuper de la gestion de ton ordinateur.
%
%%\textbf{Inconvénient :} Tu es le seul à  t'occuper de la gestion de ton ordinateur. ;-)
%%S'il devient un foyer pour virus, sache que nous avons les moyens de l'isoler
%%pour éviter toute propagation.
%
%%\begin{center}
%%  \fbox{
%%    \begin{minipage}{.7\textwidth}
%%      \begin{center}
%%Le BR te conseille \emph{très fortement} de te mettre sur le domaine
%%et de choisir l'installation simplifiée !
%%      \end{center}
%%    \end{minipage}
%%  }
%%\end{center}
%
%
%%\paragraph{Inscription sur le domaine Windows}
%
%%On te rappelle que tu ne peux t'inscrire sur le domaine que si tu utilise
%%Windows 2000, Windows XP Pro ou Windows Vista. Si tu possède Windows XP
%%Familial, Windows Vista Home ou encore une version antérieure de Windows,
%%tu dois effectuer toi-même tes réglages de pare-feu et de proxy
%%\app{Windows Update}. Réfère-toi pour cela à  l'annexe ad hoc à  la fin de
%%cette section, en page \pageref{horsdomaine}.
%
%%La procédure d'inscription est la suivante :
%%\begin{itemize}
%
%%\item \textbf{Sous Windows XP :} Clique sur le \menu{Menu Démarrer} puis fais un clic-droit sur
%%\menu{Poste de travail} et choisis \menu{Propriétés}. Ensuite, sélectionne l'onglet \menu{Nom de l'ordinateur} et clique le bouton \menu{Modifier}.
%
%%\item \textbf{Sous Windows Vista :} Clique sur \menu{Menu Démarrer}, puis fais un clic-droit sur \menu{Ordinateur}, \menu{Propriétés}. Là  sélectionne \menu{Paramètres système avancés}, onglet \menu{Nom de l'ordinateur}, puis clique sur le bouton \menu{Modifier}.
%
%%\end{itemize}
%
%%Dans la case \menu{Nom de l'ordinateur}, rentre ton pseudo, puis coche la case \menu{domaine} et
%%rentre \urllink{windows.eleves.polytechnique.fr}. Note bien que l'inscription au domaine te sera
%%refusée par le serveur si quelqu'un d'autre utilise déjà  le même nom d'ordinateur que toi. Par
%%conséquent, essaie d'opter pour un pseudo qui t'identifie de façon claire et unique, par exemple
%%\cmd{NOM\_PRENOM} \footnote{Les Jean Dupont et les Julien Thomas sont priés de trouver autre chose
%%;-)}.
%
%%\imagepos{images/win_config_domaine}{0.5}{S'inscrire sur le domaine windows}{!ht}
%
%%\begin{center}
%%\begin{tabular}{ll}
%% \parbox{.45\textwidth}{
%%  et si tu es rouje 2006 :
%%  \begin{description}
%%    \item[Nom] rouje06
%%    \item[Mot de passe] rouje.2006
%%  \end{description}
% % }
%% & \parbox{.45\textwidth}{
%%  Si tu es jône 2007, tu rentres :
%%  \begin{description}
%%    \item[Nom] jone07
%%    \item[Mot de passe] jone.2007
%%  \end{description}
%%  }
%%\\
%%\end{tabular}
%%\end{center}
%
%%\emph{Attention, ces identifiants servent juste à  t'inscrire sur le
%%domaine. Pour utiliser ton ordinateur, tu devras rentrer au
%%démarrage les mêmes nom d'utilisateur et mot de passe que tu avais
%%avant d'être sur le domaine !}
%%
%%
%%
%%%\paragraph{Installation personnalisée} --- configuration manuelle
%
%%\subparagraph{Configuration antivirus} Le BR, concerné par la
%%sécurité du réseau, te propose un antivirus pour lequel tu n'auras
%%pas à  payer la license pour obtenir les mises à  jour. Bien sà�¿½r,
%%libre à  toi d'utiliser ton anti-virus personnel ; cependant il sera
%%à  ta charge de le mettre à  jour très réguliérement. Pour cela
%%utilise comme proxy : \urllink{http://kuzh} sur le port 8080.
%
%%\emph{Installation de l'anti-virus du BR}\ : Commence par
%%désinstaller tous les antivirus ou firewalls que tu pourrais avoir
%%comme expliqué dans le paragraphe \guillemotleft~Installation simplifiée
%%--- configuration automatique~\guillemotright .
%
%%Puis ouvre ton explorateur Windows et tape :
%%\urllink{$\backslash\backslash$enez$\backslash$antivirus}
%%et double-clique sur le fichier \file{Symantec.exe}.
%
%%Ce package contient le paramétrage de la mise à  jour automatique de
%%Windows sur le serveur de l'école. Attends la fin de l'installation
%%et c'est fini ! Maintenant, tu n'as plus à  toucher à  l'antivirus,
%%normalement il sera mis à  jour automatiquement.
%
%%\subparagraph{Configuration firewall}
%
%%Si tu as Windows XP avec le SP2 installé, tu as un firewall
%%automatiquement activé et facile d'utilisation. En effet, à  chaque
%%fois qu'un programme tentera d'aller pour la première fois sur
%%Internet, il te demandera si tu veux le laisser faire ou non, comme
%%dans la capture~\ref{config:win:firewall}.
%
%%\imageref{images/win_firewall}{0.8}{Un programmme --- ici GuildFTP
%%--- demande à  accéder au réseau}{!ht}{config:win:firewall}
%
%%Le firewall commercial \app{ZoneAlarm}, indépendant de Windows,
%%fonctionne sur le même principe. Tu peux le trouver sur \xshare.
%
%%Si tu préfères utiliser le firewall intégré à  Windows XP (sans le
%%SP2) ou à  Windows Server 2003, il te faudra le configurer en détail.
%%Va dans le \menu{Menu Démarrer}, \menu{Paramètres} et clique sur
%%\menu{Connexions Réseau}. Choisis la connexion qui est utilisée par
%%ton ordinateur (souvent il n'y en a qu'une, ou alors une seule est
%%activée) et double-clique dessus. Clique sur \menu{Propriétés} en
%%bas à  gauche, puis sur l'onglet \menu{Avancé} et rentre dans le menu
%%de \menu{Paramètres} du \menu{Pare-feu Windows}. Il te faudra alors
%%ajouter manuellement tous les ports que tu veux ouvrir sur
%%l'extérieur. Pour cela, clique sur \menu{Ajouter}, et remplis la
%%boà�¿½te de dialogue en t'aidant de la capture
%%d'écran~\ref{config:win:ouvrir_port}; mets le numéro du port que tu
%%veux ouvrir, par exemple 5050, 5053 et 5055 en TCP pour \app{qRezix}
%%et 21 en TCP pour ton FTP.
%
%%\imageref{images/win_config_firewall}{0.7}{Ouvrir un port dans le firewall %Windows}{!ht}{config:win:ouvrir_port}
%
%%Comme tu peux le constater, il est beaucoup plus pratique d'aller
%%sur le domaine et de laisser le SP2 faire le gros du boulot à  ta
%%place :-).
%
%
%
%\subsubsection{Configuration \emph{web} (serveur mandataire)}
%
%\imageref{images/win_config_proxy}{0.5}{Configuration du serveur mandataire (\emph{proxy})}{!ht}{config:win:proxy}
%
%Même si tu n'utilises pas \app{Internet Explorer} comme client \emph{web}, Windows et d'autres programmes
%utilisent ses paramètres, notamment \app{Windows Update}. Par conséquent lance \app{Internet Explorer} et va
%dans le menu \menu{Outils}, \menu{Options Internet}, puis sur l'onglet \menu{Connexions} de la
%nouvelle fenêtre et enfin sur \menu{Paramètres réseau} dans le bas de la fenêtre. Tu dois être arrivé sur une fenêtre semblable à celle de la capture d'écran~\ref{config:win:proxy}. Là, coche
%uniquement la case \menu{Utiliser un script de configuration automatique}, puis remplis le champ
%\menu{Adresse} avec \urllink{http://config/proxy.pac}.
%
%Une fois que tu as fait ça, tu n'as plus forcément besoin d'\app{Internet Explorer}, tu peux donc utiliser un autre navigateur, comme \app{Mozilla
%Firefox}, disponible sur \urllink{http://www.mozilla-europe.org/fr/products/}.
%
%Même si tu ne configures pas Windows Update avec le paragraphe ci-dessous, n'oublie pas de régler ton navigateur \emph{web} et ton client \emph{mail} si tu en as un : reporte-toi page \pageref{browser}.
%
%\subsubsection{Windows Update}
%
%\label{horsdomaine} %\emph{Cette sous-section ne concerne pas les gens qui ont choisi de s'inscrire sur le domaine.}
%
%%\paragraph{Pare-feu} Si tu as Windows XP avec le SP2 installé, ou \emph{a fortiori}
%%Windows Vista, tu as un \emph{firewall} automatiquement activé et facile d'utilisation. En effet, à  chaque fois qu'un programme tentera d'aller pour
%%la première fois sur Internet, il te demandera si tu veux le laisser faire ou non. Si tu préfères une protection indépendante de Windows, le
%%\emph{firewall} commercial \app{Zone\-Alarm} fonctionne sur le même principe. Tu peux le trouver sur \xshare.
%
%%Si tu préfères utiliser le \emph{firewall} intégré à  Windows XP (sans le SP2), il te faudra le configurer en détail. Va dans le \menu{Menu Démarrer},
%%\menu{Paramètres} et clique sur \menu{Connexions Réseau}. Choisis la connexion qui est utilisée par ton ordinateur (souvent il n'y en a qu'une, ou
%%alors une seule est activée) et double-clique dessus. Clique sur \menu{Propriétés} en bas à  gauche, puis sur l'onglet \menu{Avancé} et rentre dans le
%%menu de \menu{Paramètres} du \menu{Pare-feu Windows}. Il te faudra alors ajouter manuellement tous les ports que tu veux ouvrir sur l'extérieur. Pour
%%cela, clique sur \menu{Ajouter}, et remplis la boîte de dialogue% en t'aidant de la capture d'écran~\ref{config:win:ouvrir_port} ci après
%%; mets le numéro du port que tu veux ouvrir, par exemple 5050, 5053 et 5055 en TCP pour \app{qRezix} et 21 en TCP pour ton FTP.
%
%Il reste une dernière configuration de
%serveur mandatataire indispensable pour que puissent se faire les mises à  jour automatiques
%de Windows. Il t'est fortement recommandé de le faire.
%
%\begin{description}
%
%\item[Sous Windows 7 \& Windows 8] Si tu as correctement configuré ton serveur \emph{web} mandataire, les mises à jours se font automatiquement.
%
%\item[Sous Windows XP] Fais \menu{Démarrer}, \menu{Exécuter}, puis
%tape \cmd{cmd} dans la fenêtre qui s'affiche. Une ligne de commande apparaît,
%il te suffit alors de taper : \cmd{proxycfg -p http://kuzh:8080} pour régler
%le serveur mandataire. Pour revenir à  un accès direct il faut taper \cmd{proxycfg -d}.
%
%\item[Sous Windows Vista]
%Dans le menu \menu{Démarrer}, tape \guillemotleft~Invite de commandes~\guillemotright{} dans le champ \menu{Rechercher}, puis clique droit sur le lien
%et choisis \menu{Exécuter en tant qu'administrateur}. Tape ensuite les commandes suivantes :
%\cmdline{
%C:\textbackslash{}Windows\textbackslash{}system32$>$netsh\\
%netsh>winhttp\\
%netsh winhttp>set proxy proxy-server="http://kuzh:8080"\\
%netsh winhttp>exit\\
%C:\textbackslash{}Windows\textbackslash{}system32>exit }
%
%Pour revenir à l'accès direct, il suffit de taper :
%\cmdline{
%  netsh winhttp reset proxy
%}
%
%Pour la suite de ta configuration, rendez-vous page \pageref{browser} pour faire marcher ton navigateur web préféré !
%
%\end{description}

%\clearpage

\markright{Configuration sous Mac OS}
\label{mac} \bghdr{images/fond-mac}

%\begin{center}
%\includegraphics{images/logo_Mac}
%\end{center}



\subsection{Configuration sous Mac OS X}

La configuration est d\'etaill\'ee pour Mac OS 10.5 (\app{Leopard}). %et encore un peu pour Mac OS 10.4 (Tiger).
Afin de conna\^itre la version que tu utilises, va dans le \menu{menu Pomme} puis s\'electionne \menu{\`A propos de ce Mac}.

\subsubsection{Configuration IP}

\flimage{images/mac_prefs_icone}{0.07}{l}
 \app{Pr\'ef\'erences R\'eseau}, accessible depuis l'article de menu \menu{Pr\'ef\'erences syst\`eme} du menu \menu{Pomme}, permet de configurer la connexion au r\'eseau. Par ailleurs, si au d\'emarrage un assistant te propose de configurer ton r\'eseau, refuse et utilise la proc\'edure du BR. En effet, le r\'eseau n\'ecessite une configuration particuli\`ere  \`a l'X, plus complexe que celle effectu\'ee par cet assistant.


\noindent
  \begin{figure*}[h]
    \begin{center}  
     % \subfloat[Tiger]{\includegraphics[width=0.47\textwidth]{images/mac_nouvelle_config} } 
     % \hfill
      \subfloat[Cr\'eer une nouvelle configuration r\'eseau]{ 
      \begin{minipage}{0.43 \textwidth}\begin{flushleft}
      {\includegraphics[width=0.96\textwidth]{images/mac_nouvelle_config_leopard_1}}\\ \vspace*{2cm}
      {\includegraphics[width=0.96\textwidth]{images/mac_nouvelle_config_leopard_2}} 
 		\end{flushleft}  \end{minipage}
 		 } 		
 		\subfloat[Configuration de l'interface r\'eseau \emph{Ethernet}, de l'adresse IP et du \emph{proxy}]{ 
 		 \begin{minipage}{0.43 \textwidth}\begin{flushright}
 		{\includegraphics[width=0.96 \textwidth]{images/mac_config_ethernet}} \\ 
 		{\includegraphics[width=0.96 \textwidth]{images/mac_config_ip_leopard}} \\
 		{\includegraphics[width=0.96 \textwidth]{images/mac_config_proxy_leopard}}
\end{flushright}
 		\end{minipage}
 		 	\label{config:mac:ip:leopard}	}
     	 \caption{Cr\'eer une nouvelle configuration r\'eseau}

    \end{center}
  \end{figure*}

\pagebreak

La gestion des configurations r\'eseau de Mac OS X permet de cr\'eer plusieurs configurations et de passer en un clic de l'une  l'autre gr\^ace au sous-menu \menu{Configuration R\'eseau} du menu \menu{Pomme}. Cela est tr\`es pratique pour les machines vou\'ees \`a  \^etre connect\'ees \`a  plusieurs endroits successivement --- les portables par exemple (voir la page~\pageref{wifi} puis la section \emph{Wi-Fi} pour plus de pr\'ecisions sur le \emph{Wi-Fi}). Commence donc par cr\'eer une nouvelle configuration r\'eseau dans le menu d\'eroulant \menu{Configuration}.



Une fois la nouvelle configuration cr\'e\'ee, il faut configurer l'interface r\'eseau \emph{Ethernet}.



\app{Leopard} : Dans la colonne de gauche, s\'electionne \menu{Ethernet int\'egr\'e}.

Choisis alors \menu{Configurer IPv4} :% \menu{Manuellement} (\app{Tiger}) ou 
\menu{Configurer} : \menu{Manuellement} (\app{Leopard}). Tu trouveras toutes les valeurs d'adresses IP n\'ecessaires pour la configuration en page \pageref{calcul_ip} ou en te reportant aux captures d'\'ecran~\ref{config:mac:ip:leopard}. Si une partie d'adresse IP est blanche sur ces captures, c'est qu'elle t'est personnelle et que tu dois la calculer !


  
  

%\imageref{images/mac_config_ip_leopard}{0.4}{Configuration IP (Leopard)}{!ht}{config:mac:ip:leopard}}


Pour avoir acc\`es \`a  Internet, il faut aussi configurer le \emph{proxy}.

\app{Leopard} : Clique sur le bouton \menu{Avanc\'e...} puis sur l'onglet \menu{Proxys}.


\app{Mac OS X 10.3.3 et sup\'erieur} :  choisis l'option \menu{Configuration automatique de proxy} et indique http://config/proxy.pac comme URL de fichier PAC. Pour Mac OS X 10.3 \`a  10.3.2, n'oublie pas une fois que tu as le r\'eseau de faire la mise \`a  jour de ton syst\`eme, pour pouvoir configurer de fa\c con automatique le \emph{proxy}.

\app{Mac OS X 10.3.2 et inf\'erieur} : il te faut sp\'ecifier tous les
\emph{proxies} manuellement, et mettre \server{kuzh.polytechnique.fr}, port \server{8080}. Malheureusement, cele te permettra uniquement d'acc\'eder aux sites h\'eb\'erg\'es hors du plat\^al : les sites \'el\`eves ne fonctionneront pas.


N'oublie pas d'activer le mode passif pour les transferts en FTP, en cochant la case comme dans la capture.

\subsubsection{Configuration antivirus}

Bien qu'il soit important de maintenir ton syst\`eme \`a  jour, un antivirus est pour l'instant tout \`a  fait superflu sur Mac, puisqu'aucun virus fonctionnel n'a encore vu le jour. Attention cependant, n'ouvre pas des fichiers dont tu ne te sois pas assur\'e de la provenance, et essaie de te tenir au courant des actualit\'es concernant les failles des applications que tu utilises.

\subsubsection{Configuration web}
\flimage{images/nux_firefox_icon}{0.07}{l}
Un point particulier pour la configuration du \emph{proxy} de \app{Firefox} : dans \menu{Pr\'ef\'erences}, \menu{Avanc\'e}, \menu{R\'eseau}, clique sur \menu{Param\`etres} et, dans le champ \menu{Adresse de configuration automatique du proxy}, inscris : \urllink{http://config/proxy.pac}. 
\\
\\

\flimage{images/mac_safari_icone}{0.07}{l}
\app{Safari}, le navigateur web d'Apple, est maintenant compatible avec la majorit\'e des sites \emph{web}. Tu peux donc t'en servir au quotidien, en faisant appel \`a  \app{Firefox} pour les sites r\'ecalcitrants. Un conseil : pense \`a  activer le blocage des fen\^etres \emph{pop-up} (dans le menu \menu{Safari}). \app{Safari} peut aussi servir de client RSS (voir plus bas).\\
\\

\app{Google Chrome} se règle quant à lui automatiquement sur la configuration du système. 
\\

\subsubsection{Configuration \emph{mail}}
\flimage{images/mac_mail_icone}{0.07}{l} \app{Mail} : un client \emph{mail} offrant les fonctionnalit\'es classiques d'un bon client : recherche instantan\'ee, filtre antispam, r\`egles de tri automatique des \emph{mails}, regroupement des \emph{mails} correspondant \`a  une m\^eme discussion.

Au premier lancement, \app{Mail} te demandera de remplir les informations concernant ton compte \emph{mail} sur \server{poly}, il suffit de le remplir avec les donn\'ees suivantes :
\begin{description}
  \item[Nom complet] ton nom !
  \item[Adresse \'electronique] de la forme \mail{prenom.nom@polytechnique.edu}
  \item[Serveur de r\'eception] \server{poly.polytechnique.fr}
  \item[Type de compte] \menu{POP}
  \item[Nom d'utilisateur] ton \emph{login} \server{poly} (les huit premi\`eres lettres de ton nom en g\'en\'eral)
  \item[Mot de passe] ton mot de passe \server{poly}
  \item[Serveur d'envoi (SMTP)] \server{poly.polytechnique.fr} ou \server{ssl.polytechnique.org}
\end{description}

Si tu as d\'ej\`a  cr\'e\'e un compte pr\'ec\'edemment, il faut aller dans les \menu{Pr\'ef\'erences} (accessibles depuis le menu \menu{Mail}), onglet \menu{Comptes}, pour cr\'eer un autre compte en cliquant sur la case \menu{+}.

N'oublie pas de cocher \menu{Activer le cryptage SSL} dans l'onglet \menu{Avanc\'e}, port 995. Tu souhaiteras alors certainement installer le certificat de s\'ecurit\'e de \server{poly} (tu le trouveras sur \urllink{http://poly/}). Une fois que tu as t\'el\'echarg\'e le certificat, ouvre le fichier \menu{.CRT} obtenu, et dans \app{Trousseau d'acc\`es}, installe-le dans %\menu{X509Anchors} (Tiger) ou 
\menu{session} (Leopard).

Cette configuration marche pour acc\'eder \`a  ses mails depuis l'int\'erieur de l'X mais aussi de l'ext\'erieur, sans rien changer. En revanche tu ne peux pas envoyer de \emph{mails} depuis l'ext\'erieur , car le serveur \server{poly} ne le permet pas. Nous te conseillons vivement d'utiliser le serveur SMTP \server{polytechnique.org} et de regarder la configuration propos\'ee par \urllink{Polytechnique.org}. Celle-ci permet d'envoyer des \emph{mails} s\^urs \`a  l'ext\'erieur de l'\'Ecole sans modifier ta configuration par la suite. Tu peux ajouter ce SMTP dans l'onglet \menu{Comptes} des \menu{Pr\'ef\'erences} de \emph{mail} et r\'egler dans l'onglet \menu{Avanc\'es} comme dans la capture.

\begin{figure*}[!hl]
    \begin{center}
	      \includegraphics[width=0.4\textwidth]{images/mac_config_smtp_poltechnique.png} 
      \caption{Configurer le serveur SMTP \server{Polytechnique.org}}
    \end{center}
  \end{figure*}

Le LDAP ne fonctionne pas \`a  l'heure actuelle avec \app{Mail} (mars 2009) sous \app{Leopard}, contrairement aux  autres syst\`emes d'exploitation (fonctionne cependant avec \app{Thunderbird}).
%\subsuEnfin, tu peux disposer dans \app{Mail} de l'annuaire de l'\`acole, mis \`a  disposition par la DSI. Pour cela, va dans les \menu{Pr\'ef\'erences} de Mail,
%puis dans la rubrique \menu{R\'edaction} et clique sur \menu{Configurer LDAP\ldots}. Tu peux ensuite utiliser le bouton \menu{+} pour ajouter un
%serveur, et remplir la fen\^etre comme sur la capture.

%\imagepos{images/mac_config_ldap}{0.6}{Configurer l'annuaire}{!ht}
%bsection{Logiciels additionnels}

Les logiciels suivants sont utiles pour utiliser avec Mac OS X les services propos\'es sur le r\'eseau ; ils sont t\'el\'echargeables sur \server{frankiz}, dans la rubrique \menu{T\'el\'echarger}, \menu{Mac}.

%\subsubsection{Configuration \emph{news}}
%
%\flimage{images/mac_thunderbird_icone}{0.07}{l} 
%\app{Thunderbird} : un client \emph{news} permettant d'acc\'eder aux forums de discussion des \'el\`eves (voir page~\pageref{newsgroups} pour les d\'etails sur \server{frankiz}), mais aussi \`a  ceux de \server{usenet} gr\^ace au serveur \server{polynews.polytechnique.fr}. Il est tr\`es proche d'\app{Outlook Express} dans son esprit. Dans la m\^eme cat\'egorie, il existe \app{MacSOUP}, \app{Unison} ou encore \app{MT-NewsWatcher}. La configuration se fait de la m\^eme mani\`ere.
%
%Au premier lancement, l'application te propose d'importer les param\`etres depuis une autre application. Clique sur \menu{Suivant}. Tu peux alors choisir quel type de compte tu veux configurer (tu remarqueras que tu peux aussi cr\'eer un compte courrier \'electronique, et un compte RSS). S\'electionne \menu{Compte forums de discussion} et clique sur \menu{Suivant}. Entre alors les informations suivantes :
%
%\begin{description}
%  \item[Votre nom] ton nom ou ton pseudo
%  \item[Adresse de courrier] \mail{prenom.nom@polytechnique.edu}
%  \item[Serveur de forums] \server{news}
%  \item[Nom du compte] News Frankiz
%  \item[Nom d'utilisateur] ton \emph{login }poly (les huit premi\`eres lettres de ton nom en g\'en\'eral)
%  \item[Serveur d'envoi (SMTP)] \server{poly.polytechnique.fr} ou \server{ssl.polytechnique.org}
%\end{description}
%
%
%Pour t'abonner \`a  des groupes de discussion, il te suffit de s\'electionner le compte \menu{News Frankiz} dans la fen\^etre \menu{Dossiers} de \app{Thunderbird}, puis de cliquer sur \menu{G\'erer les abonnements aux groupes de discussion}. Tu pourras ensuite s\'electionner les forums qui t'int\'eressent parmi la liste propos\'ee. Reporte-toi \`a  la page \pageref{newsgroups} pour plus d'infos sur les \emph{newsgroups} auxquels t'abonner !
%

\subsubsection{Autres logiciels utiles}

%\flimage{images/mac_qrezix_icone}{0.07}{l} \app{qRezix} : en deux mots, c'est un programme d\'evelopp\'e par le BR pour faciliter la vie sur le r\'eseau. Tu peux le r\'ecup\'erer via le lien qRezix sur \server{Frankiz} ou sur \urllink{http://br.frankiz.net/qrezix/mac/}. Pour plus de d\'etails, voir le paragraphe consacr\'e \`a  qRezix \`a  la page \pageref{qrezix}. \\

%\app{Leopard} : le pare-feu se r\`egle pour chaque application; tu n'auras qu'\`a  r\'epondre \menu{Autoriser} lorsqu'il te demandera si tu veux \menu{Autoriser les connexions entrantes}.

%\noindent  \app{Tiger} : Attention, si ton \emph{firewall} est activ\'e, tu dois ouvrir les ports 5050, 5053 et 5055 en TCP. Pour cela va dans \app{Pr\'ef\'erences Syst\`eme}, dans le module \menu{S\'ecurit\'e}, onglet \menu{Coupe-feu}. S'il est \'ecrit \menu{Coupe-feu activ\'e}, clique le bouton \menu{Nouveau} et remplis la bo\^ite de dialogue comme sur la capture d'\'ecran ci-dessous pour ouvrir les ports.

%\imagepos{images/mac_firewall}{0.5}{Ouvrir les ports pour \app{qRezix} (Tiger)}{!ht}

%\flimage{images/mac_conversation_icone}{0.1}{l}
%\noindent\app{Colloquy}, un client IRC dans le m\^eme esprit qu'\app{iChat}. Il dispose d'une interface tr\`es simple ne n\'ecessitant pas de conna\^itre les commandes IRC. Tu peux te reporter \`a  la page \pageref{irc} pour plus d'infos sur l'IRC. \app{X-Chat Aqua} est un autre client IRC, plus riche en fonctionnalit\'es, mais moins agr\'eable \`a  utiliser. \\

%\flimage{images/mac_netnewswire_icone}{0.1}{l}
%\noindent\app{NetNewsWire} est la r\'ef\'erence des clients RSS sur Mac, et est maintenant gratuit. Dans le m\^eme genre, on peut citer \app{Vienna}, un client RSS open source, dont le d\'eveloppement actif est prometteur. Les flux RSS permettent d'agr\'eger dans un seul logiciel des informations en provenance de nombreux sites web, qui peuvent provenir de forums de discussions, de mises \`a  jour de logiciels, d'informations internationales\dots \\ \\

%\flimage{images/mac_fink_icone}{0.07}{l} \app{Fink} est la mani\`ere la plus simple d'installer sur Mac OS X nombre de logiciels issus du monde Unix (Linux par exemple). Gr\^ace \`a  lui, tu pourras installer les m\^emes logiciels que dans les salles informatiques. Par exemple, tu pourras installer Scilab sans trop de peine\dots La configuration n\'ecessaire se trouve sur la page \urllink{http://frankiz/binets/reseau/Miroir\_Fink}.\\ 

\flimage{images/logo_Windows}{0.1}{l}
 \app{Windows et les Mac Intel} : Maintenant il est possible d'installer Windows gr\^ace \`a  \app{Boot Camp}, livr\'e avec \app{Leopard}. Cela te permettra de profiter des quelques applications du monde PC qui valent le coup tout en gardant ton Mac. Tu peux \'egalement virtualiser Windows (utiliser Windows en utilisant en m\^eme temps Mac OS) gr\^ace \`a  \app{VMware Fusion}, \app{VirtualBox} ou \app{Parallels Desktop}. Le  d\'efaut de cette solution est que tu n'as pas d'acc\'el\'eration 3D, donc pour les jeux il te faudra red\'emarrer. Ces trois logiciels sont disponibles sur leurs sites \emph{web} respectifs. \`A toi de choisir !
Mais v\'erifie tout de m\^eme que tu as bien un processeur Intel (\menu{Pomme} puis s\'electionne \menu{\`A propos de ce Mac}).

\clearpage


\markright{Configuration sous Linux}
\label{ubuntu} %$Id: config_nux.tex 145 2005-03-25 08:26:35Z myk $

\bghdr{images/fond-nux}



%\begin{center}
%\includegraphics{images/logo_Linux}
%\end{center}

\subsection{Configuration sous Linux}

Cette section décrit la configuration de ta connexion Internet sous Ubuntu GNU/Linux (ou une de ses variantes). Pour les autres distributions, tu peux adapter les instructions ci-dessous ou consulter la version 
en ligne de commande, page \pageref{linux_cmdline}.

\subsubsection{Configuration IP}
Tu as besoin de conna\^itre ton adresse IP, ton masque de sous-r\'eseau et ta  passerelle. Toutes les informations se trouvent page \pageref{calcul_ip}. Bien s\^ur, pour  l'ensemble des manipulations d\'ecrites ci-dessous tu auras besoin de ton  mot de passe super-utilisateur (\emph{root}) !

\label{Ubuntu:IP}
Il existe deux mani\`eres de configurer tes param\`etres r\'eseaux: l'une utilise les outils graphiques de l'environnement que tu as choisi (Gnome ou KDE), 
l'autre utilise simplement la ligne de commande. Bien sûr, les outils graphiques ne sont qu'un interm\'ediaire modifiant les fichiers dont on te parle 
plus bas. Ils te permettent parfois d'enregistrer une configuration r\'eseau, ce qui facilite la gestion si tu rentres souvent chez toi. Pour obtenir le 
m\^eme r\'esultat en ligne de commande il faut utiliser un script.
\begin{description}
\item[\'Etape 1 : configuration de la connexion au r\'eseau] \
 
\begin{itemize}
\item Va dans \menu{Syst\`eme}, \menu{Pr\'ef\'erences} puis \menu{Connexions r\'eseau}.
\item Dans l'onglet \menu{Filaire}, clique sur \menu{Ajouter}.
\item Compl\`ete le champ \menu{Nom de la connexion}  par ce que tu veux ; "Casert de Polytechnique" par exemple.
\item Puis va dans l'onglet \menu{Param\`etres IPv4}.
\item S\'electionne la m\'ethode \menu{Manuel}.
\item Clique sur \menu{Ajouter}, puis remplis les champs \menu{Adresse}, \menu{Masque de r\'eseau} et  \menu{Passerelle} par les donn\'ees qui te sont propres. 
\item Compl\`ete le  champ \menu{Serveurs DNS} par \server{129.104.201.53, 129.104.201.51} et le champ \menu{Domaines de recherche} par \server{eleves.polytechnique.fr, polytechnique.fr}. 
\item Coche enfin l'option \menu{Disponible pour tous les utilisateurs}, clique sur \menu{Appliquer} et enfin rentre ton mot de passe super-utilisateur (root).
\end{itemize}

\item[\'Etape 2 : configuration du proxy (= serveur mandataire)] \
\begin{itemize}
\item Va  dans \menu{Param\`etres Syst\`eme}, \menu{R\'eseau} puis \menu{Serveur Mandataire}.
\item S\'electionne  la \menu{M\'ethode} \menu{Automatique}.
\item Compl\`ete le champ  \menu{URL de configuration} par \urllink{http://config/proxy.pac}. 
\item Clique sur \menu{Appliquer à tout le syst\`eme...} et rentre ton mot de passe super-utilisateur si on te le demande.
\end{itemize}

\item[\'Etape 3 (\'eventuellement)] \
\begin{itemize}
\item Clique  sur l'ic\^one de l'applet R\'eseau dans la zone de notification, en forme  de fl\`eches t\^ete-b\^eche ou d'ondes. S\'electionne le r\'eseau que tu as  configur\'e dans la 1\`ere \'etape, et te voilà connect\'e à Internet !
\item Une fois ta configuration r\'eseau termin\'ee, tu peux la tester en \emph{pinguant} \fkz (dans une console), o\`u tu devrais voir quelque chose comme :
\end{itemize}

\cmdline{\$ ping frankiz\\
PING frankiz.eleves.polytechnique.fr (129.104.201.51) 56(84) bytes of data.\\
64 bytes from Frankiz.eleves.polytechnique.fr ...}

\end{description}



\subsubsection{Configuration du gestionnaire de paquets}
\label{ubuntu_mirror}

Il faut d\'esormais configurer le gestionnaire de paquets pour qu'il utilise les miroirs du BR et non les miroirs à l'ext\'erieur du campus, qui sont plus lents. \
Va  dans \menu{Applications}, \menu{Logith\`eque Ubuntu} puis menu \menu{\'edition}, \menu{Sources de logiciels...}. 
Entre ton  mot de passe super-utilisateur puis s\'electionne l'onglet \menu{Autres  logiciels}. 
D\'ecoche les cases comprenant une adresse du type \urllink{http://archive.canonical.com/ubuntu version}, o\`u \textit{version} correspond à la version d'Ubuntu install\'ee sur ton ordinateur. 
À l'impression de l'InfoBR, la version actuelle est \textbf{oneiric} et la pr\'ec\'edente est \textbf{natty}. \
Clique sur \menu{Ajouter}, puis entre dans le champ \menu{Ligne APT} :
\cmdline{deb ftp://miroir/linux/ubuntu version main restricted universe multiverse}
Tu auras bien s\^ur remplac\'e \textit{version} par ta version d'Ubuntu (\textit{maverick}/\textit{lucid}/\textit{karmic}/...). \\
Clique ensuite sur \menu{Ajouter une source de mise à jour}. Fais de même pour les lignes suivantes :
\cmdline{deb ftp://miroir/linux/ubuntu version-updates main restricted universe  multiverse \\
deb ftp://miroir/linux/ubuntu version-security main restricted universe  multiverse}
Tu peux aussi utiliser le d\'ep\^ot suivant mais attention il contient des logiciels non support\'es par Canonical, l'\'equipe de d\'eveloppement d'Ubuntu (en particulier il peut arriver que certains logiciels contiennent des erreurs) :
\cmdline{deb ftp://miroir/linux/ubuntu version-backports main restricted universe multiverse}
Le  BLL (Binet Logiciels Libres) dispose par ailleurs d'un miroir  non-officiel qui contient des paquets (flash, ...) non  inclus dans la distribution de base pour diverses raisons, en paticulier l\'egales ou \'ethiques. Pour en profiter, rajoute aussi la ligne :
\cmdline{deb ftp://miroir/linux/bll version main}
Clique enfin sur \menu{Fermer} puis r\'eponds \menu{Actualiser} à la fenêtre de dialogue qui appara\^it. \\

Note : il n'est pas n\'ecessaire de configurer Synaptic dans ses Pr\'ef\'erences pour y sp\'ecifier un proxy quelconque.

\subsubsection{Configuration antivirus}

{C'est pas non plus comme si y'en avait besoin \dots}

%\subsubsection{Configuration du pare-feu}
%
%La solution la plus simple pour se faire un \emph{firewall} sous linux est d'utiliser les \emph{iptables}. Pour ceci la premi\`ere \'etape est
%d'installer le paquet \app{iptables} pour ta distribution. Pour savoir comment configurer ton \emph{firewall} pour le r\'eseau de l'X, consulte le Wikix.

\subsubsection{Configuration navigateur web}

\flimage{images/nux_firefox_icon}{0.12}{l} Le BR te conseille d'utiliser \app{Firefox} ou
\app{Konqueror} (le navigateur fourni par d\'efaut avec KDE). Dans tous les cas, la seule
configuration \`a  effectuer est celle du serveur mandataire.

Pour \app{Firefox}, il suffit d'aller dans \menu{\'Edition}, \menu{Pr\'ef\'erences} et dans l'onglet \menu{Avanc\'e}, puis de cliquer sur l'onglet
\menu{R\'eseau}, et enfin \menu{Param\`etres de connexion} ; ensuite coche la case \menu{Adresse de configuration automatique du proxy}, et rentre 
\urllink{http://config/proxy.pac}.

Sous \app{Konqueror}, cela se trouve dans le menu \menu{Configuration}, \menu{Configurer Konqueror},
dans l'onglet \menu{Serveur mandataire}. \emph{Attention}: si tu ne configures pas le serveur mandataire dans Konqueror,
les logiciels KDE (\app{KGet}, \app{Adept},\dots) ne l'utiliseront pas!

Pour \app{Google Chrome}, la configuration se règle automatique sur celle du système. Tu n'as donc rien à faire si le reste est correctement configuré.

\imagepos{images/nux_proxy_firefox}{0.65}{Configuration du serveur mandataire sous Firefox}{ht}

\pagebreak

\subsubsection{Configuration \emph{mail}}

\flimage{images/nux_kmail_icon}{0.12}{l} Les clients \emph{mail} les plus
utilis\'es sont \app{Kmail} et \app{Thunderbird}. La configuration est semblable, quel que soit le
client utilis\'e.

Pour \app{Kmail}, va dans \menu{Configuration}, \menu{Configurer Kmail}. Choisis la
rubrique \menu{Comptes}. Commence par cr\'eer un nouveau compte dans
l'onglet \menu{R\'eception des messages} en cliquant sur le bouton
\menu{Ajouter\ldots} et choisis le type POP3.


\noindent
  \begin{figure*}[!h]
    \begin{center}  
      \subfloat[R\'eception des messages]{ 
      \includegraphics[width=0.48\textwidth]{images/nux_config_kmail_pop} }
      \hspace{\stretch{1}}
      \subfloat[Envoi des messages]{ 
 		\includegraphics[width=0.48 \textwidth]{images/nux_config_kmail_smtp} }
         	 \caption{Configuration sous \app{Kmail}}
    \end{center}
  \end{figure*}


Utilise les param\`etres suivants pour configurer l'onglet \menu{G\'en\'eral} :
\begin{description}
  \item[Nom] le nom du compte, par exemple : Mails Poly
  \item[Utilisateur] rentre le \emph{login} \server{poly} que t'a fourni la DSI \`a  ton arriv\'ee sur le plateau
  \item[Mot de passe] et l\`a  le mot de passe \server{poly}
  \item[Serveur] \server{poly.polytechnique.fr}
  \item[Port] 995
\end{description}
Ensuite, va dans l'onglet \menu{Extras} et coche la case
\menu{Utiliser SSL pour s\'ecuriser les t\'el\'echargements}.

Maintenant, dans l'onglet \menu{Envoi des messages} clique sur le
bouton \menu{Ajouter\ldots}. Utilise les param\`etres suivants pour le
configurer :
\begin{description}
  \item[Nom] le m\^eme nom de compte que pr\'ec\'edemment
  \item[Serveur] \server{poly.polytechnique.fr}
  \item[Port] 25
\end{description}
Sinon, laisse toutes les cases d\'ecoch\'ees.

%Tu peux aussi configurer l'acc\`es \`a  \app{l'annuaire LDAP} de l'\'Ecole, sorte de carnet d'adresses en ligne qui contient les adresses \emph{mail} de tout le monde sur le campus. Pour ce faire, commence par ouvrir \menu{Outils}, \menu{Carnet d'adresses}, puis va dans \menu{Configuration}, \menu{Configurer kAdressBook}, \menu{Consultation LDAP}. Clique ensuite sur \menu{Ajouter un h\^ote}, et configure comme suit: \\
%\smallskip
%\begin{minipage}[t]{0.48\textwidth}
%\begin{description}
%  \item[H\^ote] \server{ldap.eleves.polytechnique.fr}
%  \item[Port] 389
%  \item[Version de LDAP] 3
%\end{description}  
%\end{minipage} 
%\begin{minipage}[t]{0.48\textwidth}
%\begin{description}  
%  \item[DN] \server{dc=polytechnique, dc=ldap, dc=eleves, dc=fr}
%  \item[S\'ecurit\'e] Non
%  \item[Identification] Anonyme
%\end{description}
%\end{minipage} \\
%Une fois revenu dans \menu{Configuration LDAP}, coche la case \server{ldap.eleves.polytechnique.fr}. Tu as maintenant acc\`es \`a  l'annuaire LDAP lors de la
%r\'edaction de messages, avec tout au plus un red\'emarrage de \app{Kmail}. 
%
%\imagepos{images/nux_config_ldap}{0.55}{Configuration de l'annuaire LDAP sous Kmail}{pht}
%\imagepos{images/nux_config_knode}{0.45}{Configuration de Knode}{ht}
%
%\noindent
%  \begin{figure*}[!h]
%    \begin{center}  
%      \subfloat[Configuration de l'annuaire LDAP sous Kmail]{ 
%      \includegraphics[width=0.48\textwidth]{images/nux_config_ldap}}
%      \hspace{\stretch{1}}
%      \subfloat[Configuration de Knode]{ 
% 		\includegraphics[width=0.48 \textwidth]{images/nux_config_ldap} }
%	\caption{Configurations LDAP et \emph{news}}
%    \end{center}
% \end{figure*}



%\subsubsection{Configuration \emph{news}}
%
%\flimage{images/nux_knode_icon}{0.12}{l} Le client \emph{news} le plus utilis\'e est \app{Knode}. Parmi les autres clients \emph{news}, citons 
%\app{Thunderbird}, \app{Pan} ou \app{slrn}. Ici aussi, la configuration est presque ind\'ependante du logiciel choisi.
%
%
%Sous \app{Knode}, c'est dans le menu \menu{Configuration}, puis \menu{Configurer Knode}. Va dans la rubrique \menu{Comptes, Forums de discussion} et
%cr\'ee un compte en cliquant sur \menu{Ajouter\ldots}.
%
%\imagepos{images/nux_config_knode}{0.45}{Configuration de Knode}{ht}
%
%\pagebreak
%
%Remplis l'onglet \menu{Serveur} avec les informations suivantes :
%\begin{description}
%  \item[Nom] ce que tu veux pour d\'ecrire ce compte, par exemple 'News Frankiz'
%  \item[Serveur] \server{news}
%\nopagebreak  \item[Port] 119
%\end{description}
%
%\pagebreak
% 
%Ensuite occupe-toi de l'onglet \menu{Identit\'e} :
%\begin{description}
%  \item[Nom] mets ton pseudo dans ce champ
%  \item[Organisation] X, \'Ecole polytechnique, comme tu le sens
%  \item[Adresse \'electronique] ton adresse \emph{mail}, pour que les gens puissent te r\'epondre par \emph{mail}.
%\end{description}
%
%Enfin, pour que \app{Knode} puisse envoyer des \emph{mails}, il faut aller
%dans la rubrique \menu{Comptes}, sous-rubrique \menu{Serveur de
%courrier (SMTP)}, et choisir comme serveur d'envoi de \emph{mails}
%\server{poly.polytechnique.fr}, port 25 --- c'est exactement la m\^eme
%configuration SMTP que \app{Kmail}.
%
%Si tu veux mettre une signature \`a  la fin des messages que tu
%posteras, il te suffit de la mettre dans l'onglet \menu{Identit\'e}.
%Sur la plupart des clients la signature est interpr\'et\'ee comme
%ext\'erieure au message et n'est en particulier pas incluse dans le
%texte cit\'e lorsque tu r\'eponds \`a  un message. Pour d\'efinir une
%signature \`a  la main, il suffit de mettre \verb*+-- +\ (c'est \`a  dire
%-{}-<espace>) sur une ligne, et tout ce qui suivra cette ligne
%composera ta signature.
%
%Il ne te reste plus qu'\`a  t'inscrire \`a  des \emph{newsgroups} (reporte-toi \`a  la page \pageref{newsgroups} pour plus d'infos) et \`a  poster ! \\
%
%Pour te connecter aux serveurs de \emph{news} de Polytechnique.org, qui ont un acc\`es s\'ecuris\'e, avec \app{Knode}, il y a une petite subtilit\'e car il
%ne g\`ere pas le SSL. Il faut installer \app{stunnel} qui permet de d\'efinir une redirection SSL de port. Dans \file{/etc/stunnel.conf} (ou parfois \file{/etc/stunnel/stunnel.conf}), mets les lignes suivantes (les trois premi\`eres y sont en principe d\'ej\`a ) :
%\cmdline{\# location of pid file\\
%pid = /etc/stunnel/stunnel.pid\\
%\\
%\# user to run as\\
%setuid = stunnel\\
%setgid = stunnel\\
%\\
%\# Use it for client mode\\
%client = yes\\
%\\
%\# sample service-level configuration\\
%{[}nntps{]}\\
%accept  = 1119\\
%connect = ssl.polytechnique.org:563\\
%TIMEOUTclose = 0
%}
%
%Il ne te reste plus qu'\`a  lancer \app{stunnel} par :
%\cmdline{/etc/init.d/stunnel start}
%
%Et tu peux ainsi lire les \emph{news} de Polytechnique.org en mettant \server{localhost} comme serveur et
%\server{1119} comme port. Il faut aussi que tu coches \menu{Le serveur exige une identification} et
%que tu rentres ton nom d'utilisateur \`a  Polytechnique.org et ton mot de passe, que tu peux d\'efinir
%sur \urllink{https://www.polytechnique.org/Xorg/SMTPSecurise}.

\clearpage


\markright{Pour bien commencer}
\bghdr{images/fond-infobr}

\subsection{Configuration de ton navigateur \emph{Web}}
\label{browser}

\paragraph{Firefox}
\flimage{images/firefox-logo}{0.07}{l}

Lance \app{Mozilla Firefox}, et va dans le menu \menu{Outils},
\menu{Options...} (ou \menu{\'Edition}, \menu{Pr\`ef\`erences...} sous Linux). L\`a , s\'electionne la rubrique \menu{Avanc\'e}, onglet \menu{R\'eseau}, et clique sur
\menu{Param\`etres}. La case \`a  cocher est alors \menu{Adresse de configuration automatique du proxy},
et l'adresse \`a  indiquer est : \urllink{http://config/proxy.pac}.

Ensuite, il te faut accepter le certificat SSL du BR. Cela consiste \`a indiquer que tu fais confiance au BR pour authentifier les sites des binets.\\
Il suffit se se rendre sur \urllink{http://config/ca-br.crt} et de cocher toutes les cases.\\

\noindent
  \begin{figure*}[!h]
    \begin{center}  
      \subfloat[Configuration du serveur mandataire]{ 
      \includegraphics[width=0.48\textwidth]{images/nux_proxy_firefox}}
      \hspace{\stretch{1}}
      \subfloat[Acceptation du certificat BR]{ 
         \includegraphics[width=0.48 \textwidth]{images/ca-br-ff}}
           \caption{Configuration de Firefox}
    \end{center}
  \end{figure*}
%\imagepos{images/nux_proxy_firefox}{0.65}{Configuration du serveur mandataire sous Firefox}{ht}\\
%\imagepos{images/ca-br}{1}{Acceptation du certificat BR}{ht}
%
% Pour le RTFIBRp11 pour un saut de page si n\'ec\'essaire ce qui n'\'etait pas le cas en 2008

%\noindent\rule{.4\textwidth}{.4pt}

%\vfill %\pagebreak


\paragraph{Google Chrome}
\flimage{images/google-chrome-logo}{0.07}{l}

Pour \app{Google Chrome}, la configuration du serveur mandataire se r\`egle automatique sur celle du système. Tu n'es donc pas oblig\`e de r\`egler le \emph{proxy} toi-même. Cependant, pour le faire manuellement, clique sur la
clef à molette en haut à droite, choisis \menu{Options}, va dans l'onglet \menu{Paramètres avanc\`es} (ou \menu{Under the Hood}), clique sur \menu{Modifier les paramètres du Proxy...} et règle comme ci-dessus.\\

Pour accepter le certificat BR, il te faut d'abord le t\'el\'echarger sur \urllink{http://config/ca-br.crt}.
Puis, dans Chrome, rends-toi dans \menu{Pr\'eferences}, \menu{Options avanc\`ees}, \menu{G\'erer les certificats},
\menu{Autorit\'es} et clique sur \menu{Importer}. Il suffit alors de s\'electionner l'emplacement o\`u tu avais stock\'e le certificat et de cocher toutes les cases.
\imagepos{images/ca-br-chrome}{1}{Acceptation du certificat BR sous Google Chrome}{ht}

\paragraph{Konqueror}
\flimage{images/konqueror-logo}{0.07}{l}

Sous \app{Konqueror}, cela se trouve dans le menu \menu{Configuration}, \menu{Configurer Konqueror},
dans l'onglet \menu{Serveur mandataire} ; ensuite, choisis les même r\`eglages que pour \app{Firefox} ci-dessus.
Attention : si tu ne configures pas le serveur mandataire dans Konqueror,
les logiciels KDE (\app{KGet}, \app{Adept},\dots) ne l'utiliseront pas !


%%% pas besoin de configurer au niveau navigateur sur Mac OS
%\paragraph{Safari}
%\flimage{images/mac_safari_icone}{0.07}{l}

%\app{Safari}, le navigateur web d'Apple, est maintenant compatible avec la majorit\'e des sites \emph{web}. Tu peux donc t'en servir au quotidien,
%en faisant appel \`a  \app{Firefox} pour les sites r\'ecalcitrants. 

%% CE TRUC EST PAS A SA PLACE ICI.

 % \item \app{vlc} : Un logiciel qui te permettra de recevoir la t\'el\'evision directement dans ton casert, afin d'\^etre vraiment s\^{u}r d'avoir autre chose \`a  faire que travailler les veilles de p\^ales. Configuration page \pageref{TV}.



%%%%%%%%%%%%%%%%%%%%%%%%%%%%%%%%%%%%%%%%%%%%%%%%%%%%%%%%%%%%%%%%%%%%
%                            MAIL                                  %
%%%%%%%%%%%%%%%%%%%%%%%%%%%%%%%%%%%%%%%%%%%%%%%%%%%%%%%%%%%%%%%%%%%%

\subsection{Configuration de ton client \emph{mail}}

La DSI met \`a  ta disposition une bo\^{i}te aux lettres \'electronique sur
le serveur \server{poly} ; cette section t'explique comment
configurer \app{Windows Mail}, \app{Kmail} et \app{Mac OS Mail} pour y avoir acc\`es. Tu peux bien
s\^{u}r utiliser \app{Thunderbird} si tu pr\'ef\`eres, les donn\'ees \`a  rentrer
pour la configuration sont les m\^emes ; quelques d\'etails sont donn\'es
dans le WikiX sur \fkz. De plus, tu trouveras des explications plus
d\'etaill\'ees dans le manuel r\'edig\'e par la DSI.

\paragraph{Outlook et Windows Mail}
\flimage{images/outlook-logo}{0.07}{l}

La proc\'edure suivante fonctionne aussi avec \app{Windows Mail}.
Lance \app{Outlook Express} et va dans le menu \menu{Outils},
\menu{Comptes\ldots}. Clique sur le bouton \menu{Ajouter\ldots} en
haut \`a  droite, puis sur \menu{Courrier\ldots}.

Pour \app{Windows Mail} c'est sur compte de messagerie qu'il faut cliquer, avant de cliquer sur suivant.

Remplis les \'ecrans de configuration avec les donn\'ees suivantes.
\begin{description}
  \item[Nom complet : ] ton nom (\guillemotleft~Martin Durand~\guillemotright , par exemple)
  \item[Adresse de messagerie : ] de la forme \mail{prenom.nom@polytechnique.edu}
  \item[Type de serveur de messagerie pour le courrier entrant : ] \menu{POP3}
  \item[Serveur de messagerie pour le courrier entrant : ] \server{poly.polytechnique.fr}
  \item[Serveur de messagerie pour le courrier sortant : ] \server{poly.polytechnique.fr} ou \newline \server{ssl.polytechnique.org}
  \item[Nom du compte : ] ton identifiant \server{poly} (en g\`en\`eral c'est \texttt{prenom.nom}, si ça ne marche pas, va voir le bureau \emph{login} de la DSI.)
  \item[Mot de passe : ] ton mot de passe \server{poly}
       v\'erifie bien que la case \menu{M\'emoriser le mot de passe} est coch\'ee.
\end{description}

Voil\`a , clique sur \menu{Continuer}, \menu{Terminer}.

Tu te retrouves alors sur la fen\^etre \menu{Comptes Internet}. Va sur
l'onglet \menu{Courrier}, clique sur le compte que tu viens de cr\'eer
puis sur \menu{Propri\'et\'es}. Clique sur l'onglet \menu{Avanc\'e} et
configure comme sur la capture~\ref{config:win:mail} ; en
particulier, coche la seconde case \menu{Ce serveur n\'ecessite une
connexion s\'ecuris\'ee (SSL)}.

Comme \c{c}a, tu peux d\'esormais recevoir des \emph{mails} avec une liaison
s\'ecuris\'ee vers \server{poly} pour que personne ne puisse les
intercepter. Il est possible que tu aies à accepter manuellement le certificat utilis\`e par l'\'Ecole pour que cela fonctionne correctement. Tu le trouveras sur \urllink{http://poly}. Une fois t\'el\'echarg\'e, il suffit de l'importer.

\imageref{images/win_config_mail_avance}{0.5}{Configuration avanc\'ee
des serveurs \emph{mail}}{!h}{config:win:mail}



\paragraph{Kmail}
\flimage{images/kmail-logo}{0.07}{l}

Pour \app{Kmail}, va dans \menu{Configuration}, \menu{Configurer Kmail}. Choisis la
rubrique \menu{Comptes}. Commence par cr\'eer un nouveau compte dans
l'onglet \menu{R\'eception des messages} en cliquant sur le bouton
\menu{Ajouter\ldots} et choisis le type POP3.


\noindent
  \begin{figure*}[!h]
    \begin{center}  
      \subfloat[R\'eception des messages]{ 
      \includegraphics[width=0.48\textwidth]{images/nux_config_kmail_pop} }
      \hspace{\stretch{1}}
      \subfloat[Envoi des messages]{ 
 \includegraphics[width=0.48 \textwidth]{images/nux_config_kmail_smtp} }
 \caption{Configuration sous \app{Kmail}}
    \end{center}
  \end{figure*}


Utilise les param\`etres suivants pour configurer l'onglet \menu{G\'en\'eral}.
\begin{description}
  \item[Nom : ] le nom du compte, par exemple : Mails Poly
  \item[Utilisateur : ] l'identifiant \server{poly} que t'a fourni la DSI \`a  ton arriv\'ee sur le plateau
  \item[Mot de passe : ] le mot de passe \server{poly}
  \item[Serveur : ] \server{poly.polytechnique.fr}
  \item[Port : ] 995
\end{description}
Ensuite, va dans l'onglet \menu{Extras} et coche la case
\menu{Utiliser SSL pour s\'ecuriser les t\'el\'echargements}.

Maintenant, dans l'onglet \menu{Envoi des messages} clique sur le
bouton \menu{Ajouter\ldots}. Utilise les param\`etres suivants pour le
configurer :
\begin{description}
  \item[Nom] le m\^eme nom de compte que pr\'ec\'edemment
  \item[Serveur] \server{poly.polytechnique.fr} ou \server{ssl.polytechnique.org}
  \item[Port] 25
\end{description}
Sinon, laisse toutes les cases d\'ecoch\'ees.

%Tu peux aussi configurer l'acc\`es \`a  \app{l'annuaire LDAP} de l'\'Ecole, sorte de carnet d'adresses en ligne qui contient les adresses \emph{mail} de tout le monde sur le campus. Pour ce faire, commence par ouvrir \menu{Outils}, \menu{Carnet d'adresses}, puis va dans \menu{Configuration}, \menu{Configurer kAdressBook}, \menu{Consultation LDAP}. Clique ensuite sur \menu{Ajouter un h\^ote}, et configure comme suit: \\
%\smallskip
%\begin{minipage}[t]{0.48\textwidth}
%\begin{description}
%  \item[H\^ote] \server{ldap.eleves.polytechnique.fr}
%  \item[Port] 389
%  \item[Version de LDAP] 3
%\end{description}  
%\end{minipage} 
%\begin{minipage}[t]{0.48\textwidth}
%\begin{description}  
%  \item[DN] \server{dc=polytechnique, dc=ldap, dc=eleves, dc=fr}
%  \item[S\'ecurit\'e] Non
%  \item[Identification] Anonyme
%\end{description}
%\end{minipage} \\
%Une fois revenu dans \menu{Configuration LDAP}, coche la case \server{ldap.eleves.polytechnique.fr}. Tu as maintenant acc\`es \`a  l'annuaire LDAP lors de la
%r\'edaction de messages, avec tout au plus un red\'emarrage de \app{Kmail}. 
%
%\imagepos{images/nux_config_ldap}{0.55}{Configuration de l'annuaire LDAP sous Kmail}{pht}
%\imagepos{images/nux_config_knode}{0.45}{Configuration de Knode}{ht}
%
%\noindent
%  \begin{figure*}[!h]
%    \begin{center}  
%      \subfloat[Configuration de l'annuaire LDAP sous Kmail]{ 
%      \includegraphics[width=0.48\textwidth]{images/nux_config_ldap}}
%      \hspace{\stretch{1}}
%      \subfloat[Configuration de Knode]{ 
% \includegraphics[width=0.48 \textwidth]{images/nux_config_ldap} }
% \caption{Configurations LDAP et \emph{news}}
%    \end{center}
% \end{figure*}


\paragraph{Mac OS Mail}

\flimage{images/mac_mail_icone}{0.07}{l} \app{Mail} : un client \emph{mail} offrant les fonctionnalit\'es classiques d'un bon client : recherche instantan\'ee, filtre antispam, r\`egles de tri automatique des \emph{mails}, regroupement des \emph{mails} correspondant \`a  une m\^eme discussion.

Au premier lancement, \app{Mail} te demandera de remplir les informations concernant ton compte \emph{mail} sur \server{poly}, il suffit de le remplir avec les donn\'ees suivantes :
\begin{description}
  \item[Nom complet] ton nom !
  \item[Adresse \'electronique] de la forme \mail{prenom.nom@polytechnique.edu}
  \item[Serveur de r\'eception] \server{poly.polytechnique.fr}
  \item[Type de compte] \menu{POP}
  \item[Nom d'utilisateur] ton \emph{login} \server{poly} (les huit premi\`eres lettres de ton nom en g\'en\'eral)
  \item[Mot de passe] ton mot de passe \server{poly}
  \item[Serveur d'envoi (SMTP)] \server{poly.polytechnique.fr} ou \server{ssl.polytechnique.org}
\end{description}

Si tu as d\'ej\`a  cr\'e\'e un compte pr\'ec\'edemment, il faut aller dans les \menu{Pr\'ef\'erences} (accessibles depuis le menu \menu{Mail}), onglet \menu{Comptes}, pour cr\'eer un autre compte en cliquant sur la case \menu{+}.

N'oublie pas de cocher \menu{Activer le cryptage SSL} dans l'onglet \menu{Avanc\'e}, port 995. Tu souhaiteras alors certainement installer le certificat de s\'ecurit\'e de \server{poly} (tu le trouveras sur \urllink{http://poly/}). Une fois que tu as t\'el\'echarg\'e le certificat, ouvre le fichier \menu{.CRT} obtenu, et dans \app{Trousseau d'acc\`es}, installe-le dans %\menu{X509Anchors} (Tiger) ou 
\menu{session} (Leopard).

Cette configuration marche pour acc\'eder \`a  ses mails depuis l'int\'erieur de l'X mais aussi de l'ext\'erieur, sans rien changer.
En revanche tu ne peux pas envoyer de \emph{mails} depuis l'ext\'erieur, car le serveur \server{poly} ne le permet pas.
Nous te conseillons vivement d'utiliser le serveur SMTP \server{polytechnique.org} et de regarder la configuration propos\'ee par \urllink{Polytechnique.org}.
Celle-ci permet d'envoyer des \emph{mails} \`a  l'ext\'erieur de l'\'Ecole de façon s\`ecuris\`ee, sans modifier ta configuration par la suite.
Tu peux ajouter ce SMTP dans l'onglet \menu{Comptes} des \menu{Pr\'ef\'erences} de \emph{mail} et r\'egler dans l'onglet \menu{Avanc\'es} comme dans la capture.

\begin{figure*}[!hl]
    \begin{center}
            \includegraphics[width=0.4\textwidth]{images/mac_config_smtp_poltechnique.png} 
      \caption{Configurer le serveur SMTP \server{Polytechnique.org}}
    \end{center}
  \end{figure*}

%Le LDAP ne fonctionne pas \`a  l'heure actuelle avec \app{Mail} (mars 2009) sous \app{Leopard}, contrairement aux  autres syst\`emes d'exploitation (fonctionne cependant avec \app{Thunderbird}).
%\subsuEnfin, tu peux disposer dans \app{Mail} de l'annuaire de l'\`acole, mis \`a  disposition par la DSI. Pour cela, va dans les \menu{Pr\'ef\'erences} de Mail,
%puis dans la rubrique \menu{R\'edaction} et clique sur \menu{Configurer LDAP\ldots}. Tu peux ensuite utiliser le bouton \menu{+} pour ajouter un
%serveur, et remplir la fen\^etre comme sur la capture.

%\imagepos{images/mac_config_ldap}{0.6}{Configurer l'annuaire}{!ht}
%bsection{Logiciels additionnels}

%Les logiciels suivants sont utiles pour utiliser avec Mac OS X les services propos\'es sur le r\'eseau ; ils sont t\'el\'echargeables sur \server{frankiz}, dans la rubrique \menu{T\'el\'echarger}, \menu{Mac}.



%%%%%%%%%%%%%%%%%%%%%%%%%%%%%%%%%%%%%%%%%%%%%%%%
% ANCIENNE CONFIGURATION DES BR POUR CHAQUE OS %
%%%%%%%%%%%%%%%%%%%%%%%%%%%%%%%%%%%%%%%%%%%%%%%%

%% VIEILLE PAGE DE CONF NEWSGROUP POUR WINDOWS

%\subsubsection{Configuration \emph{newsgroups}}
%
%Reporte-toi a la page~\pageref{newsgroups} pour la description et des d\'etails de fonctionnement des \emph{newsgroups} \`a  l'X.
%
%Comme pour les \emph{mails}, nous d\'ecrivons la configuration de \app{Outlook Express} mais elle est sensiblement \'equivalente pour \app{Thunderbird}. Lance
%\app{Outlook Express} et va dans le menu \menu{Outils}, \menu{Comptes\ldots}. Clique sur le bouton \menu{Ajouter\ldots} en haut \`a  droite,
%\menu{News\ldots}. Remplis les \'ecrans de configuration suivants avec ces donn\'ees :
%\begin{description}
%  \item[Nom complet] ton nom !
%  \item[Adresse de messagerie] de la forme \mail{prenom.nom@polytechnique.edu}
%  \item[Serveur de news (NNTP)] \server{news} ; v\'erifie \`a  ce moment que la case
%       \menu{Connexion \`a  mon serveur de news requise} n'est pas coch\'ee.
%\end{description}
%Voil\`a , clique sur \menu{Continuer}, \menu{Terminer}; tu es abonn\'e
%au serveur \emph{news} des \'el\`eves.
%
%Quand tu fermeras la fen\^etre `Comptes Internet', il va te demander \`a 
%quels \emph{newsgroups} tu veux t'abonner, tu n'auras qu'\`a  s\'electionner
%ceux qui t'int\'eressent. Reporte-toi \`a  la page \pageref{newsgroups}
%pour plus d'infos sur les newsgroups auxquels t'abonner !
%
%Si tu veux t'inscrire \`a  d'autres serveurs \emph{news}, refais cette
%proc\'edure en rentrant le nom du serveur qui t'int\'eresse \`a  la place
%de \fkz.
%\setcounter{page}{12}

%% VIEILLE PAGE DE CONF NEWSGROUPS POUR LINUX


%\subsubsection{Configuration \emph{news}}
%
%\flimage{images/nux_knode_icon}{0.12}{l} Le client \emph{news} le plus utilis\'e est \app{Knode}. Parmi les autres clients \emph{news}, citons 
%\app{Thunderbird}, \app{Pan} ou \app{slrn}. Ici aussi, la configuration est presque ind\'ependante du logiciel choisi.
%
%
%Sous \app{Knode}, c'est dans le menu \menu{Configuration}, puis \menu{Configurer Knode}. Va dans la rubrique \menu{Comptes, Forums de discussion} et
%cr\'ee un compte en cliquant sur \menu{Ajouter\ldots}.
%
%\imagepos{images/nux_config_knode}{0.45}{Configuration de Knode}{ht}
%
%\pagebreak
%
%Remplis l'onglet \menu{Serveur} avec les informations suivantes :
%\begin{description}
%  \item[Nom] ce que tu veux pour d\'ecrire ce compte, par exemple 'News Frankiz'
%  \item[Serveur] \server{news}
%\nopagebreak  \item[Port] 119
%\end{description}
%
%\pagebreak
% 
%Ensuite occupe-toi de l'onglet \menu{Identit\'e} :
%\begin{description}
%  \item[Nom] mets ton pseudo dans ce champ
%  \item[Organisation] X, \'Ecole polytechnique, comme tu le sens
%  \item[Adresse \'electronique] ton adresse \emph{mail}, pour que les gens puissent te r\'epondre par \emph{mail}.
%\end{description}
%
%Enfin, pour que \app{Knode} puisse envoyer des \emph{mails}, il faut aller
%dans la rubrique \menu{Comptes}, sous-rubrique \menu{Serveur de
%courrier (SMTP)}, et choisir comme serveur d'envoi de \emph{mails}
%\server{poly.polytechnique.fr}, port 25 --- c'est exactement la m\^eme
%configuration SMTP que \app{Kmail}.
%
%Si tu veux mettre une signature \`a  la fin des messages que tu
%posteras, il te suffit de la mettre dans l'onglet \menu{Identit\'e}.
%Sur la plupart des clients la signature est interpr\'et\'ee comme
%ext\'erieure au message et n'est en particulier pas incluse dans le
%texte cit\'e lorsque tu r\'eponds \`a  un message. Pour d\'efinir une
%signature \`a  la main, il suffit de mettre \verb*+-- +\ (c'est \`a  dire
%-{}-<espace>) sur une ligne, et tout ce qui suivra cette ligne
%composera ta signature.
%
%Il ne te reste plus qu'\`a  t'inscrire \`a  des \emph{newsgroups} (reporte-toi \`a  la page \pageref{newsgroups} pour plus d'infos) et \`a  poster ! \\
%
%Pour te connecter aux serveurs de \emph{news} de Polytechnique.org, qui ont un acc\`es s\'ecuris\'e, avec \app{Knode}, il y a une petite subtilit\'e car il
%ne g\`ere pas le SSL. Il faut installer \app{stunnel} qui permet de d\'efinir une redirection SSL de port. Dans \file{/etc/stunnel.conf} (ou parfois \file{/etc/stunnel/stunnel.conf}), mets les lignes suivantes (les trois premi\`eres y sont en principe d\'ej\`a ) :
%\cmdline{\# location of pid file\\
%pid = /etc/stunnel/stunnel.pid\\
%\\
%\# user to run as\\
%setuid = stunnel\\
%setgid = stunnel\\
%\\
%\# Use it for client mode\\
%client = yes\\
%\\
%\# sample service-level configuration\\
%{[}nntps{]}\\
%accept  = 1119\\
%connect = ssl.polytechnique.org:563\\
%TIMEOUTclose = 0
%}
%
%Il ne te reste plus qu'\`a  lancer \app{stunnel} par :
%\cmdline{/etc/init.d/stunnel start}
%
%Et tu peux ainsi lire les \emph{news} de Polytechnique.org en mettant \server{localhost} comme serveur et
%\server{1119} comme port. Il faut aussi que tu coches \menu{Le serveur exige une identification} et
%que tu rentres ton nom d'utilisateur \`a  Polytechnique.org et ton mot de passe, que tu peux d\'efinir
%sur \urllink{https://www.polytechnique.org/Xorg/SMTPSecurise}.

%% VIEILLE PAGE DE CONF NEWSGROUP POUR MAC


%\subsubsection{Configuration \emph{news}}
%
%\flimage{images/mac_thunderbird_icone}{0.07}{l} 
%\app{Thunderbird} : un client \emph{news} permettant d'acc\'eder aux forums de discussion des \'el\`eves (voir page~\pageref{newsgroups} pour les d\'etails sur \server{frankiz}), mais aussi \`a  ceux de \server{usenet} gr\^ace au serveur \server{polynews.polytechnique.fr}. Il est tr\`es proche d'\app{Outlook Express} dans son esprit. Dans la m\^eme cat\'egorie, il existe \app{MacSOUP}, \app{Unison} ou encore \app{MT-NewsWatcher}. La configuration se fait de la m\^eme mani\`ere.
%
%Au premier lancement, l'application te propose d'importer les param\`etres depuis une autre application. Clique sur \menu{Suivant}. Tu peux alors choisir quel type de compte tu veux configurer (tu remarqueras que tu peux aussi cr\'eer un compte courrier \'electronique, et un compte RSS). S\'electionne \menu{Compte forums de discussion} et clique sur \menu{Suivant}. Entre alors les informations suivantes :
%
%\begin{description}
%  \item[Votre nom] ton nom ou ton pseudo
%  \item[Adresse de courrier] \mail{prenom.nom@polytechnique.edu}
%  \item[Serveur de forums] \server{news}
%  \item[Nom du compte] News Frankiz
%  \item[Nom d'utilisateur] ton \emph{login }poly (les huit premi\`eres lettres de ton nom en g\'en\'eral)
%  \item[Serveur d'envoi (SMTP)] \server{poly.polytechnique.fr} ou \server{ssl.polytechnique.org}
%\end{description}
%
%
%Pour t'abonner \`a  des groupes de discussion, il te suffit de s\'electionner le compte \menu{News Frankiz} dans la fen\^etre \menu{Dossiers} de \app{Thunderbird}, puis de cliquer sur \menu{G\'erer les abonnements aux groupes de discussion}. Tu pourras ensuite s\'electionner les forums qui t'int\'eressent parmi la liste propos\'ee. Reporte-toi \`a  la page \pageref{newsgroups} pour plus d'infos sur les \emph{newsgroups} auxquels t'abonner !
%


\label{wifi}
\subsection{\emph{Wi-Fi}}
La DSI propose actuellement un réseau \emph{Wi-Fi}, qui couvre le grand hall, les amphis, les salles de PC,
le bataclan (bâtiment qui va de la Kès au bâtiment des binets/langues), le bâtiment des binets/langues.

%Pour te connecter au \emph{Wi-Fi} Polytechnique avec Windows, Mac, Linux ou un iPhone, tu trouveras les instructions sur la page :
%\begin{center}
%\urllink{http://www.dsi.polytechnique.fr/fr/telecommunications/wifi/wifi-73063.kjsp}
%\end{center}

%Avec Windows, \textbf{avant Windows 8}, tu dois téléchager un logiciel appelé \app{SecureW2} qui est fourni par la DSI sur son site, \urllink{http://www.dsi.polytechnique.fr/fr/telecommunications/wifi/}. Ce n'est plus nécessaire depuis Windows 8 mais des paramètres supplémentaires portant sur l'authenfication 802.1x sont à modifier, la procédure est décrite sur la page susmentionnée.
%
%Avec Mac OS X Lion ou iOS (iPhone, iPad), il faut télécharger un fichier 
%\newline \file{Ecole-Polytechnique.mobileconfig} dont le lien se trouve sur \urllink{http://www.dsi.polytechnique.fr/fr/telecommunications/wifi/wifi-73063.kjsp}. Pour des versions plus anciennes de Mac OS, consulte \urllink{http://br.binets.fr/Configuration\_du\_WiFi\_sous\_Mac}. Attention, avec iOS7, il arrive que le fichier .mobileconfig fourni par la DSI ne marche pas, tu peux essayer avec celui là: \urllink{br.binets.fr/files/WifiPoly.mobileconfig}.
%
%Avec Linux ou Android, les noms des paramètres dépendent du système utilisé, mais voici un tableau récapitulatif :
%\begin{center}
%\begin{tabular}{r|l}
% SSID & Polytechnique \\
% Nom d'utilisateur/Mot de passe & Identifiants DSI (salle info) \\
% Sécurité & WPA1 Entreprise \\
% Gestion des clés & WPA-EAP \\
% Pairwise & TKIP \\
% Authentification & Tunneled TLS (TTLS) ou EAP-FAST \\
% Authentification interne & PAP \\
% Proxy HTTP pour tous les protocoles & 129.104.247.2 (port 8080) \\
% Serveurs DNS & 129.104.201.53, 129.104.201.51
%\end{tabular}
%\end{center}

Pour te connecter au \emph{Wi-Fi} Polytechnique avec Windows, Mac, Linux ou un iPhone, tu trouveras les instructions sur la page :
\begin{center}
\urllink{https://portail.polytechnique.edu/dsi/wifi}
\end{center}

%Avec Windows, \textbf{avant Windows 8}, tu dois téléchager un logiciel appelé \app{SecureW2} qui est fourni par la DSI sur son site, \urllink{http://www.dsi.polytechnique.fr/fr/telecommunications/wifi/}. Ce n'est plus nécessaire depuis Windows 8.\\

%Avec Mac OS X Lion ou iOS (iPhone, iPad), il faut télécharger un fichier intitulé
%\newline \file{Ecole-Polytechnique.mobileconfig} dont le lien se trouve sur :
%\begin{center}
%\urllink{http://www.dsi.polytechnique.fr/fr/telecommunications/wifi/wifi-73063.kjsp}
%\end{center}
%
%Pour des versions plus anciennes de Mac OS, consulte :\\ \urllink{http://br.binets.fr/Configuration\_du\_WiFi\_sous\_Mac}. Attention, avec iOS7, il arrive que le fichier .mobileconfig fourni par la DSI ne marche pas, tu peux essayer avec celui là : \urllink{br.binets.fr/files/WifiPoly.mobileconfig}.\\
%
%Avec Linux ou Android, les noms des paramètres dépendent du système utilisé, mais voici un tableau récapitulatif :
%\begin{center}
%\begin{tabular}{r|l}
% SSID & Polytechnique \\
% Nom d'utilisateur/Mot de passe & Identifiants DSI (salle info) \\
% Sécurité & WPA1 Entreprise \\
% Gestion des clés & WPA-EAP \\
% Pairwise & TKIP \\
% Authentification & Tunneled TLS (TTLS) ou EAP-FAST \\
% Authentification interne & PAP \\
% Proxy HTTP pour tous les protocoles & 129.104.247.2 (port 8080) \\
% Serveurs DNS & 129.104.201.53, 129.104.201.51
%\end{tabular}
%\end{center}




%Deux réseaux ont été déployés :

%\begin{description}
%  \item[keriadenn] : c'est le réseau public, qui te permet uniquement d'accéder au portail wifi (\url{http://wifi/}, accessible également depuis le réseau normal). Tu trouveras à cette adresse toutes les informations de configuration nécessaires pour te connecter au second réseau, \server{kastell}.

%  \item[kastell] : réseau protégé et caché qui permet, après authentification, de te connecter au réseau et à Internet comme si tu étais dans ton casert !
%\end{description}


% -------------------- Puis... --------------------
%\setlength{\parskip}{7pt}
\bghdr{images/fond-infobr}
\markright{Pour continuer}
%$Id: partie2.tex 110 2005-03-02 15:56:44Z myk $
\clearpage
\section{Présentation du réseau et de plusieurs services}
\label{services}

\subsubsection{Frankiz}
\label{frankiz}
La page web frankiz est la page des \'el\`eves.
Elle est visible de l'int\'erieur et de l'ext\'erieur de l'X,
en int\'egralit\'e si tu t'es identifi\'e ou partiellement pour les autres utilisateurs.
Tu peux automatiser ta connexion gr\^ace \`a un cookie d'authentification.
Nous te conseillons de faire de \url{http://frankiz/} la page d'accueil de ton navigateur Internet.

Elle permet en particulier l'acc\`es aux services suivants :
les annonces et les activit\'es du plat\^al, l'annuaire des \'el\`eves (\menu{TOL} pour Trombi-On-Line),
le t\'el\'echargement de logiciels gratuits (\menu{X-Share}), la foire aux questions (\menu{FAQ}),
la question du jour (\menu{QDJ}) et m\^eme la m\'et\'eo.
Cette page est ais\'ement personnalisable (lien \lien{Pr\'ef\'erences}).
\`A toi d'explorer tout ce que tu peux y trouver !

Les \menu{annonces}, \menu{sondages} et \menu{activit\'es} permettent d'informer les \'el\`eves
de ce qui se passe \`a l'\'ecole.
Les annonces sont tri\'ees dans un sommaire et tu peux \'eventuellement faire dispara\^itre
celles qui ne t'int\'eressent pas (en fonction des skins).
Les activit\'es appara\^issent sur la page principale le jour o\`u elles ont lieu.
Les sondages apparaissent sur la page principale et lorsque le vote est termin\'e, tu peux voir les r\'esultats.
Tu peux proposer des annonces, des sondages et des activit\'es en utilisant les liens \lien{Proposer\ldots}.

L'\menu{Annuaire} permet de trouver des renseignements utiles sur tous les \'el\`eves sur le pl\^atal.
Tu peux mettre ta fiche \`a jour en utilisant le lien \lien{Pr\'ef\'erences}.

Si tu es le pr\'esident ou le webmestre d'un binet,
tu as droit \`a un lien \lien{Administration} \`a c\^ot\'e de ton lien \lien{Pr\'ef\'erences}.
Il te permet d'exercer les terribles pouvoirs du prez ou du web,
qui sont respectivement de g\'erer la liste des membres inscrits au binet dans le TOL
et de modifier l'ic\^one et la description du binet dans la page \menu{Binets}.
Pour cela, il suffit d'avoir sign\'e la feuille de demande de droit,
qui est (normalement) disponible dans la case courrier du BR \`a la K\`es.

La \menu{Foire Aux Questions} contient les r\'eponses aux questions les plus courantes.
C'est souvent plus rapide d'y faire un petit tour que d'appeler quelqu'un,
en plus il y a un moteur de recherche.
Et si tu vois une erreur, tu peux la corriger directement !

La rubrique \menu{X-Share} permet de t\'el\'echarger des logiciels pour Windows, Mac et Linux
s\'electionn\'es par le BR et des documents importants comme cet InfoBR.
En particulier, c'est l\`a que tu trouveras les logiciels d\'evelopp\'es par le BR, dont \app{qRezix}.
Si tu cherches un logiciel pour un usage particulier, commence par l\`a !

Publi-reportage : la \menu{FAQ} et les \menu{X-Share} (entre autres) sont en cours de refonte
afin d'am\'eliorer leur ergonomie.
Si tu as des talents de programmeur, n'h\'esites pas, le d\'eveloppement de frankiz n'attend que toi !!

La \menu{QDJ} est une question binaire, s\'erieuse parfois mais le plus souvent bas\'ee sur un jeu de mots
ou sur l'activit\'e sur le campus.
Tu peux voter tous les jours et m\^eme proposer des questions au QDJMaster.

La rubrique \menu{Sites \'el\`eves} contient la liste des sites personnels des \'el\`eves h\'eb\'erg\'es sur frankiz.
Tu peux toi aussi publier ton site web en utilisant le lien dans \lien{Pr\'ef\'erences}.
De m\^eme, la rubrique \menu{Binets} contient la liste des binets, une description de celui-ci
et le lien vers le site de ce binet (\'eventuellement h\'eberg\'e sur frankiz).

M\^eme si frankiz est l'\oe uvre de tous, les webmestres se r\'eservent le droit de ne pas publier une annonce
ou d'interdire un site web si le contenu n'est pas jug\'e adapt\'e mais aussi,
dans le cas des annonces si elle g\^ene la lisibilit\'e g\'en\'erale, selon le bon principe :
``trop d'information tue l'information''.

Rappelons quelques r\`egles \'evidentes :
\begin{itemize}
 \item Tout contenu pol\'emique est banni des annonces (publie tes aigreurs dans l'IK ou sur \ngname{br.binet.polemix}).
       Les annonces d'un go\^ut douteux ne sont pas non plus les bienvenues (m\^eme recommandation).
 \item La publicit\'e n'a pas sa place sur \fkz ; toute annonce ayant un net caract\`ere publicitaire
       sera refus\'ee (y compris s'il s'agit de publicit\'e pour un sponsor).
 \item Tout contenu portant atteinte \`a une tierce personne ou \`a un groupe est interdit dans les annonces
       et les sites web, ainsi que tout lien vers un site ou document de ce type.
 \item Tout contenu ill\'egal, en particulier tout document (quel que soit son type)
       non libre de droits ou ayant un caract\`ere pornographique, est interdit,
       ainsi que tout lien vers un site ou document de ce type.
 \item Si un contenu d'un des deux types pr\'ec\'edents \'echappe toutefois \`a l'attention des webmestres,
       seuls leurs auteurs pourraient en \^etre tenus responsables.
       Tout contenu de cette sorte qui appara\^itrait sur le site doit \^etre imm\'ediatement signal\'e.
\end{itemize}

Les r\'egles \'el\'ementaires pour pr\'eserver la lisibilit\'e de frankiz :
\begin{itemize}
 \item Les annonces ne doivent pas \^etre trop longues (pas plus d'une quinzaine de lignes)
 \item L'interface des annonces utilise la syntaxe wiki, qui est expliqu\'ee
       sur une page accessible facilement depuis la page de proposition :
       n'h\'esites pas \`a y faire un petit tour pour savoir comment embellir tes annonces
       (gras, italique, liens hypertextes)
 \item Les titres des annonces ne doivent pas \^etre en majuscules ou pr\'ec\'ed\'es de signes de ponctuation.
 \item Un binet ne peut pas avoir deux annonces en m\^eme temps.
       Alors s'il te pla\^it, quand tu fais la com' de ton binet, r\'efl\'echis et \'ecris une belle annonce :
       plus elle est concise et pr\'ecise, mieux elle sera lue !
\end{itemize}

Enfin, tout ce qui est propos\'e sur frankiz doit \^etre valid\'e par les webmestres,
qui ne sont pas l\`a pour censurer mais pour maintenir la qualit\'e du site.
Inconv\'enient : \c{c}a peut parfois \^etre un petit peu long mais faut pas s'inqui\'eter.
Dans tous les cas, si tu as une demande quelconque \`a faire sur le contenu du site,
genre modifier/supprimer une annonce, un seul r\'eflexe :
envoies un mail \`a \mail{web@frankiz}, r\'eponse rapide (presque) assur\'ee !


\subsection{Le logiciel des bars d'étage : Chocapix}
\label{chocapix}

Chocapix est le tout nouveau site des bars d'étage codé par les soins du BR. Il est accessible sur tout le réseau élève, c'est-à-dire à la fois dans les bars d'étage et dans les caserts, à l'adresse \urllink{http://chocapix/}.\\
Cet outil permet de simplifier la vie au sein du bar d'étage à la fois du point de vue utilisateur que du point de vue des respos bars en permettant :
\begin{itemize}
\item De loguer facilement et rapidement les consommations ;
\item De faciliter les appros, en particulier avec l'utilisation d'un lecteur de code-barres et de la base de donnée hébergée par le BR ;
\item De simplifier la comptabilité du bar.\\
\end{itemize}
Tu trouveras plus d'informations sur ses fonctionnalités et sur son utilisation dans l'InfoBar distribué aux respos de chaque bar d'étage.



\subsection{Une source d'informations inestimable : le WikiX}
\label{WikiX}
Bien que n'\'etant pas \`a proprement parler un service du Binet R\'eseau, le BR h\'eberge sur un de ses serveurs un site un peu particulier connu sous le nom de \textbf{WikiX}. C'est un wiki qui rassemble toutes les informations dont tu peux avoir besoin sur le pl\^atal. Le plus court moyen d'y aller est de taper "wikix" (sans les ") dans la barre d'adresse de ton navigateur mais il y a aussi un lien vers le wikix sur Frankiz dans le menu navigation.
\newline
\newline
La premi\`ere fois que tu te connectes sur le WikiX il faut que tu t'identifies (identification via polytechnique.org) ensuite ton navigateur gardera un cookie qui t'identifiera \`a chaque connection si tu le souhaites. \emph{Tu ne peux modifier le WikiX que si tu es identifi\'e}!!
\newline
\newline
Tu es bien entendu encourag\'e \`a contribuer au WikiX pour faire profiter les autres de ton exp\'erience soit en modifiant/mettant \`a jour un article existant soit en cr\'eant un nouvel article qui manquait au WikiX.
\newline
\newline
Tu te rendras vite compte que \emph{peu importe l'information que tu cherches, elle est sur le WikiX.}


\subsection{Polytechnique.org}
\urllink{Polytechnique.org} est une association loi 1901 composée d'élèves et d'Anciens
 indépendante de l'administration de l'\'Ecole (donc des domaines \urllink{polytechnique.fr}
 et \urllink{polytechnique.edu}). Le but de l'association est la mise à disposition des X d'outils
ayant un rapport avec l'Internet, entre autres :
\begin{itemize}
  \item des redirections de \emph{mails} nombreuses (adresses supplémentaires) et à vie ;
  \item un serveur de \emph{news} (comme les \ngname{br.*}), ouvert notamment aux Anciens;
  \item une facilitation des contacts vers les Anciens et les camarades de promotion ;
  \item une lettre mensuelle, pour s'informer sur l'actualité de la communauté polytechnicienne ;
  \item des annonces d'événements ;
  \item des services d'hébergement pour les groupes et binets, notamment des noms de domaine (\emph{via} \server{www.polytechnique.net}) et des listes de diffusion (comme par exemple \mail{br@2008.polytechnique.org}).
\end{itemize}
Si tu veux découvrir les autres services de l'association ou savoir
comment les utiliser, tu peux aller sur la page
\urllink{https://www.polytechnique.org/Xorg/Xorg} (accessible depuis
le lien \menu{Documentations} dans le menu de \urllink{Polytechnique.org}
quand tu es connecté).

Par ailleurs, les filtres antivirus et antispam appliqués aux \emph{mails} sont très efficaces (99\,\% de repérage correct); \urllink{Polytechnique.org} te conseille donc de mettre en place la redirection suivante :
\mail{prenom.nom@polytechnique.edu}
redirigée vers \mail{prenom.nom(.promo)@polytechnique.org},
elle-même redirigée sur \mail{login@poly(.polytechnique.fr)}.
Pour réaliser ces redirections, connecte-toi sur les pages suivantes :
\begin{itemize}
  \item pour \mail{@polytechnique.edu} : \urllink{https://www.mail.polytechnique.edu} ;
  \item pour \mail{@polytechnique.org} : \urllink{https://www.polytechnique.org} ;
  \item pour \mail{@poly} : \urllink{http://poly.polytechnique.fr}.
\end{itemize}
Pour plus de détails, rends-toi sur \urllink{https://www.polytechnique.org/Xorg/RedirectionMails}.


 Ces outils sont très utiles, que
ce soit pour toi, pour tes binets ou pour garder plus tard contact avec la communauté polytechnicienne. Rejoins
les \nombre{15000} camarades déjà inscrits ! Et en cas de problème, n'hésite pas à contacter
\mail{contact@polytechnique.org}.


\subsection{Partager tes fichiers sur le réseau avec FTP}
\bghdr{images/fond-infobr}

Les échanges de fichiers sur le réseau élèves se font souvent par FTP. Rien de plus simple que de partager toi aussi tes précieuses données en installant un serveur FTP~!

\paragraph{Client FTP}
Pour une utilisation basique, taper \urllink{ftp://nom-du-ftp}  (par exemple \urllink{ftp://jtx} dans la barre d'adresse de ton navigateur suffit à parcourir les fichiers proposés par \og Gentil Vieux-Chouffe \fg.
Pour une meilleure utilisation, le BR te conseille \app{FileZilla}. Télécharge-le sur \urllink{http://www.filezilla.fr} et double-clique sur l'installeur.
Tu pourras dès la fin de l'installation aller sur tous les FTP du réseau facilement et rapidement.\\
\flimage{images/mac_cyberduck_icone}{0.07}{l} \app{Cyberduck} est un autre client FTP très simple à  utiliser et performant. Il te permettra d'aller télécharger des fichiers sur les serveurs FTP des autres élèves sans problème.\\
Pour se connecter à  un serveur, il suffit de taper son nom (exemple~: \urllink{jtx}) dans le cadre \menu{Connexion rapide} puis d'appuyer sur Entrée.\\


\paragraph{Serveur FTP}
Tu verras rapidement que tout le monde à  l'X possède un serveur FTP
afin de partager les différents projets, les films du JTX, ses
photos, etc. Il est donc quasiment indispensable que tu en installes un.\\

Parmi les plus simples on trouve \app{FileZilla Server} et \app{GuildFTP}, qui sont libres de surcroît.
Si tu es sous mac, tu peux aussi jeter un œ{}il à \app{PureFTPd Manager}, qui est très pratique à utiliser.\\
Quoiqu'il en soit, tu trouveras toutes les informations nécessaires à la configuration de ton serveur FTP sur \urllink{http://wikix.polytechnique.org/FTP}.\\
Pense à lui donner un nom sur \urllink{http://dnsapp/} (voir en dessous).


\subsection{Services propos\'es aux binets}

Le BR propose plusieurs services aux binets :
\begin{itemize}
\item le r\'ef\'erencement sur \fkz, avec tous les services associ\'es (adresse mail, annonces, mail aux membres) : consulte l'infoFrankiz pour les d\'etails ;
\item les Platalpads de binet (\urllink{http://nom\_du\_binet.platalpad.binets.fr}), accessibles aux membres du groupe \fkz\ (voir p. \pageref{platalpad}) : ce service est cr\'e\'e automatiquement lors de la cr\'eation du binet sur \fkz ;
\item l'h\'ebergement de leur site internet, qui peut \^etre interne (visible uniquement depuis l'X) ou externe (visible de l'ext\'erieur de l'X).\\
\end{itemize}

Pour disposer de ces services, tu dois d'abord d\'eclarer ton binet \`a la K\`es. Ensuite, pour la cr\'eation du binet sur \fkz, envoie un mail \`a \mail{web@frankiz.net} en pr\'ecisant le nom du binet, le nom du prez, le nom du respo com ou du respo web et que tu l'as bien d\'eclar\'e \`a la K\`es. Pour faire h\'eberger ton site web, envoie un mail \`a \mail{root@eleves.polytechnique.fr}.

\vspace{4mm}
%Le BR propose \'egalement aux binets qui le souhaitent d'h\'eberger leur site Internet. Ce site peut \^etre interne (visible uniquement depuis l'X) ou externe (visible de l'ext\'erieur de l'X). Si tu d\'esires que ton site soit accessible \`a l'ext\'erieur, il te faut remplir une fiche pour la DSI et nous.

Les sites binets disposent de PHP et MySQL, et b\'en\'eficient d'une capacit\'e de stockage (extensible) de 100 Mo.

Si tu d\'esires que ton site soit accessible \`a l'ext\'erieur, il te faut remplir une fiche pour la DSI et nous. Les sites ayant une visibilit\'e ext\'erieure doivent satisfaire aux conditions suivantes :
\begin{itemize}
   % \item aucune information ne doit \^etre diffus\'ee qui pourrait nuire \`a l'image de l'\'e�cole (photos, vid\'eos, etc.). En particulier le contenu doit respecter la loi fran\`a�aise sur les droits d'auteur ;
    \item le site doit avoir une qualit\'e visuelle, si ce n'est professionnelle, du moins tr\`es correcte ;
    \item le site ne doit pas h\'eberger de vid\'eos ou diffuser un flux vid\'eo (streaming). Toutes les vid\'eos du sites doivent \^etre h\'eberg\'ees \`a l'ext\'erieur. (Dailymotion, YouTube, etc.) ;
    \item les images pr\'esentes sur le site doivent avoir une r\'esolution suffisamment faible afin de ne pas saturer la bande passante vers l'ext\'erieur. 
\end{itemize}

Le BR offre ce service gratuitement, en partie gr\^ace \`a une subvention de la K\`es.
Il se r\'eserve le droit de refuser ou d'interrompre l'h\'ebergement d'un site, sans pr\'eavis, sans recours possible et sans avoir \`a fournir de motif.
Il s'engage \`a en informer imm\'ediatement le bureau du binet concern\'e.\\
Pour plus d'informations sur ce service, visite la page du wikiBR : \newline \urllink{https://br.binets.fr/H\%C3\%A9bergement\_des\_sites\_des\_binets}.


\subsection{Un outil de travail collaboratif : Platalpad}
\label{platalpad}

\imagepos{images/platalpad}{0.6}{Un document en cours d'édition sur un platalpad privé}{!h}

Platalpad est un éditeur de texte collaboratif par navigateur, un peu comme un google doc. Il permet de travailler à plusieurs sur un même document, chacun voyant les modification des autres. Il est donc très pratique pour s'organiser.\\
Il se décline en deux versions :
\begin{itemize}

\item \textbf{Platalpad généraliste :} rends-toi tout simplement sur \urllink{http://platalpad/} pour créer un nouveau document. Ensuite, il suffit de diffuser son adresse à toutes les personnes que tu veux voir prendre part à sa réalisation. \\

\item \textbf{Platalpad privé :} Chaque binet se voit automatiquement, une fois inscrit sur \fkz, attribué un espace privé, accessible seulement à ses membres. Tu peux y accéder à partir de la page \fkz du binet en question ou directement sur \urllink{http://nom-du-binet.platalpad/}. Une fois identifié, tu peux voir chacun des documents en cours de réalisation au sein du binet et les modifier à ta guise.

On s'y connecte en utilisant ses identifiants \fkz. Ce service fonctionne aussi à l'extérieur via : 
\urllink{https://www.polytechnique.fr/eleves/platalpad/nom\_du\_binet} ;

\end{itemize}

%\imagepos{images/platalpad}{0.6}{Un document en cours d'édition sur un platalpad privé}{!h}

%$Id: irc.tex 144 2005-03-25 01:11:37Z myk $

\subsection{IRC}

\label{irc}

IRC est un autre moyen de communication mis à ta disposition par le binet Réseau. Il s'agit d'un système de \emph{chat} (messagerie instantanée) permettant à la fois de dialoguer à plusieurs dans des salons (ou canaux), mais également d'avoir des conversations privées avec d'autres personnes connectées.


Le serveur IRC du binet Réseau est relié à RezoSup, réseau IRC des grandes écoles d'ingénieurs et universités françaises.

Pour te connecter sur IRC tu disposes de deux méthodes:

                  
  \paragraph{XChat}  Un client IRC directement issu du monde Unix.
                 Tu peux te reporter \`a  la page \pageref{irc} pour plus d'infos sur l'IRC.


\begin{description}
\item[utiliser un client IRC:] nous te conseillons \app{X-Chat} (disponible sur internet). Utilise  \server{ircserver} comme serveur, et \server{6667} (port par défaut) comme port.
  \item[passer par l'interface web:] utilise \urllink{http://ircserver/}. Tu pourras ainsi profiter d'IRC sans rien avoir à  installer.
\end{description}

 
Nous te conseillons les salons de discussion (\emph{channels}) suivants :
\begin{itemize}
  \item \ngname{\#x} le salon de tous les X
  \item \ngname{\#linux} si tu as des questions \`a poser sur linux
  \item \ngname{\#superquizz} un quizz en ligne (tape \texttt{!nick x} en arrivant)
  \item \ngname{\#br} le salon du BR !
\end{itemize}


\subsection{Avoir un nom sur le r\'eseau}
\label{dnsapp}

Tu as install\'e un serveur FTP sur ton ordinateur, voire un site web, mais pour le visiter, il faut taper ton adresse IP en entier...
Ce n'est gu\`ere pratique, et pour toi le BR a mis en place \urllink{http://dnsapp} !
En haut \`a droite de la page web, clique sur \og donne un nom \`a ton ordinateur \fg, choisis ton nom, et valide : le tour est jou\'e ! Chacun pourra d\'esormais se connecter \`a ton FTP en tapant \urllink{ftp://ton-nom/}.


%
\subsection{Le site du BR : un Wiki}
\label{siteBR}

Le site du BR est dans \server{http://gwennoz.polytechnique.fr/wiki/} et il contient plusieurs
informations sur le Binet R\'eseau.
Il est pr\'esent\'e sous la forme d'un Wiki (voir \url{http://fr.wikipedia.org/} pour l'original)
et te permettra d'acceder aux informations les plus r\'ecentes sur les diff\'erents services offerts,
sur nos projets (que tu peux d'ores et d\'ej\`a intuiter par les descriptions de cette section).

Il est compl\'ementaire \`a la {\sc FAQ} et \`a cet InfoBR : il contient des informations
de configuration pour les services offerts par le BR et des d\'etails pour les projets que
le BR m\'ene.




%%$Id: qrezix.tex 144 2005-03-25 01:11:37Z myk $

\subsection{La télévision du BR}
\label{TV}

Le BR diffuse sur le réseau plusieurs dizaines de chaines de télévision et radios. Pour les recevoir, nous recommandons \app{vlc}, disponible sur le X-Share.

\subsubsection{Configuration de vlc}

La liste des cha\^ines est diffusée sous forme d'annonces SAP. Pour voir ces annonces, ouvre ta liste de lecture (\menu{Vue}, puis \menu{Liste de lecture}), et active la découverte de services (\ref{vlc:config}).
Attention sous \app{Windows Vista} un probl\`eme de compatibilité connu entra\^ine un écran noir. Pour le résoudre le BR t'a préparé une page sur le Wikix.

\imagepos{images/vlc_config_sap.png}{0.75}{Configuration de vlc pour la télévision par le réseau}{h!}\label{vlc:config}

Tu auras ainsi dans ta liste de lecture les différents cha\^{i}nes disponibles.

\subsubsection{Autre méthode}

Si ton client préféré ne supporte pas les annonces SAP, ou que les annonces SAP ne marchent pas chez toi, tu peux récupérer la liste des cha\^ines par
PodCast, à l'adresse \urllink{http://tv.eleves.polytechnique.fr/tvbr.xml}. Sous \app{vlc}, active la découverte des services PodCast dans la liste de
lecture (\menu{Gérer}, \menu{Découverte de services}, \menu{Podcast}), puis va dans \menu{Param\`etres}, \menu{Préférences}, \menu{Liste de Lecture}, \menu{Découverte de services} et enfin \menu{Podcast} et
met l'adresse \url{http://tv.eleves.polytechnique.fr/tvbr.xml} dans le champ \guillemotleft~Liste des URLs~\guillemotright .

\subsubsection{Et si ça ne marche toujours pas?}

Vérifie que tu utilises bien la derni\`ere version de \app{vlc}. Les versions inférieures à 0.8.5 sont connues pour ne pas fonctionner.

Si rien ne marche, la raison la plus probable est un \emph{firewall} qui intercepte les flux télés. Configure ton \emph{firewall} afin d'autoriser
ces flux. Sous Linux, les r\`egles \emph{iptables} suivantes suffisent:

\cmdline{ -A INPUT -i eth0 -d 224.0.0.0/24 -j ACCEPT \\
   -A INPUT -i eth0 -d 239.255.42.0/24 -s 192.168.225.0/24 -p udp -m udp --dport 1234 -j ACCEPT\\
    -A INPUT -i eth0 -d 239.255.255.255/32 -p udp -m udp --dport 9875 -j ACCEPT\\
   -A OUTPUT -o eth0 -d 224.0.0.0/4 -j ACCEPT.}

%\subsection{Descriptions des diff�rents serveurs}
{\bf Serveurs du BR : }Voici la liste des principaux serveurs du BR,
que tu vas principalement utiliser durant tes deux ann\'ees sur le
plateau, ainsi que leurs IPs et les services qu'ils h\'ebergent.
Note que ces services peuvent � tout moment migrer d'une machine � une
autre en cas de besoin.


\begin{description}
        \item[frankiz] (\server{129.104.201.51}) : DNS secondaire,
        news, site web, sites des binets
        \item[gwennoz] (\server{129.104.201.52}) : DNS secondaire,
        d\'eveloppement, miroirs Linux
        \item[heol] (\server{129.104.201.53}) : DNS principale,
        xnetserver, ircserver
        \item[skinwel] (\server{129.104.201.54}) : DNS secondaire,
        SVN, t\'el\'e
        \item[wifi] (\server{129.104.201.56}) : Wifi
	\item[enez] (\server{129.104.201.61}) : Domaine windows
\end {description}

{\bf Serveurs de la DSI : }Etant donn\'e que le r�seau �l�ves est un
sous-r\'eseau de celui de la DSI, nous utilisons �galement les serveurs
de celle-ci et les services qu'ils h\'ebergent.

\begin{description}
        \item[kuzh] (\server{129.104.247.2}) : proxy http, pour internet
        \item[sil] (\server{129.104.247.3}) : proxy ftp, acc\`es ssh
        vers et depuis l'ext\'erieur
        \item[poly] (\server{129.104.247.5}) : mails
        \item[moned] : serveur d'authentification, permettant de
        changer ton mot de passe moned. Ce mot de passe est celui qui
        te permet de te connecter et d'utiliser n'importe
        quelle machine de salle info. Ton travail n'\'etant pas stock\'e
        en local, il t'est donc accessible, quelque soit le PC sur
        lequel tu te connectes.
\end {description}

{\bf ATTENTION : Les serveurs de la DSI sont \`a ta disposiion pour
des usages bien pr\'ecis, et ne servent pas de serveurs de
stockage. Seul sil est pr\'evu pour du transfert de fichiers. Tout
abus sera sanctionn\'e et pourra entra\^iner la perte de tes comptes
poly, moned ou sil}

% si y a moyen d'encadrer le paragraphe ci-dessus...

\newpage


%$Id: partie2.tex 110 2005-03-02 15:56:44Z myk $

\section{Référence Rapide}

\vspace*{\stretch{1}}
%$Id: questions_reponses.tex 144 2005-03-25 01:11:37Z myk $

\subsection{Questions-r�ponses}

Les questions les plus courantes sont r�pertori�es ici pour te faire gagner du temps !

\begin{description}

\item[J'ai une question sur l'informatique] je poste sur \ngname{br.informatique.[truc qui va bien]}.

\item[J'ai perdu mon mot de passe qRezix] je vais le re-d�finir dans mes  \menu{Pr�f�rences} sur \fkz : \url{http://frankiz/profil/reseau.php}.

\item[Je veux voir mon pseudo quand j'ai vot� � la QDJ] je d�finis mon pseudo sur ma fiche trombi sur la page \url{http://frankiz/profil/profil.php} (lien \menu{Pr�f�rences}).

\item[J'ai un deuxi�me ordinateur, qu'est-ce que je fais ?] je demande une deuxi�me IP sur la page \url{http://frankiz/profil/demande\_ip.php} et je la lui attribue.

\item[Je n'ai plus de r�seau] je vais voir la 4\textsuperscript{�me} de couverture.

\item[Mon client mail dit que \og l'autorit� de certification est inconnue \fg] je vais t�l�charger le certificat de s�curit� sur \url{https://poly/} et je l'installe.

\item[Je ne re�ois pas mes mails] V�rifie ta redirection sur \url{http://poly}.

\item[Je n'arrive pas � me connecter � \server{poly}] Essaye \server{poly.polytechnique.fr}.

\item[Mon ordinateur n'a pas de nom sur le r�seau] J'installe qRezix et je le configure.

\item[Je cherche des informations sur l'Ecole] Regarde sur \url{http://intranet}.

\item[Je cherche � joindre une personne de l'administration] L'annuaire de l'�cole est sur \url{http://intranet/annuaire/}.

\item[Je cherche le num�ro de portable d'un X] : \url{http://www.polytechnique.org}.

\item[J'aimerais �tre un geek moi aussi !] J'apprends par c\oe ur \url{www.copinedegeek.com}.

\end{description}

\vspace*{\stretch{1}}
\newpage


%\subsection{Membres \'eminents du binet R\'eseau}

%Chaque BR-man signale quels syst\`emes d'exploitation il conna\^it.

%% Vide pour cette année

%\vspace{\stretch{1}}

%\emph{\`A force de rester enferm\'e dans leurs caserts sombres, les BR-men sont devenus trop sensibles �  la lumière pour supporter un flash d'appareil photo. Ainsi, seule une photo de dos a pu être prise.}

%\imagepos{images/br2k7.jpg}{0.9}{L'\'equipe du BR 2k7}{h}

%\vspace{\stretch{1}}



\subsubsection*{Description rapide des postes}

\begin{description}

  \item[Prez]{(\mail{prez@eleves}) Poste fictif, qui permet toutefois d'avoir
des relations bien plac\'ees.}

  \item[Trez]{(\mail{trez@eleves}) Garde-fou du Prez, elle est là pour contrôler que l'algorithme d'utilisation des sous est bien optimisé, sans aucune fuite de mémoire budgétaire.}

%  \item[relex]{Assistant du prez pour les relations avec les \emph{gens}.}

  \item[root]{(\mail{root@eleves}) Les \emph{roots} sont les administrateurs du r\'eseau. Ce sont eux qui s'\'evertuent \`a maintenir en \'etat de marche les serveurs, \`a rajouter de nouveaux services et \`a rep\'erer les boulets qui font de la merde sur le r\'eseau. S'il s'agit de g\'erer un compte de binet, utilise plut\^ot \mail{binets@eleves}.}

  \item[admin@windows] {(\mail{windows@eleves}) Administrateurs du domaine Windows. En cas de probl\`eme avec Windows, ce sont les mieux plac\'es pour t'aider ; c'est bien s\^ur  plus facile si tu es sur le domaine ! Ils gèrent le contrat
MSDNAA avec Microsoft, et ils s'occupent de distribuer les licences
aux élèves qui les demandent.}
  \item[support@windows] {(\mail{support@eleves}) SOS d\'epannage Windows, j'\'ecoute ! Pr\^ets \`a tout pour sauver une jeune demoiselle (ou un jeune \emph{gens} \`a la rigueur) en d\'etresse avec son Windows\dots }

  \item[support@mac] {(\mail{support@eleves}) C'est un poste naturellement tranquille. Qui a besoin d'\^etre d\'epann\'e sur Mac? Ah, c'est vrai : celui qui a install\'e Windows dessus en suivant les conseils de l'InfoBR... }

%  \item[devel]{(\mail{qrezix@eleves}) Joyeux programmeurs qui sont l\`a pour am\'eliorer les logiciels du BR. Leurs efforts se concentrent principalement sur le d\'eveloppement de Frankiz 3, mais ils s'occupent \'egalement de \app{qRezix} et ses \emph{plug-ins}.}

%  \item[news] {(\mail{news@eleves}) Mainteneurs du serveur de \emph{news}, ils surveillent aussi ce que tu postes et que tu respectes les r\`egles de base comme les \emph{crossposts} (marteau-th\'erapie) \mbox{;-)}}

  \item[web] {(\mail{web@eleves}) Webmestres de \fkz, ils valident les annonces et les activit\'es et surveillent le contenu du site de ton binet ou de ton site perso.}

%  \item[X-share] {(\mail{xshare@eleves}) Personne sympathique qui cherche \`a longueur de temps de nouveaux logiciels gratuits ou mieux, libres \`a proposer aux \'el\`eves dans \xshare.}

  \item[InfoBR]{Agent de transfert du savoir des roots aux habitants du plat\^al.}

  \item[support@tv]{Charg\'es de maintenir la diffusion de la t\'el\'e sur le r\'eseau. Changeurs de cha\^ine, dieux du \emph{multicast}, ils sont les amis des \emph{switches}\dots\ ou pas.}

%  \item[QDJ Master] {(\mail{qdj@eleves}) Chaque jour un nouveau dilemne sur \fkz\dots\ n'h\'esitez pas \`a faire vos propositions \`a \urllink{qdj@eleves}.}

  \item[IRC-Op]{Responsable des relations avec RezoSup. Viendez sur IRC (\urllink{http://ircserver/}) !}

%  \item[tol] {(\mail{tol@eleves}) V\'erificateur de photos, il surveille le Trombi-On-Line.}

  \item[BR-woman]{Preuve vivante que le BR n'est pas un milieu enti\`erement masculin.}

\end{description}



\newpage
%$Id: lexique.tex 144 2005-03-25 01:11:37Z myk $

\subsection{Lexique}

\begin{description}
  \item[Avaya] Fournisseur de coupures réseau.
  \item[BR] Binet Réseau : le binet qui s'occupe d'administrer le réseau des élèves, de développer et de maintenir le site \urllink{http://frankiz/} et le logiciel \app{qRezix}.
  \item[Client] (voir serveur) : programme qui permet de se connecter à un serveur. Par exemple un client news (comme \app{Thunderbird}), un client FTP (comme \app{SmartFTP}), un client xNet (comme \app{qRezix})
  \item[Crosspost] mot préféré des newsmestres qui entraîne souvent une avalanche de mails d'insultes. Décrit l'art et la manière de ne pas pourrir les \emph{newsgroups}.
  \item[DNS] \emph{Domain Name Server} : associe un nom de machine à une IP, par exemple \server{frankiz} à 129.104.201.51.
  \item[DSI] (Direction des Système d'Information) ce sont eux qui gèrent tout le matériel informatique de l'école, ton téléphone, ton accès internet, tes mails\ldots\ un conseil : ne joue pas au plus malin avec eux.
  \item[Firewall] Logiciel de protection de ton ordinateur contre les infiltrations de vers ou de pirates informatiques.
  \item[FTP] \emph{File Transfer Protocol} : protocole réseau qui permet de s'échanger des fichiers.
  \item[IP] Adresse de ton ordinateur sur le réseau, composée de quatre nombres compris entre 0 et 255 ($129.104.xxx.xxx$). Elle identifie ta machine auprès des autres utilisateurs du réseau.
  \item[proxy] machine qui autorise (et du même coup restreint) les communications avec l'extérieur. Le \emph{proxy} protège tous les ordinateurs du réseau des attaques.
%  \item[Résolution DNS] trouver l'IP associée à un nom DNS pour pouvoir se connecter à la machine concernée.
  \item[RTF*] \emph{Read The Fucking}\ * : allusion sympathique au fait que tu aurais pu trouver l'information autre part\ldots\ tout près souvent (voir RTFIBR, RTFM, etc.)
  \item[RTFIBR] il semblerait que la réponse soit dans l'InfoBR. Souvent suivi du numéro de la page. L'usage de cette expression est traditionnellement associé au mot crosspost.
  \item[RTFM] M, c'est le manuel, le reste on l'a déjà expliqué ;-)
  \item[serveur] (Voir client) Programme qui permet d'accueillir des services. Comme par exemple le partage de fichier, le voisinage réseau, un site web\ldots\
  \item[Serveur] (Ce n'est pas le même qu'avant ;-)) Machine qui accueille des serveurs (là c'est celui d'au-dessus). Le BR possède des serveurs soigneusement cachés dans une salle secrète et blindée.
  \item[SSH] Connexion permettant de travailler sur une machine distante.
  \item[STFW] Variante : \emph{Search The Fucking Web}. \urllink{http://www.google.fr} est ton ami.
  \item[Troll] Débat polémique sans fin permettant de déployer la mauvaise foi des deux parties.
  \item[WikiBR] Le site du BR ; c'est là que tu peux trouver une description des services proposés par le BR, de ses projets, ainsi que
  d'informations plus approfondies concernant ta configuration.
  \item[Wiki*] De Wikipedia, terme d\'esignant la plus c\'el\`ebre des encycop\'edies libres en ligne, \'editables par tous. Par extension d\'esigne tous site construit sur ce mod\`ele, par exemple WikiX et WikiBR (enfin le BR ne te permet pas de tout \'editer sur ces pages).
\end{description}


\vfill


% -------------------- Enfin -------------------
%\setlength{\parskip}{8pt}
\section{Référence Rapide}

\label{faq}
\subsection{Questions-réponses}

Les questions les plus courantes sont répertoriées ici pour te faire gagner du temps !

\begin{description}

\item[J'ai une question sur l'informatique] je cherche sur le WikiBR et/ou demande \`a un BRman.

\item[Je veux voir mon pseudo quand j'ai voté à la QDJ] je définis mon pseudo sur ma fiche \linebreak trombino sur la page \menu{Administration / Compte / Profil} sur \fkz.

\item[J'ai un deuxième ordinateur, qu'est-ce que je fais ?] je demande une deuxième adresse IP en envoyant un mail à \mail{root@eleves.polytechnique.fr} en préfixant le sujet par [IP] 

\item[Je n'ai plus de réseau] je vais voir la 4\ieme de couverture.

\item[Je viens de changer de casert/section, il faudrait mettre à jour ma fiche TOL] j'envoie \linebreak un mail à \mail{tol@frankiz} avec les modifications à effectuer ainsi que la raison de ces
modifications.

\item[Mon client mail dit que \guillemotleft~l'autorité de certification est inconnue~\guillemotright ] je vais télécharger le certificat de sécurité sur \urllink{https://poly/} et je l'installe.

\item[Je ne reçois pas mes mails] je vérifie ma redirection sur \urllink{http://poly}.

\item[Si la boite mail poly est saturée] Je me connecte sur :\\
\urllink{https://webmail.enseignement.polytechnique.fr/imp/login.php}, 
login : nom de famille --
password : mdp enex ; et je supprime mes mails.


\item[Je n'arrive pas à me connecter à \server{poly}] j'essaye \server{poly.polytechnique.fr}.

\item[Mon ordinateur n'a pas de nom sur le réseau] je lui en donne un gr\^ace \`a \urllink{http://dnsapp/} .

\item[Je cherche des informations sur l'Ecole] je regarde sur \urllink{http://intranet}.

\item[Je cherche à joindre une personne de l'administration] l'annuaire de l'École est sur \linebreak{} \urllink{http://annuaire/}.

\item[Je cherche le numéro de portable d'un X] \urllink{http://www.polytechnique.org}.

\item[J'aimerais être un geek moi aussi !] j'apprends par c\oe ur \urllink{9gag.com/}.

\end{description}





\subsection{Membres \'eminents du binet R\'eseau}

%% il faudrait faire une photo chombi\`ere comme les 2k7... (marteo)

%Chaque BR-man signale quels syst\`emes d'exploitation il conna\^it.

%% Vide pour cette ann\'ee

%\vspace{\stretch{1}}

%\emph{\`A force de rester enferm\'e dans leurs caserts sombres, les BR-men sont devenus trop sensibles �  la lumi\`ere pour supporter un flash d'appareil photo. Ainsi, seule une photo de dos a pu \^etre prise.}

%\imagepos{images/br2k7.jpg}{0.9}{L'\'equipe du BR 2k7}{h}

%\vspace{\stretch{1}}

% \subsubsection*{Description rapide des postes}

Pour une liste plus compl\`ete des postes de chacun, et pour trouver un BR-man \`a contacter pour un probl\`eme bien particulier, va consulter la liste des membres sur le TOL !

\begin{description}

  \item[Le Prez]{Poste fictif, qui permet toutefois d'avoir des relations bien plac\'ees. Il est charg\'e des relations avec la DSI. Actuellement Victor ''Levans'' Berger.}
  
  \item[Le Vice-Prez]{Chef des \emph{roots} et Grand Manitou du R\'eseau de l'X. Le plus beau, le plus fort, le plus geek. Actuellement Xavier ''Havelock" Bonnetain.}

  \item[Trez]{Garde-fou du Prez, il est l\`a pour contr\^oler que l'algorithme d'utilisation des sous est bien optimis\'e, sans aucune fuite de m\'emoire budg\'etaire. Actuellement Th\'eo ''thunder'' Combe.}

  \item[Secr\'etaire]{Poste fictif, il suffit de savoir taper au clavier, et encore. Actuellement Michel ''gerard'' Blancard.}

  \item[root]{Les \emph{roots} sont les administrateurs du r\'eseau. Ce sont eux qui s'\'evertuent \`a maintenir en \'etat de marche les serveurs, \`a rajouter de 
  nouveaux services et \`a rep\'erer les boulets qui font de la merde sur le r\'eseau. S'il s'agit de g\'erer un compte de binet, utilise plut\^ot \mail{binets@eleves}.}

  \item[admin@windows] {Administrateurs du domaine Windows. En cas de probl\`eme avec Windows, ce sont les mieux plac\'es pour t'aider ;
  c'est bien s\^ur  plus facile si tu es sur le domaine ! Ils g\`erent le contrat
  MSDNAA avec Microsoft, et ils s'occupent de distribuer les licences aux \'el\`eves qui les demandent. Actuellement Pierre-Alexandre "PAM" Murena.}
  
  \item[support@windows] {SOS d\'epannage Windows, j'\'ecoute ! Pr\^ets \`a tout pour sauver une jeune demoiselle (ou un jeune \emph{gens} \`a la rigueur) en d\'etresse avec son Windows\dots }

  \item[support@mac] {C'est un poste naturellement tranquille, mais les probl\`emes sur Mac sont plus fr\'equents qu'on ne le croit... }
  
  \item[support@linux] {Tu veux l'installer, l'essayer ? Tu as malencontreusement tap\'e \texttt{rm -rf /} dans ta console ? Ils sont l\`a pour t'aider.}
  
  \item[support@latex] {Pour le plaisir de voir ton commandant de compagnie s'extasier sur la mise en page de ton rapport de stage, \LaTeX~est fait pour toi !}

  \item[dev]{Joyeux programmeurs qui sont l\`a pour am\'eliorer \fkz, avec de nouveaux \emph{skins} et de nouveaux mini-modules.}

%  \item[news] {(\mail{news@eleves}) Mainteneurs du serveur de \emph{news}, ils surveillent aussi ce que tu postes et que tu respectes les r\`egles de base comme les \emph{crossposts} (marteau-th\'erapie) \mbox{;-)}}

  \item[web] {Webmestres de \fkz, ils valident les annonces et les activit\'es et surveillent le contenu du site de ton binet ou de ton site perso.}

%  \item[X-share] {(\mail{xshare@eleves}) Personne sympathique qui cherche \`a longueur de temps de nouveaux logiciels gratuits ou mieux, libres \`a proposer aux \'el\`eves dans \xshare.}

  \item[InfoBR]{Agent de transfert du savoir des roots aux habitants du pl\^atal.}



  \item[IRC-Op]{Responsable des relations avec RezoSup. Viendez sur IRC (\urllink{http://irc/}) ! \linebreak Actuellement Jean-Matthieu ''JM'' Gallard.}

%  \item[tol] {(\mail{tol@eleves}) V\'erificateur de photos, il surveille le Trombi-On-Line.}



\end{description}


% ça prend de la place et ce n'est pas lu, ça dégage (Arcos, BR 2k11)

\subsection{Lexique}


\begin{description}
  \item[Adresse IP] Adresse de ton ordinateur sur le réseau, composée de quatre nombres compris entre 0 et 255  (\server{129.104.xxx.xxx}). Elle identifie ta machine auprès des autres utilisateurs du réseau.

  \item[Annonce] Version non-violente du \emph{mail} promo.

  \item[BR] Binet Réseau~: le binet qui s'occupe d'administrer le réseau des élèves, de développer et de maintenir le site \urllink{http://www.frankiz.net/}.

  \item[Client] (voir serveur)~: programme qui permet de se connecter à un serveur. Par exemple un client \emph{mail}
	(comme \app{Thunderbird}) ou un client FTP (comme \app{FileZilla}).
%   \item[Crosspost] Paradis perdu. Voir page 11.

  \item[Chocapix] Site des bars d'étages, qui permet d'aider au bon fonctionnement de ceux-ci. Fonctionne bien mieux avec un lecteur de codes-barres. Accessible à \urllink{http://chocapix/}.

  \item[DNS] \emph{Domain Name Server}~: associe un nom de machine à une adresse IP, par exemple \server{frankiz} à  $129.104.201.51$.

  \item[DSI] (Direction des Systèmes d'Information) ce sont eux qui gèrent tout le matériel informatique de l'École, ton téléphone, ton accès Internet, tes \emph{mails}\ldots\ un conseil~: ne joue pas au plus malin avec eux.

  \item[Firewall] Logiciel de protection de ton ordinateur contre les infiltrations de vers ou de pirates informatiques.

  \item[FTP] \emph{File Transfer Protocol}~: Protocole réseau qui permet d'échanger des fichiers en toute “simplicité”, délaissé au profit de Phœnix, le cloud promo.

  \item[Frankiz] Le portail \emph{web} des élèves, édité par le BR.

  \item[Fruit] Personne ayant un haut potentiel de fabrication semi-volontaire de merdasse.% Permet aussi de voir des films et des séries. % NtBR: Ceci n'a rien à faire ici, on subit assez les remarques comme ça~:)

  \item[Git] Gestionnaire de versions arborescent. Permet de bosser à plusieurs sur le même code source de façon efficace. Fait aussi le café.

 % \item[InfoGeek et Inf'Autiste] Respectivement INF422~: \og Composants d'un système informatique \fg, et INF423~: \og Fondements de l'informatique~: logique, modèles, calculs \fg.
%Deux cours courts que le BR t'encourage vivement à choisir pour ta 2A.

%   \item[Iooss] Personnage mythique du BR. Certains racontent que son âme hante toujours le réseau de l'École...

   \item[Java] Langage de programmation utilisé dans le cadre des cours d'informatique. Ses qualités sont sujettes à débat.

   \item[Mail Promo] Voir Spam Promo

  \item[Phoenix] Cloud promo — voir page \pageref{phoenix} — permet de partager des fichiers, et de stocker quelques fichiers persos accessibles partout.

  \item[SCP] Équivalent de SSH pour les téléchargements~: permet de récupérer des fichiers depuis une machine distante.

  \item[Serveur] (Voir client) Programme qui permet d'accueillir des services. Comme par exemple le partage de fichiers, le voisinage réseau, un site \emph{web}\ldots\

  \item[Serveur] (Ce n'est pas le même qu'avant~;-)) Machine qui accueille des serveurs (là c'est celui d'au-dessus).

  \item[Serveur mandataire (\emph{proxy})] Machine qui autorise
(et du même coup restreint) les communications avec l'extérieur. Le \emph{proxy} protège tous les ordinateurs du réseau contre les attaques.

  \item[Spam Promo] Moyen de communication violent. À l'X, il est reservé aux informations importantes pour qu'elles ne soient pas noyées par du bruit. Voir page \pageref{mails}. Les sanctions à un abus peuvent être brutales (voir \emph{DSI}), vous aurez été prévenus.

  \item[SSH] Connexion permettant de travailler sur une machine distante. Télécharge \app{PuTTY}~!

  \item[Titus] le chat du plâtal. Voir page \pageref{titus}

  \item[Troll] Débat polémique sans fin permettant de déployer la mauvaise foi des deux parties.

  %\item[VPN] Permet de jouer à des jeux en ligne. L'usage en étant interdit, le BR ne peut vous le recommander.%Mêmes raisons que Fruit.

  %\item[WikiBR] \urllink{https://br.binets.fr}, le site du BR~; c'est là que tu peux trouver une description des services proposés par le BR, de ses projets, ainsi que
  %d'informations plus approfondies concernant ta configuration.

  \item[Wiki*] Mot hawa\"ien pour \og rapide \fg, a donné son nom à Wikipédia, célèbre encyclopédie éditable par tous.
  Par extension, désigne tout site construit sur ce modèle, par exemple le WikiX.% et le WikiBR (enfin, le BR ne te permet pas de tout éditer sur ses pages).

  \item[Xelnor] Divinité du BR.

  \item[thunder, IooNag, Aniem] Divinités mineures du BR.

  \item[Windows] Moyen le plus efficace de télécharger Linux.

  \item[Internet Explorer] Seul moyen de télécharger Firefox.

\end{description}



\subsection{Comment contacter le BR}

Lorsque tu veux contacter le BR, il ne faut surtout pas contacter un BRman particulier, utilise une des adresses ci-dessous (en rajoutant \`a chaque fois \server{.polytechnique.fr}).

\begin{description}

\item[\mail{web@eleves}] Pour contacter les personnes qui valident les annonces et activit\'es sur \fkz, cr\'eent des groupes et des comptes.

\item[\mail{dev@eleves}] Si tu as trouv\'e un \emph{bug} sur \fkz ou si tu as une id\'ee pour l'am\'eliorer (minimodule, th\`eme\dots).

\item[\mail{root@eleves}] Pour tout probl\`eme concernant le r\'eseau ou les machines du BR.

\item[\mail{msdnaa@eleves}] Lorsque tu as un probl\`eme en rapport aux demandes de licence Windows (par contre les demandes se font sur \fkz).

\item[\mail{infobr@eleves}] De mani\`ere \'evidente lorsque tu veux faire des remarques sur l'InfoBR.

\item[\mail{bureau@eleves}] Surtout pour discuter de probl\'ematiques de binets et d'association (ou si tu te demandes pourquoi on fait la distinction). 

\item[\mail{irc@eleves}] Tu t'es fait bannir de \#x ?

\item[\mail{support@eleves}] Si tu as un probl\`eme avec ton ordinateur.

\item[\mail{matos@eleves}] Pour emprunter du mat\'eriel informatique au BR.

\item[\mail{br@eleves}] Lorsque tu veux contacter le BR tout entier.

\end{description}




% -------------------- Pour rajouter une page -------------------

%\newpage

%Enlevage des watermarks
%\leftwatermark{}
\rightwatermark{}

%\thispagestyle{plain}
%\hrulefill \, \begin{Large}Notes personnelles\end{Large} \hrulefill

% -------------------- Pour rajouter une deuxième page ----------
%\newpage 
%\thispagestyle{plain} 
%\hrulefill \, \begin{Large}Fais dédicacer ton InfoBR !\end{Large} \hrulefill 


\end{document}

