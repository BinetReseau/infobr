\clearpage
\pagebreak

\bghdr{images/fond-mac}

%\begin{center}
%\includegraphics{images/logo_Mac}
%\end{center}

\subsection{Configuration sous Mac OS X}

\subsubsection{Configuration IP}

\flimage{images/mac_prefs_icone}{0.1}{l}
\app{Pr\'ef\'erences R\'eseau}, accessible depuis l'article de menu \menu{Pr\'ef\'erences Syst\`eme} du \menu{menu Pomme}, permet de configurer la connexion au r\'eseau. Par ailleurs, si au d\'emarrage un assistant te propose de configurer ton r\'eseau, refuse gentiment et utilise la proc\'edure que le BR te propose. En effet, le r\'eseau n\'ecessite une configuration particuli\`ere � l'X, plus complexe que celle autoris\'ee par cet assistant.

La gestion des configurations r\'eseau de Mac OS X permet de cr\'eer plusieurs configurations et de passer en un clic de l'une \`a l'autre gr\^ace au sous-menu \menu{Configuration R\'eseau} du menu \menu{Pomme}. Cela est tr�s pratique pour les machines vou\'ees \`a �tre connect\'ees \`a plusieurs endroits successivement --- les portables par exemple (voir la page~\pageref{wifi} pour plus de pr\'ecisions sur le Wifi). Commence donc par cr\'eer une nouvelle configuration r\'eseau dans le menu d\'eroulant \menu{Configuration}.

\imagepos{images/mac_nouvelle_config}{0.5}{Cr\'eer une nouvelle configuration r\'eseau}{!hb}

Une fois la nouvelle configuration cr\'e\'ee, il faut configurer l'interface r\'eseau Ethernet. Dans le menu d\'eroulant \menu{Afficher}, s\'electionne \menu{Ethernet int\'egr\'e}.

\imagepos{images/mac_config_ethernet}{0.5}{Configurer l'interface r\'eseau Ethernet}{!h}

Choisis alors \menu{Configurer IPv4} : \menu{Manuellement}.
Tu trouveras toutes les valeurs d'IP n\'ecessaires pour la configuration en page \pageref{calcul_ip}
ou en te reportant � la capture d'\'ecran~\ref{config:mac:ip}.
Si une partie d'IP est blanche sur cette capture, c'est qu'elle t'est personnelle et que tu dois la calculer !

\imageref{images/mac_config_ip}{0.66}{Configuration IP}{!ht}{config:mac:ip}

Pour avoir acc\`es \`a Internet, il faut aussi configurer le proxy. Clique sur l'onglet \menu{Proxys}.
Regarde dans le \menu{menu Pomme} : \menu{\`A Propos de ce Mac}, tu y trouveras la version pr\'ecise de Mac OS que tu as. Pour Mac OS X 10.3.2 et inf�rieur, il te faut sp\'ecifier tous les proxies manuellement, et mettre \server{kuzh.polytechnique.fr}, port \server{8080}. Pour Mac OS X 10.3.3 et sup\'erieur, choisis l'option \menu{Configuration automatique de proxy}. Pour Mac OS X 10.3 � 10.3.2, n'oublie pas une fois que tu as le r\'eseau de faire la mise � jour de ton syst\`eme, pour pouvoir configurer de fa\c{c}on automatique le proxy.

N'oublie pas d'activer le mode passif pour les transferts en FTP, en cochant la case comme dans la capture.

\imagepos{images/mac_config_proxy}{0.7}{Configurer le proxy}{!ht}


\subsubsection{Configuration antivirus}

Bien qu'il soit important de maintenir ton syst\`eme � jour, un anti-virus est pour l'instant tout � fait superflu sur Mac, puisqu'aucun virus fonctionnel n'a encore vu le jour. Attention cependant, n'ouvre pas des fichiers dont tu ne te sois pas assur\'e de la provenance, et essaie de te tenir au courant des actualit\'es concernant les failles des applications que tu utilises.

\subsubsection{Configuration web}

\flimage{images/mac_safari_icone}{0.1}{l}
\app{Safari}, le navigateur web d'Apple, est maintenant compatible avec la majorit\'e des sites web. Tu peux donc t'en servir au quotidien, en faisant appel \`a \app{Firefox} ou \app{Camino} pour les sites r\'ecalcitrants.
Un conseil : pense \`a activer le blocage des fen\^etres pop-up (dans le menu \menu{Safari}). \app{Safari} peut aussi servir de client RSS (voir plus bas).\\

\subsubsection{Configuration mail}

\flimage{images/mac_mail_icone}{0.1}{l}
\app{Mail} : un client mail, offrant les fonctionnalit\'es classiques d'un bon client : recherches instantan\'ees, filtre antispam, r\`egles de tri automatique des mails, regroupement des mails correspondant \`a une m�me discussion.

Au premier lancement, \app{Mail} te demandera de remplir les informations concernant ton compte mail sur \server{poly}, il suffit de le remplir avec les donn\'ees suivantes :

\begin{description}
  \item[Nom complet] ton nom !
  \item[Adresse \'electronique] de la forme \mail{prenom.nom@polytechnique.edu}
  \item[Serveur de r\'eception] \server{poly.polytechnique.fr}
  \item[Type de compte] \menu{POP}
  \item[Nom d'utilisateur] ton login \server{poly} (les huit premi\`eres lettres de ton nom en g\'en\'eral)
  \item[Mot de passe] ton mot de passe \server{poly}
  \item[Serveur d'envoi (SMTP)] \server{poly.polytechnique.fr}
\end{description}

Si tu as d\'ej\`a cr\'e\'e un compte pr\'ec\'edemment, il faut aller dans les \menu{Pr\'ef\'erences} (accessibles depuis le menu \menu{Mail}), onglet \menu{Comptes}, pour cr\'eer un autre compte en cliquant sur la case \menu{+}.

N'oublie pas de cocher \menu{Activer le cryptage SSL} dans l'onglet \menu{Avanc\'e}, port 995. Tu souhaiteras alors certainement installer le certificat de s\'ecurit\'e de \server{poly} (tu le trouveras sur \url{http://poly/}). Une fois que tu as t�l�charg� le certificat, ouvre le fichier \menu{.crt} obtenu, et dans \app{Trousseau d'acc\`es}, installe le dans \menu{X509Anchors}.

Cette configuration marche pour acc\'eder \`a ses mails depuis l'int\'erieur de l'X mais aussi de l'ext\'erieur, sans rien changer. Par contre depuis l'ext\'erieur tu ne peux pas envoyer de mails, car le serveur \server{poly} ne le permet pas. Tu peux regarder la configuration propos\'ee par \url{Polytechnique.org} pour surmonter cette difficult\'e.

Enfin, tu peux disposer dans Mail de l'annuaire de l'�cole, mis � disposition par la DSI. Pour cela, va dans les \menu{Pr�f�rences} de Mail, puis dans la rubrique \menu{R�daction} et clique sur \menu{Configurer LDAP\ldots}. Tu peux ensuite utiliser le bouton \menu{+} pour ajouter un serveur, et remplir la fen�tre comme sur la capture.

\imagepos{images/mac_config_ldap}{0.6}{Configurer l'annuaire}{!ht}


\subsubsection{Logiciels Additionnels}

Les logiciels suivants sont utiles pour utiliser avec Mac OS X les services propos\'es sur le r\'eseau, ils sont t\'el\'echargeables sur \server{frankiz}, dans la rubrique \menu{T\'el\'echarger -> Mac}.


\subsubsection{Configuration news}

\flimage{images/mac_thunderbird_icone}{0.1}{l}
\app{Thunderbird} : un client news permettant d'acc\'eder aux forums de discussion des \'el\`eves
                   (voir page~\ref{newsgroups} pour les d\'etails sur \server{frankiz}
                   (mais aussi \`a ceux de \server{usenet} gr�ce au serveur \server{polynews.polytechnique.fr}).
Il est tr\`es proche d'\app{Outlook Express} dans son esprit. Dans la m�me cat\'egorie, il existe \app{MacSOUP}, \app{Unison} ou encore \app{MT-NewsWatcher}. La configuration se fait de la m�me mani\`ere.

Au premier lancement, l'application te propose d'importer les param\`etres depuis une autre application. Clique sur \menu{Suivant >}. Tu peux alors choisir quel type de compte tu veux configurer (tu remarqueras que tu peux aussi cr\'eer un compte courrier \'electronique, et un compte RSS). S\'electionne \menu{Compte forums de discussion} et clique sur \menu{Suivant >}. Entre alors les informations suivantes :

\begin{description}
  \item[Votre nom] ton nom ou ton pseudo
  \item[Adresse de courrier] \mail{prenom.nom@polytechnique.edu}
  \item[Serveur de forums] \server{frankiz}
  \item[Nom du compte] News Frankiz
  \item[Nom d'utilisateur] ton login poly (les huit premi\`eres lettres de ton nom en g\'en\'eral)
  \item[Serveur d'envoi (SMTP)] \server{poly.polytechnique.fr}
\end{description}

Pour t'abonner \`a des groupes de discussion, il te suffit de s\'electionner le compte \menu{News Frankiz} dans la fen�tre \menu{Dossiers} de \app{Thunderbird}, puis de cliquer sur \menu{G\'erer les abonnements aux groupes de discussion}. Tu pourras ensuite s\'electionner les forums qui t'int\'eressent parmi la liste propos\'ee. Reporte-toi \`a la page \pageref{newsgroups} pour plus d'infos sur les newsgroups auxquels t'abonner !

\subsubsection{Client FTP}

\flimage{images/mac_cyberduck_icone}{0.1}{l}
\app{Cyberduck} : un client FTP tr\`es simple \`a utiliser mais performant. Il te permettra d'aller t\'el\'echarger des fichiers sur les serveurs FTP des autres \'el\`eves. Il existe aussi \app{Fugu}, que certains pr\'ef\`erent.

Pour se connecter \`a un serveur, il suffit de taper son nom (exemple : \url{ftp://jtx}) dans le cadre \menu{Connexion rapide} et appuyer sur Entr\'ee.\\

\subsubsection{Autres logiciels utiles}

Les logiciels cit\'es ici sont t\'el\'echargeables sur \server{frankiz} : rubrique \menu{T\'el\'echarger -> Mac -> R\'eseau}.

\flimage{images/mac_qrezix_icone}{0.1}{l}
\noindent\app{qRezix} : En deux mots, c'est un programme d\'evelopp\'e par le BR pour faciliter la vie sur le r\'eseau. Tu peux le r\'ecup\'erer dans la partie Mac de \xshare.

\noindent Pour plus de d\'etails, voir le paragraphe consacr\'e \`a qRezix \`a la page \pageref{qrezix}.

\noindent Attention, si ton firewall est activ\'e, tu dois ouvrir les ports 5050, 5053 et 5055 en TCP. Pour cela va dans \app{Pr\'ef\'erences Syst\`eme}, dans le module \menu{S\'ecurit\'e}, onglet \menu{Coupe-feu}. S'il est \'ecrit \menu{Coupe-feu activ\'e}, clique le bouton \menu{Nouveau} et remplis la bo�te de dialogue comme sur la capture d'�cran ci-dessous pour ouvrir les ports.

\imagepos{images/mac_firewall}{0.5}{Ouvrir les ports pour \app{qRezix}}{!ht}

%\flimage{images/mac_cocoaxnet_icone}{0.1}{l}
%\noindent\app{CocoaXNet} est un \'equivalent plus l�ger de qRezix, avec une meilleure int\'egration \`a Mac OS X. Pour l'instant, il est sans d\'eveloppeur, et donc ne contient pas les fonctionalit\'es de la version 2 de \app{qRezix} telles les plans de l'\'ecole donc c'est \`a toi de d\'ecider lequel tu pr\'ef\`eres !

%\noindent Si le d\'eveloppement en Objective-C et Cocoa (les langages utilis�s pour CocoaXNet) te tente, n'h\'esites pas \`a contacter les d\'eveloppeurs via \mail{qrezix@frankiz}. \\

\flimage{images/mac_conversation_icone}{0.1}{l}
\noindent\app{Colloquy}, un client IRC dans le m�me esprit qu'\app{iChat}. Il dispose d'une interface tr\`es simple ne n\'ecessitant pas de conna�tre les commandes IRC. Tu peux te reporter \`a la page \pageref{irc} pour plus d'infos sur l'IRC. \app{X-Chat Aqua} est un autre client IRC, plus riche en fonctionnalit�s. \\

\flimage{images/mac_rss_icone}{0.1}{l}
\noindent\app{Vienna} est un client RSS gratuit et agr\'eable, dont le d\'eveloppement actif est prometteur. Les flux RSS permettent d'agr\'eger dans un seul logiciel des informations en provenance de nombreux sites web, qui peuvent provenir de forums de discussions, de mises \`a jour de logiciels, d'informations internationales...\\ \\

\flimage{images/mac_fink_icone}{0.1}{l}
\noindent\app{Fink} est la mani�re la plus simple d'installer sur Mac OS X nombre de logiciels issus du monde UNIX (Linux par exemple). Gr�ce � lui, tu pourras installer les m�mes logiciels qu'en salles informatiques. Par exemple, tu pourras installer Scilab sans trop de peine...\\ \\
