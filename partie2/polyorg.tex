\subsection{Le site des anciens : Polytechnique.org}
\urllink{Polytechnique.org} est une association loi 1901 composée d'élèves et d'Anciens
 indépendante de l'administration de l'\'Ecole (donc des domaines \urllink{polytechnique.fr}
 et \urllink{polytechnique.edu}). Le but de l'association est la mise à disposition des X d'outils
ayant un rapport avec Internet, entre autres :
\begin{itemize}
  \item des redirections de \emph{mails} nombreuses (adresses supplémentaires) et à vie ;
  \item un serveur de \emph{news}, ouvert notamment aux Anciens;
  \item une facilitation des contacts vers les Anciens et les camarades de promotion ;
  \item une lettre mensuelle, pour s'informer sur l'actualité de la communauté polytechnicienne ;
  \item des annonces d'événements ;
  \item des services d'hébergement pour les groupes et binets, notamment des noms de domaine (\emph{via} \server{www.polytechnique.net}) et des listes de diffusion (comme par exemple \mail{br@2009.polytechnique.org}).
\end{itemize}
Si tu veux découvrir les autres services de l'association ou savoir
comment les utiliser, tu peux aller sur la page
\urllink{https://www.polytechnique.org/Xorg/Xorg} (accessible depuis
le lien \menu{Documentations} dans le menu de \urllink{Polytechnique.org}
quand tu es connecté).

Par ailleurs, les filtres antivirus et antispam appliqués aux \emph{mails} sont très efficaces (99\,\% de repérage correct); \urllink{Polytechnique.org} te conseille donc de mettre en place la redirection suivante :
\mail{prenom.nom@polytechnique.edu}
redirigée vers \mail{prenom.nom(.promo)@polytechnique.org},
elle-même redirigée sur \mail{login@poly(.polytechnique.fr)}.
Pour réaliser ces redirections, connecte-toi sur les pages suivantes :
\begin{itemize}
  \item pour \mail{@polytechnique.edu} : \urllink{https://www.mail.polytechnique.edu} ;
  \item pour \mail{@polytechnique.org} : \urllink{https://www.polytechnique.org} ;
  \item pour \mail{@poly} : \urllink{http://poly.polytechnique.fr}.
\end{itemize}
Pour plus de détails, rends-toi sur \urllink{https://www.polytechnique.org/Xorg/RedirectionMails}.


 Ces outils sont très utiles, que
ce soit pour toi, pour tes binets ou pour garder plus tard contact avec la communauté polytechnicienne. Rejoins
les 18\,000 camarades déjà inscrits ! Et en cas de problème, n'hésite pas à contacter
\mail{contact@polytechnique.org}.
