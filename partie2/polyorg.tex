\subsection{Polytechnique.org}
\url{Polytechnique.org} est une association d'anciens �l�ves affili�e � l'AX.
Elle est destin�e � promouvoir l'image des Polytechniciens sur Internet.

L'association n'a pas de lien particulier avec les sites officiels de l'\'Ecole
sur \server{polytechnique.fr} et \server{polytechnique.edu}.

Les domaines \server{polytechnique.org} et \server{polytechnique.net} servent
exclusivement \`a parler des X, \'el\`eves et anciens \'el\`eves, sur Internet par Internet.

L'autre but est d'offrir le maximum de services de communication par Internet aux inscrits
volontaires \`a notre site. Il s'agit l\`a de favoriser la vie des promotions, des associations
polytechniciennes (groupes X, binets, ...) et de la communaut\'e en g\'en\'eral.

L'association propose de nombreux services aux X, qu'ils soient ou non sur le plateau.
En g\'en\'eral, ils sont peu connus, et pourtant souvent tr\`es utiles.

Inscrivez-vous, et n'h\'esitez
pas \`a vous y connecter pour exploiter ces services, dont voici les principaux :
\begin{itemize}
  \item des redirections mails nombreuses (adresses suppl\'ementaires)
  \item des services de news comme le binet r\'eseau, mais ouverts aux anciens, et aux non plat\^aliens
  \item des contacts ais\'es vers les anciens, les camarades de promotion
  \item une newsletter, pour publier des informations de groupes X, des informations qui toucheront tous les polytechniciens
  \item des annonces d'\'ev\'enements
  \item des services d'h\'ebergement pour les groupes et binets, des noms de domaine
  \item des listes de diffusion de mails (br2004@polytechnique.org, par exemple)
\end{itemize}
