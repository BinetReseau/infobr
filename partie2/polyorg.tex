\subsection{Polytechnique.org}
\urllink{Polytechnique.org} est une association loi 1901 composée d'élèves et d'Anciens élèves
 indépendante de l'administration de l'\'Ecole (et donc des domaines \urllink{polytechnique.fr}
 et \urllink{polytechnique.edu}). Le but de l'association est la mise à disposition des X d'outils
ayant un rapport avec l'Internet, entre autres :
\begin{itemize}
  \item des redirections mails nombreuses (adresses supplémentaires) et à vie ;
  \item un serveur de news (comme les br.*), ouvert aux Anciens et aux non-platâliens ;
  \item une facilitation des contacts vers les Anciens et les camarades de promotion ;
  \item une lettre mensuelle, pour s'informer sur l'actualité de la communauté polytechnicienne ;
  \item des annonces d'événements ;
  \item des services d'hébergement pour les groupes et binets, notamment des noms de domaine (via \server{www.po\-ly\-tech\-ni\-que.net}) et des listes de diffusion (\mail{br2007@po\-ly\-tech\-ni\-que.org}, par exemple).
\end{itemize}
Si tu veux découvrir les autres services de l'association ou savoir
comment les utiliser, tu peux aller sur la page
\urllink{https://www.polytechnique.org/Xorg/Xorg} (accessible depuis
le lien \menu{Documentations} dans le menu de \urllink{Po\-ly\-tech\-ni\-que.org}
quand tu est connecté).

Par ailleurs, les filtres antivirus et antispam appliqués à sur les mails sont très efficaces (99\% de repérage correct), et polytechnique.org te conseille donc de mettre en place la redirection suivante :
\mail{prenom.nom@poly\-technique.edu}
redirigée sur \mail{prenom.nom(.promo)@po\-ly\-tech\-ni\-que.org},
elle-même redirigée vers \mail{login@poly(.po\-ly\-tech\-ni\-que.fr)}.
Pour effectuer ces redirections, connecte-toi sur les pages suivantes :
\begin{itemize}
  \item pour \mail{@polytechnique.edu} : \urllink{https://www.mail.polytechnique.edu} ;
  \item pour \mail{@polytechnique.org} : \urllink{https://www.polytechnique.org} ;
  \item pour \mail{@poly} : \urllink{http://poly.polytechnique.fr}.
\end{itemize}
Cela est expliqué plus en détails sur la page
\urllink{https://www.polytechnique.org/Xorg/Re\-di\-rec\-tion\-Mails}. \\
%
\nopagebreak
%
\indent Ces outils sont très utiles, et faciles à s'approprier, que
ce soit pour toi, pour tes binets, ou pour (dans le
futur) garder contact avec la communauté polytechnicienne. Rejoins
les \nombre{15000} camarades déjà inscrits ! Et en cas de problème, n'hésite pas à contacter
\mail{contact@polytechnique.org}.
