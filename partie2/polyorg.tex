\subsection{Polytechnique.org}
Pour pr\'esenter l'association polytechnique.org, rien de mieux que de citer leur site web :
  Nous avons cr\'e\'e une association afin de promouvoir l'image des Polytechniciens sur Internet.
  Il est important de noter que nous n'avons aucun mandat sp\'ecial pour le faire, ni non plus
  de contre-indication d'ailleurs. Cependant, les vues officielles de l'\'ecole peuvent \^etre
  trouv\'ees sur www.polytechnique.fr et www.polytechnique.edu. Le domaine polytechnique.org sert
  exclusivement \`a parler des X, \'el\`eves et anciens \'el\`eves, sur Internet par Internet.

  L'autre but est d'offrir le maximum de services de communication par Internet aux inscrits
  volontaires \`a notre site. Il s'agit l\`a de favoriser la vie des promotions, des associations
  polytechniciennes (groupes X, binets, ...) et de la communaut\'e en g\'en\'eral.

L'association propose de nombreux services aux X, qu'ils soient ou non sur le plateau.
En g\'en\'eral, ils sont peu connus, et pourtant souvent tr\`es utiles. Inscrivez-vous, et n'h\'esitez
pas \`a vous y connecter pour exploiter ces services, dont voici les principaux :
  des redirections mails nombreuses (adresses suppl\'ementaires)
  des services de news comme le binet r\'eseau, mais ouverts aux anciens, et aux non plat\^aliens
  des contacts ais\'es vers les anciens, les camarades de promotion
  une newsletter, pour publier des informations de groupes X, des informations qui toucheront tous les polytechniciens
  des annonces d'\'ev\'enements
  des services d'h\'ebergement pour les groupes et binets, des noms de domaine
  des listes de diffusion de mails (br2004@polytechnique.org, par exemple)
