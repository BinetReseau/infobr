\subsection{Frankiz}
\label{frankiz} La page web frankiz est la page des élèves. Elle est visible de l'intérieur et de
l'extérieur de l'X, en intégralité si tu t'es identifié ou partiellement pour les autres
utilisateurs. Tu peux automatiser ta connexion grâce à un \emph{cookie} d'authentification. Nous te
conseillons de faire de \urllink{http://frankiz/} la page d'accueil de ton navigateur Internet.

Elle permet en particulier l'accès aux services suivants : les
annonces et les activités du platâl, l'annuaire des élèves
(\menu{TOL} pour Trombino On Line), le téléchargement de logiciels
gratuits (\menu{X-Share}), la base de connaissances (\menu{WikiX}) et
la question du jour (\menu{QDJ}). Cette page est
aisément personnalisable (lien \lien{Préférences}). À toi d'explorer
tout ce que tu peux y trouver !

Les \menu{annonces}, \menu{sondages} et \menu{activités} permettent d'informer les élèves de ce qui se passe à l'école. Les annonces sont triées dans
un sommaire et tu peux éventuellement faire disparaître celles qui ne t'intéressent pas (en fonction des \emph{skins}). Les activités apparaissent
sur la page principale le jour où elles ont lieu. Les sondages apparaissent sur la page principale et tu peux voir les
résultats lorsque le vote est terminé. Tu peux proposer des annonces, des sondages et des activités en utilisant les liens \lien{Proposer\ldots}; la syntaxe à utiliser est
rappelée en bas de page. Avant de valider, utilise la fonction \lien{tester} pour vérifier que tu n'as pas fait d'erreur, oublié l'image, \dots

Le \menu{Trombino} permet de trouver des renseignements utiles sur
tous les élèves sur le plâtal. Tu peux mettre ta fiche à jour en
utilisant le lien \lien{Préférences}.

Si tu es le président ou le webmestre d'un binet, tu as droit à un
lien \lien{Administration} à c\^oté de ton lien \lien{Préférences}.
Il te permet d'exercer les terribles pouvoirs du prez ou du web, qui
sont respectivement de gérer la liste des membres inscrits au binet
dans le TOL et de modifier l'ic\^one et la description du binet dans
la page \menu{Binets}. Pour cela, il suffit d'avoir signé la feuille
de demande de droit, qui est (normalement) disponible dans la case
courrier du BR à la Kès.

Le \menu{WikiX} contient les réponses aux questions les plus courantes, sur des sujets aussi divers que l'histoire et les traditions de l'\'Ecole, la
vie pratique sur le plâtal, les binets, l'informatique, etc. C'est souvent plus rapide d'y faire un petit tour que d'appeler quelqu'un, d'autant plus qu'il y a
un moteur de recherche. Et si tu vois une erreur, tu peux la corriger directement !

La rubrique \menu{X-Share} permet de télécharger des logiciels pour
Windows, Mac et Linux sélectionnés par le BR et des documents
importants comme cet InfoBR. En particulier, c'est là que tu
trouveras les logiciels développés par le BR, dont \app{qRezix}. Si
tu cherches un logiciel pour un usage particulier, commence par là !

%Publi-reportage : les \menu{X-Share} (entre autres) sont en cours de refonte afin d'améliorer leur ergonomie. Si tu as des talents de programmeur, n'hésite pas, le développement de frankiz n'attend que toi !!

La \menu{QDJ} est une question binaire, parfois sérieuse mais le
plus souvent basée sur un jeu de mots ou sur l'activité sur le
campus. Tu peux voter tous les jours et même proposer des questions
au QDJmestre.

La rubrique \menu{Sites élèves} contient la liste des sites personnels des élèves hébergés sur frankiz. Tu peux toi aussi publier ton site \emph{web} en
utilisant le lien dans \lien{Préférences}. De même, la rubrique \menu{Binets} contient la liste des binets, une description de celui-ci et le lien
vers le site de ce binet (éventuellement hébergé sur frankiz).

Même si \fkz est l'\oe uvre de tous, les webmestres se réservent le droit de ne pas publier une annonce ou d'interdire un site web si le contenu
n'est pas jugé adapté mais aussi, dans le cas des annonces si elle gêne la lisibilité générale, selon le bon principe : \guillemotleft~trop
d'information tue l'information~\guillemotright .

Rappelons quelques règles évidentes :
\begin{itemize}
 \item Tout contenu polémique est banni des annonces (publie tes aigreurs dans l'IK ou sur \ngname{br.binet.polemix}).
       Les annonces d'un go\^ut douteux ne sont pas non plus les bienvenues (même recommandation);
 \item La publicité n'a pas sa place sur \fkz ; toute annonce ayant un net caractère publicitaire
       sera refusée (y compris s'il s'agit de publicité pour un sponsor);
 \item Tout contenu portant atteinte à une tierce personne ou à un groupe est interdit dans les annonces
       et les sites \emph{web}, ainsi que tout lien vers un site ou document de ce type;
 \item Tout contenu illégal, en particulier tout document (quel que soit son type)
       non libre de droits ou ayant un caractère pornographique, est interdit,
       ainsi que tout lien vers un site ou document de ce type;
 \item Si un contenu d'un des deux types précédents échappe toutefois à l'attention des webmestres,
       seuls leurs auteurs pourraient en être tenus responsables.
       Tout contenu de cette sorte qui apparaîtrait sur le site doit être immédiatement signalé.
\end{itemize}

Les régles élémentaires pour préserver la lisibilité de frankiz :
\begin{itemize}
 \item Les annonces ne doivent pas être trop longues (pas plus d'une quinzaine de lignes);
 \item L'interface des annonces utilise la syntaxe wiki, qui est rappelée en bas de page et expliquée plus en détail
        sur une page accessible facilement depuis la page de proposition :
        n'hésite pas à y faire un petit tour pour savoir comment embellir tes annonces
        (gras, italique, liens hypertextes);
 \item Les titres des annonces ne doivent pas être en capitales ou précédés de signes de ponctuation.
 \item Un binet ne peut pas avoir deux annonces en même temps;
       Alors s'il te plaît, quand tu fais la com' de ton binet, réfléchis et écris une belle annonce :
       plus elle est concise et précise, mieux elle sera lue !
\end{itemize}

Enfin, tout ce qui est proposé sur \fkz doit être validé par les
webmestres, qui ne sont pas là pour censurer mais pour maintenir la
qualité du site. Inconvénient : ça peut parfois être un petit peu
long mais il ne faut pas s'inquiéter. Dans tous les cas, si tu as une
demande quelconque à faire sur le contenu du site, par exemple
modifier/supprimer une annonce, un seul réflexe : envoie un mail à
\mail{web@frankiz.polytechnique.fr}, réponse rapide (presque) assurée !
