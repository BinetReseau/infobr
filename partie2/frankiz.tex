\subsection{Frankiz}

\bghdr{images/logo-Frankiz}

\subsubsection{Présentation}

\label{frankiz} La page web \fkz est la page des élèves. Elle est visible de l'intérieur et de
l'extérieur de l'X, en intégralité si tu t'es identifié ou partiellement pour les autres
utilisateurs. Nous te conseillons de faire de \urllink{http://www.frankiz.net/} la page d'accueil de ton navigateur Internet.

\fkz permet notamment l'accès aux services suivants : les
annonces et les activités du plâtal, l'annuaire des élèves
(\menu{TOL} pour Trombino On Line), l'emploi du temps (\menu{EdT}), et la question du jour (\menu{QDJ}). Tu peux accéder à ces services sous forme de mini-modules que tu peux déplacer sur ta page d'accueil pour adopter la configuration que tu préfères.
Tout est aisément personnalisable (lien \lien{Administration}). ÀA toi d'explorer tout ce que tu peux y trouver !

Les \menu{annonces} et \menu{activités} permettent d'informer les élèves de ce qui se passe à  l'École. Les annonces sont triées dans
un sommaire et disparaissent une fois que tu les as lues. Tu peux également en mettre certaines en favoris pour ne garder que celles qui t'intéressent. Les activités sont listées sur la page dédiée et apparaissent
sur la page principale le jour où elles ont lieu. Tu peux proposer des annonces et des activités en utilisant les liens sur leurs pages respectives \lien{Annonces} et \lien{Activités}. Tu peux proposer des activités ou annonces en ton nom propre ou au nom d'un binet dont tu es membre pour assurer une meilleure visibilité à celles-ci.

Le \menu{Trombino} permet de trouver des renseignements utiles sur
tous les élèves sur le plâtal. Tu peux mettre ta fiche à  jour en
utilisant le lien \lien{Administration}.

En rejoignant un groupe ou un binet, tu peux le faire en tant que \menu{membre} (participant aux activités) ou simple \menu{sympathisant} (ce qui te permet d'être mieux informé sur les différentes activités du binet). Chaque groupe ou binet a également ses \menu{administrateurs} qui gèrent les droits des membres, valident les annonces ou les activités faites au nom du groupe et renseignent les diverses informations concernant le binet (description, logo, site web, \dots). Si tu es administrateur, tu peux accéder à tous ces droits depuis ta page \lien{Administration}.

Il est très important que tu deviennent rapidement membre ou sympathisant des binets qui peuvent t'intéresser.
En effet, cela te donnera accès à leurs annonces internes et permettra aux autres membres du binet de te 
contacter beaucoup plus facilement pour organiser des activités.

%Le \menu{WikiX} contient les réponses aux questions les plus courantes, sur des sujets aussi divers que l'histoire et les traditions de l'École, la
%vie pratique sur le plâtal, les binets, l'informatique, etc. C'est souvent plus rapide d'y faire un petit tour que d'appeler quelqu'un, d'autant plus qu'il y a
%un moteur de recherche. Et si tu vois une erreur, tu peux la corriger directement !

Tu pourras aussi retrouver de nombreux liens sur le bandeau gris en haut de l'écran, que ce soit vers les différents sites de l'École (la direction des études \menu{DE}, ton emploi du temps avec tes salles de cours \menu{ENEX}, la fondation de l'X \menu{FX}, \dots), le \menu{WikiX}, les archives du JTX, et plein d'autres, \dots

%Publi-reportage : les \menu{X-Share} (entre autres) sont en cours de refonte afin d'améliorer leur ergonomie. Si tu as des talents de programmeur, n'hésite pas, le développement de frankiz n'attend que toi !!

La \menu{QDJ} est une question binaire, parfois sérieuse mais le
plus souvent basée sur un jeu de mots ou sur l'activité sur le
campus. Tu peux voter tous les jours et même proposer des questions
au QDJmestre.

\subsubsection{Règles de modération}

Même si \fkz est l'œ uvre de tous, les webmestres se réservent le droit de ne pas publier une annonce ou d'interdire un site web si le contenu
n'est pas jugé adapté mais aussi, dans le cas des annonces si elle gêne la lisibilité générale, selon le bon principe : \guillemotleft~trop
d'information tue l'information~\guillemotright .

Rappelons quelques règles évidentes :
\begin{itemize}
 \item tout contenu polémique est banni des annonces (publie tes aigreurs dans l'IK !).
       Les annonces d'un goût douteux ne sont pas non plus les bienvenues (même recommandation);
 \item la publicité n'a pas sa place sur \fkz ; toute annonce ayant un net caractère publicitaire
       sera refusée (y compris s'il s'agit de publicité pour un sponsor);
 \item tout contenu portant atteinte à  une tierce personne ou à  un groupe est interdit dans les annonces
       et les sites \emph{web}, ainsi que tout lien vers un site ou document de ce type;
 \item tout contenu illégal, en particulier tout document (quel que soit son type)
       non libre de droits ou ayant un caractère pornographique, est interdit,
       ainsi que tout lien vers un site ou document de ce type;
 \item si un contenu d'un des deux types précédents échappe toutefois à  l'attention des webmestres,
       seuls leurs auteurs pourraient en être tenus responsables.
       Tout contenu de cette sorte qui apparaîtrait sur le site doit être immédiatement signalé.
\end{itemize}

Les annonces doivent être rédigées avec la syntaxe wiki. Celle-ci est expliquée en détail sur la page \urllink{https://www.frankiz.net/wiki\_help/notitle} (ce lien est accessible depuis la page de proposition). N'hésite pas à  y faire un petit tour pour savoir comment embellir tes annonces (gras, italique, liens hypertextes).

Pour préserver la lisibilité de \fkz, une annonce :
\begin{itemize}
 \item ne doit pas être trop longues (pas plus d'une quinzaine de lignes) ;
 \item doit être écrite dans un français correct : pas de langage sms, pas de fautes d'orthographe ;
 \item doit être écrite avec une typographie correcte : pas de majuscules ou de points d'exclamation partout, particulièrement dans le titre ;
 \item doit être publiée \textbf{au nom du bon binet} : tu ne devrais presque jamais avoir à écrire une annonce en ton nom ;
 \item doit être postée \textbf{à destination du groupe qu'elle concerne} ;
 \item doit concerner un nombre non négligeable de personnes ;
 \item doit contenir les informations essentielles comme la date ou le lieu pour un événement ;
\end{itemize}


Enfin, tout ce qui est proposé sur \fkz doit être validé par les
webmestres, qui ne sont pas là  pour censurer mais pour maintenir la
qualité du site. Inconvénient : \c ca peut parfois être un petit peu
long mais il ne faut pas s'inquiéter. 
Par ailleurs, il est possible de poster une annonce visible non pas par tout le monde,
mais uniquement par les membres d'un certain binet dont tu es sympathisant ou membre.
Cette annonce peut alors être validée par les administrateurs du binet, et sera donc visible 
beaucoup plus rapidement qu'une annonce globale.
Dans tous les cas, si tu as une demande quelconque à  faire sur le contenu du site, par exemple
modifier/supprimer une annonce, un seul réflexe : envoie un mail à 
\mail{web@eleves.polytechnique.fr}, réponse rapide (presque) assurée !

\subsubsection{Mails promo}

Il existe un cas particulier, celui des \textbf{\emph{mails} promo} : il est possible d'envoyer un \emph{mail} à toute
ta promotion, voire aux deux promotions, de préférence au nom d'un binet.
Cette procédure doit cependant rester exceptionnelle, sinon les gens ne lisent plus les
mails promo parce qu'ils pensent que cela ne les concerne pas. Par
conséquent :
\begin{itemize}
  \item ton seul interlocuteur valable est la Kès. Il est strictement interdit, sous peine de sanctions douloureuses, d'envoyer un \emph{mail} promo \og à la main \fg~en utilisant la liste de toutes les adresses \emph{e-mail},
  ou de faire envoyer ton \emph{mail} par ton compromo ou une secrétaire ;
  \item ton \emph{mail} doit vraiment concerner la promo (ou en tout cas une forte majorité) ;
  \item ton \emph{mail} doit être suffisamment important pour que les autres moyens de diffusion (\fkz\ et forums) ne soient pas des moyens suffisants.
\end{itemize}

Pour envoyer des mails promo, il est impératif de les soumettre via l'interface prévue pour cela sur \fkz (lien \lien{Administration}). Dans ce cas ce ne sont pas les webmestres mais la Kès qui choisit de les valider ou non. Le BR n'ayant qu'un rôle strictement technique, toute réclamation est à transmettre à la Kès.


%\subsubsection{Quel moyen pour quel besoin ?}
%\emph{Ton binet organise une activité sur le plateau} (ex : Spectacle,
%Soirées, Dégustations, \dots): pour annoncer l'activité sur frankiz, utilise
%le lien \lien{Proposer une activité} sur la page \lien{Activités}. Si tu veux donner plus de
%précisions sur cette activité, crée une page web adaptée sur le site
%de ton binet et fais pointer le lien de ton activité vers cette
%page. N'oublie pas non plus de préciser à qui est destinée cette activité (tout le monde, membres du binet, simplement toi pour l'afficher sur ton emploi du temps, %\dots).
%
%\emph{Ton binet organise une activité exceptionnelle} (ex : sortie le week-end, invitation à l'extérieur de l'X, ou plus gros: JSP ou Dez) 
%\emph{ou veut informer les élèves sur son activité} (Ex : recruter des élèves, diffuser une liste de spectacle) : pour avoir une annonce sur \fkz, utilise le lien \lienù{Proposer une annonce} sur la page correspondante.
%
%\emph{Attention :} si l'annonce n'est pas justifiée et qu'une activité aurait suffi, elle ne sera pas acceptée, ce qui fait perdre du temps à tout le
%monde. Si l'annonce que tu veux écrire est manifestement trop longue --- et risque donc grandement d'être refusée --- crée une page web adaptée sur le
%site de ton binet et fais pointer un lien dans ton annonce vers cette page.
%
%Pour continuer à communiquer avec les personnes intéressées, utilise
%le forum de ton binet (\ngname{br.binet.ton\_binet}) et les \emph{mails}
%(en créant éventuellement une liste de diffusion).
%
%Si ton annonce ne concerne pas toute les élèves, par
%exemple pour le prochain weekend spéleo, règle correctement la
%visibilité de l'annonce : les sympatisants de ton binet seront mieux
%avertis et ceux qui ne sont pas intéressés ne verront pas ton message.
