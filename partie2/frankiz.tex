\subsubsection{Frankiz}
\label{frankiz}
La page web frankiz est la page des \'el\`eves.
Elle est visible de l'int\'erieur et de l'ext\'erieur de l'X,
en int\'egralit\'e si tu t'es identifi\'e ou partiellement pour les autres utilisateurs.
Tu peux automatiser ta connexion gr\^ace \`a un cookie d'authentification.
Nous te conseillons de faire de \url{http://frankiz/} la page d'accueil de ton navigateur Internet.

Elle permet en particulier l'acc\`es aux services suivants :
les annonces et les activit\'es du plat\^al, l'annuaire des \'el\`eves (\menu{TOL} pour Trombi-On-Line),
le t\'el\'echargement de logiciels gratuits (\menu{X-Share}), la base de connaissances (\menu{Wikix}),
la question du jour (\menu{QDJ}) et m\^eme la m\'et\'eo.
Cette page est ais\'ement personnalisable (lien \lien{Pr\'ef\'erences}).
\`A toi d'explorer tout ce que tu peux y trouver !

Les \menu{annonces}, \menu{sondages} et \menu{activit\'es} permettent d'informer les \'el\`eves
de ce qui se passe \`a l'\'ecole.
Les annonces sont tri\'ees dans un sommaire et tu peux \'eventuellement faire dispara\^itre
celles qui ne t'int\'eressent pas (en fonction des skins).
Les activit\'es appara\^issent sur la page principale le jour o\`u elles ont lieu.
Les sondages apparaissent sur la page principale et lorsque le vote est termin\'e, tu peux voir les r\'esultats.
Tu peux proposer des annonces, des sondages et des activit\'es en utilisant les liens \lien{Proposer\ldots}.

L'\menu{Annuaire} permet de trouver des renseignements utiles sur tous les \'el\`eves sur le pl\^atal.
Tu peux mettre ta fiche \`a jour en utilisant le lien \lien{Pr\'ef\'erences}.

Si tu es le pr\'esident ou le webmestre d'un binet,
tu as droit \`a un lien \lien{Administration} \`a c\^ot\'e de ton lien \lien{Pr\'ef\'erences}.
Il te permet d'exercer les terribles pouvoirs du prez ou du web,
qui sont respectivement de g\'erer la liste des membres inscrits au binet dans le TOL
et de modifier l'ic\^one et la description du binet dans la page \menu{Binets}.
Pour cela, il suffit d'avoir sign\'e la feuille de demande de droit,
qui est (normalement) disponible dans la case courrier du BR \`a la K\`es.

La \menu{Foire Aux Questions} contient les r\'eponses aux questions les plus courantes.
C'est souvent plus rapide d'y faire un petit tour que d'appeler quelqu'un,
en plus il y a un moteur de recherche.
Et si tu vois une erreur, tu peux la corriger directement !

La rubrique \menu{X-Share} permet de t\'el\'echarger des logiciels pour Windows, Mac et Linux
s\'electionn\'es par le BR et des documents importants comme cet InfoBR.
En particulier, c'est l\`a que tu trouveras les logiciels d\'evelopp\'es par le BR, dont \app{qRezix}.
Si tu cherches un logiciel pour un usage particulier, commence par l\`a !

Publi-reportage : les \menu{X-Share} (entre autres) sont en cours de refonte
afin d'am\'eliorer leur ergonomie.
Si tu as des talents de programmeur, n'h\'esites pas, le d\'eveloppement de frankiz n'attend que toi !!

La \menu{QDJ} est une question binaire, s\'erieuse parfois mais le plus souvent bas\'ee sur un jeu de mots
ou sur l'activit\'e sur le campus.
Tu peux voter tous les jours et m\^eme proposer des questions au QDJMaster.

La rubrique \menu{Sites \'el\`eves} contient la liste des sites personnels des \'el\`eves h\'eb\'erg\'es sur frankiz.
Tu peux toi aussi publier ton site web en utilisant le lien dans \lien{Pr\'ef\'erences}.
De m\^eme, la rubrique \menu{Binets} contient la liste des binets, une description de celui-ci
et le lien vers le site de ce binet (\'eventuellement h\'eberg\'e sur frankiz).

M\^eme si frankiz est l'\oe uvre de tous, les webmestres se r\'eservent le droit de ne pas publier une annonce
ou d'interdire un site web si le contenu n'est pas jug\'e adapt\'e mais aussi,
dans le cas des annonces si elle g\^ene la lisibilit\'e g\'en\'erale, selon le bon principe :
``trop d'information tue l'information''.

Rappelons quelques r\`egles \'evidentes :
\begin{itemize}
 \item Tout contenu pol\'emique est banni des annonces (publie tes aigreurs dans l'IK ou sur \ngname{br.binet.polemix}).
       Les annonces d'un go\^ut douteux ne sont pas non plus les bienvenues (m\^eme recommandation).
 \item La publicit\'e n'a pas sa place sur \fkz ; toute annonce ayant un net caract\`ere publicitaire
       sera refus\'ee (y compris s'il s'agit de publicit\'e pour un sponsor).
 \item Tout contenu portant atteinte \`a une tierce personne ou \`a un groupe est interdit dans les annonces
       et les sites web, ainsi que tout lien vers un site ou document de ce type.
 \item Tout contenu ill\'egal, en particulier tout document (quel que soit son type)
       non libre de droits ou ayant un caract\`ere pornographique, est interdit,
       ainsi que tout lien vers un site ou document de ce type.
 \item Si un contenu d'un des deux types pr\'ec\'edents \'echappe toutefois \`a l'attention des webmestres,
       seuls leurs auteurs pourraient en \^etre tenus responsables.
       Tout contenu de cette sorte qui appara\^itrait sur le site doit \^etre imm\'ediatement signal\'e.
\end{itemize}

Les r\'egles \'el\'ementaires pour pr\'eserver la lisibilit\'e de frankiz :
\begin{itemize}
 \item Les annonces ne doivent pas \^etre trop longues (pas plus d'une quinzaine de lignes)
 \item L'interface des annonces utilise la syntaxe wiki, qui est expliqu\'ee
       sur une page accessible facilement depuis la page de proposition :
       n'h\'esites pas \`a y faire un petit tour pour savoir comment embellir tes annonces
       (gras, italique, liens hypertextes)
 \item Les titres des annonces ne doivent pas \^etre en majuscules ou pr\'ec\'ed\'es de signes de ponctuation.
 \item Un binet ne peut pas avoir deux annonces en m\^eme temps.
       Alors s'il te pla\^it, quand tu fais la com' de ton binet, r\'efl\'echis et \'ecris une belle annonce :
       plus elle est concise et pr\'ecise, mieux elle sera lue !
\end{itemize}

Enfin, tout ce qui est propos\'e sur frankiz doit \^etre valid\'e par les webmestres,
qui ne sont pas l\`a pour censurer mais pour maintenir la qualit\'e du site.
Inconv\'enient : \c{c}a peut parfois \^etre un petit peu long mais faut pas s'inqui\'eter.
Dans tous les cas, si tu as une demande quelconque \`a faire sur le contenu du site,
genre modifier/supprimer une annonce, un seul r\'eflexe :
envoies un mail \`a \mail{web@frankiz}, r\'eponse rapide (presque) assur\'ee !
