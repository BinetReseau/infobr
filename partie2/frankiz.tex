\subsection{Frankiz}

\bghdr{images/logo-Frankiz}

\label{frankiz} La page web \fkz est la page des \'el\`eves. Elle est visible de l'int\'erieur et de
l'ext\'erieur de l'X, en int\'egralit\'e si tu t'es identifi\'e ou partiellement pour les autres
utilisateurs. Nous te conseillons de faire de \urllink{http://www.frankiz.net/} la page d'accueil de ton navigateur Internet.

\fkz permet notamment l'acc\`es aux services suivants : les
annonces et les activit\'es du pl\^atal, l'annuaire des \'el\`eves
(\menu{TOL} pour Trombino On Line), ton emploi du temps personnalis\'e (\menu{EdT}), la base de connaissances (\menu{WikiX}) et
la question du jour (\menu{QDJ}). Tu peux accéder à ces services sous forme de mini-modules que tu peux d\'eplacer sur ta page d'accueil pour adopter la configuration que tu préfères.
Tout est ais\'ement personnalisable (lien \lien{Administration}). \`A toi d'explorer tout ce que tu peux y trouver !

Les \menu{annonces} et \menu{activit\'es} permettent d'informer les \'el\`eves de ce qui se passe \`a  l'\'Ecole. Les annonces sont tri\'ees dans
un sommaire et disparaissent une fois que tu les as lues. Tu peux \'egalement en mettre certaines en favoris pour ne garder que celles qui t'int\'eressent. Les activit\'es sont list\'ees sur la page d\'edi\'ee et apparaissent
sur la page principale le jour o\`u elles ont lieu. Tu peux proposer des annonces et des activit\'es en utilisant les liens sur leurs pages respectives \lien{Annonces} et \lien{Activit\'es}. Tu peux proposer des activit\'es ou annonces en ton nom propre ou au nom d'un binet dont tu es membre pour assurer une meilleure visibilit\'e \`a celles-ci.

Le \menu{Trombino} permet de trouver des renseignements utiles sur
tous les \'el\`eves sur le pl\^atal. Tu peux mettre ta fiche \`a  jour en
utilisant le lien \lien{Administration}.

En rejoignant un groupe ou un binet, tu peux le faire en tant que \menu{membre} (participant aux activit\'es) ou simple \menu{sympathisant} (ce qui te permet d'être mieux informé sur les différentes activités du binet). Chaque groupe ou binet a \'egalement ses \menu{administrateurs} qui g\`erent les droits des membres, valident les annonces ou les activit\'es faites au nom du groupe et renseignent les diverses informations concernant le binet (description, logo, site web, \dots). Si tu es administrateur, tu peux acc\'eder \`a tous ces droits depuis ta page \lien{Administration}.

Il est tr\`es important que tu deviennent rapidement membre ou sympathisant des binets qui peuvent t'int\'eresser.
En effet, cela te donnera acc\`es \`a leurs annonces internes et permettra aux autres membres du binet de te 
contacter beaucoup plus facilement pour organiser des activit\'es.

%Le \menu{WikiX} contient les r\'eponses aux questions les plus courantes, sur des sujets aussi divers que l'histoire et les traditions de l'\'Ecole, la
%vie pratique sur le pl\^atal, les binets, l'informatique, etc. C'est souvent plus rapide d'y faire un petit tour que d'appeler quelqu'un, d'autant plus qu'il y a
%un moteur de recherche. Et si tu vois une erreur, tu peux la corriger directement !

Tu pourras aussi retrouver de nombreux liens sur le bandeau gris en haut de l'écran, que ce soit vers les différents sites de l'\'Ecole (la direction des études \menu{DE}, ton emploi du temps avec tes salles de cours \menu{ENEX}, la fondation de l'X \menu{FX}, \dots), le \menu{WikiX}, les archives du JTX, et plein d'autres, \dots

%Publi-reportage : les \menu{X-Share} (entre autres) sont en cours de refonte afin d'am\'eliorer leur ergonomie. Si tu as des talents de programmeur, n'h\'esite pas, le d\'eveloppement de frankiz n'attend que toi !!

La \menu{QDJ} est une question binaire, parfois s\'erieuse mais le
plus souvent bas\'ee sur un jeu de mots ou sur l'activit\'e sur le
campus. Tu peux voter tous les jours et m\^eme proposer des questions
au QDJmestre.

M\^eme si \fkz est l'\oe uvre de tous, les webmestres se r\'eservent le droit de ne pas publier une annonce ou d'interdire un site web si le contenu
n'est pas jug\'e adapt\'e mais aussi, dans le cas des annonces si elle g\^ene la lisibilit\'e g\'en\'erale, selon le bon principe : \guillemotleft~trop
d'information tue l'information~\guillemotright .

Rappelons quelques r\`egles \'evidentes :
\begin{itemize}
 \item Tout contenu pol\'emique est banni des annonces (publie tes aigreurs dans l'IK !).
       Les annonces d'un go\^ut douteux ne sont pas non plus les bienvenues (m\^eme recommandation);
 \item La publicit\'e n'a pas sa place sur \fkz ; toute annonce ayant un net caract\`ere publicitaire
       sera refus\'ee (y compris s'il s'agit de publicit\'e pour un sponsor);
 \item Tout contenu portant atteinte \`a  une tierce personne ou \`a  un groupe est interdit dans les annonces
       et les sites \emph{web}, ainsi que tout lien vers un site ou document de ce type;
 \item Tout contenu ill\'egal, en particulier tout document (quel que soit son type)
       non libre de droits ou ayant un caract\`ere pornographique, est interdit,
       ainsi que tout lien vers un site ou document de ce type;
 \item Si un contenu d'un des deux types pr\'ec\'edents \'echappe toutefois \`a  l'attention des webmestres,
       seuls leurs auteurs pourraient en \^etre tenus responsables.
       Tout contenu de cette sorte qui appara\^itrait sur le site doit \^etre imm\'ediatement signal\'e.
\end{itemize}

Les r\`egles \'el\'ementaires pour pr\'eserver la lisibilit\'e de \fkz :
\begin{itemize}
 \item Les annonces ne doivent pas \^etre trop longues (pas plus d'une quinzaine de lignes);
 \item L'interface des annonces utilise la syntaxe wiki, qui est expliqu\'ee en d\'etail sur une page accessible facilement depuis la page de proposition. N'h\'esite pas \`a  y faire un petit tour pour savoir comment embellir tes annonces (gras, italique, liens hypertextes);
 \item Les titres des annonces ne doivent pas \^etre en capitales ou pr\'ec\'ed\'es de signes de ponctuation ;
 \item Chaque annonce doit \^etre publi\'ee au nom du bon binet : tu ne devrais presque jamais avoir \`a \'ecrire une annonce en ton nom ;
 \item Une seule annonce est en principe autoris\'ee par \'ev\'enement, afin que le nombre total d'annonces reste raisonnable (une annonce de rappel peut \^etre tol\'er\'ee si la premi\`ere date de plus d'une semaine).
\end{itemize}

Enfin, tout ce qui est propos\'e sur \fkz doit \^etre valid\'e par les
webmestres, qui ne sont pas l\`a  pour censurer mais pour maintenir la
qualit\'e du site. Inconv\'enient : \c ca peut parfois \^etre un petit peu
long mais il ne faut pas s'inqui\'eter. 
Par ailleurs, il est possible de poster une annonce visible non pas par tout le monde,
mais uniquement par les membres d'un certain binet dont tu es sympathisant ou membre.
Cette annonce peut alors \^etre valid\'ee par les administrateurs du binet, et sera donc visible 
beaucoup plus rapidement qu'une annonce globale.
Dans tous les cas, si tu as une demande quelconque \`a  faire sur le contenu du site, par exemple
modifier/supprimer une annonce, un seul r\'eflexe : envoie un mail \`a 
\mail{web@eleves.polytechnique.fr}, r\'eponse rapide (presque) assur\'ee !


%\subsubsection{Quel moyen pour quel besoin ?}
%\emph{Ton binet organise une activité sur le plateau} (ex : Spectacle,
%Soirées, Dégustations, \dots): pour annoncer l'activité sur frankiz, utilise
%le lien \lien{Proposer une activité} sur la page \lien{Activit\'es}. Si tu veux donner plus de
%précisions sur cette activité, crée une page web adaptée sur le site
%de ton binet et fais pointer le lien de ton activité vers cette
%page. N'oublie pas non plus de pr\'eciser \`a qui est destin\'ee cette activit\'e (tout le monde, membres du binet, simplement toi pour l'afficher sur ton emploi du temps, %\dots).
%
%\emph{Ton binet organise une activité exceptionnelle} (ex : sortie le week-end, invitation à l'extérieur de l'X, ou plus gros: JSP ou Dez) 
%\emph{ou veut informer les élèves sur son activité} (Ex : recruter des élèves, diffuser une liste de spectacle) : pour avoir une annonce sur \fkz, utilise le lien \lienù{Proposer une annonce} sur la page correspondante.
%
%\emph{Attention :} si l'annonce n'est pas justifiée et qu'une activité aurait suffi, elle ne sera pas acceptée, ce qui fait perdre du temps à tout le
%monde. Si l'annonce que tu veux écrire est manifestement trop longue --- et risque donc grandement d'être refusée --- crée une page web adaptée sur le
%site de ton binet et fais pointer un lien dans ton annonce vers cette page.
%
%Pour continuer à communiquer avec les personnes intéressées, utilise
%le forum de ton binet (\ngname{br.binet.ton\_binet}) et les \emph{mails}
%(en créant éventuellement une liste de diffusion).
%
%Si ton annonce ne concerne pas toute les élèves, par
%exemple pour le prochain weekend spéleo, règle correctement la
%visibilité de l'annonce : les sympatisants de ton binet seront mieux
%avertis et ceux qui ne sont pas intéressés ne verront pas ton message.
