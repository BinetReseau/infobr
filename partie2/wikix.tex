
\subsection{Une source d'informations inestimable : le WikiX}
\label{WikiX}
Bien que n'\'etant pas \`a proprement parler un service du binet R\'eseau, le BR h\'eberge sur un de ses serveurs un site un peu particulier connu sous le nom de \textbf{WikiX}. C'est un wiki qui rassemble toutes les informations dont tu peux avoir besoin sur le pl\^atal. Le plus court moyen d'y aller est de taper \urllink{wikix} dans la barre d'adresse de ton navigateur mais il y a aussi un lien vers le WikiX sur Frankiz dans le menu navigation.

La premi\`ere fois que tu te connectes sur le WikiX il faut que tu t'identifies (identification \emph{via} \urllink{polytechnique.org}); ensuite ton navigateur gardera un cookie qui t'identifiera \`a chaque connection si tu le souhaites. \emph{Tu ne peux lire et modifier le WikiX que si tu es identifi\'e} !

Tu es bien entendu encourag\'e \`a contribuer au WikiX pour faire profiter les autres de ton exp\'erience soit en modifiant ou en mettant \`a jour un article existant, soit en cr\'eant un nouvel article qui manquait au WikiX.


Tu te rendras vite compte que \emph{peu importe l'information que tu cherches, elle est sur le WikiX.}
