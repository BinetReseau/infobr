
\subsection{Une source d'informations inestimable : le WikiX}
\label{WikiX}

Bien que n'étant pas à proprement parler un service du Binet Réseau, un site un peu particulier connu sous le nom de \textbf{WikiX} est hébérgé sur un des serveurs du BR.
C'est un wiki qui rassemble toutes les informations dont tu peux avoir besoin sur le plâtal.
Le plus court moyen d'y aller est de taper \urllink{wikix} dans la barre d'adresse de ton navigateur, mais il y a aussi un lien vers le WikiX sur \fkz dans le menu navigation.

La première fois que tu te connectes sur le WikiX il faut que tu t'identifies (identification \emph{via} \urllink{polytechnique.org});
ensuite ton navigateur gardera un cookie qui t'identifiera à chaque connection si tu le souhaites. \emph{Tu ne peux lire et contribuer au WikiX que si tu es identifié !}

Tu es bien entendu encouragé à écrire dans le WikiX pour faire profiter les autres de ton expérience,
soit en mettant à jour un article existant, soit en créant un nouvel article qui manquait au WikiX.

Tu te rendras vite compte que \emph{peu importe l'information que tu cherches, elle est sur le WikiX.}
