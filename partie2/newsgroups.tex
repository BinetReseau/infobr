\subsection{Les newsgroups}
\subsubsection{Pr\'esentation}
	Le BR fournit aux \'el\`eves un service de newsgroups, souvent surnomm\'e ��les br��. Ils fonctionnent un peu comme un forum: chacun peut poster une annonce, poser une question, r\'epondre \`a un sujet post\'e par un autre \'el\`eve... ils sont tr\`es utiles pour faire de la pub pour une activit\'e organis\'ee par un binet, poser une question en cas de probl\`eme, ou tout simplement savoir ce qu'il se passe sur le platal. Pour qu'ils puissent remplir pleinement ce r\^ole, le BR a \'edicter un certain nombre de r\`egles que les newsmestres sont charg\'es de faire respecter.

\subsubsection{Les r\`egles}
-Pas d'insultes, attaques personnelles,calomnies et autre.
-Ne pas dissimuler son identit\'e, et donc utiliser une adresse mail valide et un pseudo qui permet de t'identifier par l'intrerm\'ediaire du TOL.
-Pour que les br restent lisibles, faire des crossposts propres (cf infoBR p11).
-Dans le m\^eme ordre d'id\'ee, poster sur le forum adapt\'e: les petites annonces vont sur le br.pa et nulle part ailleurs, seules les questions concernant directement le BR ont leur place sur le br.binet.br  (pour les questions d'informatique il y a les br.informatique.*).
-Eviter de troller abusivement (sauf sur br.binet.polemix qui sert \`a \c{c}a ;))
-Garder son calme (ou \'eviter de poster), mieux vaut une explication dans la vraie vie que par br interpos\'e.

\subsubsection{Les diff\'erents newsgroups de frankiz}
	Frankiz h\'eberge un certain nombre de newsgroups (plus de 200...) mais il est probable qu'ils ne t'int\'eressent pas tous. Pour faire un choix au d\'ebut, en voici une liste non exhaustive (tu peux aussi t'amuser \`a tous les lire si tu veux devenir newsmestre ;)):
br.eleves: posts g\'en\'eraux int\'eressant potentiellement un  grand nombre d'\'el\`eve
br.promo.* : pour les posts ne concernant \`a priori qu'une seule promo (Rouje, Jone ou Oranje)
br.kes: le newsgroup de la K\`es. Suite \`a un pourrissage abusif de ce br, des restrictions ont \'et\'e mises en place: tout le monde peut lancer une nouvelle discussion sur ce newsgroup, mais seuls les kessiers peuvent r\'epondre. Il sert donc essentiellement \`a la K\`es pour faire une annonce, ou aux \'el\`eves pour poser une question aux kessiers (note: il n'est pas non plus interdit de se d\'eplacer \`a la K\`es pour poser ta question directement)
br.binet.ton\_binet: chaque binet a son newsgroup. Il sert le plus souvent pour la communication interne du binet, ses annonces ou \`a poser une question au dit binet. Dans le cas d'un nouveau binet, le BR peut lui cr\'eer un newsgroup, mais uniquement apr\`es que le binet ait \'et\'e cr\'e\'e dans les r\`egles \`a la K\`es.
br.binet.br: comme son nom l'indique, il s'agit du newsgroup du binet r\'eseau. A utiliser pour les questions ayant un lien DIRECT avec le BR. Pour les autres questions li\'ees \`a l'informatique, il convient d'utiliser les newsgroups suivants:
br.informatique.reseau: pour tout ce qui a trait au r\'eseau
br.informatique.windows/linux/mac : selon ton OS
br.informatique.divers
br.binet.lose et br.binet.suba\"isse: pour raconter tes meilleures loses/suba\"isses et d\'ecouvrir que tu n'es pas le seul \`a loser. Pour la diff\'erence subtile entre une lose et une suba\"isse, contacter le binet suba\"isse (nous on a pas compris).
br.binet.polemix: le newsgroup pour troller par excellence. Attention, essayer de suivre les discussions qui y ont trouv\'e refuge peut prendre \'enorm\'ement de temps.
br.communaut\'e.*: les newsgroups de diff\'erentes communaut\'es (religieuses, musicales, g\'eographiques ou autres)
br.enseignement: pour tout ce qui a trait \`a l'enseignement.
br.section.ta\_section\_sportive: le newsgroup de ta section. Utile pour savoir ce qu'il s'y passe et planifier les activit\'es.
br.pa: pour les petites annonces. Merci de les poster ici et pas ailleurs.
br.test: quand tu veux tester un truc sur les newsgroups, viens le faire ici plut\^ot que de pourrir un br utile.
br.trash: pour les craquages.
Public.*: les newsgroups accessibles \`a l'ensemble du personnel de l'\'ecole. Il y r\`egne assez peu d'activit\'e en g\'en\'eral, mais ils peuvent parfois avoir leur utilit\'e.

	Il arrive parfois que certains newsgroups temporaires soient cr\'e\'es pour des \'ev\`enements particuliers (campagne K\`es). Leur cr\'eation sera annonc\'e le plus souvent sur frankiz ou sur le br.eleves.

\subsubsection{Autres newsgroups}
	Le site polytechnique.org dispose de son propre service de newsgroups. Il sont accessibles \`a tous les polytechniciens, qu'ils soient actuellement sur le campus ou membres de promos pr\'ec\'edentes. Pour plus d'informations, www.polytechnique.org.
	Il est aussi possible d'acc\'eder \`a certains newsgroups ext\'erieurs \`a l'\'ecole via polynews. Ce serveur de la DSI est synchronis\'e avec l'ext\'erieur. Si tu cherches un newsgroup qui ne s'y trouve pas, n'h\'esite pas \`a demander aux newsmestres (cf section suivante).

\subsubsection{Les newsmestres}
	Les newsmestres sont les membres du BR charg\'es de l'administration et de la mod\'eration des newsgroups. Leur but est de maintenir les newsgroups dans un \'etat correct, pour que tout le monde puisse en profiter et trouver ce qu'il y cherche. Ne prend donc pas une remarque de leur part comme une attaque personnelle...
Pour les contacter: news@frankiz.polytechnique.fr

\subsection{Poster sur plusieurs newsgroups...}
\subsubsection{Qu'est-ce qu'un crosspost ?}
	Un crosspost permet de poster le m\^eme message sur diff\'erents newsgroups, et de renvoyer toutes les r\'eponses sur le m\^eme newsgroup, quelque soit le newsgroup sur lequel elles ont \'et\'e post\'ees.

\subsubsection{Avantage d'un crosspost}
- Toutes les r\'eponses que les gens te font sont centralis\'ees sur le 
m\^eme forum, ce qui t'\'evite de perdre du temps \`a lire des r\'eponses 
diss\'emin\'ees sur les diff\'erents forums o\`u tu as post\'e. De plus, ceux qui te 
r\'epondent peuvent eux aussi lire facilement toutes les r\'eponses que tu as d\'ej\`a 
re\c{c}ues.
- Sur certains clients news, il suffit de lire le message cross-post\'e
sur un forum pour qu'il soit marqu\'e comme lu sur tous les autres forums
o\`u il a \'et\'e post\'e, ce qui \'evite ainsi aux gens de devoir lire plusieurs
fois le m\^eme message.
- Cela t'\'evite de recevoir comme r\'eponse un "RTFIBRp11" hargneux de la
part d'un(e) newsmestre.

\subsubsection{Comment faire un crosspost ?}
	Il suffit de mettre dans le premier en-t\^ete ('groupe de discussion' ou 'newsgroup') la liste de tous les newsgroups o\`u tu veux poster, s\'epar\'es par des virgules.
Exemple: br.eleves, br.lose, br.binet.bob,br.promo.rouje
	Il faut ensuite mettre dans l'en-t\^ete "Transf\'erer \`a" le newsgroup o\`u tu veux que les r\'eponses apparaissent. Attention, ce doit \^etre l'un des newsgroups o\`u tu postes le message.
Voir les screenshots (exemple de Outlook Express et Thunderbird).\\

%\image{images/cross_post_TB}{0.3}{Thunderbird}
\image{images/cross_post_OE}{0.5}{Outlook Express}


Voil\`a, c'est quand m\^eme bien plus simple que de faire des copier/coller de ton message sur tous les newsgroups, et \c{c}a permet de consercer un peu de clart\'e sur les newsgroups. 

