% \subsection{Installer et mettre à jour Windows ou Linux}
\subsection{Installer et mettre à jour Linux}


% \subsubsection{Licences de produits Microsoft}
% \label{msdnaa} Les accords négociés par le BR avec Microsoft dans le cadre de MSDNAA donnent à  chaque X le droit de posséder une version de Windows
% XP Pro, de Windows Vista Business ou de Windows 7 Pro gratuite et légale, ainsi que les licences pour la plupart des logiciels de la société (quasiment tous, sauf
% Office et les jeux). La seule condition à  remplir est d'être étudiant sur le plâtal au moment de l'installation du logiciel ; tu pourras ensuite le
% garder sur ton PC même après ton départ de l'X.
% 
% La procédure pour obtenir les logiciels et les clés correspondantes
% est la suivante :
% \begin{itemize}
% 
% \item Va d'abord sur \fkz et connecte-toi, puis clique sur le lien \menu{Licences MSDNAA} qui se trouve dans la rubrique \menu{Administration}. Sélectionne le logiciel que tu souhaites installer et valide ta demande, tu recevras ta clé par e-mail. Facile ! Si jamais le logiciel n'est pas dans la liste proposée, c'est soit qu'il n'y a pas besoin de clé --- c'est le cas de beaucoup des logiciels autres que Windows, soit qu'on a oublié de l'y mettre ; dans ce cas, écris à  \mail{msdnaa@eleves} pour qu'on t'attribue manuellement une clé.
% 
% \item Maintenant que tu as ta clé, il faut télécharger le logiciel proprement
% dit. Pour cela, connecte-toi par FTP sur \urllink{ftp://miroir/windows/msdnaa/} avec ton client FTP préféré. Tu pourras, selon le logiciel, y récupérer soit une image du CD de type \file{logiciel.iso} (à 
% graver ou à  utiliser avec \app{Daemon Tools}), soit directement le contenu du CD.
%  
% \end{itemize}

\subsubsection{Miroirs GNU/Linux}

Le BR propose sur \urllink{ftp://miroir/} des miroirs de plusieurs distributions GNU/Linux. Leur avantage principal est que le téléchargement se fait directement sur le réseau local, donc \emph{très} rapidement.
% Les miroirs suivants sont disponibles:
% 
% \begin{itemize}
% \item \distrib{Cygwin} (Environnement Unix pour Windows);
% \item \distrib{Debian} (CDs, distribution et mises à jour);
% \item \distrib{Gentoo} (CDs, distribution et mises à jour);
% \item \distrib{Ubuntu} (CDs, distribution et mises à jour);
% \item \distrib{Fink} (Nombreux logiciels Unix/Linux adaptés pour Mac OS).
% \end{itemize}

Les miroirs évoluent régulièrement. Tu peux te référer à la page \pageref{ubuntu_mirror} pour leur configuration sous \distrib{Ubuntu} ;
pour les autres, toutes les informations pour leur utilisation sont disponibles sur \urllink{https://br.binets.fr/Miroir\_FTP}.

%Pour Windows, les fichiers ISO à  télécharger sont les suivants :
%\begin{description}
%\item[Pour Windows XP]
%\urllink{ftp://miroir/windows/msdnaa/Windows\%20XP/Francais/fr\_windows\_xp\_professional\_with\_service\_pack\_3\_x86\_cd\_x14-80440.iso}.
%
%\item[Pour Windows Vista]
%Là , il y a deux possibilités :
%\begin{description}
%\item[Si tu as un processeur 64 bits :] L'image à  télécharger est dans\\
%\urllink{ftp://miroir/msdnaa/win\_vista\_business\_avec\_sp1/dvd\_64bits/francais/}. Attention, cette
%image ISO est à  graver sur un DVD. On remarque aussi que parfois, même sur un processeur 64 bits il
%vaut mieux choisir la version 32 bits de Windows (celle qui est dans le prochain \emph{item}) pour des
%raisons de disponibilité de pilotes.
%\item[Si tu as un processeur 32 bits :] Là , tu peux choisir de graver :
%\begin{itemize}
%\item soit un DVD, dans \\
%\urllink{ftp://miroir/msdnaa/win\_vista\_business\_avec\_sp1/dvd\_32bits/francais/};
%\item soit quatre CDs, que tu trouveras dans \\
%\urllink{ftp://miroir/msdnaa/win\_vista (version CD)/cd\_32bits/french/}.
%
%\end{itemize}
%\end{description}
%
%\item[Pour Windows 7]
%Comme pour Vista, deux possibilités selon ton processeur. Attention, les images sont à graver sur DVD :
%\begin{description}
%\item[Si tu as un processeur 64 bits :] L'image à  télécharger est dans\\
%\urllink{ftp://miroir/msdnaa/win\_7/french/64 bits/}.
%\item[Si tu as un processeur 32 bits :] L'image à  télécharger est dans\\
%\urllink{ftp://miroir/msdnaa/win\_7/french/32 bits/}.
%
%\end{description}
%\end{description}
%
%Les versions anglophones de Windows XP, Vista et 7 sont également disponibles sur \server{miroir}.
%Ainsi, si tu as acheté un ordinateur sans OS (et ainsi économisé environ 150 \euro), tu vas chez un copain, fais les demandes et graves le CD chez lui. Si tu as encore des questions, plus de détails sont donnés dans le Wikix et le WikiBR de \fkz.
%

