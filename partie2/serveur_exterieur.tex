\subsection{Se connecter à des serveurs à travers le proxy}

\subsubsection{Socat}

\app{Socat} est un utilitaire qui permet d'établir des connections à travers le proxy. Pour l'installer la procédure dépend de ton système d'exploitation~:

\paragraph{Sous Windows} Télécharge-le sur \urllink{http://blog.gentilkiwi.com/programmes/socat}
\paragraph{Sous Linux} Fais \cmd{sudo apt-get install socat} dans une console (si tu n'es pas sous Ubuntu/Kubuntu il faudra adapter la ligne)
\paragraph{Sous Mac} Il faut avoir installé \app{Fink} (page \pageref{mac-fink}) puis faire dans une console
\cmdline{fink selfupdate \\
fink update-all \\
fink install socat
}

\subsubsection{Ssh et Git}

Sous Windows, il n'y a pas de problème car \app{Putty} permet de paramétrer le proxy en allant dans \menu{Connection} puis \menu{Proxy} et en mettant les options comme ci-dessous~:

\imagepos{images/putty_proxy}{0.5}{Connexion à travers le proxy avec \app{Putty}}{!h}

Pour les autres, le mieux est de définir une commande sshExt qui fonctionnera comme un ssh normal mais en passant à travers le \emph{proxy}~:
\cmdline{alias scpExt="scp -o ProxyCommand='socat - PROXY:kuzh:\%h:\%p,proxyport=8080'" \\
alias sshExt="ssh -o ProxyCommand='socat - PROXY:kuzh:\%h:\%p,proxyport=8080'"}

Lorsque tu as la flemme de taper sshExt ou que tu veux utiliser git il faut rajouter deux lignes dans \file{/home/user/.ssh/config}, par exemple pour toujours passer par le \emph{proxy} lorsque tu te connectes à \server{github.com}~:
\cmdline{Host github.com \\
    ProxyCommand socat - PROXY:kuzh:\%h:\%p,proxyport=8080
}

\subsubsection{Mail}

Si tu dois absolument utiliser un serveur \emph{mail} et que tu ne peux pas faire de redirections mail alors tu peux créer un tunnel vers le serveur \emph{mail} en utilisant dans une console~:
\cmdline{socat TCP-L:6663,fork,reuseaddr PROXY:kuzh:<nom de ton serveur>:993}
Ensuite tu devras configurer ton client \emph{mail} pour qu'il se connecte à \server{127.0.0.1} sur le port 6663.
