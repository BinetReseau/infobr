%$Id: irc.tex 144 2005-03-25 01:11:37Z myk $

\subsection{IRC}

\label{irc}

IRC est un autre moyen de communication mis à ta disposition par le Binet Réseau. Il s'agit d'un système de \emph{chat} (messagerie instantanée) permettant à la fois de dialoguer à plusieurs dans des salons (ou canaux), mais également d'avoir des conversations privées avec d'autres personnes connectées.


Le serveur IRC du binet Réseau est relié à RezoSup, réseau IRC des grandes écoles d'ingénieurs et universités françaises (Centrale, Supélec ou par exemple l'Univeristé Paris-Sud y sont aussi connectés).

Pour rejoindre IRC tu disposes de deux méthodes:

                  
\begin{description}
  \item[passer par l'interface web :] utilise \urllink{http://irc/}. Tu pourras ainsi profiter d'IRC directement depuis ton navigateur.
  \item[installer un client IRC :] nous te conseillons \app{KVIrc} (disponible gratuitement sur \urllink{www.kvirc.net}). Utilise  \server{ircserver} comme serveur, et \server{6667} comme port.
\end{description}

 
Une fois connecté, tu pourras rejoindre de nombreux canaux de discussions (\emph{channels}) dont :
\begin{itemize}
  \item \ngname{\#x2011} Le canal de la promotion 2011 (d'horribles jônes)
  \item \ngname{\#x2010} Le canal de la promotion 2010 (de merveilleux roûjes)
  \item \ngname{\#br} Le canal du Binet Réseau.
  \item \ngname{\#jtx} Le canal du Jtx. Spoilers sur les projs à venir garantis.
  \item \ngname{\#faerix} Le canal du binet fa\"erix.
  \item \ngname{\#linux} Pour toutes les discussions sur linux
  \item \ngname{\#babe} Le canal du Binet d'Aide aux Bars d'Etages.
  \item \ngname{\#physique} Besoin d'aide en physique.
  \item \ngname{\#x} Le canal public de tous les X
  \item \ngname{\#help} Un problème avec IRC ? C'est ici.
\end{itemize}

 Tu veux en savoir plus ? Enregistrer ton pseudo ? Créer ton propre canal ? Va voir sur \urllink{https://br.binets.fr/Aide\_utilisateur\_d'IRC} ou contacte \textbf{irc@eleves.polytechnique.fr}.

\imagepos{images/irc}{1}{Viens discuter sur IRC !}{!h}

