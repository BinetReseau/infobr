%$Id: irc.tex 144 2005-03-25 01:11:37Z myk $

\subsubsection{IRC}

\label{irc}

IRC est un autre moyen de communication mis � ta disposition par le Binet R�seau. Il s'agit d'un syst�me de chat (messagerie instantan�e) permettant � la fois de dialoguer � plusieurs dans des \emph{salons}, mais �galement d'avoir des conversations priv�es avec d'autres personnes connect�es.

Le serveur IRC du Binet R�seau est reli� � RezoSup, r�seau IRC des grandes �coles d'ing�nieurs et universit� fran�aise.

Pour te connecter sur IRC tu disposes de deux m�thodes:

\begin{description}
  \item[utiliser un client IRC]: nous te conseillons \app{X-Chat} (disponible dans la partie \emph{T�l�charger} sur \fkz). Utilises  \server{ircserver} comme serveur, et \server{6667} (port par d�faut) comme port.
  \item[passer par l'interfa�e web]: utilise \url{http://ircserver/}, ou suit le lien \emph{Acc�der � IRC} sur \fkz. Tu pourra ainsi profiter d'IRC sans rien avoir � installer.
\end{description}

Nous te conseillons les salons de discussion (\textit{`channels'}) suivants :
\begin{itemize}
  \item \ngname{\#x} le chan de tous les X
  \item \ngname{\#linux} si tu as des questions \`a poser sur linux
  \item \ngname{\#superquizz} un quizz en ligne (tape \texttt{!nick x} en arrivant)
\end{itemize}
