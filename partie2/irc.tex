%$Id: irc.tex 144 2005-03-25 01:11:37Z myk $

\subsubsection{IRC}

\label{irc}

IRC est un autre moyen de communication mis à ta disposition par le Binet Réseau. Il s'agit d'un système de \emph{chat} (messagerie instantanée)
permettant à la fois de dialoguer à plusieurs dans des \emph{salons}, mais également d'avoir des conversations privées avec d'autres personnes
connectées.

Le serveur IRC du Binet Réseau est relié à RezoSup, réseau IRC des grandes écoles d'ingénieurs et université française.

Pour te connecter sur IRC tu disposes de deux méthodes:

\begin{description}
  \item[utiliser un client IRC:] nous te conseillons \app{X-Chat} (disponible dans la partie \emph{Télécharger} sur \fkz). Utilise  \server{ircserver} comme serveur, et \server{6667} (port par défaut) comme port.
  \item[passer par l'interface web:] utilises \urllink{http://ircserver/}, ou suis le lien \menu{Accéder à IRC} sur \fkz. Tu pourras ainsi profiter d'IRC sans rien avoir à installer.
\end{description}

Nous te conseillons les salons de discussion (\emph{channels}) suivants :
\begin{itemize}
  \item \ngname{\#x} le chan de tous les X
  \item \ngname{\#linux} si tu as des questions \`a poser sur linux
  \item \ngname{\#superquizz} un quizz en ligne (tape \texttt{!nick x} en arrivant)
  \item \ngname{\#br} le chan du BR !
\end{itemize}
