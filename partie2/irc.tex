%$Id: irc.tex 144 2005-03-25 01:11:37Z myk $

\subsection{IRC}

\label{irc}

Le BR te fournit un autre moyen de communication : l'IRC sur RezoSup. IRC veut dire Internet Relay Chat ; RezoSup est le r\'eseau des grandes \'Ecoles et universit\'es auquel est rattach\'e le BR. C'est le moyen id\'eal de discuter avec un groupe de personnes, car l'IRC est organis\'e par salles de discussion. Et comme RezoSup est inter-\'ecoles, tu pourras peut-\^etre y retrouver des potes de pr\'epa !

Pour te connecter \`a RezoSup, tu peux cliquer sur le lien IRC sur \fkz --- attention c'est assez long, c'est normal ---, ou installer un programme sp\'ecifique comme \app{X-Chat}. Avec \app{X-Chat}, il suffit que tu te connectes sur \server{ircserver}, port \server{6667} (par d\'efaut). Nous te conseillons les salons de discussion (\textit{`channels'}) suivants :
\begin{itemize}
  \item \ngname{\#x} le chan de tous les X
  \item \ngname{\#linux} si tu as des questions \`a poser sur linux
  \item \ngname{\#superquizz} un quizz en ligne (tape !nick x en arrivant)
\end{itemize}
