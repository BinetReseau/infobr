%$Id: irc.tex 144 2005-03-25 01:11:37Z myk $

\subsubsection{IRC}

\label{irc}

%IRC est un autre moyen de communication mis à ta disposition par le Binet Réseau. Il s'agit d'un système de \emph{chat} (messagerie instantanée) permettant à la fois de dialoguer à plusieurs dans des \emph{salons}, mais également d'avoir des conversations privées avec d'autres personnes connectées.

IRC is a communication device provided by the Binet Reseau. It is a \emph{chat} software
enabling a multi-user conversation within \emph{channels}, as well as private chats with other connected people.

The Binet Reseau's IRC server is connected to RezoSup, an IRC network gathering lost of french engineering schools and universities.

%Le serveur IRC du Binet Réseau est relié à RezoSup, réseau IRC des grandes écoles d'ingénieurs et université française.

%Pour te connecter sur IRC tu disposes de deux méthodes:
There are two ways for you to connect to IRC :


\begin{description}
  \item[using an IRC client:] we recommend the use of \app{X-Chat} (available in the \emph{Télécharger} part on  \fkz). Use \server{ircserver} as server, and \server{6667} (default port) as port.
  \item[using the web interface:] go to \urllink{http://ircserver/}, or follow the link \menu{Accéder à IRC} on \fkz. Thus you'll be able to use the IRC without installing anything.
 \end{description}

Here are some channels you might find usefull :
 \begin{itemize}

  \item \ngname{\#x} the channel for all students at the Ecole Polytechnique.
  \item \ngname{\#linux} if ever you have questions about linux
  \item \ngname{\#superquizz} an online quizz(type \texttt{!nick x} when login in)
  \item \ngname{\#br} the Binet Reseau's channel
 \end{itemize}
