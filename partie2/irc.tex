%$Id: irc.tex 144 2005-03-25 01:11:37Z myk $
%
\subsection{Discute en instantané avec IRC}

\label{irc}

IRC est un autre moyen de communication mis à ta disposition par le Binet Réseau.
Il s'agit d'un système de \emph{chat} (messagerie instantanée) permettant à la fois de dialoguer à plusieurs dans des salons (ou canaux),
mais également d'avoir des conversations privées avec d'autres personnes connectées.


Le serveur IRC du Binet Réseau est relié à RezoSup, réseau IRC des grandes écoles d'ingénieurs et universités françaises (Centrale, Supélec ou par exemple l'Univeristé Paris-Sud y sont aussi connectés).

Pour rejoindre IRC tu disposes de deux méthodes:

\begin{description}
  \item[passer par l'interface web :] utilise \urllink{http://irc/}. Tu pourras ainsi profiter d'IRC directement depuis ton navigateur ;
  \item[installer un client IRC :] nous te conseillons \app{KVIrc} (disponible gratuitement sur \linebreak \urllink{www.kvirc.net}). Utilise  \server{ircserver} comme serveur, et \server{6667} comme port.
\end{description}

Une fois connecté, tu pourras rejoindre de nombreux canaux de discussions (\emph{channels}) dont :
\begin{itemize}
  \item \ngname{\#x2015} Le canal de la promotion 2015 (d'horribles jônes) ;
   \item \ngname{\#x2014} Le canal de la promotion 2014 (de merveilleux roûjes) ;
  \item \ngname{\#br-actif} Le canal du Binet Réseau ;
  \item \ngname{\#br-support} Le canal dédié au discussions de support.
  %\item \ngname{\#jtx} Le canal du JTX. Spoilers sur les projs à venir garantis ;
  %\item \ngname{\#faerix} Le canal du binet fa\"erix ;
%   \item \ngname{\#physique} Besoin d'aide en physique ;
  \item \ngname{\#platal} Le canal public des deux promotions du Platal ;
  \item \ngname{\#pasloin} Le chan des anciens qui trainent encore sur le platal.
  \item \ngname{\#help} Un problème avec IRC ? C'est ici.
\end{itemize}

 Tu veux en savoir plus ? Enregistrer ton pseudo ? Créer ton propre canal ? Va voir sur \urllink{https://br.binets.fr/Aide\_utilisateur\_IRC} ou contacte \mail{irc@eleves.polytechnique.fr}.
 Le BR fournit aussi un accès persistant — historique des conversations, pseudo réservé, etc. — aux élèves utilisant régulièrement ce service, et ce sur demande.
 %(Si tu en as besoin, le BR peut te fournir un accès persistant qui te permettra d'avoir tous les logs de ce qui se dit.)
%\imagepos{images/irc}{1}{Viens discuter sur IRC !}{!h}

