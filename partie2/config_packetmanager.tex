\subsection{Configurer le gestionnaire de paquets}



% \paragraph{Sous Archlinux}
% Pour pouvoir utiliser \app{pacman} à travers le \emph{proxy} de l'École, il suffit de configurer les variables d'environnement ''\_proxy''.
% \noindent \cmdline{
% export http\_proxy="http://kuzh.polytechnique.fr:8080/"\\
% export https\_proxy="http://kuzh.polytechnique.fr:8080/"
% }

\paragraph{Sous Debian}
\label{debian_mirror} Pour pouvoir utiliser \app{APT} (via \app{apt-get} ou \app{aptitude}) à travers le \emph{proxy} de l'École, il faut ajouter dans \file{/etc/apt/apt.conf} la ligne~:
\cmdline{Acquire::http::Proxy "http://129.104.247.2:8080/";}
Le cas échéant, il faudra créer ce fichier, qui est vide la plupart du temps.

\paragraph{Sous Gentoo}
\label{gentoo_mirror} Pour pouvoir utiliser \app{emerge} à  travers le serveur mandataire (\emph{proxy}) de l'École, il faut définir %les variables d'environnement
ci-dessous dans le fichier \file{/etc/make.conf}~:
\cmdline{http\_proxy=http://kuzh.polytechnique.fr:8080\\
ftp\_proxy=http://kuzh.polytechnique.fr:8080\\
no\_proxy=.eleves.polytechnique.fr\\
GENTOO\_MIRRORS="http://gentoo.osuosl.org/"\\
SYNC="rsync://rsync/gentoo-portage"}

%Tu peux évidemment ajouter d'autres miroirs (séparés par des espaces) dans ta liste mais \urllink{ftp://miroir} étant interne, il sera toujours beaucoup plus rapide que les autres.


Pour le reste, le paramétrage des dépôts se fait comme toujours dans \file{/etc/apt/sources.list} ou via ton gestionnaire graphique préféré.
