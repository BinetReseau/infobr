\subsection{Platalpad}
\label{platalpad}

Platalpad est un éditeur de texte collaboratif par navigateur, un peu comme un google doc. Il permet de travailler à plusieurs sur un même document, chacun voyant les modification des autres. Il est donc très pratique pour s'organiser.\\
Il se décline en deux versions :
\begin{itemize}

\item \textbf{Platalpad généraliste :} rend toi tout simplement sur \urllink{http://platalpad/} pour créer un nouveau document. Ensuite, il suffit de diffuser son adresse à toutes les personnes que tu veux voir prendre part à sa réalisarion. \\

\item \textbf{Platalpad privé :} Chaque binet se voit automatiquement, une fois inscrit sur \fkz, attribué un espace privé, accessible seulement à ses membres. Tu peux y accéder à partir de la page \fkz du binet en question ou directement sur \urllink{http://nom-du-binet.platalpad/}. Une fois identifié, tu peux voir chacun des documents en cours de réalisation au sein du binet et les modifier à ta guise.

\end{itemize}

\imagepos{images/platalpad}{0.8}{Un document en cours d'édition sur un platalpad privé}{!h}


