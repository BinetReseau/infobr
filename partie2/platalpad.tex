\subsection{Un outil de travail collaboratif : Platalpad}
\label{platalpad}

Platalpad est un \'editeur de texte collaboratif par navigateur, un peu comme un google doc. Il permet de travailler \`a plusieurs sur un m\^eme document, chacun voyant les modification des autres. Il est donc tr\`es pratique pour s'organiser.\\
Il se d\'ecline en deux versions :
\begin{itemize}

\item \textbf{Platalpad g\'en\'eraliste :} rends-toi tout simplement sur \urllink{http://platalpad/} pour cr\'eer un nouveau document. Ensuite, il suffit de diffuser son adresse \`a toutes les personnes que tu veux voir prendre part \`a sa r\'ealisation. \\

\item \textbf{Platalpad priv\'e :} Chaque binet se voit automatiquement, une fois inscrit sur \fkz, attribu\'e un espace priv\'e, accessible seulement \`a ses membres. Tu peux y acc\'eder \`a partir de la page \fkz du binet en question ou directement sur \urllink{http://nom-du-binet.platalpad/}. Une fois identifi\'e, tu peux voir chacun des documents en cours de r\'ealisation au sein du binet et les modifier \`a ta guise.

On s'y connecte en utilisant ses identifiants \fkz. Ce service fonctionne aussi \`a l'ext\'erieur via : 
\urllink{https://www.polytechnique.fr/eleves/platalpad/nom\_du\_binet} ;

\end{itemize}

\imagepos{images/platalpad}{0.7}{Un document en cours d'\'edition sur un platalpad priv\'e}{!h}