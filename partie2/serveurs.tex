\subsection{Descriptions des différents serveurs}
{\bf Serveurs du BR :} Voici la liste des serveurs du BR que tu vas
utiliser le plus durant tes deux années sur le plateau, ainsi que
leurs IPs et les services qu'ils hébergent. Note que ces services
peuvent à tout moment migrer d'une machine à une autre en cas de
besoin.


\begin{description}
        \item[frankiz] (\server{129.104.201.51}) : DNS secondaire,
        news, portail des élèves, sites des binets
        \item[gwennoz] (\server{129.104.201.52}) : DNS secondaire,
        développement, miroirs FTP
        \item[heol] (\server{129.104.201.53}) : DNS principale,
        serveur xNet (pour \app{qRezix}, voir page~\pageref{qrezix}), serveur IRC
        \item[skinwel] (\server{129.104.201.54}) : DNS secondaire,
        dépôt SVN, télévision
	\item[pellwel] (\server{129.104.201.57}) : télévision
    \item[enez] (\server{129.104.201.61}) : Domaine windows, MSDNAA (logiciels Microsoft gratuits)
\end {description}

{\bf Serveurs de la DSI : }Etant donné que le réseau élèves est un
sous-réseau de celui de la DSI, nous utilisons également les
serveurs de celle-ci et les services qu'ils hébergent.

\begin{description}
        \item[kuzh] (\server{129.104.247.2}) : proxy http (pour l'Internet), proxy ssh pour sortir du plateau, futur proxy ftp
        \item[sil] (\server{129.104.247.3}) : ancien proxy ssh double sens et proxy ftp en sursit, il est conservé tant qu'il survit, et ne sera pas réparé s'il a des problèmes.
        \item[poly] (\server{129.104.247.5}) : mails (réception et envoi), tu trouveras en t'y connectant en http (\urllink{http://poly/}) le certificat de sécurité pour l'authentification sécurisée.
        \item[moned] : serveur d'authentification, permettant de
        changer ton mot de passe \server{moned}. Ce mot de passe est celui qui
        te permet de te connecter et d'utiliser n'importe
        quelle machine de salle info. Ton travail n'étant pas stocké
        en local, il t'est donc accessible, quel que soit le PC des salles info depuis
        lequel tu te connectes.
    \item[milou] : serveur ntp (\server{ntp.polytechnique.fr})
\end {description}

\fbox{
\begin{minipage}{0.9\textwidth}
  \bf ATTENTION : Les serveurs de la DSI sont à ta disposition pour
  des usages bien précis, et ne servent pas de serveurs de
  stockage. La DSI est assez vigilante,
  et elle a pour habitude de sanctionner les abus; cela peut inclure la perte de tes comptes
  \server{poly}, \server{moned} ou \server{sil}.
\end{minipage}
}
