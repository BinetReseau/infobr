\subsection{Descriptions des diff�rents serveurs}
{\bf Serveurs du BR :} Voici la liste des serveurs du BR que tu vas
utiliser le plus durant tes deux ann�es sur le plateau, ainsi que
leurs IPs et les services qu'ils h�bergent. Note que ces services
peuvent � tout moment migrer d'une machine � une autre en cas de
besoin.


\begin{description}
        \item[frankiz] (\server{129.104.201.51}) : DNS secondaire,
        news, portail des �l�ves, sites des binets
        \item[gwennoz] (\server{129.104.201.52}) : DNS secondaire,
        d�veloppement, miroirs FTP
        \item[heol] (\server{129.104.201.53}) : DNS principale,
        serveur xNet (pour \app{qRezix}, voir page~\pageref{qrezix}), serveur IRC
        \item[skinwel] (\server{129.104.201.54}) : DNS secondaire,
        d�p�t SVN, t�l�vision
    \item[enez] (\server{129.104.201.61}) : Domaine windows, MSDNAA (logiciels Microsoft gratuits)
\end {description}

{\bf Serveurs de la DSI : }Etant donn� que le r�seau �l�ves est un
sous-r�seau de celui de la DSI, nous utilisons �galement les
serveurs de celle-ci et les services qu'ils h�bergent.

\begin{description}
        \item[kuzh] (\server{129.104.247.2}) : proxy http (pour l'Internet)
        \item[sil] (\server{129.104.247.3}) : proxy ftp, acc�s ssh
        vers et depuis l'ext�rieur
        \item[poly] (\server{129.104.247.5}) : mails (r�ception et envoi), tu trouveras en t'y connectant en http (\url{http://poly/}) le certificat de s�curit� pour l'authentification s�curis�e.
        \item[moned] : serveur d'authentification, permettant de
        changer ton mot de passe \server{moned}. Ce mot de passe est celui qui
        te permet de te connecter et d'utiliser n'importe
        quelle machine de salle info. Ton travail n'�tant pas stock�
        en local, il t'est donc accessible, quelque soit le PC des salles info depuis
        lequel tu te connectes.
    \item[milou] : serveur ntp (\server{ntp.polytechnique.fr})
\end {description}

\fbox{
\begin{minipage}{0.9\textwidth}
  \bf ATTENTION : Les serveurs de la DSI sont � ta disposition pour
  des usages bien pr�cis, et ne servent pas de serveurs de
  stockage. Seul \server{sil} est pr�vu pour du transfert de fichiers. La DSI est assez vigilante,
  et elle a pour habitude de sanctionner les abus; cela peut inclure la perte de tes comptes
  \server{poly}, \server{moned} ou \server{sil}.
\end{minipage}
}
