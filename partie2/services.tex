\subsection{Services du BR}
\subsubsection{Miroirs}
Le BR facilite pour les utilisateurs de Macs la possibilit\'e d'utiliser les logiciels faits pour le monde linux, la suite KDE comme le logiciel scilab ou subversion pour les projets communs de code. Pour cela, nous mettons \`a disposition des miroirs qui se trouve derri\`ere le firewall de l'\'ecole, ce qui permet ais\'ement et tr\`es rapidement de r\'ecup\'erer les paquets. Le BR propose les miroirs suivants:
\begin{itemize}
\item Cygwin (Windows)
\item Debian
\item Fink (MacOS)
\item Gentoo
\item Knoppix
\item Mandriva
\item Ubuntu
\end{itemize}

La configuration, sp\'ecifique \`a chaque distribution et mise \`a jour r\'eguli\`erement est expliqu\'ee sur le wiki du binet r\'eseau : 
http://gwennoz/wiki/Miroir\_Fink

<<<<<<< .mine
%\subsubsection{FedeRez}
%FedeRez est un projet \'etudiant qui vise \`a regrouper des associations de
%grandes \'ecoles et d'universit\'es d\'evolues \`a l'informatique, aux r\'eseaux
%et aux t\'el\'ecommunications. FedeRez compte ainsi une quinzaine
%d'associations \'etudiantes, nombre de celles-ci g\`erent les r\'eseaux des
%campus de leur \'ecole. L'objectif de FedeRez est l'entraide, le partage
%d'exp\'erience et de connaissances, ainsi que le d\'eveloppement de projets
%communs.
%Chaque ann\'ee, l'association facilite l'organisation de commandes group\'ees pour les associations la %composant, et, depuis deux ans, organise une journ\'ee de rencontres et de conf\'erences sur des th\'ematiques informatiques, ToIP (t\'el\'ephonie sur IP) ou THD (tr\`es haut d\'ebit) par exemple.

=======
>>>>>>> .r195
