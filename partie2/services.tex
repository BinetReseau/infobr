\subsection{Services du BR}
\subsubsection{Commandes group\'ees}
Le Binet R\'eseau organise tous les ans une commande group\'ee d'ordinateurs \`a la rentr\'ee
pr\'ec\'edant le tronc commun, d\'ebut Mai environ. Cette commande est ouverte \`a tous les \'el\`eves,
et les r\'eductions, qui ne sont pas n\'ecessairement meilleures que les tarifs \'etudiants
classiques, sont compens\'ees par le fait que les ordinateurs soient livr\'es sur le plateau.
Le Binet organise parfois, lorsqu'il y a une demande importante, des commandes group\'ees
de mat\'eriel informatique \`a d'autres moments de l'ann\'ee : d\'ebut 2006, il y ainsi eu une
commande d'onduleurs d'organis\'ee, parce que le courant dans les nouveaux kaserts \'etait
peu fiable. Ce sont cependant des \'ev\'enements ponctuels car lourds \`a organiser.
Renseignez-vous sur les brs (news) !

\subsubsection{Polytechnique.org}
Pour pr\'esenter l'association polytechnique.org, rien de mieux que de citer leur site web :
  Nous avons cr\'e\'e une association afin de promouvoir l'image des Polytechniciens sur Internet.
  Il est important de noter que nous n'avons aucun mandat sp\'ecial pour le faire, ni non plus
  de contre-indication d'ailleurs. Cependant, les vues officielles de l'\'ecole peuvent \^etre
  trouv\'ees sur www.polytechnique.fr et www.polytechnique.edu. Le domaine polytechnique.org sert
  exclusivement \`a parler des X, \'el\`eves et anciens \'el\`eves, sur Internet par Internet.

  L'autre but est d'offrir le maximum de services de communication par Internet aux inscrits
  volontaires \`a notre site. Il s'agit l\`a de favoriser la vie des promotions, des associations
  polytechniciennes (groupes X, binets, ...) et de la communaut\'e en g\'en\'eral.

L'association propose de nombreux services aux X, qu'ils soient ou non sur le plateau.
En g\'en\'eral, ils sont peu connus, et pourtant souvent tr\`es utiles. Inscrivez-vous, et n'h\'esitez
pas \`a vous y connecter pour exploiter ces services, dont voici les principaux :
  des redirections mails nombreuses (adresses suppl\'ementaires)
  des services de news comme le binet r\'eseau, mais ouverts aux anciens, et aux non plat\^aliens
  des contacts ais\'es vers les anciens, les camarades de promotion
  une newsletter, pour publier des informations de groupes X, des informations qui toucheront tous les polytechniciens
  des annonces d'\'ev\'enements
  des services d'h\'ebergement pour les groupes et binets, des noms de domaine
  des listes de diffusion de mails (br2004@polytechnique.org, par exemple)

\subsubsection{miroirs}
Le BR facilite pour les utilisateurs de macs la possibilit\'e d'utiliser les logiciels faits pour le monde linux, la suite KDE comme le logiciel scilab ou subversion pour les projets communs de code. Pour cela, nous mettons \`a disposition des miroirs qui se trouve derri\`ere le firewall de l'\'ecole, ce qui permet ais\'ement et tr\`es rapidement de r\'ecup\'erer les paquets. Le BR propose les miroirs suivants:
	- Cygwin (Windows)
	- Debian
	- Fink (MacOS)
	- Gentoo
	- Knoppix
	- Mandriva
	- Ubuntu
	
La configuration, sp\'ecifique \`a chaque distribution et mise \`a jour r\'eguli\`erement est expliqu\'ee sur le wiki du binet r\'eseau : 
http://gwennoz/wiki/Miroir\_Fink

\subsubsection{FedeRez}
FedeRez est un projet \'etudiant qui vise \`a regrouper des associations de
grandes \'ecoles et d'universit\'es d\'evolues \`a l'informatique, aux r\'eseaux
et aux t\'el\'ecommunications. FedeRez compte ainsi une quinzaine
d'associations \'etudiantes, nombre de celles-ci g\`erent les r\'eseaux des
campus de leur \'ecole. L'objectif de FedeRez est l'entraide, le partage
d'exp\'erience et de connaissances, ainsi que le d\'eveloppement de projets
communs.
Chaque ann\'ee, l'association facilite l'organisation de commandes group\'ees pour les associations la composant, et, depuis deux ans, organise une journ\'ee de rencontres et de conf\'erences sur des th\'ematiques informatiques, ToIP (t\'el\'ephonie sur IP) ou THD (tr\`es haut d\'ebit) par exemple.

