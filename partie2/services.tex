\subsection{Services du BR}
\subsubsection{Miroirs}
Le BR facilite pour les utilisateurs de Macs la possibilit\'e d'utiliser les logiciels faits pour le monde linux, la suite KDE comme le logiciel scilab ou subversion pour les projets communs de code. Pour cela, nous mettons \`a disposition des miroirs qui se trouve derri\`ere le firewall de l'\'ecole, ce qui permet ais\'ement et tr\`es rapidement de r\'ecup\'erer les paquets. Le BR propose les miroirs suivants:
\begin{itemize}
\item Cygwin (Windows)
\item Debian
\item Fink (MacOS)
\item Gentoo
\item Knoppix
\item Mandriva
\item Ubuntu
\end{itemize}

La configuration, sp\'ecifique \`a chaque distribution et mise \`a jour r\'eguli\`erement est expliqu\'ee sur le wiki du binet r\'eseau : 
http://gwennoz/wiki/Miroir\_Fink
