\subsection{Par quel moyen communiquer~?}

Le BR propose plusieurs services aux binets~:
\begin{itemize}
\item une adresse \emph{mail} en \mail{nom\_du\_binet@binets.polytechnique.fr} qui permet de contacter les administrateurs \fkz\ du binet~;
\item l'hébergement d'un site sur \fkz\~;
\item le référencement des membres grâce au TOL~;
\item des annonces (une seule visible par binet) sur \fkz\ quand tu veux faire de la pub~;
\item Platalpad (\urllink{http://platalpad}) qui te permet de partager des documents textes en ligne~;
\item les Platalpads de binet (\urllink{http://nom\_du\_binet.platalpad}), accessibles aux membres du groupe \fkz\
(connexion en utilisant les login/mdp \fkz). Fonctionne aussi \`a l'ext\'erieur via
\urllink{https://www.polytechnique.fr/eleves/platalpad/nom\_du\_binet}.
\end{itemize}

Pour disposer de ces services, il te suffit de remplir les fiches qui sont dans la case du BR à la Kès. Il faut à la création et à la passation du binet
donner au BR les noms du prez et du webmestre de ton binet ainsi que, si tu désires que ton site soit accessible à l'extérieur, une fiche pour la DSI et nous. Les modalités d'utilisation de ces différents moyens de communication sont décrites ci-après.

\subsubsection{Des moyens de communication divers et nombreux}

\textbf{Les \emph{newsgroups}~:} cf. page \pageref{newsgroups}.

\textbf{Frankiz~:} cf. page \pageref{frankiz}~; cela inclut les annonces, les activités et les sondages.

\textbf{Le mail promo~:} il est possible d'envoyer un mail à toute
ta promotion, voire aux deux promotions. Cette procédure doit
cependant rester exceptionnelle, sinon les gens ne lisent plus les
mails promo parce qu'ils pensent que cela ne les concerne pas. Par
conséquent~:
\begin{itemize}
  \item ton seul interlocuteur valable est la Kès; tu ne dois pas demander à ton commandant de promotion
        ou toute autre personne de transmettre ton mail;
  \item ton mail doit vraiment concerner la promo (ou en tout cas une forte majorité);
  \item ton mail doit être suffisamment important pour que les autres moyens de diffusion (\fkz\ et forums) ne soient pas des moyens suffisants.
\end{itemize}
C'est la Kès qui valide ou non les mails promo. Le BR n'ayant qu'un rôle strictement technique, toute réclamation est à transmettre à la Kès. Il est
impératif de les soumettre par l'interface prévue pour cela sur \fkz.

  Le mail c'est plus efficace avec des listes de diffusion~; tu peux en créer dans la rubrique \lien{Listes de diffusion} de \urllink{www.polytechnique.org}.

\textbf{qRezix, IRC} ou tout autre messagerie instantanée.

\subsubsection{Quel moyen pour quel besoin~?}
\emph{Ton binet organise une activité sur le plateau} (ex~: Spectacle,
Soirées, Dégustations, \dots): pour annoncer l'activité sur frankiz, utilise
le lien \lien{Proposer une activité} sur la page \lien{Activit\'es}. Si tu veux donner plus de
précisions sur cette activité, crée une page web adaptée sur le site
de ton binet et fais pointer le lien de ton activité vers cette
page. N'oublie pas non plus de pr\'eciser \`a qui est destin\'ee cette activit\'e (tout le monde, membres du binet, simplement toi pour l'afficher sur ton emploi du temps, \dots).

\emph{Ton binet organise une activité exceptionnelle} (ex~: sortie le week-end, invitation à l'extérieur de l'X, ou plus gros: JSP ou Dez)
\emph{ou veut informer les élèves sur son activité} (Ex~: recruter des élèves, diffuser une liste de spectacle)~: pour avoir une annonce sur \fkz, utilise le lien \lien{Proposer une annonce} sur la page correspondante.

\emph{Attention~:} si l'annonce n'est pas justifiée et qu'une activité aurait suffi, elle ne sera pas acceptée, ce qui fait perdre du temps à tout le
monde. Si l'annonce que tu veux écrire est manifestement trop longue --- et risque donc grandement d'être refusée --- crée une page web adaptée sur le
site de ton binet et fais pointer un lien dans ton annonce vers cette page.

Pour continuer à communiquer avec les personnes intéressées, utilise
le forum de ton binet (\ngname{br.binet.ton\_binet}) et les \emph{mails}
(en créant éventuellement une liste de diffusion).

Si ton annonce ne concerne pas toute les élèves, par
exemple pour le prochain weekend spéleo, règle correctement la
visibilité de l'annonce~: les sympatisants de ton binet seront mieux
avertis et ceux qui ne sont pas intéressés ne verront pas ton message.
