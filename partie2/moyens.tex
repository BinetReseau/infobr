\subsection{Par quel moyen communiquer ?}
\subsubsection{De nombreux moyens diff\'erents}

\textbf{Les newsgroups :} cf. page \pageref{newsgroups}.

\textbf{Frankiz :} cf. page \pageref{frankiz}; cela inclut les annonces, les activit�s, et les sondages.

\textbf{Le mail promo :} il est possible d'envoyer un mail \`a toute ta promotion, voire aux deux promotions.
Cette proc\'edure doit cependant rester exceptionnelle, sinon les gens ne lisent plus les mails promo parce qu'ils pensent que
cela ne les concerne pas. Par cons\'equent :
\begin{itemize}
  \item Ton seul interlocuteur valable est la K\`es, tu ne dois pas demander \`a ton commandant de promotion ou toute autre personne
    de transmettre ton mail.
  \item Ton mail doit vraiment concerner la promo (ou en tout cas une forte majorit\'e).
  \item Ton mail doit \^etre suffisamment important pour que les autres moyens de diffusion (frankiz et forums) ne soient pas des moyens suffisants.
\end{itemize}
C'est la K\`es qui valide ou non les mails promo. Le BR n'ayant qu'un r\^ole strictement technique, toute r\'eclamation est \`a transmettre \`a
la K\`es. Il est imp\'eratif de les soumettre par l'interface pr\'evue pour cela sur frankiz.

  Le mail c'est plus efficace avec des listes de diffusion ; tu peux en cr\'eer dans la rubique \lien{Listes de diffusion} de \url{www.polytechnique.org}.

\textbf{qRezix, IRC} ou tout autre messagerie instantan�e.

\subsubsection{Quel moyen pour quel besoin ?}
\emph{Ton binet organise une activit\'e sur le plateau} (Ex : BRC, Binet rock, soir\'ees):
pour annoncer l'activit\'e sur frankiz, utilise le lien \lien{Proposer une activit\'e}.
Si tu veux donner plus de pr\'ecisions sur cette activit\'e, cr\'ee une page web adapt\'ee sur le site de
ton binet et fais pointer le lien de ton activit\'e vers cette page.

\emph{Ton binet organise une activit\'e exceptionnelle} (Ex : sortie le week-end, invitation \`a l'ext\'erieur de l'X, ou plus gros, genre JSP ou Dez)
ou \emph{veut informer les \'el\`eves sur son activit\'e}	 (Ex : recruter des \'el\`eves, diffuser une liste de spectacle) : pour avoir une annonce sur frankiz,
utilise le lien \lien{Proposer une annonce}.

\emph{Attention:} si l'annonce n'est pas justifi\'ee et qu'une activit\'e aurait suffi, elle ne sera pas accept\'ee, ce qui fait perdre
du temps \`a tout le monde.
Si l'annonce que tu veux \'ecrire est manifestement trop longue - et risque donc grandement d'\^etre refus\'ee, cr\'ee une page web adapt\'ee sur
le site de ton binet et fais pointer un lien dans ton annonce vers cette page.

Si tu veux traiter deux sujets (deux activit\'es) pour un m\^eme binet, synth\'etise tes annonces en une seule, en utilisant \'eventuellement
le site web de ton binet.

Pour continuer \`a communiquer avec les personnes int\'eress\'ees, utilise le forum de ton binet et les mails (en cr\'eant \'eventuellement
une liste de diffusion).

