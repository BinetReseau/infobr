%$Id: qrezix.tex 144 2005-03-25 01:11:37Z myk $

\subsection{\app{qRezix} : connecte-toi}
\label{qrezix}

\app{qRezix} est un programme développé par le BR pour simplifier la vie sur le réseau.
Il permet en particulier de fixer le nom que ta machine portera sur le réseau (nom DNS),
qui permettra aux autres de se connecter à ton serveur FTP (et HTTP, news, etc.).

\imagepos{images/qrezix.png}{0.75}{qRezix}{ht}

\subsubsection{Installation de \app{qRezix}}

\app{qRezix} est disponible sur \xshare. L'installation se fait en général simplement, quel que soit ton
OS~\footnote{Sous Linux tu pourrais avoir à compiler le programme à la main ;
en cas de problème, n'hésite pas à demander de l'aide aux développeurs.}.

Pour que le \emph{chat} et le transfert de fichiers fonctionnent, n'oublie pas d'ouvrir les ports 5050, 5053 et 5055 en TCP de ton \emph{firewall} si
tu en as un.

\app{qRezix} évolue régulièrement. N'hésite donc pas à installer les dernières mises à jour et les nouveaux \emph{plug-ins}, qui sont en général
annoncés sur les \emph{newsgroups}.

\subsubsection{Que fait \app{qRezix} ?}

\app{qRezix} te permet de connaître la liste de toutes les personnes connectées, de voir où elles habitent, de définir des favoris et des groupes, de
\emph{chatter} avec tes amis avec une interface conviviale. Il y a plein d'options marrantes, alors n'hésite pas à explorer les \menu{Préférences}.

\app{qRezix} te permet également de connaître l'état des serveurs des personnes connectées\ldots\
ainsi d'un simple coup d'\oe il tu sais si une personne a un serveur FTP,
un site web ou un serveur de news, tu sais quelle est sa promo, etc.

En bref : \app{qRezix} c'est \emph{le} logiciel indispensable sur le réseau !

\subsubsection{En savoir plus}

La gestion du domaine \urllink{.eleves.polytechnique.fr} est faite par le BR grâce à \app{xNet},
dont \app{qRezix} est le principal client. C'est donc grâce à \app{qRezix} qu'on peut facilement
donner un nom à chaque ordinateur, en réalité à son adresse IP.

Si tu te sens l'âme d'un développeur et qu'un travail multi-plateforme en Q{}t t'intéresse, n'hésite pas à prendre contact avec les développeurs (par
\emph{mail} : \mail{qrezix@frankiz}, ou sur \ngname{br.binet.br.devel}) pour participer ! Si tu veux créer un thème d'icône, ou une traduction de
\app{qRezix}, \emph{idem}, n'hésite pas à nous contacter. On n'attend que ça ;-) !
