\subsection{Partager tes fichiers sur le réseau avec FTP}
\bghdr{images/fond-infobr}

Les échanges de fichiers sur le réseau élèves se font souvent par FTP. Rien de plus simple que de partager toi aussi tes précieuses données en installant un serveur FTP~!

\paragraph{Client FTP}
Pour une utilisation basique, taper \urllink{ftp://nom-du-ftp}  (par exemple \urllink{ftp://jtx} dans la barre d'adresse de ton navigateur suffit à parcourir les fichiers proposés par \og Gentil Vieux-Chouffe \fg.
Pour une meilleure utilisation, le BR te conseille \app{FileZilla}. Télécharge-le sur \urllink{http://www.filezilla.fr} et double-clique sur l'installeur.
Tu pourras dès la fin de l'installation aller sur tous les FTP du réseau facilement et rapidement.\\
\flimage{images/mac_cyberduck_icone}{0.07}{l} \app{Cyberduck} est un autre client FTP très simple à  utiliser et performant. Il te permettra d'aller télécharger des fichiers sur les serveurs FTP des autres élèves sans problème.\\
Pour se connecter à  un serveur, il suffit de taper son nom (exemple~: \urllink{jtx}) dans le cadre \menu{Connexion rapide} puis d'appuyer sur Entrée.\\


\paragraph{Serveur FTP}
Tu verras rapidement que tout le monde à  l'X possède un serveur FTP
afin de partager les différents projets, les films du JTX, ses
photos, etc. Il est donc quasiment indispensable que tu en installes un.\\

Parmi les plus simples on trouve \app{FileZilla Server} et \app{GuildFTP}, qui sont libres de surcroît.
Si tu es sous mac, tu peux aussi jeter un œ{}il à \app{PureFTPd Manager}, qui est très pratique à utiliser.\\
Quoiqu'il en soit, tu trouveras toutes les informations nécessaires à la configuration de ton serveur FTP sur \urllink{http://wikix.polytechnique.org/FTP}.\\
Pense à lui donner un nom sur \urllink{http://dnsapp/} (voir en dessous).
