\subsection{FTP : partage de fichiers sur le r\'eseau}

Les \'echanges de fichiers sur le r\'eseau \'el\`eves se font souvent par FTP. Rien de plus simple que de partager toi aussi tes pr\'ecieuses donn\'ees en installant un client FTP !

\paragraph{Client FTP}
Pour une utilisation basique, taper \urllink{ftp://nom-du-ftp}  (par exemple \urllink{ftp://jtx} dans la barre d'adresse de ton navigateur suffit \`a parcourir les fichiers propos\'es par \og Gentil Vieux-Chouffe \fg.
Pour une meilleure utilisation, le BR te conseille \app{FileZilla}. T\'el\'echarge-le sur \urllink{http://www.filezilla.fr} et double-clique sur l'installeur.
Tu pourras d\`es la fin de l'installation aller sur tous les FTP du r\'eseau
facilement et rapidement.

\paragraph{Serveur FTP}
Tu verras rapidement que tout le monde \`a  l'X poss\`ede un serveur FTP
afin de partager les diff\'erents projets, les films du JTX, ses
photos, etc. Il est donc quasiment indispensable que tu en installes
un.

Parmi les plus simples on trouve \app{FileZilla Server} et \app{GuildFTP}, qui sont libres de surcro\^{i}t. Pour les d\'etails de l'installation, rendez-vous sur le WikiX : \urllink{http://wikix/FTP}.
Pense \`a lui donner un nom sur \urllink{http://dnsapp/} (voir aussi page \pageref{dnsapp}.)

\flimage{images/mac_cyberduck_icone}{0.07}{l} \app{Cyberduck} : un client FTP tr\`es simple \`a  utiliser mais performant. Il te permettra d'aller t\'el\'echarger des fichiers sur les serveurs FTP des autres \'el\`eves.
 %Il existe aussi \app{Transmit} (partagiciel), ou encore \app{Fugu}, que certains pr\'ef\`erent.
Pour se connecter \`a  un serveur, il suffit de taper son nom (exemple : \urllink{jtx}) dans le cadre \menu{Connexion rapide} puis d'appuyer sur Entr\'ee.\\
Tu verras rapidement que tout le monde \`a  l'X poss\`ede un serveur FTP afin de partager les diff\'erents projets, les films du JTX, ses photos, etc. Donc il est quasiment indispensable que tu en installes un. Nous te conseillons \app{PureFTPd Manager}, qui dispose d'une interface tr\`es facile \`a  utiliser (m\^eme pour un d\'ebutant) et en m\^eme temps de fonctionnalit\'es avanc\'ees tr\`es puissantes. Tu trouveras le d\'etail de sa configuration sur le WikiX.


