\subsection{FTP : partage de fichiers sur le r\'eseau}

\paragraph{Client FTP}
Le BR te conseille \app{FileZilla} ou \app{SmartFTP}. Pour installer l'un des deux, t\'el\'echarge-le sur \urllink{http://www.filezilla.fr} et double-clique sur l'installeur.
Tu pourras d\`es la fin de l'installation aller sur tous les FTP du r\'eseau
facilement et rapidement.

\paragraph{Serveur FTP}
Tu verras rapidement que tout le monde \`a  l'X poss\`ede un serveur FTP
afin de partager les diff\'erents projets, les films du JTX, ses
photos, etc. Il est donc quasiment indispensable que tu en installes
un.

Parmi les plus simples on trouve \app{FileZilla Server} et \app{GuildFTP}, qui sont libres de surcro\^{i}t. Expliquer les d\'etails de la configuration est un peu long pour l'InfoBR, mais tu trouveras sur le WikiX un tutoriel expliquant cela : \urllink{http://wikix.polytechnique.org/FTP}.

\flimage{images/mac_cyberduck_icone}{0.07}{l} \app{Cyberduck} : un client FTP tr\`es simple \`a  utiliser mais performant. Il te permettra d'aller t\'el\'echarger des fichiers sur les serveurs FTP des autres \'el\`eves.
 %Il existe aussi \app{Transmit} (partagiciel), ou encore \app{Fugu}, que certains pr\'ef\`erent.
Pour se connecter \`a  un serveur, il suffit de taper son nom (exemple : \urllink{jtx}) dans le cadre \menu{Connexion rapide} puis d'appuyer sur Entr\'ee.\\
Tu verras rapidement que tout le monde \`a  l'X poss\`ede un serveur FTP afin de partager les diff\'erents projets, les films du JTX, ses photos, etc. Donc il est quasiment indispensable que tu en installes un. Nous te conseillons \app{PureFTPd Manager}, qui dispose d'une interface tr\`es facile \`a  utiliser (m\^eme pour un d\'ebutant) et en m\^eme temps de fonctionnalit\'es avanc\'ees tr\`es puissantes. Tu trouveras le d\'etail de sa configuration sur le WikiX.


