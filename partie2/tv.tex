%$Id: qrezix.tex 144 2005-03-25 01:11:37Z myk $

\subsection{La télévision du BR}
\label{TV}

Le BR diffuse sur le réseau plusieurs dizaines de chaines de télévision et radios. Pour les recevoir, nous recommandons \app{vlc}, disponible sur le X-Share.

\subsubsection{Configuration de vlc}

La liste des cha\^ines est diffusée sous forme d'annonces SAP. Pour voir ces annonces, ouvre ta liste de lecture (\menu{Vue}, puis \menu{Liste de lecture}), et active la découverte de services (\ref{vlc:config}).
Attention sous \app{Windows Vista} un probl\`eme de compatibilité connu entra\^ine un écran noir. Pour le résoudre le BR t'a préparé une page sur le Wikix.

\imagepos{images/vlc_config_sap.png}{0.75}{Configuration de vlc pour la télévision par le réseau}{h!}\label{vlc:config}

Tu auras ainsi dans ta liste de lecture les différents cha\^{i}nes disponibles.

\subsubsection{Autre méthode}

Si ton client préféré ne supporte pas les annonces SAP, ou que les annonces SAP ne marchent pas chez toi, tu peux récupérer la liste des cha\^ines par
PodCast, à l'adresse \urllink{http://tv.eleves.\linebreak{}polytechnique.fr/tvbr.xml}. Sous \app{vlc}, active la découverte des services PodCast dans la liste de
lecture (\menu{Gérer}, \menu{Découverte de services}, \menu{Podcast}), puis va dans \menu{Param\`etres}, \menu{Préférences}, \menu{Liste de Lecture}, \menu{Découverte de services} et enfin \menu{Podcast} et
met l'adresse \url{http://tv.eleves.polytechnique.fr/tvbr.xml} dans le champ \guillemotleft~Liste des URLs~\guillemotright .

\subsubsection{Et si ça ne marche toujours pas?}

Vérifie que tu utilises bien la derni\`ere version de \app{vlc}. Les versions inférieures à 0.8.5 sont connues pour ne pas fonctionner.

Si rien ne marche, la raison la plus probable est un \emph{firewall} qui intercepte les flux télés. Configure ton \emph{firewall} afin d'autoriser
ces flux. Sous Linux, les r\`egles \emph{iptables} suivantes suffisent:

\cmdline{ -A INPUT -i eth0 -d 224.0.0.0/24 -j ACCEPT \\
   -A INPUT -i eth0 -d 239.255.42.0/24 -s 192.168.225.0/24 -p udp -m udp --dport 1234 -j ACCEPT\\
    -A INPUT -i eth0 -d 239.255.255.255/32 -p udp -m udp --dport 9875 -j ACCEPT\\
   -A OUTPUT -o eth0 -d 224.0.0.0/4 -j ACCEPT.}