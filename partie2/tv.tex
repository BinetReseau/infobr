%$Id: qrezix.tex 144 2005-03-25 01:11:37Z myk $

\subsection{La t\'el\'evision du BR}
\label{TV}

Disponible pour les roujes, infos sur \urllink{http://tv} ; un jour peut-\^etre pour les j\^ones \dots
%   \begin{quote}
%   Attention : suite \`a  des probl\`emes au niveau du mat\'eriel de la DSI, il est possible que tu constates des probl\`emes de r\'eception. Si tel est le cas, n'h\'esite pas \`a  envoyer un mail d\'ecrivant le probl\`eme (nature, heure, chaine, ton num\'ero de prise, ton batiment) \`a  \mail{tv@frankiz.polytechnique.fr}.
%   \end{quote}
%Le BR propose depuis 2006 un service de t\'el\'evision. Cependant, celui-ci conna\`a�t actuellement des probl\`emes, et il est possible que tu ne puisses pas acc\'eder aux cha\`a�nes propos\'ees. Pour conna\`a�tre la configuration qui te permettrait th\'eoriquement de recevoir les cha\`a�nes, rends-toi sur: \urllink{http://tv.eleves.polytechnique.fr/}, qui sera maintenue \`a  jour autant que possible.

%Le BR diffuse sur le r\'eseau plusieurs dizaines de chaines de t\'el\'evision et radios. Pour les recevoir, nous recommandons \app{vlc}, disponible sur le X-Share.

%\subsubsection{Configuration de vlc}

%La liste des cha\^ines est diffus\'ee sous forme d'annonces SAP. Pour voir ces annonces, ouvre ta liste de lecture (\menu{Vue}, puis \menu{Liste de lecture}), et active la d\'ecouverte de services. Attention sous \app{Windows Vista} un probl\`eme de compatibilit\'e connu entra\^ine un \'ecran noir. Pour le r\'esoudre le BR t'a pr\'epar\'e une page sur le WikiX.

%\imagepos{images/vlc_config_sap.png}{0.75}{Configuration de vlc pour la t\'el\'evision par le r\'eseau}{h!}

%Tu auras ainsi dans ta liste de lecture les diff\'erents cha\^{i}nes disponibles.

%\subsubsection{Autre m\'ethode}

%Si ton client pr\'ef\'er\'e ne supporte pas les annonces SAP ou que les annonces SAP ne marchent pas chez toi, il t'est aussi possible de r\'ecup\'erer la liste des cha\^ines par
%\emph{PodCast}, \`a  l'adresse \urllink{http://tv.eleves.polytechnique.fr/tvbr.xml}. Sous \app{vlc}, active la d\'ecouverte des services \emph{PodCast} dans la liste de
%lecture (\menu{G\'erer}, \menu{D\'ecouverte de services}, \menu{Podcast}), puis va dans \menu{Param\`etres}, \menu{Pr\'ef\'erences}, \menu{Liste de Lecture}, \menu{D\'ecouverte de services} et enfin \menu{Podcast} et
%mets l'adresse \urllink{http://tv.eleves.polytechnique.fr/tvbr.xml} dans le champ \guillemotleft~Liste des URLs~\guillemotright .

%\subsubsection{Et si \c ca ne marche toujours pas?}

%V\'erifie que tu utilises bien la derni\`ere version de \app{vlc}. Les versions inf\'erieures \`a  0.8.5 sont connues pour ne pas fonctionner.

%Si rien ne marche, la raison la plus probable est un \emph{firewall} qui intercepte les flux TV. Configure ton \emph{firewall} afin d'autoriser
%ces flux. Sous Linux, les r\`egles \emph{iptables} suivantes suffisent:

%\cmdline{ -A INPUT -i eth0 -d 224.0.0.0/24 -j ACCEPT \\
%   -A INPUT -i eth0 -d 239.255.42.0/24 -s 192.168.225.0/24 -p udp -m udp --dport 1234 -j ACCEPT\\
%    -A INPUT -i eth0 -d 239.255.255.255/32 -p udp -m udp --dport 9875 -j ACCEPT\\
%   -A OUTPUT -o eth0 -d 224.0.0.0/4 -j ACCEPT.}
   
