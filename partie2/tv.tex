%$Id: qrezix.tex 144 2005-03-25 01:11:37Z myk $

\subsection{La t\'el\'evision du BR}
\label{TV}

Le BR diffuse sur le r\'eseau plusieurs dizaines de chaines de t\'el\'evision et radios. Pour les recevoir, nous recommandons \app{vlc}, disponible sur le X-Share.

\subsubsection{Configuration de vlc}

La liste des cha\^ines est diffus\'ee sous forme d'annonces SAP. Pour voir ces annonces, ouvre ta liste de lecture (Vue -> Liste de lecture), et active la d\'ecouverte de services (\ref{vlc:config}).
Attention sous \app{Windows Vista} un probl\`eme de compatibilit\'e connu entra\^ine un \'ecran noir. Pour le r\'esoudre le BR t'a pr\'epar\'e une page sur le Wikix.

\imagepos{images/vlc_config_sap.png}{0.75}{Configuration de vlc pour la t\'el\'evision par le r\'eseau}{h!}\label{vlc:config}

Tu auras ainsi dans ta liste de lecture les diff\'erents cha\^{i}nes disponibles.

\subsubsection{Autre m\'ethode}

Si ton client pr\'ef\'er\'e ne supporte pas les annonces SAP, ou que les annonces SAP ne marchent pas chez toi, tu peux r\'ecup\'erer la liste des cha\^ines par
PodCast, � l'adresse \url{http://tv.eleves.polytechnique.fr/tvbr.xml}. Sous \app{vlc}, active la d\'ecouverte des services PodCast dans la liste de
lecture (G\'erer > D\'ecouverte de services > Podcast), puis va dans Param\`etres > Pr\'ef\'erences > Liste de Lecture > D\'ecouverte de services > Podcast et
met l'adresse \url{http://tv.eleves.polytechnique.fr/tvbr.xml} dans le champ \guillemotleft~Liste des URLs~\guillemotright .

\subsubsection{Et si �a ne marche toujours pas?}

V\'erifie que tu utilises bien la derni\`ere version de \app{vlc}. Les versions inf\'erieures � 0.8.5 sont connues pour ne pas fonctionner.

Si rien ne marche, la raison la plus probable est un \emph{firewall} qui intercepte les flux t\'el\'es. Configure ton \emph{firewall} afin d'autoriser
ces flux. Sous Linux, les r\`egles \emph{iptables} suivantes suffisent:

  \cmdline[0.85] {
   -A INPUT -i eth0 -d 224.0.0.0/24 -j ACCEPT
   -A INPUT -i eth0 -d 239.255.42.0/24 -s 192.168.225.0/24 -p udp -m udp --dport 1234 -j ACCEPT
   -A INPUT -i eth0 -d 239.255.255.255/32 -p udp -m udp --dport 9875 -j ACCEPT
   -A OUTPUT -o eth0 -d 224.0.0.0/4 -j ACCEPT.
  }
