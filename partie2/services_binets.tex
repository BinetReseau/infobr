\subsection{Services proposés aux binets}

Le BR propose plusieurs services aux binets :
\begin{itemize}
\item le référencement sur \fkz, avec tous les services associés (adresse mail, annonces, mail aux membres) : consulte l'infoFrankiz pour les détails ;
\item les Platalpads de binet (\urllink{http://nom\_du\_binet.platalpad.binets.fr}), accessibles aux membres du groupe \fkz\ (voir p. \pageref{platalpad}) : ce service est créé automatiquement lors de la création du binet sur \fkz ;
\item l'hébergement de leur site internet, qui peut être interne (visible uniquement depuis l'X) ou externe (visible de l'extérieur de l'X).\\
\end{itemize}

Pour disposer de ces services, tu dois d'abord déclarer ton binet à la Kès. Ensuite, pour la création du binet sur \fkz, envoie un mail à \mail{web@frankiz.net} en précisant le nom du binet, le nom du prez, le nom du respo com ou du respo web et que tu l'as bien déclaré à la Kès. Pour faire héberger ton site web, envoie un mail à \mail{root@eleves.polytechnique.fr}.

\vspace{4mm}
%Le BR propose également aux binets qui le souhaitent d'héberger leur site Internet. Ce site peut être interne (visible uniquement depuis l'X) ou externe (visible de l'extérieur de l'X). Si tu désires que ton site soit accessible à l'extérieur, il te faut remplir une fiche pour la DSI et nous.

Les sites binets disposent de PHP et MySQL, et bénéficient d'une capacité de stockage (extensible) de 100 Mo.

Si tu désires que ton site soit accessible à l'extérieur, il te faut remplir une fiche pour la DSI et nous. Les sites ayant une visibilité extérieure doivent satisfaire aux conditions suivantes :
\begin{itemize}
   % \item aucune information ne doit être diffusée qui pourrait nuire à l'image de l'é‰cole (photos, vidéos, etc.). En particulier le contenu doit respecter la loi franà§aise sur les droits d'auteur ;
    \item le site doit avoir une qualité visuelle, si ce n'est professionnelle, du moins très correcte ;
    \item le site ne doit pas héberger de vidéos ou diffuser un flux vidéo (streaming). Toutes les vidéos du sites doivent être hébergées à l'extérieur. (Dailymotion, YouTube, etc.) ;
    \item les images présentes sur le site doivent avoir une résolution suffisamment faible afin de ne pas saturer la bande passante vers l'extérieur. 
\end{itemize}

Le BR offre ce service gratuitement, en partie grâce à une subvention de la Kès.
Il se réserve le droit de refuser ou d'interrompre l'hébergement d'un site, sans préavis, sans recours possible et sans avoir à fournir de motif.
Il s'engage à en informer immédiatement le bureau du binet concerné.\\
Pour plus d'informations sur ce service, visite la page du wikiBR : \newline \urllink{https://br.binets.fr/H\%C3\%A9bergement\_des\_sites\_des\_binets}.
