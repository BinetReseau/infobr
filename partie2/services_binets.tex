\subsection{Services propos\'es aux binets}

Le BR propose plusieurs services aux binets :
\begin{itemize}
\item une adresse \emph{mail} en \mail{nom\_du\_binet@binets.polytechnique.fr} qui permet de contacter les administrateurs \fkz\ du binet ;
\item le référencement des membres grâce au TOL ;
\item des annonces (une seule visible par binet) sur \fkz\ quand le binet veut faire de la pub ;
\item les Platalpads de binet (\urllink{http://nom\_du\_binet.platalpad}), accessibles aux membres du groupe \fkz\ (voir p. \pageref{platalpad}).
On s'y connecte en utilisant ses identifiants \fkz. Ce service fonctionne aussi \`a l'ext\'erieur via : \newline
\urllink{https://www.polytechnique.fr/eleves/platalpad/nom\_du\_binet}.
\end{itemize}

Pour disposer de ces services, il te suffit de remplir les fiches qui sont dans la case du BR à la Kès. Il faut à la création et à la passation du binet
donner au BR les noms du prez et du webmestre de ton binet ainsi que, si tu désires que ton site soit accessible à l'extérieur, une fiche pour la DSI et nous.\\

\vspace{4mm}
Le BR propose également aux binets qui le souhaitent d'héberger leur site Internet. Ce site peut être interne (visible uniquement depuis l'X) ou externe (visible de l'extérieur de l'X).
Les sites binets disposent de PHP et MySQL, et bénéficient d'une capacité de stockage (extensible) de 100 Mo.

Les sites ayant une visibilité extérieure doivent satisfaire aux conditions suivantes :
\begin{itemize}
    \item Aucune information ne doit être diffusée qui pourrait nuire à l'image de l’École (photos, vidéos, etc.). En particulier le contenu doit respecter la loi française sur les droits d'auteur.
    \item Le site doit avoir une qualité visuelle, si ce n'est professionnelle, du moins très correcte.
    \item Le site ne doit pas héberger de vidéos ou diffuser un flux vidéo (streaming). Toutes les vidéos du sites doivent être hébergées à l'extérieur. (Dalymotion, YouTube, etc.)
    \item Les images présentes sur le site ne doivent avoir une résolution suffisamment faible afin de ne pas saturer la bande passante vers l'extérieur. 
\end{itemize}

Le BR offre ce service gratuitement, en partie grâce à une subvention de la Kès.
Il se réserve le droit de refuser ou d'interrompre l'hébergement d'un site, sans préavis, sans recours possible et sans avoir à fournir de motif.
Il s'engage à en informer immédiatement le bureau du binet concerné.\\
Pour plus d'informations sur ce service, visite \urllink{https://br.binets.fr/H\'ebergement}
