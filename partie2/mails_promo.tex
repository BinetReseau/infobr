\subsection{Les mails promos}

% J'ai � peine chang� l'ancienne version.

\textbf{Le mail promo :} il est possible d'envoyer un mail � toute
ta promotion, voire aux deux promotions, de pr�f�rence au nom d'un binet.
Cette proc�dure doit cependant rester exceptionnelle, sinon les gens ne lisent plus les
mails promo parce qu'ils pensent que cela ne les concerne pas. Par
cons�quent :
\begin{itemize}
  \item ton seul interlocuteur valable est la K�s. Il est strictement interdit, sous peine de sanctions douloureuses, d'envoyer un mail promo \og � la main \fg~en utilisant la liste de toutes les adresses e-mail,
  ou de faire envoyer ton mail par ton compromo ou une secr�taire.
  \item ton mail doit vraiment concerner la promo (ou en tout cas une forte majorit�);
  \item ton mail doit �tre suffisamment important pour que les autres moyens de diffusion (\fkz\ et forums) ne soient pas des moyens suffisants.
\end{itemize}
C'est la K�s qui valide ou non les mails promo. Le BR n'ayant qu'un r�le strictement technique, toute r�clamation est � transmettre � la K�s. Il est
imp�ratif de les soumettre par l'interface pr�vue pour cela sur \fkz.

Plut�t que d'entrer � la main les adresses e-mail de chaque membre de ton binet, pense � cr�er une liste de diffusion dans la rubrique \lien{Listes de diffusion} de \urllink{www.polytechnique.org}.
Elle ressemblera � \texttt{2011@mon_binet.polytechnique.org} ou \texttt{mon_binet@polytechnique.org}. V�rifie qu'elle n'existe pas d�j�...