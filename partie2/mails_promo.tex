\subsection{Les \emph{mails} promo}

% J'ai \`a peine chang\'e l'ancienne version.

\textbf{Le \emph{mail} promo :} il est possible d'envoyer un \emph{mail} \`a toute
ta promotion, voire aux deux promotions, de pr\'ef\'erence au nom d'un binet.
Cette proc\'edure doit cependant rester exceptionnelle, sinon les gens ne lisent plus les
mails promo parce qu'ils pensent que cela ne les concerne pas. Par
cons\'equent :
\begin{itemize}
  \item ton seul interlocuteur valable est la K\`es. Il est strictement interdit, sous peine de sanctions douloureuses, d'envoyer un \emph{mail} promo \og \`a la main \fg~en utilisant la liste de toutes les adresses \emph{e-mail},
  ou de faire envoyer ton \emph{mail} par ton compromo ou une secr\'etaire ;
  \item ton \emph{mail} doit vraiment concerner la promo (ou en tout cas une forte majorit\'e) ;
  \item ton \emph{mail} doit \^etre suffisamment important pour que les autres moyens de diffusion (\fkz\ et forums) ne soient pas des moyens suffisants.
\end{itemize}
C'est la K\`es qui valide ou non les \emph{mails} promo. Le BR n'ayant qu'un r\^ole strictement technique, toute r\'eclamation est \`a transmettre \`a la K\`es. Il est
imp\'eratif de les soumettre par l'interface pr\'evue pour cela sur \fkz.

Plut\^ot que d'entrer \`a la main les adresses \emph{e-mail} de chaque membre de ton binet, pense \`a cr\'eer une liste de diffusion dans la rubrique \lien{Listes de diffusion} de \urllink{www.polytechnique.org}.
Elle ressemblera \`a \url{2011@mon_binet.polytechnique.org} ou \url{mon_binet@polytechnique.org}. V\'erifie qu'elle n'existe pas d\'ej\`a...
