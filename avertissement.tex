\newskip\oldpar
\oldpar=\parskip
\advance\parskip by \baselineskip

\section*{Avertissement}

Tu tiens entre tes mains l'InfoBR, document pr\'ecieux qui te permettra de te connecter facilement
--- enfin, on l'esp\`ere --- au r\'eseau.
Nous te conseillons gentiment d'\'eviter de le paumer sous un meuble ; il pourra te resservir le jour o\`u ton ordi te cr\`evera mis\'erablement entre les
mains. Surtout si on te r\'epond que la solution se trouve \`a telle page :-/.

Si tu rencontres un probl\`eme, une proc\'edure typique \`a suivre pour le r\'esoudre est expliqu\'ee en quatri\`eme de couverture.
Bien s�r, si tu ne t'en sors pas, tu peux appeler un des membres du binet --- la liste est \`a l'int\'erieur.
Nous sommes l\`a pour te rendre service.

Cependant, essaye d'abord de bien tout re-v\'erifier avant d'appeler un BR-man, et si possible, adresse toi d'abord \`a quelqu'un de ton \'etage qui s'y
conna�t. D'abord \c{c}a va plus vite. Ensuite, le BR-man moyen, m\^eme s'il est de bonne volont\'e, trouve ca un peu abusif d'\^etre d\'erang\'e si ton r\'eseau ne
marche pas parce que tu as \'ecrit \texttt{polytechnqiue} au lieu de \texttt{polytechnique}.

Ceci dit, en avant pour la configuration !

\parskip=\oldpar
