\thispagestyle{empty}

\section*{Avertissement}

Tu tiens entre tes mains l'InfoBR, document pr\'ecieux qui te permettra de te connecter facilement
--- enfin, on l'esp\`ere --- au r\'eseau.
Nous te conseillons gentiment d'\'eviter d'en faire une liti\`ere pour ton lapin ; il pourra te resservir le jour o\`u ton ordi te cr\`evera mis\'erablement entre les
mains. Surtout si on te r\'epond que la solution se trouve \`a telle page :-/.

Si tu rencontres un probl\`eme, une proc\'edure typique \`a suivre pour le r\'esoudre est expliqu\'ee en quatri\`eme de couverture.
Bien sûr, si tu ne t'en sors pas, tu peux toujours contacter un des membres du binet --- la liste est \`a l'int\'erieur.
Nous sommes l\`a pour te rendre service.

Cependant, essaye d'abord de bien tout re-v\'erifier avant d'appeler un BR-man, et si possible, 
adresse toi d'abord \`a quelqu'un de ton \'etage qui s'y
connaît, ça te permettra de faire connaissance, de dire bonjour \`a tes voisins et 
de faire sortir un geek de sa chambre pour venir t'aider. 
Et m\^eme s'il est de bonne volont\'e, le BR-man moyen trouve ça un peu abusif d'\^etre d\'erang\'e si ton r\'eseau ne
marche pas parce que tu as \'ecrit \texttt{polytechnqiue} au lieu de \texttt{polytechnique}.

Ceci dit, en avant pour la configuration !

\vfill

\begin{itemize}
\item Pour te renseigner sur le WikiX, sur Polytechnique.org, ou sur Frankiz et les autres services du BR, rendez-vous page \pageref{services}
\item Pour te connecter au plus vite à Internet, calcule ton IP page \pageref{ip} puis consulte, selon ton système d'exploitation, la section Windows page \pageref{windows},
la section Ubuntu page \pageref{ubuntu} ou la section Mac page \pageref{mac}.
\item Tu ne comprends rien, ça ne marche pas, tu as des questions : beaucoup de réponses sont données page \pageref{faq} et suivantes.

\end{itemize}

\begin{center}
\'Ecris ci-dessous les différents nombres que tu auras calculés, ils te serviront plus d'une fois.
\end{center}

\begin{center}
  \begin{tabular}{|rp{5cm}|}
  \hline
  \rule[-8pt]{0pt}{24pt} \textbf{Mon adresse IP :} \ungaramond 129.104. & \\ \hline
  \rule[-8pt]{0pt}{24pt} \textbf{Ma passerelle :} \ungaramond 129.104. & \\ \hline
  \rule[-8pt]{0pt}{24pt} \textbf{Mon adresse de diffusion :} \ungaramond 129.104. & \\ \hline
  \rule[-8pt]{0pt}{24pt} \textbf{Mon masque de sous-r\'eseau :} \ungaramond 255.255. & \\ \hline
  \end{tabular}
  \label{tableau:mon_IP}
\end{center}
