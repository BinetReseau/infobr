\indent 
Salut \`a toi, l'intr\'epide qui a ouvert ce manuel \`a l'aspect \'etrange et d\'eroutant ! Tu tiens entre tes mains l'\textbf{InfoBR}, texte de r\'ef\'erence qui te sauvera la vie plus d'une fois pendant ton s\'ejour sur le plateau...\\

Pas \`a pas, il te guide \`a travers les m\'eandres des diff\'erentes configurations de ton ordinateur, pour finalement te permettre de profiter de la multitude de services offerts par le r\'eseau de l'\'Ecole: internet et messagerie, bien s\^{u}r, mais aussi forums, chat, \emph{peer to peer}, t\'el\'evision, pour n'en citer que quelques-uns.\\

Tous les membres du Binet R\'eseau sont disponibles pour r\'epondre aux questions et convaincre le cas \'ech\'eant les ordinateurs r\'ecalcitrants, mais v\'erifie bien auparavant que la solution \`a ton probl\`eme ne se trouve pas expos\'ee ici-m\^{e}me. Une liste des \mbox{BR-men} pr\'esente, \`a la fin de cet InfoBR, leurs diff\'erentes sp\'ecialit\'es.\\

Bonne lecture, le r\'esultat en vaut la peine !\\

\begin{flushright}
\textbf{Thifu}
\end{flushright}
 