\newskip\oldpar
\oldpar=\parskip
\advance\parskip by \baselineskip

\section*{Avertissement}

Tu tiens entre tes mains l'InfoBR, document pr�cieux qui te permettra de te connecter facilement
--- enfin, on l'esp�re --- au r�seau.
Nous te conseillons gentiment d'�viter de le paumer sous un meuble ;
il pourra te resservir le jour o� ton ordi te cr�vera mis�rablement entre les mains.
Surtout si on te r�pond que la solution se trouve � telle page :-�.

Si tu rencontres un probl�me, une proc�dure typique � suivre pour le r�soudre est expliqu�e en quatri�me de couverture.
Bien s�r, si tu ne t'en sors pas, tu peux appeler un des membres du binet --- la liste est � l'int�rieur.
Nous sommes l� pour te rendre service.

Cependant, essaye d'abord de bien tout re-v�rifier avant d'appeler un BR-man, et si possible,
adresse toi d'abord � quelqu'un de ton �tage qui s'y conna�t.
D'abord \c{c}a va plus vite.
Ensuite, le BR-man moyen, m�me s'il est de bonne volont�, trouve ca un peu abusif d'�tre d�rang�
si ton r�seau ne marche pas parce que tu as �crit \texttt{polytechnqiue} au lieu de \texttt{polytechnique}.

Ceci dit, en avant pour la configuration !

\parskip=\oldpar
