\subsection{Polytechnique.org}

\emph{Ce n'est pas un service du BR mais de l'association Polytechnique.org !}

Polytechnique.org fournit des services de communications aux �l�ves et anciens �l�ves de l'Ecole : adresse e-mail � vie avec filtre anti-spam et anti-virus, annuaire en ligne sur \url{www.polytechnique.org} avec plus de 12 000 polytechniciens inscrits, forums de discussion\ldots\ Ton adresse mail --- sauf conflit avec un autre X --- est de la forme \mail{prenom.nom@polytechnique.org}, avec les alias \mail{prenom.nom@m4x.org}, \mail{ton\_pseudo@melix.net} et \mail{ton\_pseudo@melix.org} Et ceci est absolument gratuit ! C'est la meilleure m�thode de toujours rester en contact avec la grande communaut� des X.

Cette association est g�r�e par des �l�ves et de jeunes anciens qui sont �galement � l'origine du site \url{www.manageurs.com}, site mettant en relation les dipl�m�s des tr�s grandes �coles de prestige --- X, HEC, Centrale Paris, ENSAE pour l'instant --- avec les recruteurs et les entreprises. Ce site propose, au travers d'un moteur de recherche puissant, des recherches parmi les offres d'emploi de toute exp�rience, ou de stage. Il te propose �galement de d�poser un ou plusieurs de tes CV, consultables par les recruteurs dans le moteur de recherche, pour que tu puisses �tre contact� directement par les entreprises ou les cabinets de recrutement.

Les serveurs mail de Polytechnique.org ne stockent pas r�ellement tes mails comme le fait \server{poly}  par exemple. Ils servent � rediriger tes mails vers la ou les adresses de ton choix, par exemple \mail{prenom.nom@polytechnique.fr} (pendant ta scolarit� sur le plateau) ou \mail{blah.chombier@gmail.com}. Outre l'adresse unique � vie pour tes correspondants, l'int�r�t est de pouvoir modifier la redirection n'importe quand de n'importe o�, par exemple quand tu pars en vacances, en stage, ou quand tu quitteras l'X\ldots (snif). En plus \mail{@polytechnique.org} , �a en jette ! Les filtres anti-spams (contre ceux qui envoient plein de pub) sont tr�s performants, avec une configuration � trois niveaux (\og je prend tout \fg, \og je marque [spam probable] dans le sujet \fg, \og je supprime tous les spams sans demander \fg).

Comment acc�der � tous ces services ? Tout est sur \url{http://www.polytechnique.org} ! Alors si tu n'es pas encore inscrit, n'attends pas ! Les diff�rents services (serveur d'envoi de mails, serveur de news, annuaires, etc.) sont expliqu�s et accessibles par le lien \menu{Services propos�s} � gauche, une fois que tu t'es authentifi� par mot de passe, ainsi que par le lien \menu{Documentation} en bas � gauche.
 