\subsection{Polytechnique.org}

\emph{Attention, ce n'est pas un service du BR mais de l'association Polytechnique.org !}
Cette association regroupe des anciens X qui d�veloppent b�n�volement le site des anciens X : \url{http://www.polytechnique.org}, qui regroupe 12 000 X sur le web, ainsi que \url{http://www.manageurs.com}, qui propose de mettre en relation dipl�m�s des grandes �coles et recruteurs.

Polytechnique.org te propose aussi une adresse mail � vie --- sauf conflit avec un autre X --- de la forme \mail{prenom.nom@polytechnique.org}, avec les alias \mail{prenom.nom@m4x.org}, \mail{ton\_pseudo@melix.net} et \mail{ton\_pseudo@melix.org} ainsi de que de nombreux autres services : trombino, forums, etc. Et ceci est absolument gratuit ! C?est la meilleure m�thode de toujours rester en contact avec la grande communaut� des X du monde.

Les serveurs mail de Polytechnique.org ne stockent pas r�ellement tes mails comme le fait \server{poly}  par exemple. Ils servent � rediriger tes mails vers la ou les adresses de ton choix, par exemple \mail{prenom.nom@polytechnique.fr} (pendant ta scolarit� sur le plateau) ou \mail{blah.chombier@gmail.com}. Outre l'adresse unique � vie pour tes correspondants, l'int�r�t est �vident :
\begin{itemize}
  \item tu peux modifier la redirection n'importe quand de n'importe o�, par exemple quand tu pars en vacances, en stage\ldots quand tu quitteras l'X\ldots (snif) 
  \item et \mail{@polytechnique.org} , �a en jette !
\end{itemize}

Les filtres anti-spams (contre ceux qui envoient plein de pub) ont �t� perfectionn�s l'ann�e derni�re et sont tr�s performants, avec une configuration � trois niveaux ("je prend tout", "je marque [spam probable] dans le sujet", "je supprime tous les spams sans demander").

Comment acc�er � tous ces services ? Tout est sur \url{http://www.polytechnique.org} ! Alors si tu n'es pas encore inscrit, n'attends pas ! 

Les diff�rents services (serveur d'envoi de mails, serveur de news, annuaires, ...) sont expliqu�s et accessibles par le lien \menu{Services propos�s} � gauche, une fois que tu t'es authentifi� par mot de passe, ainsi que par le lien \menu{Documentation} en bas � gauche.
 