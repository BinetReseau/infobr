%$Id$

\newcommand{\dscserver}[3]{\item \server{#1} (#2) : #3}

\subsection{Description des serveurs}

\subsubsection{Serveurs du BR}

Voici quelques-uns des serveurs du BR --- ceux que vous avez le droit de conna�tre ;) Les services h�berg�s sur chaque serveurs sont donn�s � titre informatif, et peuvent �tre chang�s � tout moment. Pour ne pas avoir de mauvaise surprise, utilise les alias qui te sont donn�s dans cet infoBR.
\begin{itemize}
	\dscserver{frankiz}{129.104.201.51}{Site web, newsgroups, DNS secondaire}
	\dscserver{gwennoz}{129.104.201.52}{CVS, miroirs, DNS secondaire}
	\dscserver{heol}{129.104.201.50}{xnetserver, DNS principal}
	\dscserver{skinwel}{129.104.201.53}{T�l�, DNS secondaire}
	\dscserver{enez}{129.104.201.61}{Domaine windows}
\end{itemize}

\subsubsection{Serveurs de la DSI}

Les �l�ves vivent dans un sous-r�seau de celui de la DSI. Et donc, tu risques un jour o� l'autre d'avoir affaire � l'un des serveurs de la DSI.
\begin{itemize}
	\dscserver{poly}{129.104.247.5}{mails, tu le connais certainement}
	\dscserver{kuzh}{129.104.247.2}{proxy http, pour internet :)}
	\dscserver{sil}{129.104.247.3}{proxy, ta seule passerelle ssh vers/depuis l'ext�rieur}
	\dscserver{milou}{129.104.30.41}{ntp, DNS de l'Ecole}
	\dscserver{rackham}{129.104.32.41}{DNS de l'Ecole}
\end{itemize}

La DNS du BR comprend la DNS de l'Ecole et des noms en .eleves.polytechnique.fr pour toutes les machines des �l�ves connect�es sur xNet, gr�ce � qRezix en g�n�ral.
