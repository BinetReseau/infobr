%$Id$

\newcommand{\dscserver}[3]{\item #1 (#2) : #3}

\subsection{Description des serveurs}

\subsubsection{Serveurs du BR}

Voici quelques-uns des serveurs du BR... ceux que vous avez le droit de conna�tre... Les services h�berg�s sur chaque serveurs sont donn�s � titre informatif, et peuvent �tre chang�s � tout moment. Pour ne pas avoir de mauvaise surprise, utilises les alias qui te sont donn�s dans cet infoBR.
\begin{itemize}
	\dscserver{Frankiz}{129.104.201.51}{Site web, newsgroups, DNS secondaire}
	\dscserver{Gwennoz}{129.104.201.52}{CVS, miroirs, DNS secondaire}
	\dscserver{Heol}{129.104.201.53}{T�l�, DNS secondaire}
	\dscserver{Frankiz2}{129.104.201.50}{xnetserver, DNS principal}
	\dscserver{Enez}{129.104.201.61}{Domaine windows}
\end{itemize}

\subsubsection{Serveurs de la DSI}

Les �l�ves vivent dans un sous-r�seau de celui de la DSI. Et donc, tu risques un jour o� l'autre d'avoir affaire � l'un des serveurs de la DSI.
\begin{itemize}
	\dscserver{Poly}{129.104.247.5}{Mails}
	\dscserver{Kuzh}{129.104.247.2}{Proxy http, pour internet :)}
	\dscserver{Sil}{129.104.247.3}{Proxy, ta seule passerelle ssh vers/depuis l'ext�rieur}
	\dscserver{Milou}{129.104.30.41}{ntp, DNS de l'Ecole}
	\dscserver{Rackham}{129.104.32.41}{DNS de l'Ecole}
\end{itemize}

La dns du BR inclus la DNS de l'Ecole en y ajoutant des noms en .eleves.polytechnique.fr pour toutes les machines des �l�ves connect�es sur le xNet (gr�ce par exemple � qRezix)
