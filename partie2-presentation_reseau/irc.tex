%$Id$

\subsection{IRC}

\label{irc}

Le BR te fournit un autre moyen de communication : l'IRC sur RezoSup. IRC veut dire Internet Relay Chat ; RezoSup est le r�seau des grandes �coles et universit�s auquel est rattach� le BR. C'est le moyen id�al de discuter avec un groupe de personnes, car l'IRC est organis� par salles de discussion. Et comme RezoSup est inter-�cole, tu pourras peut-�tre y retrouver des potes de pr�pa !

\imagepos{partie2-presentation_reseau/irc}{1}{IRC}{hb}

Pour te connecter � RezoSup, tu peux cliquer sur le lien IRC sur \fkz --- attention c'est assez long, c'est normal ---, ou installer un programme sp�cifique comme \app{X-Chat}. Avec \app{X-Chat}, il suffit que tu te connectes sur \server{ircserver}, port \server{6667} (par d�faut). Nous te conseillons les chans suivants :
\begin{itemize}
  \item \ngname{\#x} le chan de tous les X
  \item \ngname{\#linux} si tu as des questions � poser sur linux
  \item \ngname{\#superquizz} un quizz en ligne (tape !nick x en arrivant)
\end{itemize}
