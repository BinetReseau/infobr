%$Id$

\subsection{Relations entre IP, prises et cartes r�seau}

Pour all�ger la charge de la matrice (la machine par laquelle passent toutes les communications internes � l'X), la DSI fixe la relation entre les num�ros de casert et les cartes r�seau des ordinateurs, identifi�es par une adresse appel�e MAC (secure MAC). Par ailleurs, pour un fonctionnement sain du r�seau, le BR surveille que c'est toujours la m�me carte r�seau qui correspond � une IP (arpwatch). Ceci a plusieurs cons�quences :
\begin{itemize}
  \item prends l'IP qu'on te donne et ne t'amuse pas � en changer --- et sache que de toute fa�on, on le voit tout de suite !
  \item tu ne peux pas te connecter avec ton ordinateur chez un copain m�me en prenant son IP � cause du secure MAC.
  \item si tu as un nouvel ordinateur, tu dois demander une nouvelle IP dans tes \menu{Pr�f�rences} sur \fkz puis demander au BR d'autoriser ta carte r�seau � se connecter.
  \item si tu changes d'ordinateur ou m�me seulement de carte r�seau, tu dois �galement demander au BR d'autoriser ta nouvelle carte r�seau.
\end{itemize}

Si tu as une nouvelle carte r�seau, mets un post avec les informations suivantes sur \ngname{br.binet.br} afin qu'un BR-man fasse le n�cessaire pour l'autoriser � se connecter :

\begin{itemize}
  \item ton num�ro de casert
  \item l'adresse MAC de ta carte r�seau. Pour l'obtenir :
  \begin{itemize}
    \item sous Windows : va dans \menu{Menu D�marrer -> Ex�cuter}, rentre 'cmd' puis clique OK. Tape alors
    \cmdlineshort{ipconfig /all}
    et fais Entr�e. Tu dois voir ton adresse MAC, de la forme \texttt{\NoAutoSpaceBeforeFDP XX-XX-XX-XX-XX-XX\AutoSpaceBeforeFDP}, o� \texttt{X} est un caract�re hexad�cimal.
     \item sous Linux : dans une console tape
     \cmdlineshort{ifconfig}
     et fais Entr�e. L'adresse MAC est de la forme \texttt{\NoAutoSpaceBeforeFDP xx:xx:xx:xx:xx:xx\AutoSpaceBeforeFDP}, o� \texttt{x} est un caract�re hexad�cimal. Elle se trouve dans le paragraphe o� tu vois aussi ton IP. 
	\item sous Mac OS X : lance \app{Terminal} dans \rep{Applications, Utilitaires} puis fais exactement comme sous Linux.
  \end{itemize}
  \item une gentille formule de politesse :)
\end{itemize}

Ensuite, il suffit qu'un BR-man fasse un tour par l�, d'attendre une demi-heure la mise � jour des switches et c'est bon !


\vfill
 