%$Id$

\subsection{Comment calculer votre IP ?}

Une adresse IP est une suite de quatre chiffres compris entre 0 et 255 s�par�s par des points ; en gros, elle identifie de mani�re unique toute machine connect�s au r�seau mondial. Exemple : l'IP de frankiz est 209.104.201.13.

L'X \og posss�de \fg toutes les IP de la forme 129.104.AAA.BBB. Pour ton ordinateur il va falloir calculer quoi mettre � la place de AAA et BBB afin qu'il ait son adresse unique.

\subsubsection{Calcul du sous-r�seau}

Il s'agit de remplacer le AAA ; le sous-r�seau correspond en gros � ton �tage ou � ton b�timent.

\begin{itemize}

\item pour Fayolle, Maunoury, Joffre et Foch.

Regarde sur ta prise r�seau et note les deux premiers caract�res --- par exemple�: A0. Il suffit alors d'utiliser la table de correspondance suivante pour trouver le AAA :

\begin{tabular}{cccc}
AO \Rightarrow AAA=205 & A1 \Rightarrow AAA=206 & A2 \Rightarrow AAA=207 & A3 \Rightarrow AAA=208 \\
B0 \Rightarrow AAA=215 & B1 \Rightarrow AAA=216 & B2 \Rightarrow AAA=217 & B3 \Rightarrow AAA=218 \\
C0 \Rightarrow AAA=209 & C1 \Rightarrow AAA=210 & C2 \Rightarrow AAA=211 & C3 \Rightarrow AAA=212 \\
D0 \Rightarrow AAA=219 & D1 \Rightarrow AAA=220 & D2 \Rightarrow AAA=221 & D3 \Rightarrow AAA=222 \\
\end{tabular}

\item Pour le PEM

C�est simple AAA=214�!

\item Pour le BEM

Si tu es au b�timent A, alors AAA=203 ; si tu es au b�timent D, alors AAA=204.

\end{itemize}

\subsubsection{}

Rempla�ons le BBB.

\begin{itemize}

\item pour Fayolle, Maunoury, Joffre et Foch.

Tu rel�ves les 2 derniers chiffres de ta prise r�seau et tu rajoutes 120.
Par exemple, si ta prise est la B145 alors ton num�ro ip est le 129.104.216.165 o� $165 = 120 + 45$.

\item pour le PEM


\begin{itemize}

\item Si tu habites au rez-de-chauss�e : BBB = 11 + le num�ro de ta chambre 
\item Si tu habites au premier �tage : BBB = 66 + le num�ro de ta chambre. 

\end{itemize}

Exemple :  si ton num�ro de chambre est 14 et au rez-de-chauss�e, ton IP est 129.104.214.25.

\item pour le BEM

Pour toi BBB = 50 + les deux derniers chiffres de ton num�ro de chambre.
Par exemple l'IP de la chambre D6604 sera 129.104.204.54.

\end{itemize}

\subsubsection{Quelle est l'IP de ma passerelle ?}

La passerelle est une machine par laquelle passent toutes les communications entre ton ordinateur et l'ext�rieur. Elle se situe logiquement au niveau de ton sous-r�seau et porte habituellement le num�ro 13.

Ainsi, seul le dernier nombre de ton IP --- le groupe not� BBB ci-dessus --- change en 13 pour obtenir celle de ta passerelle.

Exemple : si ton IP est 129.104.205.125, alors l'adresse de ta passerelle est 129.104.205.13.
