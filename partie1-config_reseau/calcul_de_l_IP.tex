%$Id$

\subsection{Comment calculer ton IP ?}

\label{calcul_ip}

Une adresse IP est une suite de quatre chiffres compris entre $0$ et $255$ s�par�s par des points ; en gros, elle identifie de mani�re unique toute machine connect�s au r�seau mondial. Exemple : l'IP de \fkz est $209.104.201.13$.

L'X \og posss�de \fg toutes les IP de la forme $129.104.AAA.BBB$. Tu vas devoir calculer quoi mettre � la place de $AAA$ et $BBB$ afin que ton ordinateur ait son adresse unique.

\subsubsection{Calcul du sous-r�seau}

Il s'agit de remplacer le $AAA$ ; le sous-r�seau correspond en g�n�ral � ton �tage ou � ton b�timent.

\begin{itemize}

\item pour Fayolle, Maunoury, Joffre et Foch.

Regarde sur ta prise r�seau et note les deux premiers caract�res --- exemple�: $A0$. Il suffit alors d'utiliser la table de correspondance suivante pour trouver le $AAA$ :

\begin{figure*}[!h]
    \begin{center}
\begin{tabular}{cccc}
$A0 \Rightarrow 205$ & $A1 \Rightarrow 206$ & $A2 \Rightarrow 207$ & $A3 \Rightarrow 208$ \\
$B0 \Rightarrow 215$ & $B1 \Rightarrow 216$ & $B2 \Rightarrow 217$ & $B3 \Rightarrow 218$ \\
$C0 \Rightarrow 209$ & $C1 \Rightarrow 210$ & $C2 \Rightarrow 211$ & $C3 \Rightarrow 212$ \\
$D0 \Rightarrow 219$ & $D1 \Rightarrow 220$ & $D2 \Rightarrow 221$ & $D3 \Rightarrow 222$ \\
\end{tabular}
    \caption{Correspondance prise - sous-r�seau}
    \end{center}
\end{figure*}

\item Pour les b�timents A, B, C et D (les nouveaux caserts)

{\Huge ????}

\item Pour le PEM

C'est simple, $AAA=214$.

\item Pour le BEM

\begin{itemize}
  \item si tu es au b�timent A, alors $AAA=203$.
  \item si tu es au b�timent D, alors $AAA=204$.
\end{itemize}


\end{itemize}

\subsubsection{Calcul du num�ro de machine}

Rempla�ons maintenant le groupe $BBB$.

\begin{itemize}

\item pour Fayolle, Maunoury, Joffre et Foch : tu rel�ves les 2 derniers chiffres de ta prise r�seau et tu rajoutes $120$.
Par exemple, si ta prise est la $B145$ alors ton adresse IP est $129.104.216.165$ o� $165 = 120 + 45$.

\item pour le PEM :

\begin{itemize}

\item Si tu habites au rez-de-chauss�e : $BBB = 11 +$le num�ro de ta chambre 
\item Si tu habites au premier �tage : $BBB = 66 +$le num�ro de ta chambre. 

\end{itemize}

Exemple :  si tu as la chambre $14$ du au rez-de-chauss�e, ton IP est $129.104.214.25$.

\item pour le BEM : $BBB = 50 +$les deux derniers chiffres de ton num�ro de chambre.
Par exemple l'IP de la chambre $D6604$ sera $129.104.204.54$.

\end{itemize}

\subsubsection{Calcul de l'IP de la passerelle}

La passerelle est une machine par laquelle passent toutes les communications entre ton ordinateur et l'ext�rieur. Elle se situe logiquement au niveau de ton sous-r�seau et porte habituellement le num�ro $13$. Ainsi, seul le dernier nombre de ton IP --- le groupe not� $BBB$ ci-dessus --- change en $13$ pour obtenir celle de ta passerelle.

Exemple : si ton IP est $129.104.205.125$, alors l'adresse de ta passerelle est $129.104.205.13$.

\subsubsection{IP des serveurs DNS}

L� il n'y a rien de compliqu� ; le BR offre quatre serveurs DNS redondants qui ont les IP suivantes :
\begin{itemize}
  \item 129.104.201.51
  \item 129.104.201.52
  \item 129.104.201.53
  \item 129.104.201.54
\end{itemize}
Tu choisis celui que tu veux !


\vfill
