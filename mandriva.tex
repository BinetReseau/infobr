\paragraph{\distrib{Sous Mandriva}}

Il existe dans cette distribution une interface graphique pour configurer le r\'eseau, bien qu'il soit possible,
comme sur tout syst\`eme linux, de le faire \`a la main \`a partir des fichiers de configuration dans \file{/etc/}.

Lance l'utilitaire \app{Drakconf] en tant que root, puis clique sur l'onglet \menu{R\'eseau & Internet} et enfin sur \menu{Configurer une nouvelle connexion Internet (LAN, ISDN, ADSL, ...)}. 
Choisis ensuite \menu{Connexion � travers un r\'eseau local (LAN)}, puis la carte r\'eseau sur laquelle tu as branch\'e ton c\^able r\'eseau. (Tu as autant de eth0, eth1, ... que de cartes r\'eseau d\'etect\'ees).  Dans l'exemple ci-dessous, il s'agit de eth0.

Sur la fen\^etre suivante, choisis l'option \menu{configuration manuelle}.  La configuration du r\'eseau commence alors. 
Dans la premi�re fen\^etre, entre :
\begin{description}
        \item[ton adresse IP] entre celle que tu as calcul\'ee : \server{129.104.AAA.BBB} (cf. page \pageref{calcul_ip})
	\item[ton masque de sous-r\'eseau] : \server{255.255.FFF.DDD} (cf. page \pageref{calcul_ip})
\end{description}

Dans la fen\^etre suivante, entre les informations suivantes : 
\begin{description}
        \item[Nom d'h�te (non obligatoire)] : ton_pseudo
        \item[Serveur DNS 1] : \server{129.104.201.53}
        \item[Serveur DNS 2] : \server{129.104.201.54}
        \item[Serveur DNS 3] : \server{129.104.201.51}
        \item[Domaine recherch\'e] : eleves.polytechnique.fr polytechnique.fr
        \item[Passerelle] : tu l'as aussi calcul\'ee : \server{129.104.GGG.CCC} (cf. page \pageref{calcul_ip})
        \item[P\'eriph\'erique passerelle] : aucun
\end {description}

Clique ensuite deux fois sur suivant. (Tu n'as pas besoin d'entrer de nom d'h\^ote ZeroConf.)
L�, tu choisis l'option oui, puis tu valides et tu as normalement un message qui t'annonce que ta configuration r\'eseau est termin\'ee.

	\item[DHCP] laisse cette case d\'ecoch\'ee
\end{description}
 
