\paragraph{\distrib{Sous Mandriva}}
{\bf 1. Configuration du r\'eseau}
Il existe dans cette distribution une interface graphique pour configurer le r\'eseau, bien qu'il soit possible,
comme sur tout syst\`eme linux, de le faire \`a la main \`a partir des fichiers de configuration dans \file{/etc/}.

Lance l'utilitaire \app{Drakconf] en tant que root, puis clique sur l'onglet \menu{R\'eseau & Internet} et enfin sur \menu{Configurer une nouvelle connexion Internet (LAN, ISDN, ADSL, ...)}. 
Choisis ensuite \menu{Connexion � travers un r\'eseau local(LAN)},puis la carte r\'eseau sur laquelle tu as branch\'e ton c\^able r\'eseau.
(Tu as autant de eth0, eth1, ... que de cartes r\'eseau d\'etect\'ees).  Dans l'exemple ci-dessous, il s'agit de eth0.

\imagepos{images/nux_mdk_interface.png}{0.66}{S\'election de l'interface r\'eseau}{!h}

Sur la fen\^etre suivante, choisis l'option \menu{configuration manuelle}.  La configuration du r\'eseau commence alors. 
Dans la premi�re fen\^etre, entre :
\begin{description}
        \item[ton adresse IP] entre celle que tu as calcul\'ee : \server{129.104.AAA.BBB} (cf. page \pageref{calcul_ip})
	\item[ton masque de sous-r\'eseau] : \server{255.255.FFF.DDD} (cf. page \pageref{calcul_ip})
\end{description}

\imagepos{images/nux_mdk_ip.png}{0.66}{Configuration de l'IP}{!h}

Dans la fen\^etre suivante, entre les informations suivantes : 
\begin{description}
        \item[Nom d'h�te (non obligatoire)] : ton_pseudo
        \item[Serveur DNS 1] : \server{129.104.201.53}
        \item[Serveur DNS 2] : \server{129.104.201.54}
        \item[Serveur DNS 3] : \server{129.104.201.51}
        \item[Domaine recherch\'e] : eleves.polytechnique.fr polytechnique.fr
        \item[Passerelle] : tu l'as aussi calcul\'ee : \server{129.104.GGG.CCC} (cf. page \pageref{calcul_ip})
        \item[P\'eriph\'erique passerelle] : aucun
\end {description}

\imagepos{nux_mdk_reseau.png}{0.66}{Configuration du r\'eseau}{!h}

Clique ensuite deux fois sur suivant. (Tu n'as pas besoin d'entrer de nom d'h\^ote ZeroConf.)
L�, tu choisis l'option oui, puis tu valides et tu as normalement un message qui t'annonce que ta configuration r\'eseau est termin\'ee.

{\bf 2. Configuration du gestionnaire de paquets :}
Dans l'utilitaire \app{Drakconf], choisis l'onglet \menu{Gestionnaire
de logiciels}, puis clique sur \menu{S\'electionner la source de
t\'el\'echargement des paquetages pour la mise \`a jour du syst\`eme}.
Clique ensuite sur l'onglet \menu{mandataire} � droite, coche la case
\menu{Nom du serveur mandataire} et inscrit dans le champ correspondant :
kuzh:8080 , puis valide. Clique ensuite sur \menu{ajouter} pour
ajouter les sources de paquets. On te conseille de choisir deux
miroirs pour les mises \`a jour officielles et deux miroirs pour les
sources pour la distribution.
On te conseille entre autres comme miroirs
\url{ftp://ftp.lip6.fr/pub/linux/distributions/Mandrakelinux/official/updates/}
et
\url{http://anorien.csc.warwick.ac.uk/mirrors/Mandriva/official/updates/}

{\bf NB :} Il est possible qu'entre le moment o\`u j'\'ecris ces
lignes et le moment o\`u tu arriveras sur le plateau le BR dispose de son
propre miroir mandriva, auquel cas toutes les \'etapes pr\'ec\'edentes
du 2. seront inutiles et tu auras juste � cliquer sur \menu{ajouter la source
personalis\'ee}, \`a choisir serveur ftp et � mettre comme url \url{ftp://miroir/mandriva}.