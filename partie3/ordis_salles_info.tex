\subsection{Acc\`es \`a des serveurs distants depuis chez toi}

\subsubsection{Putty et FileZilla (SFTP) pour Windows}

Pour ouvrir une ligne de commande à distance tu vas devoir utiliser \app{Putty} TODO

Lorsque tu voudras télécharger des fichiers depuis un serveur (comme dans les salles infos) tu pourras utiliser \app{FileZilla} TODO

\subsubsection{Ssh pour Linux et Mac}

Sous Linux et Mac pour te connecter en ligne de commande à un serveur distant il suffit d'ouvrir une console puis de taper en remplaçant utilisateur et serveur comme il faut : \cmd{ssh utilisateur@serveur} Pour récupérer des fichiers des ordis des salles infos tu peux utiliser \app{FileZilla} de la même façon que sous Windows.

\subsubsection{Se connecter aux salles infos}

Les machines des salles infos ont des noms sympathiques à base de noms d'oiseaux, marques de voitures, nom d'os, nom de poisson\dots Essaye un serveur avec un nom de ce type avec ton utilisateur et mot de passe des salles infos (les mêmes que pour ENEX). Tu n'as pas besoin de te connecter à chaque fois à la même machine, ton compte sera le même sur toutes. 

%  \paragraph{WinSCP} Un logiciel pratique qui te permet de te connecter en salle info.
 %                 Son fonctionnement est expliqu\'e en d\'etails dans le WikiX (\urllink{http://winscp.net}).  \\
  %                Voir aussi \app{Putty} (\urllink{http://www.putty.org/}).
