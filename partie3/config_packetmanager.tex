\subsection{Configurer le gestionnaire de paquets}

La fa\c con la plus simple de param\'etrer son gestionnaire de paquets est d'utiliser les miroirs du BR. Cependant, si tu as besoin d'utiliser des d\'ep\^ots suppl\'ementaires \`a l'ext\'erieur de l'X, tu peux choisir
la configuration pr\'esent\'ee ici.

\paragraph{Sous Gentoo}
\label{gentoo_mirror} Pour pouvoir utiliser \app{emerge} \`a  travers le serveur mandataire (\emph{proxy}) de l'\'Ecole, il faut d\'efinir %les variables d'environnement
ci-dessous dans le fichier \file{/etc/make.conf} :
\cmdline{http\_proxy=http://kuzh.polytechnique.fr:8080\\
ftp\_proxy=http://kuzh.polytechnique.fr:8080\\
no\_proxy=.eleves.polytechnique.fr\\
GENTOO\_MIRRORS="ftp://miroir.eleves.polytechnique.fr/linux/gentoo http://gentoo.osuosl.org/"\\
SYNC="rsync://rsync/gentoo-portage"}

Tu peux \'evidemment ajouter d'autres miroirs (s\'epar\'es par des espaces) dans ta liste mais \urllink{ftp://miroir} \'etant interne, il sera toujours beaucoup plus rapide que les autres.

\paragraph{Sous Archlinux}

\paragraph{Sous Debian}
\label{debian_mirror} Pour pouvoir utiliser \app{APT} (via \app{apt-get} ou \app{aptitude}) \`a travers le \emph{proxy} de l'\'Ecole, il faut ajouter dans \file{/etc/apt/apt.conf} la ligne :
\cmdline{Acquire::http::Proxy "http://129.104.247.2:8080/";}}
Le cas \'ech\'eant, il faudra cr\'eer ce fichier, qui est vide la plupart du temps.

Pour le reste, le param\'etrage des d\'ep\^ots se fait comme toujours dans \file{/etc/apt/sources.list} ou via ton gestionnaire graphique pr\'ef\'er\'e.
