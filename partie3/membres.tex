\subsection{Membres \'eminents du Binet R\'eseau}

Chaque BR-man signale quels syst\`emes d'exploitation il conna\^it.

\mbr{Sigma}{71 66}{\wins \nuxs}{Prez, root, news, devel qrezix, support@windows, support@linux}
\mbr{Xelnor}{61 47}{\nuxs}{Prez, root, IRCop, devel frankiz, support@linux, WiFi}{Depuis qu'il a \'et\'e \'elu (si, si !), il essaie de suivre tout ce qui se passe au BR.}

\mbr{Flafla}{63 33}{\wins}{Trez, support@windows, devel qrezix}{Poste exigeant s'il en est : toute la subtilit\'e r\'eside dans le fait de financer des petits plaisirs de geeks en les faisant passer pour des d\'epenses n\'ecessaires...}

\mbr{J-Philippe}{61 58}{\nuxs}{root, web, tol, news, InfoBR, support@linux, support@latex}{Passionné de typographie et de \LaTeX{}, il était logique que ce soit lui qui s'occupe de l'InfoBR. Ne parle surtout pas de majuscule devant lui, il te prendrait la tête pendant 2 heures pour t'expliquer la différence avec la capitale. \`A part ça, il essayera toujours de trouver un peu de temps pour t'aider si tu as un problème en linux (c'est pas le meilleur) ou en \LaTeX{} (c'est le meilleur!).}

\mbr{Totor}{71 96}{\macs}{root, web, tol, news, support@mac, Xshare@mac}{Toujours prêt à aider les macqueux, même en plein milieu de la nuit, et les autres aussi même si il ne comprend pas leur problèmes. Pour les macs addicts, ils trouveront leur bonheur et tous les petits programmes qui vont bien sur son ftp.}

\mbr{Marsu}{61 12}{\wins}{admin@windows}{Si ta clef de licence microsoft met longtemps \`a arriver, c'est qu'il est quelque part en Europe, et t'as plus qu'\`a attendre qu'il revienne.}

\mbr{Claire}{65 16}{\wins}{News, BRwoman}{Arriv\'ee au BR par accident, pour elle l'info c'est java. Son ordinateur, elle ne l'utilise que pour des applications simples: la preuve, elle est sous vista. Dort entre 22h et 7h m\^eme les week-end.}

\mbr{Pauline}{60 33}{\wins}{Tol, web, BRwoman}{Respo je ne connais rien a l'informatique mais je suis au BR. Toujours dispo pour corriger vos annonces, valider vos activit\'es et admirer vos photos tol. Bref, je ne sais pas r\'epondre \`a vos questions, mais je le ferai toujours avec le sourire. On m'a dit que la BRwoman ne sert qu'\`a ça de toute façon.}

\mbr{Mingguo}{61 35}{\wins}{TV}{Le BRman qui essaie d'apprendre la ligne de commande sous linux pour travailler.}

\mbr{Bingxiong}{63 41}{\wins}{InfoBR(anglais)}{}

\mbr{Bogdan}{60 42}{\wins}{Support@windows, x-share@windows, web, web devel, admin@windows}{C'est le BRman que tu appelles quand tu ne trouves pas la solution \`a tes probl\`emes sous Windows dans l'InfoBR et quand google ne veut plus \^etre ton ami.}

\mbr{La buche}{61 41}{\wins}{Support@windows, x-share windows, web devel}{Consultations sans rendez-vous au Binet B\^oBar, tous les soirs \`a partir de 20h30 (surtout le mardi) ...}

\mbr{John Peter}{64 61}{\wins}{QDJmestre}{Au BR plus formellement que r\'eellement, ne le contacte pas si tu as un probl\`eme avec ton PC. Par contre envoie lui des QDJ et viens lui r\'eclamer tes binouzes si jamais il oublie...}

\mbr{Sandalphon}{65 24}{\nuxs \wins}{Relex, support@windows, support@linux}{Pas assez geek pour \^etre devel, trop m\'echant pour s'occuper des forums, mais suffisamment pipo pour le poste de relex. N\'eanmoins toujours pr\^et \`a filer un coup de main informatique dans la joie et la bonne humeur !}

\mbr{Ccpasteur}{60 30}{\nuxs}{Root, xshare@linux, support@linux, bll}{Bien que linuxien acharn\'e dans l'\^ame, il lui arrive \`a son plus grand d\'esarroi de repasser sous Windows plusieurs fois par jour pour jouer \`a PES. Il pourra tout t'expliquer sur le RAID et sur les diff\'erentes pannes des disques durs gr\^ace \`a son exp\'erience de respo info du JTX.}

\mbr{Insane0}{63 30}{\nuxs}{root, news, TV, WiFi, BLL, devel@qrezix, support@linux}{Adepte du clavier dvorak et de gentoo. Si vous ne savez pas ce que c'est, demandez lui. Il sera ravi de vous expliquer et de vous convertir.}

\mbr{Xavier}{63 34}{\nuxs}{Root, news, infoBR, support@windows}{En cas de probl\`eme il est toujours dispo mais attention, ce n'est jamais pour tr\`es longtemps. Fascin\'e par les pingouins, il pourra te montrer comment faire danser la valse \`a une otarie par une simple ligne de commande.}

\mbr{Zuzuf}{61 74}{\nuxs}{root, devel qrezix, BLL, support@Linux, hardware}{Ad\`epte de Kubuntu et autres Linux grand publics, il viendra \`a bout de tous tes probl\`emes de mat\'eriel que ça concerne un overclocking hors norme, un watercooling ou juste des n\'eons
et \`a tous tes probl\`emes logiciels gr\^ace \`a l'antivirus ultime : Linux \dots}

\mbr{Patrick}{61 73}{\macs}{Mac, support@mac, Xshare@mac}{Grand gourou des Macs sur le plateau. Viendra te configurer ton Airport \`a n'importe quelle heure du jour et de la nuit !}

\subsubsection*{Description rapide des postes}

\begin{description}

  \item[Prez]{(\mail{prez@frankiz}) Poste fictif, qui permet toutefois d'avoir
des relations bien plac\'ees.}

  \item[Trez]{(\mail{trez@frankiz}) Escroc qui cherche uniquement \`a remplir le compte en banque pour organiser un voyage de geeks \`a Redmond, ou plut\^ot Cupertino\dots}

  \item[relex]{Assistant du prez pour les relations avec les \emph{gens}.}

  \item[root]{(\mail{root@frankiz}) Les roots sont les administrateurs du r\'eseau. Ce sont eux qui s'\'evertuent \`a maintenir en \'etat de marche les serveurs, \`a rajouter de nouveaux services et \`a rep\'erer les boulets qui font de la merde sur le r\'eseau. S'il s'agit de g\'erer un compte de binet, utilise plut\^ot \mail{binets@frankiz}.}

  \item[admin@windows] {(\mail{windows@frankiz}) Administrateurs du domaine Windows. En cas de probl\`eme avec Windows, en particulier avec l'antivirus, ce sont les mieux plac\'es pour t'aider ; bien s\^ur c'est plus facile si tu es sur le domaine !
}
  \item[support@windows] {(\mail{support@frankiz}) SOS d\'epannage windows, j'\'ecoute ! Pr\^ets \`a tout pour sauver une jeune demoiselle (ou un jeune \emph{gens} \`a la rigueur) en d\'etresse avec son windows\dots }

  \item[support@mac] {(\mail{support@frankiz}) C'est un poste naturellement tranquille. Qui a besoin d'\^etre d\'epann\'e sur Mac? Ah, c'est vrai : celui qui a install\'e Windows dessus en suivant les conseils de l'InfoBR... }

  \item[devel]{Joyeux programmeurs qui sont l\`a pour am\'eliorer les logiciels du BR --- \app{qRezix} et ses plug-ins (\mail{qrezix@frankiz}), le site web \urllink{http://frankiz/}. Ce sont eux qui tous les deux mois te disent que ton \app{qRezix} n'est pas \`a jour.}

  \item[news] {(\mail{news@frankiz}) Mainteneurs du serveur de news, ils surveillent aussi ce que tu postes et que tu respectes les r\`egles de base comme les crossposts (marteau-th\'erapie) \mbox{;-)}}

  \item[web@frankiz] {(\mail{web@frankiz}) Webmestres de \fkz, ils valident les annonces et les activit\'es et surveillent le contenu du site de ton binet ou de ton site perso.}

  \item[X-share] {(\mail{xshare@frankiz}) Personne sympathique qui cherche \`a longueur de temps de nouveaux logiciels gratuits, ou mieux, libres, \`a proposer aux \'el\`eves dans \xshare.}

  \item[InfoBR]{L'art du travail distribu\'e: il dit \`a tous les autres d'\'ecrire. Le probl\`eme majeur \'etant la synchronisation des diff\'erentes parties avant les dates limites.}

  \item[TV]{Charg\'es de maintenir la diffusion de la t\'el\'e sur le r\'eseau. Changeurs de cha\^ine, dieux du multicast, ils sont les amis des switches\dots\ ou pas.}

  \item[QDJ Master] {(\mail{qdj@frankiz}) Chaque jour un nouveau dilemne sur \fkz\dots\ n'h\'esitez pas \`a faire vos propositions \`a \urllink{qdj@frankiz}.}

  \item[IRCop]{Responsable des relations avec RezoSup. Viendez sur IRC (\urllink{http://ircserver/}) !}

  \item[TOL] {(\mail{tol@frankiz}) V\'erificateur de photos, il surveille le Trombi On Line.}

  \item[BRwoman]{Preuve vivante que le BR n'est pas un milieu enti\`erement masculin.}

\end{description}
