

\subsection{Membres \'eminents du binet R\'eseau}

Chaque BR-man signale quels syst\`emes d'exploitation il conna\^it.

\mbr{Qt}{64 85}{\macs \wins}{Prez, root, support@windows, support@mac, support@\LaTeX}{Il a d\'ecouvert les Mac en d\'ebut d'ann\'ee et --- apr\`es avoir perdu son premier Macbook dans le RER --- en est devenu compl\`etement fan. Il peut te donner un coup de main sur Mac, sur Windows ou en \LaTeX. Indisponible avant midi, il est sp\'ecialis\'e dans l'envoi massif de mails entre 1\,h et 4\,h.}

\mbr{Pj}{63 66}{\nuxs}{Vice-prez, root, devel@qrezix, support@linux, support@wifi}{Parlant spontan\'ement plus aux machines qu'aux gens, il sera toutefois toujours disponible pour t'aider en cas de probl\`eme. \'Evite Windows quand m\^eme, \c ca fait quelques temps qu'il ne l'utilise plus. N'h\'esite pas en revanche \`a le taper pour qu'il am\'eliore \app{qRezix}.}

\mbr{Brice}{62 51}{\wins}{Trez, root, web, tol, IRC-Op, support@windows, support@wifi}{Toujours sous Windows, il est le touche \`a tout et bon \`a rien du BR. Souvent dans son casert, il t'aidera volontiers tant qu'il ne s'agit pas de questions trop techniques.}

\mbr{Almageste}{69 17}{\nuxs \wins}{root, support@tv, support@linux, support@windows, BLL}{Le BR-man le plus \'eloign\'e de ton casert, il n'en reste pas moins disponible. Si tu veux discuter logiciels libres ou formats ouverts il r\'epond toujours pr\'esent ! Il peut \'eventuellement te donner un coup de main en \LaTeX\ si besoin.}

\mbr{Feandil}{63 27}{\nuxs \wins}{root, web, tol, news, IRC-Op, support@windows, admin@windows}{Windowsien pas encore pass\'e \`a Linux, barbu, il s'occupe surtout de ce qui n'est pas visible de l'ext\'erieur (sauf quand il casse Frankiz). Si tu rencontres un probl\`eme avec l'un des services du BR, n'h\'esite pas \`a le contacter, il ne mord pas.}

\mbr{Pinal}{61 68}{\nuxs}{root, support@linux}{S'\'etant lanc\'e dans l'exploration des secrets du pingouin d\`es son arriv\'ee sur le plat\^al, il pourra surtout te donner des conseils sur ce qu'il ne faut pas faire. Tu le trouveras entre le local JTX et le B\^oB. PS: il n'est pas geek.}

\mbr{Riton}{65 26}{\wins}{root, devel@frankiz, support@windows}{Pour tout ce qui touche aux sites binets et \`a Frankiz, c'est \`a lui qu'il faut s'adresser. Si un jour Frankiz 3 (a.k.a. Frankiz Nukem Forever) sort, vous saurez qui remercier ! Root utilisant irr\'eductiblement Windows, n'h\'esite pas \`a le contacter si tu as des probl\`emes avec ce fabuleux OS.}

\mbr{Wilyo}{62 01}{\nuxs \wins}{root, news, support@linux, secr\'etaire}{Le BSDiste, le seul, l'unique. Bien qu'il passe son temps \`a utiliser des OS dont personne ne comprend l'utilit\'e, il est quand m\^eme capable de donner un coup de main en cas de besoin sous Windows ou Linux. Sauf le matin.}

\mbr{Anna}{64 93}{\wins}{web, tol, news, BR-woman}{Elle apporte une touche de f\'eminit\'e \`a ce monde de mecs en contribuant au BR \`a coups de validations sur Frankiz, de mod\'eration des brs et de bisous aux install-parties.}

\mbr{Benoit et Mathias}{63 20 et 60 83}{\macs}{l\'egal}{Ils marchent par paire et tournent tous les deux sous Mac OSX 10.6.2. Ces MacAholic s'occupent des aspects l\'egaux du BR. Ils sont toujours pr\^ets \`a aider mais n'y connaissent pas grand-chose !}

\mbr{Camille}{64 96}{\wins}{com, BR-woman}{BR-woman num\'ero 2, qui poss\`ede \`a peu pr\`es autant de paires de chaussures que le reste du BR r\'euni. Arm\'ee de son fid\`ele Photoshop, elle essaye de tenir les objectifs du prez en termes de "communication graphique".}

\mbr{Forbin}{62 22}{\nuxs \wins}{devel@frankiz, support@\LaTeX, \'electronique}{Paul de son pr\'enom, converti depuis peu \`a Arch Linux. Il r\'edige tout ce qu'il peut en \LaTeX, et en plus de d\'evelopper pour Frankiz 3, il aime communiquer avec la machine \`a l'aide de son fer \`a souder. Affectionne particuli\`erement les m\'etaux lourds.}

\mbr{Haroine}{63 68}{\nuxs}{support@linux}{Partisan du libre jusqu'au bout des ongles (il para\^itrait m\^eme qu'il monte ses clips JTX sur KdenLive \dots), il pourra toujours essayer de t'aider si tu rencontres un probl\`eme avec ton Tux.}

\mbr{Juan}{61 64}{\wins}{devel@frankiz}{Il se focalise surtout sur le d\'eveloppement de Frankiz 3, mais il trouvera toujours du temps pour r\'epondre \`a un appel au secours !}

\mbr{Karlos}{61 56}{\macs}{news, web, tol, support@mac, support@\LaTeX, relex}{Empereur de Frankiz aux heures o\`u les honn\^etes gens dorment, et o\`u seuls les bots r\^odent encore, fasciste des newsgroups et grand amateur de la firme \`a la Pomme, tu peux lui demander conseil pour pas mal de manips sur ton Mac, si tant est que tu rencontres le moindre probl\`eme, bien s\^ur. Il ma\^itrise \'egalement l'installation de Windows en plus (mais qui serait assez fou pour faire \c ca ?), sauf sur son propre ordinateur \dots}

\mbr{Lad}{61 42}{\wins}{web, tol, InfoBR}{Il trouve son bonheur dans les t\^aches qui ne n\'ecessitent aucune qualification. Il sera n\'eanmoins toujours pr\^et \`a te donner un coup de main, si c'est dans son domaine de comp\'etence !}

\mbr{Penangol}{64 11}{\nuxs}{support@linux}{Nouvellement sous Ubuntu, il est l\`a principalement pour tra\^iner aux IP. Il est toujours ravi de donner un coup de main --- dans la limite rapidement atteinte de ses comp\'etences.}

\mbr{Tonio}{62 03}{\macs \wins}{support@mac, devel@frankiz}{Toujours pr\^et \`a aider ceux qui rencontrent des probl\`emes sur Mac ou ceux qui ont des questions sur le fonctionnement de Photoshop. Le plus simple reste de passer \`a Fa\"eriX pour le trouver.}

\mbr{TyPHooN}{60 03}{\wins}{admin@windows, support@windows}{Un souci avec les licences Microsoft ? Un probl\`eme avec Windows ou le serveur Enez ? N'h\'esite pas \`a le contacter, il se fera un plaisir de t'aider et de r\'epondre \`a tes questions. Il est toujours pr\^et \`a d\'epanner un PC \dots surtout lorsqu'il faut mettre les mains dans le cambouis !}

%\vspace{\stretch{1}}

%\emph{\`A force de rester enferm\'e dans leurs caserts sombres, les BR-men sont devenus trop sensibles �  la lumière pour supporter un flash d'appareil photo. Ainsi, seule une photo de dos a pu être prise.}

%\imagepos{images/br2k7.jpg}{0.9}{L'\'equipe du BR 2k7}{h}

%\vspace{\stretch{1}}



\subsubsection*{Description rapide des postes}

\begin{description}

  \item[Prez]{(\mail{prez@frankiz}) Poste fictif, qui permet toutefois d'avoir
des relations bien plac\'ees.}

  \item[Trez]{(\mail{trez@frankiz}) Escroc qui cherche uniquement \`a remplir le compte en banque pour organiser un voyage de geeks\dots}

  \item[relex]{Assistant du prez pour les relations avec les \emph{gens}.}

  \item[root]{(\mail{root@frankiz}) Les \emph{roots} sont les administrateurs du r\'eseau. Ce sont eux qui s'\'evertuent \`a maintenir en \'etat de marche les serveurs, \`a rajouter de nouveaux services et \`a rep\'erer les boulets qui font de la merde sur le r\'eseau. S'il s'agit de g\'erer un compte de binet, utilise plut\^ot \mail{binets@frankiz}.}

  \item[admin@windows] {(\mail{windows@frankiz}) Administrateurs du domaine Windows. En cas de probl\`eme avec Windows, en particulier avec l'antivirus, ce sont les mieux plac\'es pour t'aider ; c'est bien s\^ur  plus facile si tu es sur le domaine !
}
  \item[support@windows] {(\mail{support@frankiz}) SOS d\'epannage Windows, j'\'ecoute ! Pr\^ets \`a tout pour sauver une jeune demoiselle (ou un jeune \emph{gens} \`a la rigueur) en d\'etresse avec son Windows\dots }

  \item[support@mac] {(\mail{support@frankiz}) C'est un poste naturellement tranquille. Qui a besoin d'\^etre d\'epann\'e sur Mac? Ah, c'est vrai : celui qui a install\'e Windows dessus en suivant les conseils de l'InfoBR... }

  \item[devel]{(\mail{qrezix@frankiz}) Joyeux programmeurs qui sont l\`a pour am\'eliorer les logiciels du BR. Leurs efforts se concentrent principalement sur le d\'eveloppement de Frankiz 3, mais ils s'occupent \'egalement de \app{qRezix} et ses \emph{plug-ins}.}

  \item[news] {(\mail{news@frankiz}) Mainteneurs du serveur de \emph{news}, ils surveillent aussi ce que tu postes et que tu respectes les r\`egles de base comme les \emph{crossposts} (marteau-th\'erapie) \mbox{;-)}}

  \item[web] {(\mail{web@frankiz}) Webmestres de \fkz, ils valident les annonces et les activit\'es et surveillent le contenu du site de ton binet ou de ton site perso.}

%  \item[X-share] {(\mail{xshare@frankiz}) Personne sympathique qui cherche \`a longueur de temps de nouveaux logiciels gratuits ou mieux, libres \`a proposer aux \'el\`eves dans \xshare.}

  \item[InfoBR]{L'art du travail distribu\'e: il dit \`a tous les autres d'\'ecrire. Le probl\`eme majeur \'etant la synchronisation des diff\'erentes parties avant les dates limites.}

  \item[support@tv]{Charg\'es de maintenir la diffusion de la t\'el\'e sur le r\'eseau. Changeurs de cha\^ine, dieux du \emph{multicast}, ils sont les amis des \emph{switches}\dots\ ou pas.}

%  \item[QDJ Master] {(\mail{qdj@frankiz}) Chaque jour un nouveau dilemne sur \fkz\dots\ n'h\'esitez pas \`a faire vos propositions \`a \urllink{qdj@frankiz}.}

  \item[IRC-Op]{Responsable des relations avec RezoSup. Viendez sur IRC (\urllink{http://ircserver/}) !}

  \item[tol] {(\mail{tol@frankiz}) V\'erificateur de photos, il surveille le Trombi-On-Line.}

  \item[BR-woman]{Preuve vivante que le BR n'est pas un milieu enti\`erement masculin.}

\end{description}

