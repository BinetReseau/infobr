\subsection{Membres \'eminents du Binet R\'eseau}

Chaque BR-man signale quels syst\`emes d'exploitation il conna\^it.

\mbr{Sigma}{71 66}{\wins \nuxs}{Prez, root, news, devel qrezix, support@windows, support@linux}{Touche à tout, il essaie de suivre tout ce qui se passe (même s'il subit parfois ce qui se passe sur les machines). Il a découvert les Virtualbox et s'amuse à les multiplier. Il se charge de la permanence BR matinale (dès 5-6 heures du matin !)}

\mbr{Nathaniel}{76 83}{\nuxs \wins}{Trez, root, web, tol, news, IRCop, devel@frankiz}{Tyran bisounours des news. Si tu as une question concernant le BR (sauf le support), appelle le, y a de grandes chances que ça le concerne. Sa vie a changé depuis qu'il a découvert la puissance du ssh (puis, plus tard, la puissance du -Y).}

\mbr{J-Philippe}{76 42}{\nuxs}{root, web, tol, news, InfoBR, support@linux, support@\LaTeX{}}{Passionné de typographie et de \LaTeX{}, il était logique que ce soit lui qui s'occupe de l'InfoBR. Ne parle surtout pas de majuscule devant lui, il te prendrait la tête pendant 2 heures pour t'expliquer la différence avec la capitale. \`A part ça, il essayera toujours de trouver un peu de temps pour t'aider si tu as un problème en linux (c'est pas le meilleur) ou en \LaTeX{} (c'est le meilleur!).}

\mbr{simon}{73 75}{\nuxs}{root, IRCop, bll}{Depuis assez peu de temps sous linux, il essaie d'en maîtriser les arcanes les plus t\'en\'ebreuses. Toujours à la rue, mais à un degr\'e variable selon l'heure et le jour.}

\mbr{Totor}{71 96}{\macs}{root, web, tol, news, support@mac, Xshare@mac}{Toujours prêt à aider les macqueux, même en plein milieu de la nuit, et les autres aussi même si il ne comprend pas leur problèmes. Tu le trouveras à coup sûr entre entre son kazert et le BôB dès la fin des cours. Pour les macs addicts, ils trouveront leur bonheur et tous les petits programmes qui vont bien sur son ftp.}

\mbr{Andrei}{76 98}{}{devel@root, devel@frankiz}{}

\mbr{Kithyane}{76 22}{\nuxs \wins}{web, tol, news, BRwoman, devel@frankiz}{Elle a découvert son côté geek en entrant au BR, est passée sous Linux depuis, et s'émerveille des nouvelles possibilités qui s'offrent à elle. Extension spirituelle de Nathaniel pour les validations sur Frankiz et la modération des br, en un peu moins bisounours, peut-être, mais avec le sourire de la BRWoman en plus !}

\mbr{Mikado}{70 90}{\wins}{admin@windows}{Problème avec Windows ou les licences Microsoft ? C'est pour lui!}

\mbr{Jiherr}{72 60}{\macs}{support@mac, respo mac, relex}{}

\mbr{RogerTroutman}{77 00}{\wins}{root, admin@windows, support@windows}{}

\mbr{Boris}{73 70}{\nuxs}{root, devel@qrezix}{Gentooiste, il a choisi la voie de la concaténation des Saints Manuels. Il a abandonné son humanité depuis peu pour devenir une page de manuel à  part entière. Tapez `\texttt{/}' pour rechercher.  }

\mbr{Bruno}{71 29}{\nuxs}{support@linux}{N'hésite pas à m'appeler si tu as un problème avec Linux ou le réseau.}

\mbr{Pom}{77 12}{\wins \nuxs}{web, tol, support@windows, support@linux, support@\LaTeX{}}{Coucou ! Moi c'est Pom (ah mince c'est marqué à gauche... bon ben
sinon, c'est Thomas, ça c'est une info exclusive !). Je suis toujours
prêt à t'aider si tu as le moindre problème ou la moindre question.
Mon but c'est rendre service, donc n'hésite pas ! Ma porte est
toujours ouverte (certes elle est perdue au milieu du 3ème étage de
Maunoury, mais quand même). Je sais comment fonctionne Windows :
clic gauche sur le menu démarrer, puis encore un clic au bon
endroit, et ... Quant à linux, et bien je sais que si je tape
"firefox" j'arrive sur internet ! C'est déjà ça. Sinon je m'y connais
en latex, enfin \LaTeX{} hein! Pas de mauvaise compréhension ;) et je
suis chargé de valider tes annonces frankiz, c'est toujours bon à
savoir ! Bisous !}

\mbr{Ninja}{70 70}{\nuxs}{devel@root, BLL, support@linux}{}

\mbr{Shuba}{71 32}{\nuxs}{devel@root}{}

\mbr{fulanor}{72 56}{\macs}{news}{}

\mbr{Buu-Minh}{72 17}{\nuxs \wins}{devel@root, support@windows, support@linux}{Même si ça ne se voit pas, il sera toujours prêt à t'aider si tu rencontres un problème avec ta machine.}


\subsubsection*{Description rapide des postes}

\begin{description}

  \item[Prez]{(\mail{prez@frankiz}) Poste fictif, qui permet toutefois d'avoir
des relations bien plac\'ees.}

  \item[Trez]{(\mail{trez@frankiz}) Escroc qui cherche uniquement \`a remplir le compte en banque pour organiser un voyage de geeks \`a Redmond, ou plut\^ot Cupertino\dots}

  \item[relex]{Assistant du prez pour les relations avec les \emph{gens}.}

  \item[root]{(\mail{root@frankiz}) Les roots sont les administrateurs du r\'eseau. Ce sont eux qui s'\'evertuent \`a maintenir en \'etat de marche les serveurs, \`a rajouter de nouveaux services et \`a rep\'erer les boulets qui font de la merde sur le r\'eseau. S'il s'agit de g\'erer un compte de binet, utilise plut\^ot \mail{binets@frankiz}.}

  \item[admin@windows] {(\mail{windows@frankiz}) Administrateurs du domaine Windows. En cas de probl\`eme avec Windows, en particulier avec l'antivirus, ce sont les mieux plac\'es pour t'aider ; bien s\^ur c'est plus facile si tu es sur le domaine !
}
  \item[support@windows] {(\mail{support@frankiz}) SOS d\'epannage windows, j'\'ecoute ! Pr\^ets \`a tout pour sauver une jeune demoiselle (ou un jeune \emph{gens} \`a la rigueur) en d\'etresse avec son windows\dots }

  \item[support@mac] {(\mail{support@frankiz}) C'est un poste naturellement tranquille. Qui a besoin d'\^etre d\'epann\'e sur Mac? Ah, c'est vrai : celui qui a install\'e Windows dessus en suivant les conseils de l'InfoBR... }

  \item[devel]{Joyeux programmeurs qui sont l\`a pour am\'eliorer les logiciels du BR --- \app{qRezix} et ses plug-ins (\mail{qrezix@frankiz}), le site web \urllink{http://frankiz/}. Ce sont eux qui tous les deux mois te disent que ton \app{qRezix} n'est pas \`a jour.}

  \item[news] {(\mail{news@frankiz}) Mainteneurs du serveur de news, ils surveillent aussi ce que tu postes et que tu respectes les r\`egles de base comme les crossposts (marteau-th\'erapie) \mbox{;-)}}

  \item[web@frankiz] {(\mail{web@frankiz}) Webmestres de \fkz, ils valident les annonces et les activit\'es et surveillent le contenu du site de ton binet ou de ton site perso.}

  \item[X-share] {(\mail{xshare@frankiz}) Personne sympathique qui cherche \`a longueur de temps de nouveaux logiciels gratuits, ou mieux, libres, \`a proposer aux \'el\`eves dans \xshare.}

  \item[InfoBR]{L'art du travail distribu\'e: il dit \`a tous les autres d'\'ecrire. Le probl\`eme majeur \'etant la synchronisation des diff\'erentes parties avant les dates limites.}

  \item[TV]{Charg\'es de maintenir la diffusion de la t\'el\'e sur le r\'eseau. Changeurs de cha\^ine, dieux du multicast, ils sont les amis des switches\dots\ ou pas.}

  \item[QDJ Master] {(\mail{qdj@frankiz}) Chaque jour un nouveau dilemne sur \fkz\dots\ n'h\'esitez pas \`a faire vos propositions \`a \urllink{qdj@frankiz}.}

  \item[IRCop]{Responsable des relations avec RezoSup. Viendez sur IRC (\urllink{http://ircserver/}) !}

  \item[TOL] {(\mail{tol@frankiz}) V\'erificateur de photos, il surveille le Trombi On Line.}

  \item[BRwoman]{Preuve vivante que le BR n'est pas un milieu enti\`erement masculin.}

\end{description}
