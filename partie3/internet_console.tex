\subsection{Param\'etrer la connexion Internet avec la console}
\label{linux_cmdline}
Si tu utilises une distribution de Linux dans laquelle il est plus simple de configurer le r\'eseau en ligne de commande, voici la marche \`a suivre :

Tout d'abord renseigne ta configuration r\'eseau dans le fichier \file{/etc/network/interfaces}. Il doit ressembler \`a ceci :
\noindent \cmdline{
auto lo eth0\\
iface lo inet loopback\\
\\
iface eth0 inet static\\
address 129.104.xxx.xxx\\
netmask 255.255.xxx.xxx\\
broadcast 129.104.xxx.xxx\\
gateway 129.104.xxx.xxx
}
''gateway'' d\'esigne la passerelle et ''broadcast'' l'adresse de diffusion.
\newline

Ensuite, configure la r\'esolution DNS : tu dois avoir ceci dans le fichier \file{/etc/resolv.conf} :
\noindent \cmdline{
search eleves.polytechnique.fr polytechnique.fr\\
nameserver 129.104.201.53\\
nameserver 129.104.201.51}

N'oublie pas de r\'egler ensuite le proxy dans ton navigateur web pr\'ef\'er\'e.
