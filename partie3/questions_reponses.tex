%$Id: questions_reponses.tex 144 2005-03-25 01:11:37Z myk $

\subsection{Questions-r�ponses}

Les questions les plus courantes sont r�pertori�es ici pour te faire gagner du temps !

\begin{description}

\item[J'ai une question sur l'informatique] je poste sur \ngname{br.informatique.[truc qui va bien]}.

\item[J'ai perdu mon mot de passe qRezix] je vais le re-d�finir dans mes  \menu{Pr�f�rences} sur \fkz : \url{http://frankiz/profil/reseau.php}.

\item[Je veux voir mon pseudo quand j'ai vot� � la QDJ] je d�finis mon pseudo sur ma fiche trombi sur la page \url{http://frankiz/profil/profil.php} (lien \menu{Pr�f�rences}).

\item[J'ai un deuxi�me ordinateur, qu'est-ce que je fais ?] je demande une deuxi�me IP sur la page \url{http://frankiz/profil/demande\_ip.php} et je la lui attribue.

\item[Je n'ai plus de r�seau] je vais voir la 4\textsuperscript{�me} de couverture.

\item[Je viens de changer de casert/section, il faudrait mettre � jour ma fiche TOL] j'envoie un mail � \mail{tol@frankiz} avec les modifications � effectuer ainsi que la raison de ces
modifications.

\item[Mon client mail dit que \guillemotleft~l'autorit� de certification est inconnue~\guillemotright ] je vais t�l�charger le certificat de s�curit� sur \url{https://poly/} et je l'installe.

\item[Je ne re�ois pas mes mails] V�rifie ta redirection sur \url{http://poly}.

\item[Je n'arrive pas � me connecter � \server{poly}] Essaye \server{poly.polytechnique.fr}.

\item[Mon ordinateur n'a pas de nom sur le r�seau] J'installe qRezix et je le configure.

\item[Je cherche des informations sur l'Ecole] Regarde sur \url{http://intranet}.

\item[Je cherche � joindre une personne de l'administration] L'annuaire de l'�cole est sur \url{http://intranet/annuaire/}.

\item[Je cherche le num�ro de portable d'un X] : \url{http://www.polytechnique.org}.

\item[J'aimerais �tre un geek moi aussi !] J'apprends par c\oe ur \url{www.copinedegeek.com}.

\end{description}
