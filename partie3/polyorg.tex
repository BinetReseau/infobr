\subsection{Polytechnique.org}
\urllink{Polytechnique.org} est une association loi 1901 compos�e d'�l�ves et d'Anciens �l�ves
 ind�pendante de l'administration de l'�cole (et donc des domaines \server{polytechnique.fr}
 et \server{polytechnique.edu}).

Le but de l'association est la mise � disposition des X d'outils
ayant un rapport avec l'Internet, entre autres :
\begin{itemize}
  \item des redirections mails nombreuses (adresses suppl�mentaires) et � vie ;
  \item un serveur de news (comme les br.*), ouvert aux Anciens et aux non-plat�liens ;
  \item une facilitation des contacts vers les Anciens et les camarades de promotion ;
  \item une lettre mensuelle, pour s'informer sur l'actualit� de la communaut� polytechnicienne ;
  \item des annonces d'�v�nements ;
  \item des services d'h�bergement pour les groupes et binets, notemment des noms de domaine (via \server{www.po\-ly\-tech\-ni\-que.net}) et des listes de diffusion (\mail{br2005@po\-ly\-tech\-ni\-que.org}, par exemple).
\end{itemize}
Si tu veux d�couvrir les autres services de l'association ou savoir
comment les utiliser, tu peux aller sur la page
\urllink{https://www.polytechnique.org/Xorg/Xorg} (accessible depuis
le lien Documentations dans le menu de \urllink{Po\-ly\-tech\-ni\-que.org}
quand tu est connect�).

Par ailleurs, les filtres antivirus et antispam appliqu�s � sur les mails sont tr�s efficaces (99\% de rep�rage correct), et polytechnique.org te conseille donc de mettre en place la redirection suivante :
\mail{prenom.nom@polytechnique.edu} 
redirig�e sur \mail{prenom.nom(.promo)@po\-ly\-tech\-ni\-que.org}, 
elle-m�me redirig�e vers \mail{login@poly(.po\-ly\-tech\-ni\-que.fr)}. 
Pour effectuer ces redirections, connecte-toi sur les pages suivantes :
\begin{itemize}
  \item pour \mail{@polytechnique.edu} : \urllink{https://www.mail.polytechnique.edu} ;
  \item pour \mail{@polytechnique.org} : \urllink{https://www.polytechnique.org} ;
  \item pour \mail{@poly} : \urllink{http://poly.polytechnique.fr}.
\end{itemize}
Cela est expliqu� plus en d�tails sur la page
\urllink{https://www.polytechnique.org/Xorg/Re\-di\-rec\-tion\-Mails}.

Ces outils sont tr�s utiles, et faciles � s'approprier, que
ce soit pour toi, pour tes binets, ou pour (dans le
futur) garder contact avec la communaut� polytechnicienne. Rejoins
les 15000 camarades d�j� inscrits !

En cas de probl�me, n'h�site pas � contacter
\mail{contact@polytechnique.org}.
