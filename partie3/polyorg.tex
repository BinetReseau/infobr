\subsection{Polytechnique.org}
\urllink{Polytechnique.org} est une association loi 1901 compos�e d'�l�ves et d'anciens �l�ves
 ind�pendante de l'administration de l'�cole (et donc des domaines \server{polytechnique.fr}
 et \server{polytechnique.edu}).

Le but de l'association est la mise � disposition des X d'outils
ayant un rapport avec l'Internet. En particulier, parmi ces outils
il y a :
\begin{itemize}
  \item des redirections mails nombreuses (adresses suppl�mentaires) et � vie ;
  \item des services de news comme le binet r�seau, mais ouverts aux anciens, et aux non plat�liens ;
  \item des contacts ais�s vers les anciens, les camarades de promotion ;
  \item une lettre mensuelle, pour publier des informations qui toucheront tous les polytechniciens ;
  \item des annonces d'�v�nements ;
  \item des services d'h�bergement pour les groupes et binets, des noms de domaine (via \server{www.po\-ly\-tech\-ni\-que.net}) ;
  \item des listes de diffusion de mails (\mail{br2005@polytechnique.org}, par exemple).
\end{itemize}

Par ailleurs, les filtres antivirus et antispams de polytechnique.org �tant assez efficaces, nous te conseillons de mettre en place la redirection
suivante : adresse \mail{prenom.nom@polytechnique.edu} redirig�e sur l'adresse \mail{prenom.nom(.promo)@polytechnique.org}, elle-m�me redirig�e vers
l'adresse \mail{login@poly(.polytechnique.fr)}. Pour effectuer ces redirections, connecte-toi sur les pages suivantes :
\begin{itemize}
  \item pour l'adresse \mail{@polytechnique.edu} : \urllink{https://www.mail.polytechnique.edu/} ;
  \item pour l'adresse \mail{@polytechnique.org} : \urllink{http://www.polytechnique.org} (elle est dr�le celle-l�, hein ?) ;
  \item pour l'adresse \mail{@poly} : \urllink{http://poly.polytechnique.fr}.
\end{itemize}

Tu remarqueras tr�s rapidement que ces outils sont tr�s utiles, que
ce soit pour toi personnellement ou pour tes binets et pour, dans le
futur, garder contact avec la communaut� polytechnicienne: rejoins
les 13000 camarades d�j� inscrits !
