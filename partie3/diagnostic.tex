\label{diagnostic}
Vous n'avez plus Internet ? Voici quelques tests usuels qui permettent de déterminer tout seul précisément d'où vient le problème. Cela est utile avant de contacter un BRman.\footnote{Ces instructions sont identiques à celles disponibles sur l'InfoBR en ligne, qui seront peut-être plus à jour.}

\subsection{Obtenir son adresse IP}
Pour savoir quelle étape de la connexion à Internet pose problème, il vous faut votre adresse IP. Il s'agit d'une suite de 4 nombres entre $0$ et $255$ séparés par des points : par exemple l'adresse IP du proxy est $129.104.247.2$. Après avoir pris soin de ne pas être connecté en Wifi et en filaire à la fois :

\subsubsection{Linux}
Ouvrir un terminal et lance la commande :
\cmdline{ip addr | grep "inet " | sed "s/\textasciicircum{}.*inet \textbackslash{}([\textasciicircum\space]*\textbackslash{}) .*\$/\textbackslash{}1/" | grep -v 127.0.0.}

Si ip n'est pas installé, alors tu peux utiliser :
\cmdline{ifconfig | grep "inet adr" | sed "s/\textasciicircum{}.*inet adr:\textbackslash{}([\textasciicircum{}\space{}]*\textbackslash{}) .*\$/\textbackslash{}1/" | grep -v 127.0.0.}
si nécessaire en tant que root.
 
\subsubsection{Windows}
Ouvrir l'invite de commande (Windows+R, taper cmd puis entrée.)
\cmdline{ipconfig | findstr /I ipv4}
 
\subsubsection{Mac}
Par l'interface graphique
Ouvrir les Préférences réseaux, et chercher la ligne qui contient l'adresse ip
En ligne de commande
Ouvre un terminal et tape la commande suivante :
\cmdline{ifconfig | grep "inet " | sed "s/\textasciicircum.*inet \textbackslash{}([\textasciicircum\space]*\textbackslash{}) .*\$/\textbackslash{}1/" | grep -v 127.0.}

\subsection{Diagnostic}
Le tableau suivant permet de savoir quelle est l'étape qui bloque:

\begin{tabu}{|X[1,l,m,$]|X[4,j,m]|X[3,j,m]|}%$
\hline 
IP & Autres symptômes & Procédure à suivre \\ 
\hline 
169.… & Pas d'Internet du tout & Débrancher et rebrancher les deux bouts du câbles ethernet. Oui ça paraît bête, mais ça marche. \\ 
\hline 
192.168.… & Redirection systématique vers l'InfoBR & Configurer le $802.1$X (voir InfoBR en ligne) \\ 
\hline 
129.104.… & Seul frankiz et quelques sites binets fonctionnent & Configurer le proxy (voir InfoBR en ligne) \\ 
\hline 
129.104.… & Tout Internet fonctionne y compris \urllink{http://chocapix.eleves.polytechnique.fr} mais pas \urllink{http://chocapix/} & Configurer les domaines de recherche. \\ 
\hline 
\end{tabu}

Si vous avez un doute, rien de tel que de reprendre la procédure depuis le début.

\subsection{Toujours un problème ?}
Cette page est trop synthétique pour être exhaustive. Si malgré les tests proposés des problèmes réseaux persistent, n'hésitez pas à contacter :
\begin{itemize}
    \item pour les élèves, le support du BR à \mail{support@eleves.polytechnique.fr}
    \item pour les autres, le support de la DSI à \mail{support@polytechnique.fr}
\end{itemize}
en détaillant ce qui fonctionne et ce qui est cassé, et en citant les résultats des tests de cette page.
