\bghdr{images/fond-win}

%\begin{center}
%\includegraphics{images/logo_Windows}
%\end{center}

\subsection{Configuration sous Microsoft Windows}

Cette section d\'ecrit comment configurer un ordinateur tournant sous Windows XP ou sous Windows Vista. Si tu poss\`ede une autre version de Windows,
nous t'invitons \`a regarder directement la section sur les licences MSDNAA en page \pageref{msdnaa}, ou alors\dots \`a te d\'ebrouiller ! ;-)

\subsubsection{Configuration IP}

\begin{itemize}

\item \textbf{Sous Windows XP :} va dans le \menu{Menu D\'emarrer}, \menu{Panneau de configuration} et double-clique sur \menu{Connexions r\'eseau} puis sur \menu{Connexion au r\'eseau local}. Clique enfin sur \menu{Propri\'et\'es}.

\item \textbf{Sous Windows Vista :} va dans le \menu{Menu D\'emarrer}, \menu{Panneau de configuration}, \menu{R\'eseau et Internet}, \menu{Centre r\'eseau et partage}. L\`a, dans le menu \`a gauche, clique sur \menu{G\'erer les connexions r\'eseau}, puis clique droit sur \menu{Connexion au r\'eseau local}, et enfin \menu{Propri\'et\'es}\footnote{A ce stade, ainsi qu'\`a plusieurs autres \'etapes de ce tutoriel, Windows Vista doit normalement t'afficher un message te demandant de confirmer l'action que tu viens d'effectuer. Donc, tu confirmes, et cela \`a chaque fois !}.

\end{itemize}



%\flimage{images/win_connexion_icone}{0.15}{l} Va dans le \menu{Menu
%D\'emarrer}, \menu{Panneau de configuration} et double-clique sur
%\menu{Connexions r\'eseau} puis sur \menu{Connexion au r\'eseau local}.
%Clique enfin sur \menu{Propri\'et\'es}.\\

%Dans cette fen\^etre, coche les trois cases \menu{Client pour les
%r\'eseaux Microsoft}, \menu{Partage de fichiers} et \menu{Protocole
%Internet (TCP/IP)}:

\imagepos{images/win_config_connexion2}{0.5}{Configurer la connexion au r\'eseau local}{!h}

\imageref{images/win_config_ip}{0.5}{Configuration IP --- Propri\'et\'es de protocole Internet (TCP/IP)}{!ht}{config:win:IP1}
%\imageref{images/win_config_ip2}{0.71}{Configuration de la connexion
%au r\'eseau local et propri\'et\'es du TCP/IP}{!ht}{config:win:IP1}

S\'electionne ensuite la ligne \menu{Protocole Internet Version 4 (TCP/IPv4)}
\footnote{\menu{Protocole Internet (TCP/IP)} pour certaines versions de Windows XP},
puis clique sur le bouton \menu{Propri\'et\'es} qui vient de se
d\'egriser. Tu tombes alors sur l'\'ecran de configuration de ta
connexion vers l'ext\'erieur.

%% \newpage
Coche alors les cases \menu{Utiliser l'adresse IP suivante} et \menu{Utiliser l'adresse de serveur DNS suivante} et remplis les cinq champs d'IP. Tu
trouveras toutes les valeurs d'IP n\'ecessaires pour la configuration en page~\pageref{tableau:mon_IP} ; aide-toi de la capture
d'\'ecran~\ref{config:win:IP1} pour les placer. Si une partie d'IP est blanche sur cette capture, c'est qu'elle t'est personnelle et que tu dois la
calculer !

Ensuite, clique sur le bouton \menu{Avanc\'e}, puis sur l'onglet
\menu{DNS} en haut.

\imageref{images/win_config_dns2}{0.5}{Configuration DNS}{!ht}{config:win:IP2}

Il n'y a plus qu'\`a remplir les diff\'erents champs comme sur la
capture d'\'ecran suivante, avec le bouton \menu{Ajouter} et les
fl\`eches pour r\'eordonner les \'el\'ements.


\subsubsection{Le domaine Windows}

\paragraph{Qu'est ce que c'est ?}
Le domaine Windows est un syst\`eme d'automatisation de la
configuration de plusieurs ordinateurs sous Windows situ\'es sur le
m\^eme r\'eseau. En fait, c'est un outil d'administration, con\c{c}u par
exemple pour des entreprises o\`u un service informatique doit g\'erer
de nombreuses machines; il permet d'appliquer des modifications de
configuration \`a toutes les machines du domaine directement depuis un
serveur. Le BR poss\`ede un serveur d\'edi\'e au domaine Windows,
\server{enez}.

Le domaine met \`a jour automatiquement Windows et l'antivirus \`a partir d'\server{enez} (tr\`es rapide car tu n'as pas besoin de r\'ecup\'erer des fichiers
en dehors de l'\'ecole!). Il configure le \emph{firewall} (pare-feu: syst\`eme de protection contre les \'eventuelles attaques par le r\'eseau) Windows, mais
il est toujours possible de le d\'esactiver si tu pr\'ef\`eres un autre \emph{firewall}. En bref, il permet de simplifier \`a l'extr\^eme la mise \`a jour
continuelle de l'ordinateur.


\paragraph{Alors, domaine ou pas domaine ?} Soit tu choisis de te
mettre sur le domaine Windows, et tu vas alors au paragraphe
\guillemotleft~Inscription sur le domaine Windows~\guillemotright.

%% \newpage
\textbf{Avantages :}
\begin{itemize}
\item Windows est mis \`a jour automatiquement ; tu as toujours les
derni\`eres corrections de s\'ecurit\'e et un pare-feu correctement
configur\'e. Donc tu es mieux prot\'eg\'e contre les intrusions.
\item Surtout, tu n'as plus \`a t'en occuper, presque tout est automatique.
\end{itemize}

\textbf{Inconv\'enients :}
\begin{itemize}
  \item Tu d\'el\`egues une partie des droits d'administration de ta machine au BR
        (tout ce qui concerne la s\'ecurit\'e du r\'eseau en particulier).
        Cependant, si tu ne sais pas le faire, c'est plut\^ot un avantage
        de laisser le BR s'en occuper \`a ta place.
  \item Cela ne marche qu'avec Windows 2000, Windows XP Pro ou Windows Vista Business.
        On te rappelle que tu peux facilement, gratuitement et l\'egalement passer \`a
        Windows XP Pro ou bien \`a Windows Vista Business (section sur les licences
        MSDNAA en page \pageref{msdnaa}).
\end{itemize}

Bien s\^{u}r, tu peux sortir du domaine \`a tout instant, et effectuer manuellement les r\'eglages n\'ecessaires \`a la s\'ecurit\'e de ton ordinateur.

Soit tu choisis de configurer toi-m\^eme ton ordinateur, et tu peux passer
directement \`a la section \guillemotleft Installation de l'antivirus
\guillemotright. Tu trouveras les informations n\'ecessaires \`a la configuration
manuelle du pare-feu et du proxy pour \app{Windows Update} en annexe \`a la
fin de cette section, en page \pageref{horsdomaine}.

\textbf{Avantage :} Tu es le seul \`a t'occuper de la gestion de ton ordinateur.

\textbf{Inconv\'enient :} Tu es le seul \`a t'occuper de la gestion de ton ordinateur. ;-)
S'il devient un foyer pour virus, sache que nous avons les moyens de l'isoler
pour \'eviter toute propagation.

\begin{center}
  \fbox{
    \begin{minipage}{.7\textwidth}
      \begin{center}
Le BR te conseille \emph{tr\`es fortement} de te mettre sur le domaine
et de choisir l'installation simplifi\'ee !
      \end{center}
    \end{minipage}
  }
\end{center}


\paragraph{Inscription sur le domaine Windows}

On te rappelle que tu ne peux t'inscrire sur le domaine que si tu utilise
Windows 2000, Windows XP Pro ou Windows Vista. Si tu poss\`ede Windows XP
Familial, Windows Vista Home ou encore une version ant\'erieure de Windows,
tu dois effectuer toi-m\^eme tes r\'eglages de pare-feu et de proxy
\app{Windows Update}. R\'ef\`ere-toi pour cela \`a l'annexe ad hoc \`a la fin de
cette section, en page \pageref{horsdomaine}.

La proc\'edure d'inscription est la suivante :
\begin{itemize}

\item \textbf{Sous Windows XP :} Clique sur le \menu{Menu D\'emarrer} puis fais un clic-droit sur
\menu{Poste de travail} et choisis \menu{Propri\'et\'es}. Ensuite, s\'electionne l'onglet \menu{Nom de l'ordinateur} et clique le bouton \menu{Modifier}.

\item \textbf{Sous Windows Vista :} Clique sur \menu{Menu D\'emarrer}, puis fais un clic-droit sur \menu{Ordinateur}, \menu{Propri\'et\'es}. L\`a s\'electionne \menu{Param\`etres syst\`eme avanc\'es}, onglet \menu{Nom de l'ordinateur}, puis clique sur le bouton \menu{Modifier}.

\end{itemize}

Dans la case \menu{Nom de l'ordinateur}, rentre ton pseudo, puis coche la case \menu{domaine} et
rentre \urllink{windows.eleves.polytechnique.fr}. Note bien que l'inscription au domaine te sera
refus\'ee par le serveur si quelqu'un d'autre utilise d\'ej\`a le m\^eme nom d'ordinateur que toi. Par
cons\'equent, essaie d'opter pour un pseudo qui t'identifie de fa\c{c}on claire et unique, par exemple
\cmd{NOM\_PRENOM} \footnote{Les Jean Dupont et les Julien Thomas sont pri\'es de trouver autre chose
;-)}.

\imagepos{images/win_config_domaine}{0.5}{S'inscrire sur le domaine windows}{!ht}

\begin{center}
\begin{tabular}{ll}
 \parbox{.45\textwidth}{
  et si tu es rouje 2006 :
  \begin{description}
    \item[Nom] rouje06
    \item[Mot de passe] rouje.2006
  \end{description}
  }
 & \parbox{.45\textwidth}{
  Si tu es j\^one 2007, tu rentres :
  \begin{description}
    \item[Nom] jone07
    \item[Mot de passe] jone.2007
  \end{description}
  }
\\
\end{tabular}
\end{center}

\emph{Attention, ces identifiants servent juste \`a t'inscrire sur le
domaine. Pour utiliser ton ordinateur, tu devras rentrer au
d\'emarrage les m\^emes nom d'utilisateur et mot de passe que tu avais
avant d'\^etre sur le domaine !}



%\paragraph{Installation personnalis\'ee} --- configuration manuelle

%\subparagraph{Configuration antivirus} Le BR, concern\'e par la
%s\'ecurit\'e du r\'eseau, te propose un antivirus pour lequel tu n'auras
%pas \`a payer la license pour obtenir les mises \`a jour. Bien s�r,
%libre \`a toi d'utiliser ton anti-virus personnel ; cependant il sera
%\`a ta charge de le mettre \`a jour tr\`es r\'eguli\'erement. Pour cela
%utilise comme proxy : \urllink{http://kuzh} sur le port 8080.

%\emph{Installation de l'anti-virus du BR}\ : Commence par
%d\'esinstaller tous les antivirus ou firewalls que tu pourrais avoir
%comme expliqu\'e dans le paragraphe \guillemotleft~Installation simplifi\'ee
%--- configuration automatique~\guillemotright .

%Puis ouvre ton explorateur Windows et tape :
%\urllink{$\backslash\backslash$enez$\backslash$antivirus}
%et double-clique sur le fichier \file{Symantec.exe}.

%Ce package contient le param\'etrage de la mise \`a jour automatique de
%Windows sur le serveur de l'\'ecole. Attends la fin de l'installation
%et c'est fini ! Maintenant, tu n'as plus \`a toucher \`a l'antivirus,
%normalement il sera mis \`a jour automatiquement.

%\subparagraph{Configuration firewall}

%Si tu as Windows XP avec le SP2 install\'e, tu as un firewall
%automatiquement activ\'e et facile d'utilisation. En effet, \`a chaque
%fois qu'un programme tentera d'aller pour la premi\`ere fois sur
%Internet, il te demandera si tu veux le laisser faire ou non, comme
%dans la capture~\ref{config:win:firewall}.

%\imageref{images/win_firewall}{0.8}{Un programmme --- ici GuildFTP
%--- demande \`a acc\'eder au r\'eseau}{!ht}{config:win:firewall}

%Le firewall commercial \app{ZoneAlarm}, ind\'ependant de Windows,
%fonctionne sur le m\^eme principe. Tu peux le trouver sur \xshare.

%Si tu pr\'ef\`eres utiliser le firewall int\'egr\'e \`a Windows XP (sans le
%SP2) ou \`a Windows Server 2003, il te faudra le configurer en d\'etail.
%Va dans le \menu{Menu D\'emarrer}, \menu{Param\`etres} et clique sur
%\menu{Connexions R\'eseau}. Choisis la connexion qui est utilis\'ee par
%ton ordinateur (souvent il n'y en a qu'une, ou alors une seule est
%activ\'ee) et double-clique dessus. Clique sur \menu{Propri\'et\'es} en
%bas \`a gauche, puis sur l'onglet \menu{Avanc\'e} et rentre dans le menu
%de \menu{Param\`etres} du \menu{Pare-feu Windows}. Il te faudra alors
%ajouter manuellement tous les ports que tu veux ouvrir sur
%l'ext\'erieur. Pour cela, clique sur \menu{Ajouter}, et remplis la
%bo�te de dialogue en t'aidant de la capture
%d'\'ecran~\ref{config:win:ouvrir_port}; mets le num\'ero du port que tu
%veux ouvrir, par exemple 5050, 5053 et 5055 en TCP pour \app{qRezix}
%et 21 en TCP pour ton FTP.

%\imageref{images/win_config_firewall}{0.7}{Ouvrir un port dans le firewall %Windows}{!ht}{config:win:ouvrir_port}

%Comme tu peux le constater, il est beaucoup plus pratique d'aller
%sur le domaine et de laisser le SP2 faire le gros du boulot \`a ta
%place :-).


\subsubsection{Installation de l'antivirus}
\label{antivirus} Le BR, concern\'e par la s\'ecurit\'e du r\'eseau, te propose un antivirus \footnote{En
l'occurence, l'antivirus utilis\'e est \app{Symantec Antivirus 10.2}.} pour lequel tu n'auras pas \`a
payer la licence pour obtenir les mises \`a jour. Bien s\^{u}r, libre \`a toi d'utiliser ton antivirus
personnel ; cependant il sera \`a ta charge de le mettre \`a jour tr\`es r\'eguli\`erement. Pour cela,
utilise comme proxy : \urllink{kuzh.polytechnique.fr} sur le port 8080 (Il y a aussi des fois o\`u il 
faut rentrer \urllink{http://kuzh.polytechnique.fr}).

%Version pour vista \`a taper
Tout d'abord, \emph{d\'esinstalle tous les antivirus que tu pourrais
avoir !} Dans le \menu{Menu D\'emarrer}, va dans \menu{Panneau de
Configuration}, \menu{Ajout/Suppression de Programmes} et
d\'esinstalle si tu l'as Symantec Antivirus, McAfee Antivirus, Norton
Antivirus, ou tout autre antivirus.

Ensuite, il s'agit de t\'el\'echarger la bonne version de l'antivirus :
\begin{itemize}

\item Si tu as Windows Vista\dots
\begin{itemize}
\item en version 32 bits : \urllink{ftp://enez/antivirus/Win32\_Vista.zip}.
\item en version 64 bits : \urllink{ftp://enez/antivirus/Win64\_Vista.zip}.
\end{itemize}

\item Si tu as Windows XP ou une autre version de Windows\dots
\begin{itemize}
\item en version 32 bits : \urllink{ftp://enez/antivirus/Win32\_PasVista.zip}.
\item en version 64 bits : \urllink{ftp://enez/antivirus/Win64\_PasVista.zip}.
\end{itemize}
\end{itemize}

Si tu ne sais pas si ton Windows est 64 ou 32 bits, le plus problable est qu'il
soit 32 bits, donc tu peux commencer \`a essayer avec cette version.

Une fois le fichier ad\'equat t\'el\'echarg\'e, ouvre l'archive compress\'ee \file{Win\?\?\_*.zip}, d\'ecompresse \emph{tous les fichiers} dans un dossier, et
enfin ex\'ecute le fichier \file{setup.exe} qui vient d'apparaitre dans ce dossier. A partir de l\`a, tu n'as plus qu'\`a suivre les instructions du
programme d'installation. Tu n'auras alors plus \`a toucher \`a l'antivirus, celui-ci sera mis \`a jour automatiquement.\footnote{Pour des raisons historiques, cette page porte le num\'ero 31. L'explication se trouve en page 11.}

\subsubsection{Configuration web (proxy)}

\imageref{images/win_config_proxy}{0.5}{Configuration du proxy}{!ht}{config:win:proxy}

M\^eme si tu n'utilises pas \app{Internet Explorer} comme client web, Windows et d'autres programmes
utilisent ses param\`etres, notamment \app{Windows Update}. Donc lance \app{Internet Explorer} et va
dans le menu \menu{Outils}, \menu{Options Internet}, puis sur l'onglet \menu{Connexions} de la
nouvelle fen\^etre et enfin sur \menu{Param\`etres r\'eseau} dans le bas de la fen\^etre. L\`a, coche
uniquement la case \menu{Utiliser un script de configuration automatique}, puis remplis le champs
\menu{Adresse} avec \urllink{http://frankiz/proxy.pac}. Tu dois alors avoir quelque chose qui
ressemble \`a la capture d'\'ecran~\ref{config:win:proxy}.
% Pour le RTFIBRp11
\setcounter{page}{33}
Une fois que tu as fait \c{c}a, tu n'as plus forc\'ement besoin d'\app{Internet Explorer}, tu peux donc utiliser un autre navigateur, comme \app{Mozilla
Firefox}, disponible sur \xshare, qui est plus s\'ecuris\'e. Mais ce n'est pas une garantie ultime --- tu es le premier garant de la s\'ecurit\'e de ton
ordinateur, en n'ouvrant pas tous les fichiers qui te passent sous la main
--- mais tu seras sensiblement plus en s\'ecurit\'e.


Si tu utilises \app{Mozilla Firefox}, il faut que tu fasses, en plus, le m\^eme r\'eglage de proxy pour
ce navigateur. Pour ce faire, lance \app{Mozilla Firefox}, et va dans le menu \menu{Outils},
\menu{Options...}. L\`a, s\'electionne la rubrique \menu{Avanc\'e}, onglet \menu{R\'eseau}, et clique sur
\menu{Param\`etres}. La case \`a cocher est alors \menu{Adresse de configuration automatique du proxy},
et l'adresse \`a indiquer est la m\^eme que pr\'ec\'edemment : \urllink{http://frankiz/proxy.pac}.


% Pour le RTFIBRp11 pour un saut de page si nécéssaire ce qui n'était pas le cas en 2008

%\noindent\rule{.4\textwidth}{.4pt}

%\vfill %\pagebreak

\subsubsection{Configuration mail}

La DSI met \`a ta disposition une bo\^{i}te aux lettres \'electronique sur
le serveur \server{poly} ; cette section t'explique comment
configurer \app{Outlook Express}, et \app{Windows Mail} pour y avoir acc\`es. Tu peux bien
s\^{u}r utiliser \app{Thunderbird} si tu pr\'ef\`eres, les donn\'ees \`a rentrer
pour la configuration sont les m\^emes ; quelques d\'etails sont donn\'es
dans le Wikix sur \fkz. De plus, tu trouveras des explications plus
d\'etaill\'ees dans le manuel r\'edig\'e par la DSI.
La proc\'edure suivante fonctionne aussi avec \app{Windows Mail}.
Lance \app{Outlook Express} et va dans le menu \menu{Outils},
\menu{Comptes\ldots}. Clique sur le bouton \menu{Ajouter\ldots} en
haut \`a droite, puis sur \menu{Courrier\ldots}, pour \app{Windows Mail} c'est sur compte de messagerie qu'il faut cliquer, avat d'appuyer sur suivant.

% Pour le RTFIBRp11

\setcounter{page}{12}

Remplis les \'ecrans de configuration suivants avec ces donn\'ees :
\begin{description}
  \item[Nom complet] ton nom (\guillemotleft~Martin Durand~\guillemotright , par exemple)
  \item[Adresse de messagerie] de la forme \mail{prenom.nom@polytechnique.edu}
  \item[Type de serveur de messagerie pour le courrier entrant] \menu{POP3}
  \item[Serveur de messagerie pour le courrier entrant] \server{poly.polytechnique.fr}
  \item[Serveur de messagerie pour le courrier sortant] \server{poly.polytechnique.fr}
  \item[Nom du compte] ton login \server{poly} (les huit premi\`eres lettres de ton nom en g\'en\'eral, si tu ne t'en souviens pas ne t'en fait pas on devrait te le redonner en cours d'informatique lors de ton premier TD. Si tu es vraiment press\'e va voir le bureau login de la DSI.)
  \item[Mot de passe] ton mot de passe \server{poly} ;
       v\'erifie bien que la case \menu{M\'emoriser le mot de passe} est coch\'ee.
\end{description}

Voil\`a, clique sur \menu{Continuer}, \menu{Terminer}.

Tu te retrouves alors sur la fen\^etre \menu{Comptes Internet}. Va sur
l'onglet \menu{Courrier}, clique sur le compte que tu viens de cr\'eer
puis sur \menu{Propri\'et\'es}. Clique sur l'onglet \menu{Avanc\'e} et
configure comme sur la capture~\ref{config:win:mail} ; en
particulier, coche la seconde case \menu{Ce serveur n\'ecessite une
connexion s\'ecuris\'ee (SSL)}.

Comme \c{c}a, tu peux d\'esormais recevoir des mails, avec une liaison
s\'ecuris\'ee vers \server{poly} pour que personne ne puisse les
intercepter.


\imageref{images/win_config_mail_avance}{0.5}{Configuration avanc\'ee
des serveurs mail}{!h}{config:win:mail}

\subsubsection{Configuration newsgroups}
Reporte-toi a la page~\pageref{newsgroups} pour la description et des d\'etails de fonctionnement des newsgroups \`a l'X.

Comme pour les mails, nous d\'ecrivons la configuration de \app{Outlook Express}, mais elle est sensiblement \'equivalent pour \app{Thunderbird}. Lance
\app{Outlook Express} et va dans le menu \menu{Outils}, \menu{Comptes\ldots}. Clique sur le bouton \menu{Ajouter\ldots} en haut \`a droite,
\menu{News\ldots}. Remplis les \'ecrans de configuration suivants avec ces donn\'ees :

\begin{description}
  \item[Nom complet] ton nom !
  \item[Adresse de messagerie] de la forme \mail{prenom.nom@polytechnique.edu}
  \item[Serveur de news (NNTP)] \fkz ; v\'erifie \`a ce moment que la case
       \menu{Connexion \`a mon serveur de news requise} n'est pas coch\'ee.
\end{description}

Voil\`a, clique sur \menu{Continuer}, \menu{Terminer} et tu es abonn\'e
au serveur news des \'el\`eves.

Quand tu fermeras la fen\^etre `Comptes Internet', il va te demander \`a
quels newsgroups tu veux t'abonner, tu n'auras qu'\`a s\'electionner
ceux qui t'int\'eressent. Reporte-toi \`a la page \pageref{newsgroups}
pour plus d'infos sur les newsgroups auquels t'abonner !

Si tu veux t'inscrire \`a d'autres serveurs news, refais cette
proc\'edure en rentrant le nom du serveur qui t'int\'eresse \`a la place
de \fkz, par exemple pour acc\'eder aux news externes
\server{polynews.polytechnique.fr}.


\subsubsection{Configuration FTP}

\paragraph{Client FTP}

Le BR te conseille \app{FileZilla} ou \app{SmartFTP}. Pour installer l'un des deux, t\'el\'echarge-le sur \xshare et double-clique sur l'installeur.

Finis l'installation, et tu peux aller sur tous les FTP du r\'eseau
facilement et rapidement.

\paragraph{Serveur FTP}

Tu verras rapidement que tout le monde \`a l'X poss\`ede un serveur FTP
afin de partager les diff\'erents projets, les films du JTX, ses
photos, etc. Donc il est quasiment indispensable que tu en installes
un.

Parmi les plus simples on trouve \app{FileZilla Server} et \app{GuildFTP}, qui sont libres de surcro\^{i}t. Expliquer les d\'etails de la configuration est un peu long pour l'InfoBR, mais tu trouveras sur le Wikix un tutoriel expliquant cela : \\
\urllink{http://wikix.polytechnique.org/eleves/wikix/FTP}.

\subsubsection{Autres logiciels utiles}

\begin{itemize}
  \item qRezix : Un programme d\'evelopp\'e par le BR pour faciliter la vie sur le r\'eseau,
                  \`a r\'ecup\'erer sur \xshare. Pour plus de d\'etails, voir le paragraphe consacr\'ee
                  \`a \app{qRezix} \`a la page \pageref{qrezix}.
  \item XChat : Un client IRC directement issu du monde Unix.
                 Tu peux te reporter \`a la page \pageref{irc} pour plus d'infos sur l'IRC.
  \item WinSCP : Un logiciel pratique qui te permet de te connecter en salle info.
                  Tu peux le r\'ecup\'erer lui aussi sur \xshare ;
                  son fonctionnement est expliqu\'e en d\'etails dans le Wikix. Voir aussi \app{Putty}, aussi dans le \xshare.
  \item vlc : Un logiciel qui te permettra de recevoir la t\'el\'evision directement dans ton casert, afin d'\^etre vraiment s\^{u}r d'avoir autre chose \`a faire que travailler les veilles de p\^ales. Configuration page \pageref{TV}.
\end{itemize}


\subsubsection{Obtenir un Windows gr\^{a}ce aux licences MSDNAA}

\label{msdnaa} Les accords n\'egoci\'es par le BR avec Microsoft dans le cadre de MSDNAA donnent \`a chaque X le droit de poss\'eder une version de Windows
XP Pro ou de Windows Vista Business gratuite et l\'egale, ainsi que les licences pour la plupart des logiciels de la soci\'et\'e (quasiment tous, sauf
Office et les jeux). La seule condition \`a remplir est d'\^etre \'etudiant sur le plat\^{a}l au moment de l'installation du logiciel ; tu pourras ensuite le
garder sur ton PC m\^eme apr\`es ton d\'epart de l'X.

La proc\'edure pour obtenir les logiciels et les cl\'es correspondantes
est la suivante :
\begin{itemize}

\item Va d'abord sur \fkz, et connecte-toi, puis clique sur le lien \menu{Licences MSDNAA} qui se trouve dans la bo\^ite \menu{Liens utiles}. S\'electionne le logiciel que tu souhaites installer et valide ta demande, tu recevras ta cl\'e par e-mail. Facile ! Si jamais le logiciel n'est pas dans la liste propos\'ee, c'est soit qu'il n'y a pas besoin de cl\'e --- c'est le cas de beaucoup des logiciels autres que Windows, soit qu'on a oubli\'e de l'y mettre ; dans ce cas, \'ecris \`a \mail{msdnaa@frankiz} pour qu'on t'attribue manuellement une cl\'e.

\item Maintenant que tu as ta cl\'e, il faut t\'el\'echarger le logiciel proprement
dit. Pour cela, connecte-toi par FTP sur \urllink{ftp://enez/msdnaa/} avec ton client FTP pr\'ef\'er\'e.
Alors, selon le logiciel, tu vas r\'ecup\'erer soit une image du CD de type \file{logiciel.iso} (\`a
graver ou \`a utiliser avec \app{Daemon Tools}), soit directement le contenu du CD.
\end{itemize}

Pour Windows, les fichiers iso t\'el\'echarger sont les suivants :
\begin{itemize}
\item \textbf{Pour Windows XP :}
\urllink{ftp://enez/msdnaa/win\_xp/french/win\_xp\_pro\_sp2\_fr.iso}.

\item \textbf{Pour Windows Vista :}
L\`a, il y a deux possibilit\'es :
\begin{description}
\item[Si tu as un processeur 64 bits :] L'image \`a t\'el\'echarger est \\
\urllink{ftp://enez/msdnaa/win\_vista/dvd\_64bits/french/vista\_dvd\_x64\_fr.iso}. Attention, cette
image iso est \`a graver sur un DVD. On remarque aussi que parfois, m\^eme sur un processeur 64 bits il
vaut mieux choisir la version 32 bits de Windows (celle qui est dans le prochain item) pour des
raisons de disponibilit\'e de pilotes.
\item[Si tu as un processeur 32 bits :] L\`a, tu peux choisir de graver :
\begin{itemize}
\item soit un DVD : \\
\urllink{ftp://enez/msdnaa/win\_vista/dvd\_32bits/french/vista\_dvd\_x86\_fr.iso}.
\item soit quatre CDs (!) : \\
\urllink{ftp://enez/msdnaa/win\_vista/cd\_32bits/french/vista\_cd\( N\) \_x86\_fr.iso}, N variant
de 1 \`a 4.

\end{itemize}
\end{description}

\end{itemize}

Les versions anglophones de Windows XP et Vista sont \'egalement disponibles sur \server{enez}.

Ainsi, si tu as achet\'e un ordinateur sans OS (et ainsi \'economis\'e environ 150 \euro), tu vas chez un copain, et fais les demandes et grave le CD chez lui. Si tu as encore des questions, plus de d\'etails sont donn\'es dans le Wikix et le WikiBR de \fkz.


\subsubsection{Annexe pour ceux n'utilisant pas le domaine Windows}

\label{horsdomaine} \emph{Cette sous-section ne concerne pas les gens qui ont choisi de s'inscrire sur le domaine.}

\paragraph{Pare-feu} Si tu as Windows XP avec le SP2 install\'e, ou \emph{a fortiori}
Windows Vista, tu as un \emph{firewall} automatiquement activ\'e et facile d'utilisation. En effet, \`a chaque fois qu'un programme tentera d'aller pour
la premi\`ere fois sur Internet, il te demandera si tu veux le laisser faire ou non. Si tu pr\'ef\`eres une protection ind\'ependante de Windows, le
\emph{firewall} commercial \app{ZoneAlarm} fonctionne sur le m\^eme principe. Tu peux le trouver sur \xshare.

Si tu pr\'ef\`eres utiliser le \emph{firewall} int\'egr\'e \`a Windows XP (sans le SP2), il te faudra le configurer en d\'etail. Va dans le \menu{Menu D\'emarrer},
\menu{Param\`etres} et clique sur \menu{Connexions R\'eseau}. Choisis la connexion qui est utilis\'ee par ton ordinateur (souvent il n'y en a qu'une, ou
alors une seule est activ\'ee) et double-clique dessus. Clique sur \menu{Propri\'et\'es} en bas \`a gauche, puis sur l'onglet \menu{Avanc\'e} et rentre dans le
menu de \menu{Param\`etres} du \menu{Pare-feu Windows}. Il te faudra alors ajouter manuellement tous les ports que tu veux ouvrir sur l'ext\'erieur. Pour
cela, clique sur \menu{Ajouter}, et remplis la bo\^ite de dialogue en t'aidant de la capture d'\'ecran~\ref{config:win:ouvrir_port} ci apr\`es; mets le num\'ero du
port que tu veux ouvrir, par exemple 5050, 5053 et 5055 en TCP pour \app{qRezix} et 21 en TCP pour ton FTP.

\paragraph{Proxy pour windows update} Il reste une derni\`ere configuration de
proxy indispensable pour que puissent se faire les mises-\`a-jour automatiques
de Windows. Il t'est fortement recommandé de le faire.

\begin{itemize}

\item \textbf{Sous Windows XP :} Fais \menu{D\'emarrer}, \menu{Ex\'ecuter}, puis
tape \cmd{cmd} dans la fen\^etre qui s'affiche. Une ligne de commande apparait,
il te suffit alors de taper : \cmd{proxycfg -p http://kuzh:8080} pour r\'egler
le proxy. Pour revenir \`a un acc\`es direct il faut taper \cmd{proxycfg -d}.

\item \textbf{Sous Windows Vista :}
Dans le menu \menu{D\'emarrer}, tape \guillemotleft~Invite de commandes~\guillemotright dans le champ \menu{Rechercher}, puis clique droit sur le lien
et choisis \menu{Ex\'ecuter en tant qu'administrateur}. Tape ensuite les commandes suivantes :

\noindent \cmdline{
C:\textbackslash{}Windows\textbackslash{}system32$>$netsh\\
netsh>winhttp\\
netsh winhttp>set proxy proxy-server="http://kuzh:8080"\\
netsh winhttp>exit\\
C:\textbackslash{}Windows\textbackslash{}system32>exit }



\end{itemize}
