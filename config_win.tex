\clearpage
\pagebreak

\begin{center}
\includegraphics{images/logo_Windows}
\end{center}

\subsection{Configuration sous Microsoft Windows}

\subsubsection{Configuration IP}

\flimage{images/win_connexion_icone}{0.15}{l}
Va dans le \menu{Menu D\'emarrer}, \menu{Panneau de configuration} et double-clique sur \menu{Connexions r\'eseau} puis sur \menu{Connexion au r\'eseau local}. Clique enfin sur \menu{Propri\'et\'es}.\\

Dans cette fen�tre, coche les trois cases \menu{Client pour les r\'eseaux Microsoft}, \menu{Partage de fichiers} et \menu{Protocole Internet (TCP/IP)}:

\vspace{1cm}

\imagepos{images/win_config_connexion2}{0.71}{Configurer la connexion au r\'eseau local}{!h}

\imagepos{images/win_config_ip}{0.71}{Configuration IP --- Propri\'et\'es de protocole Internet (TCP/IP)}{!h}

S\'electionne ensuite la ligne \menu{Protocole Internet (TCP/IP)}, puis clique sur le bouton \menu{Propri\'et\'es} qui vient de se d\'egriser. Tu tombes alors sur l'\'ecran de configuration de ta connexion vers l'ext\'erieur.

\newpage
Tu coches alors les cases \menu{Utiliser l'adresse IP suivante} et \menu{Utiliser l'adresse de serveur DNS suivante} et tu remplis les cinq champs d'IP. Tu trouveras toutes les valeurs d'IP n\'ecessaires pour la configuration en page ?? ; aide-toi du screenshot ci-dessus pour les placer. Si une partie d'IP est blanche sur le screenshot, c'est qu'elle t'est personnelle et que tu dois la calculer !

Ensuite, clique sur le bouton \menu{Avanc\'e}, puis sur l'onglet \menu{DNS} en haut.

\imagepos{images/win_config_dns2}{0.50}{Configuration IP}{!h}

Il n'y a plus qu'\`a remplir les diff\'erents champs comme sur le screenshot suivant, avec le bouton \menu{Ajouter} et les fl�ches pour r\'eordonner les \'el\'ements.

\subsubsection{Le domaine Windows}

\paragraph{Qu'est ce que c'est ?}Le domaine Windows est un syst�me d'automatisation de la configuration de plusieurs ordinateurs sous Windows situ\'es sur le m�me r\'eseau. En fait, c'est un outil d'administration, con\c{c}u par exemple pour des entreprises o� un service informatique doit g\'erer de nombreuses machines; il permet d'appliquer des modifications de configuration \`a toutes les machines du domaine directement depuis un serveur. Le BR poss�de un serveur d\'edi\'e au domaine Windows, \server{enez}.

Le domaine met \`a jour automatiquement Windows et l'antivirus \`a partir d'\server{enez} (tr�s rapide !). Il configure le firewall (pare-feu: syst\`eme de protection contre les \'eventuelles attaques par le r\'eseau) Windows, mais il est toujours possible de le d\'esactiver si on pr\'ef�re un autre firewall. En bref, il permet de simplifier \`a l'extr\^eme la mise \`a jour continuelle de l'ordinateur.

\paragraph{Alors, domaine ou pas domaine ?}
\subparagraph{Soit tu choisis de te mettre sur le domaine Windows}...\ et tu vas alors au paragraphe \og Installation simplifi\'ee --- configuration automatique \fg.

\newpage
\textbf{Avantages :}
\begin{itemize}
  \item Windows est mis \`a jour automatiquement ; tu as toujours les derniers patches de s\'ecurit\'e et un antivirus \`a jour. Donc tu es mieux prot\'eg\'e contre les virus.
  \item Surtout, tu n'as plus \`a t'en occuper, presque tout est automatique.
\end{itemize}

\textbf{Inconv\'enients :}
\begin{itemize}
  \item Tu d\'el�gues une partie des droits d'administration de ta machine au BR (tout ce qui concerne la s\'ecurit\'e du r\'eseau en particulier). Cependant, si tu ne sais pas le faire, c'est plut�t un avantage de laisser le BR s'en occuper \`a ta place.
  \item Cela ne marche qu'avec Windows 2000, Windows XP Pro ou Windows Server 2003, mais tu peux facilement et gratuitement passer \`a Windows XP Pro.
\end{itemize}

Bien s�r, tu peux sortir du domaine \`a tout instant en reprenant ta configuration et en suivant l'installation personnalis\'ee.

\subparagraph{Soit tu choisis de configurer toi-m�me ton ordinateur}...\ et tu vas alors au paragraphe \og Installation personnalis\'ee --- configuration manuelle \fg.

\textbf{Avantages :}
\begin{itemize}
  \item Tu es responsable de ton ordinateur. Si ta machine est toujours \`a jour et que tu n'attrapes aucun virus, tant mieux pour toi.
\end{itemize}

\textbf{Inconv\'enients :}
\begin{itemize}
  \item Tu es responsable de ton ordinateur. S'il devient un foyer pour virus, sache que nous avons les moyens de l'isoler pour \'eviter toute propagation.
  \item Si tu ne ma\^itrises pas le firewall, l'antivirus et autres Windows Update, \c{c}a ne sert \`a rien d'envisager cette solution.
\end{itemize}

\begin{center}
  \fbox{
    \begin{minipage}{.7\textwidth}
      \begin{center}
Le BR te conseille \emph{tr�s fortement} de te mettre sur le domaine et de choisir l'installation simplifi\'ee !
      \end{center}
    \end{minipage}
  }
\end{center}

\subsubsection{Configuration de l'ordinateur}

\paragraph{Installation simplifi\'ee} --- configuration automatique

Tout d'abord : \emph{d\'esinstalle tous les antivirus que tu pourrais avoir !} Dans le \menu{Menu D\'emarrer}, va dans \menu{Panneau de Configuration}, \menu{Ajout/Suppression de Programmes} et d\'esinstalle si tu l'as Symantec Antivirus, McAfee Antivirus, Norton Antivirus, et tout autre antivirus ou firewall.

Ensuite, tu vas t'inscrire sur le domaine. Pour cela, il faut avoir Windows XP Pro que le BR te fournit gratuitement et l\'egalement (cf. page~\pageref{xp}).

Clique sur le \menu{Menu D\'emarrer} puis fais un clic-droit sur \menu{Poste de travail} et choisis \menu{Propri\'et\'es}. Ensuite, s\'electionne l'onglet \menu{Nom de l'ordinateur} et clique le bouton \menu{Modifier}. Dans la case \menu{Nom de l'ordinateur}, rentre ton pseudo, puis coche la case \menu{domaine} et rentre \url{windows.eleves.polytechnique.fr}.

\imagepos{images/win_config_domaine}{0.56}{S'inscrire sur le domaine windows}{!h}

\begin{center}
\begin{tabular}{ll}
  \parbox{.45\textwidth}{
  Si tu es j\^one 2005, tu rentres :
  \begin{description}
    \item[Nom] jone05
    \item[Mot de passe] jone.2005
  \end{description}
  }
& \parbox{.45\textwidth}{
  et si tu es rouJe 2004 :
  \begin{description}
    \item[Nom] rouje04
    \item[Mot de passe] rouje.2004
  \end{description}
  }
\\
\end{tabular}
\end{center}

\emph{Attention, ces identifiants servent juste \`a t'inscrire sur le domaine. Pour utiliser ton ordinateur, tu devras rentrer au d\'emarrage les m�mes nom d'utilisateur et mot de passe que tu avais avant d'�tre sur le domaine !}

\paragraph{Installation personnalis\'ee} --- configuration manuelle

Commence par d\'esinstaller tous les antivirus ou firewalls que tu pourrais avoir comme expliqu\'e dans le paragraphe \og Installation simplifi\'ee --- configuration automatique \fg.

\subparagraph{Configuration antivirus} Installe l'antivirus que le BR te propose. Et l\`a on ne rigole plus ! Quand la s\'ecurit\'e du r\'eseau est en jeu, on prend cinq minutes pour installer proprement tout \c{c}a !

Ouvre ton explorateur Windows et tape : 
\url{$\backslash\backslash$enez$\backslash$antivirus}
et double-clique sur le fichier \file{Symantec.exe}.

Ce package contient le param\'etrage de la mise \`a jour automatique de Windows sur le serveur de l'\'ecole. Attends la fin de l'installation et c'est fini ! Maintenant, tu n'as plus \`a toucher \`a l'antivirus, normalement il sera mis \`a jour automatiquement.

\subparagraph{Configuration firewall}

Si tu as Windows XP avec le SP2 install\'e, tu as un firewall automatiquement activ\'e et facile d'utilisation. En effet, \`a chaque fois qu'un programme tentera d'aller pour la premi�re fois sur Internet, il te demandera si tu veux le laisser faire ou non.

\imagepos{images/win_firewall}{0.8}{Un programmme --- ici GuildFTP --- demande \`a acc\'eder au r\'eseau}{!h}

Le firewall commercial \app{ZoneAlarm}, ind\'ependant de Windows, fonctionne sur le m�me principe. Tu peux le trouver sur \xshare.

Si tu pr\'ef�res utiliser le firewall int\'egr\'e \`a Windows XP (sans le SP2) ou \`a Windows Server 2003, il te faudra le configurer en d\'etail. Va dans le \menu{Menu D\'emarrer}, \menu{Param�tres} et clique sur \menu{Connexions R\'eseau}. Choisis la connexion qui est utilis\'ee par ton ordinateur (souvent il n'y en a qu'une, ou alors une seule est activ\'ee) et double-clique dessus. Clique sur \menu{Propri\'et\'es} en bas \`a gauche, puis sur l'onglet \menu{Avanc\'e} et rentre dans le menu de \menu{Param�tres} du \menu{Pare-feu Windows}. Il te faudra alors ajouter manuellement tous les ports que tu veux ouvrir sur l'ext\'erieur. Pour cela, clique sur \menu{Ajouter}, et remplis la bo\^ite de dialogue en t'aidant de la capture d'\'ecran ; mets le num\'ero du port que tu veux ouvrir, par exemple 5050, 5053, 5054 et 5055 en TCP pour \app{qRezix} et 21 en TCP pour ton FTP.

\imagepos{images/win_config_firewall}{0.7}{Ouvrir un port dans le firewall Windows}{!h}

Comme tu peux le constater, il est beaucoup plus pratique d'aller sur le domaine et de laisser le SP2 faire le gros du boulot \`a ta place :-).

\subsubsection{Configuration web (proxy)}

\imagepos{images/win_config_proxy}{0.7}{Configuration du proxy}{!h}

Lance \app{Internet Explorer}. Clique sur \menu{Options Internet} dans le menu \menu{Outils}, puis sur l'onglet \menu{Connexions} de la nouvelle fen�tre et enfin sur \menu{Param�tres r\'eseau} dans le bas de la fen�tre. Remplis le champ \menu{Adresse} avec \url{http://frankiz/proxy.pac} pour finaliser ta connexion vers Internet ; tu dois alors avoir quelque chose qui ressemble au screenshot.

Une fois que tu as fait \c{c}a, tu n'as plus forc\'ement besoin d'\app{Internet Explorer}, qui est soumis \`a certaines failles de s\'ecurit\'e importantes. M�me sur le r\'eseau de l'X, les risques ne sont pas nuls. Le BR t'invite donc \`a installer un navigateur alternatif, \app{Mozilla Firefox}, disponible sur \xshare. Ce n'est pas une garantie ultime --- de toute facon, tu es le premier garant de la s\'ecurit\'e de ton ordinateur, en n'ouvrant pas tous les fichiers qui te passent sous la main --- mais tu seras sensiblement plus en s\'ecurit\'e.

Tes param�tres (en particulier tes favoris) seront directement import\'es depuis \app{Internet Explorer}, la seule configuration \`a effectuer \'etant le proxy. Pour cela, clique sur \menu{Options} dans le menu \app{Outils} puis sur \app{Param�tres de connexion} en bas \`a droite. La case \`a remplir est alors la m�me qu'avec \app{Internet Explorer}, si ce n'est qu'elle se situe en bas de la fen�tre et pas au milieu.

\subsubsection{Configuration mail}

La DSI met \`a ta disposition une bo\^ite aux lettres \'electronique sur le serveur \server{poly} ; cette section t'explique comment configurer \app{Outlook Express} pour y avoir acc�s. Tu peux bien s�r utiliser \app{Thunderbird} si tu pr\'ef�res, les donn\'ees \`a rentrer pour la configuration sont les m�mes ; quelques d\'etails sont donn\'es dans la FAQ sur \fkz. De plus, tu trouveras des explications plus d\'etaill\'ees dans le manuel r\'edig\'e par la DSI.

Lance \app{Outlook Express} et va dans le menu \menu{Outils}, \menu{Comptes\ldots}. Clique sur le bouton \menu{Ajouter\ldots} en haut \`a droite, \menu{Courrier\ldots}.

Remplis les \'ecrans de configuration suivants avec ces donn\'ees :
\begin{description}
  \item[Nom complet] ton nom !
  \item[Adresse de messagerie] de la forme \mail{prenom.nom@polytechnique.edu}
  \item[Type de serveur de messagerie pour le courrier entrant] \menu{POP3}
  \item[Serveur de messagerie pour le courrier entrant] \server{poly.polytechnique.fr}
  \item[Serveur de messagerie pour le courrier sortant] \server{poly.polytechnique.fr}
  \item[Nom du compte] ton login \server{poly} (les huit premi�res lettres de ton nom en g\'en\'eral)
  \item[Mot de passe] ton mot de passe \server{poly} ; v\'erifie bien que la case \menu{M\'emoriser le mot de passe} est coch\'ee.
\end{description}

Voil\`a, clique sur \menu{Continuer}, \menu{Terminer}.

Tu te retrouves alors sur la fen�tre \menu{Comptes Internet}. Va sur l'onglet \menu{Courrier}, clique sur le compte que tu viens de cr\'eer puis sur \menu{Propri\'et\'es}. Clique sur l'onglet \menu{Avanc\'e} et configure comme sur le screenshot suivant ; en particulier, coche la seconde case \menu{Ce serveur n\'ecessite une connexion s\'ecuris\'ee (SSL)}.

\imagepos{images/win_config_mail_avance}{0.66}{Configuration avanc\'ee des serveurs mail}{!h}

Comme ca, tu peux d\'esormais recevoir des mails, avec une liaison s\'ecuris\'ee vers \server{poly} pour que personne ne puisse les intercepter.

\subsubsection{Configuration news}

Comme pour les mails, ca se passe dans \app{Outlook Express}, mais \app{Thunderbird} offre une solution alternative tout \`a fait convenable. Lance \app{Outlook Express} et va dans le menu \menu{Outils}, \menu{Comptes\ldots}. Clique sur le bouton \menu{Ajouter\ldots} en haut \`a droite, \menu{News\ldots}. Remplis les \'ecrans de configuration suivants avec ces donn\'ees :
\begin{description}
  \item[Nom complet] ton nom !
  \item[Adresse de messagerie] de la forme \mail{prenom.nom@polytechnique.edu}
  \item[Serveur de news (NNTP)] \fkz ; v\'erifie \`a ce moment que la case \menu{Connexion \`a mon serveur de news requise} n'est pas coch\'ee. 
\end{description}

Voil\`a, clique sur \menu{Continuer}, \menu{Terminer} et tu es abonn\'e au serveur news des \'el�ves.

Quand tu fermeras la fen�tre `Comptes Internet', il va te demander \`a quels newsgroups tu veux t'abonner, tu n'auras qu'\`a s\'electionner ceux qui t'int\'eressent. Reporte-toi \`a la page \pageref{newsgroups} pour plus d'infos sur les newsgroups auquels t'abonner !

Si tu veux t'inscrire \`a d'autres serveurs news, refais cette proc\'edure en rentrant le nom du serveur qui t'int\'eresse \`a la place de \fkz, par exemple pour acc\'eder aux news externes \server{polynews.polytechnique.fr}.

\subsubsection{Configuration FTP}

\paragraph{Client FTP}

Le BR te conseille \app{SmartFTP} ou \app{FileZilla}. Pour installer l'un des deux, t\'el\'echarge-le sur \xshare et double-clique sur l'installateur.
 
Finis l'installation, et tu peux aller sur tous les FTP du r\'eseau facilement et rapidement. 

\paragraph{Serveur FTP}

Tu verras rapidement que tout le monde \`a l'X poss�de un serveur FTP afin de partager les diff\'erents projets, les films du JTX, ses photos, etc. Donc il est quasiment indispensable que tu en installes un.

Parmi les plus simples on trouve \app{FileZilla} et \app{GuildFTP}, qui sont libres de surcro\^it. Expliquer les d\'etails de la configuration est un peu long pour l'InfoBR, mais il y a une FAQ sur \fkz o� tout ca est d\'ecrit en d\'etail !

\subsubsection{Autres logiciels utiles}

\begin{description}
  \item[qRezix] : Un programme d\'evelopp\'e par le BR pour faciliter la vie sur le r\'eseau, \`a r\'ecup\'erer sur \xshare. Pour plus de d\'etails, voir le paragraphe consacr\'ee \`a \app{qRezix} \`a la page \pageref{qrezix}.
  \item[XChat] : Un client IRC directement issu du monde Unix. Tu peux te reporter \`a la page \pageref{irc} pour plus d'infos sur l'IRC.
  \item[WinSCP] : Un logiciel pratique qui te permet de te connecter en salle info. Tu peux le r\'ecup\'erer lui aussi sur \xshare ; son fonctionnement est expliqu\'e en d\'etails dans la FAQ. Voir aussi \textbf{Putty}
\end{description}

\subsubsection{Obtenir Windows XP Pro et les licences MSDNAA}

\label{xp}
Les accords n\'egoci\'es par le BR avec Microsoft dans le cadre de MSDNAA donnent \`a chaque X le droit de poss\'eder une version de Windows XP Pro gratuite et l\'egale, ainsi que les licences pour la plupart des logiciels de la soci\'et\'e --- quasiment tous, sauf Office et les jeux --- La seule condition \`a remplir est d'�tre \'etudiant sur le plat�l au moment de l'installation du logiciel; tu pourras ensuite le garder sur ton PC m�me apr�s ton d\'epart de l'X.

La proc\'edure pour obtenir les logiciels et les cl\'es correspondantes est la suivante :
\begin{itemize}
  \item Vas d'abord sur \fkz, et connectes toi, puis cliques sur le lien \menu{Licences MSDNAA} qui se trouve dans la bo\^ite \menu{Liens utiles}. S\'electionne le logiciel que tu souhaites installer et valide ta demande, tu recevras ta cl\'e par e-mail. Facile ! Si jamais le logiciel n'est pas dans la liste propos\'ee, c'est soit qu'il n'y a pas besoin de cl\'e --- c'est le cas de beaucoup des logiciels autres que Windows, soit qu'on a oubli\'e de le mettre; dans ce cas, \'ecris \`a \mail{msdnaa@frankiz} pour qu'on t'attribue manuellement une cl\'e.
  \item Maintenant que tu as ta cl\'e, il faut t\'el\'echarger le logiciel proprement dit. Pour cela, deux m\'ethodes : se connecter par FTP sur \url{ftp://enez/} avec ton client FTP pr\'ef\'er\'e, ou y aller en tapant \url{$\backslash\backslash$enez$} dans l'\app{Explorateur Windows} ou \app{Internet Explorer}. Dans les deux cas, tu peux r\'ecup\'erer soit une image du CD (\`a graver ou \`a utiliser avec \app{Daemon Tools}), soit directement le contenu du CD. Pour Windows XP Pro r�cup�re l'image du CD ''fr\_winxp\_pro\_with\_sp2.iso'' qui te permettra d'avoir le dernier Service Pack de Windows et grave la pour avoir un CD identique � un CD Microsoft. Si tu as achet\'e un ordinateur sans OS (et ainsi \'economis\'e environ 150 \euro), fais tout ceci chez un copain!
\end{itemize}

Si tu as encore des questions, plus de d\'etails sont donn\'es dans la FAQ de \fkz.

\vfill

\pagebreak

 







