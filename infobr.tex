%$Id$

\documentclass[11pt,twoside,a4paper,openright]{article}
\usepackage[latin1]{inputenc}
\usepackage[T1]{fontenc}
\usepackage[francais]{babel}
\AddThinSpaceBeforeFootnotes
\FrenchFootnotes
\usepackage{xspace}
\usepackage{lmodern}
\usepackage{amsfonts}
\usepackage{amsmath}
\usepackage{amssymb}
\usepackage{eurosym}
\usepackage{ifthen}
\usepackage{alltt}
\usepackage{wallpaper}
\usepackage{watermark}
\usepackage{wrapfig}
\usepackage{graphicx}
\usepackage{color}
\usepackage{epstopdf}
\DeclareGraphicsRule{.tif}{png}{.png}{`convert #1 `dirname #1`/`basename #1 .tif`.png}


\usepackage{fancyhdr}
\pagestyle{fancy}
\addtolength{\headwidth}{80pt}
\addtolength{\headheight}{3pt}
\renewcommand{\sectionmark}[1]{\markboth{#1}{}}
\renewcommand{\subsectionmark}[1]{\markright{\thesubsection\ #1}}
\fancyhf{}
\fancyhead[LE,RO]{\bfseries{\Large\thepage}}
\fancyhead[LO]{\bfseries\rightmark}
\fancyhead[RE]{\bfseries\MakeUppercase{\leftmark}}
\fancypagestyle{plain}{%
  \fancyhead{} % get rid of headers
  \renewcommand{\headrulewidth}{0pt} % and the line
} 

\addtolength{\evensidemargin}{-60pt}
\addtolength{\voffset}{-24pt}
\addtolength{\textheight}{85pt}
\addtolength{\textwidth}{60pt}

\title{InfoBR 2k4}
\author{BR 2k3}

\date{\today}

% Commandes speciales definies ici

% pour les images de fond sur les headers
\newcommand{\bghdr}[1]{
  \leftwatermark{
    \raisebox{-1.5cm}{
      \hspace{-2.4cm}
      \includegraphics{#1}
    }
  }
  \rightwatermark{
    \raisebox{-1.5cm}{
      \hspace{13.2cm}
      \includegraphics{#1}
    }
  }
}

% pour inclure des images
\newcommand{\image}[3]{ % arguments : path, largeur (entre 0 et 1), l�gende
  \begin{figure*}
    \begin{center}
      \includegraphics[width=#2\textwidth]{#1}
      \caption{#3}
    \end{center}
  \end{figure*}
}
\newcommand{\imagepos}[4]{ % arguments : path, largeur (entre 0 et 1), l�gende, positionnement
  \begin{figure*}[#4]
    \begin{center}
      \includegraphics[width=#2\textwidth]{#1}
      \caption{#3}
    \end{center}
  \end{figure*}
}
\newcommand{\flimage}[3]{ % arguments : path, largeur (entre 0 et 1), position
  \begin{wrapfigure}{#3}{0pt}
    \includegraphics[width=#2\textwidth]{#1}
  \end{wrapfigure}
}

% texttt : ligne de commande, serveurs
% textsf : tout les reste : url, email, dossiers

% pour les serveurs
\definecolor{DarkBlue}{cmyk}{0.95,0.8,0.0,0.0}
\newcommand{\server}[1]{\texttt{\color{DarkBlue}#1}\xspace}
% pour le ligne de commande
\definecolor{LightRed}{cmyk}{0.,0.1,0.15,0.0}
\definecolor{BrightGreen}{cmyk}{0.8,0.,1.0,0.0}
\newcommand{\cmdline}[1]{
  \vspace{4pt}
  \noindent
��\colorbox{LightRed}{
    \parbox[c]{.9\textwidth}{
    \NoAutoSpaceBeforeFDP
    \texttt{\color{BrightGreen}#1}
    \AutoSpaceBeforeFDP
    }
  }
  \vspace{4pt}
}
\newcommand{\cmdlineshort}[1]{
  \noindent
��\colorbox{LightRed}{
    \parbox{.2\textwidth}{
    \NoAutoSpaceBeforeFDP
    \texttt{\color{BrightGreen}#1}
    \AutoSpaceBeforeFDP
    }
  }
  \vspace{4pt}
}

% pour les URLs
\definecolor{Blue}{cmyk}{0.6,0.0,0.05,0.0}
\newcommand{\url}[1]{\NoAutoSpaceBeforeFDP\textsf{\color{Blue}#1}\AutoSpaceBeforeFDP\xspace}
% pour les mails
\definecolor{Blue2}{cmyk}{0.6,0.2,0.0,0.0}
\newcommand{\mail}[1]{\textsf{\color{Blue2}#1}\xspace}
% pour les newsgroups
\definecolor{DarkGreen}{cmyk}{0.8,0.1,0.9,0.4}
\newcommand{\ngname}[1]{\textsf{\color{DarkGreen}#1}\xspace}

% pour les applications
\definecolor{GreenApp}{cmyk}{0.7,0.0,1.0,0.8}
\newcommand{\app}[1]{\textbf{\color{GreenApp}#1}\xspace}
% pour les menus et les elements de menu
\definecolor{GrayMenu}{cmyk}{0.7,0.5,0.6,0.4}
\newcommand{\menu}[1]{\textsf{\color{GrayMenu}'#1'}\xspace}
% pour les repertoires
\definecolor{Orange}{cmyk}{0.,0.2,1.0,0.2}
\newcommand{\rep}[1]{\textsf{\color{Orange}#1}\xspace}
% pour les fichiers
\definecolor{DarkOrange}{cmyk}{0.,0.4,1.0,0.4}
\newcommand{\file}[1]{\textsf{\color{DarkOrange}#1}\xspace}
% pour les distributions Linux
\newcommand{\distrib}[1]{{\color{green}#1}}


% images des OS
\newcommand{\nux}{\includegraphics{nux}\xspace}
\newcommand{\win}{\includegraphics{win}\xspace}
\newcommand{\mac}{\includegraphics{mac}\xspace}
\newcommand{\nuxs}{\includegraphics{nux_s}\xspace}
\newcommand{\wins}{\includegraphics{win_s}\xspace}
\newcommand{\macs}{\includegraphics{mac_s}\xspace}
% divers
\newcommand{\fkz}{\server{frankiz}}
\newcommand{\xshare}{la rubrique \menu{T�l�charger} de \fkz}


\begin{document}

%\maketitle
\thispagestyle{empty}
%$Id$

\ThisCenterWallPaper{.6}{linux_rulez}

\begin{minipage}{.84\textwidth}


\begin{center}
{\LARGE Le mot du prez}
\end{center}
\vspace{2em}

\addtolength{\parskip}{.8em}

\hspace{4em}Salut � tous !

\setlength{\parindent}{2em}

\vspace{1em}

Voil� la nouvelle version de l'InfoBR ! Ce petit fascicule t'aidera � d�couvrir comment fonctionne le r�seau � l'X et � configurer le mieux possible ton ordinateur. Il a �t� r�dig� dans le but de faciliter la vie aux d�butants et d'aider les gros geeks � approfondir leurs connaissances. FTP, qRezix, firewall et cross-post seront bient�t tes meilleurs amis !

Sache que le r�seau �l�ves est g�r� par un groupe de personnes b�n�voles qui investit beaucoup de temps --- d'ailleurs, il faudra penser � passer aux 35 heures un jour --- afin que tu puisses acc�der � Internet et aux services que nous proposons aux X du campus. Alors prends soin du mat�riel r�seau, respecte la charte d'utilisation de la DSI (Direction des Syst�mes d'Information, les gourous informatiques de l'X), et celle du BR ; ainsi tout se passera dans les meilleures conditions. Cependant s'il s'av�rait que tu posais des probl�mes, nous n'aurions aucun scrupule � prendre les mesures qui s'imposent --- et qui pourraient te priver de ton acc�s au r�seau pour quelque temps !

Si tu utilises Windows, nous te sugg�rons de faire tout particuli�rement attention aux virus et aux spywares ! Si tu n'as pas beaucoup l'habitude de l'informatique, et il n'y a pas de honte � cela, nous te conseillons d'utiliser notre domaine Windows qui assurera la s�curit� de ton PC, tu seras ainsi prot�g� automatiquement. Sinon, installe l'antivirus que nous te proposons et mets-le � jour r�guli�rement. Dans tous les cas, ne clique pas sur les pi�ces jointes de mails sans �tre s�r que la personne qui te l'envoie est une personne de confiance et que le contenu du mail est sens�. Enfin, on dit ca pour toi :-)

Pour finir, nous proposons aux X int�ress�s de passer sous Linux pour d�couvrir ce syst�me d'exploitation. Si tu aimes un peu l'informatique ce n'est pas vraiment difficile, alors n'h�site pas ! Nous sommes pour cela � ton enti�re disposition, au cours d'une des \og install-parties \fg que nous organisons r�guli�rement en coop�ration avec le Binet Logiciels Libres. Pendant ces soir�es, des linuxiens exp�riment�s viennent t'aider � installer un Linux qui tourne bien sur ton ordinateur, avec lequel tu pourras faire tout ce que tu faisais sous Windows !

Enfin, on te rassure, si on est un peu stressants c'est surtout pour prot�ger le r�seau des ordinateurs mal configur�s qui permettent aux virus de se propager ; si tu fais ce qu'on te dit, tout se passera tr�s bien pour toi ! Amuse toi bien !

\vspace{1em}

\begin{flushright}
\bsc{dei}, pour le Binet R�seau 2k3
\end{flushright}

\addtolength{\parskip}{-.8em}

\end{minipage}

\pagebreak

\addtolength{\oddsidemargin}{-30pt}


\bghdr{fond-infobr}

\tableofcontents

\vfill{}

\addtolength{\parskip}{.8em}

\section*{Avertissement}

Tu tiens entre tes mains l'InfoBR, document pr�cieux qui te permettra de te connecter facilement --- enfin, on l'esp�re --- au r�seau. Nous te conseillons gentiment d'�viter de le paumer sous un meuble ; il pourra te resservir le jour o� ton ordi te cr�vera mis�rablement entre les mains. Surtout si on te r�pond que la solution se trouve � telle page. :p

Si tu rencontres un probl�me, la proc�dure � suivre pour le r�soudre est expliqu�e en quatri�me de couverture. Bien s�r, si tu ne t'en sors pas, tu peux appeler un des membres du binet --- la liste est � l'int�rieur. Nous sommes l� pour te rendre service.

Mais essaye d'abord de bien tout re-v�rifier avant de le faire, et si possible, adresse toi d'abord � quelqu'un de ton �tage qui s'y conna�t. En effet le BR-man moyen, bien que de bonne volont�, appr�cie moyennement d'�tre d�rang� si ton r�seau ne marche pas parce que tu as �crit \texttt{polytechnqiue} au lieu de \texttt{polytechnique}.

Ceci dit, en avant pour la configuration !

\addtolength{\parskip}{-.8em}

\setlength{\parindent}{2em}

\vfill{}

\pagebreak
%$Id$

\section{Configurer le r�seau}

%$Id$

\subsection{Comment calculer votre IP ?}
\pagebreak
%$Id$

\subsection{Configuration sous Microsoft Windows}
\pagebreak
%$Id$

\subsection{Configuration sous Linux}
\pagebreak
%$Id$

\subsection{Configuration sous Mac OS X}

Voici une pr�sentation des divers logiciels utiles pour utiliser les diff�rents services propos�s sur le r�seau avec MacOS X (les logiciels non int�gr�s  MacOS X et cit�s ici sont quasiment tous t�l�chargeables sur frankiz : lien T�l�charger -> Mac -> R�seau), ainsi que leur configuration.

\subsubsection{Configuration IP/DHCP}

\includegraphics[width=0.5in]{partie1-config_reseau/mac_prefs_icone}
\app{Pr�f�rences R�seau}, accessible depuis l'article de menu Pr�f�rences Syst�me du menu pomme, permet, comme son nom l'indique de configurer la connexion au r�seau. Par ailleurs, si au d�marrage un assistant te propose de configurer ton r�seau, refuse gentiment et utilise la proc�dure que le BR te propose.\\

La gestion des configuration r�seau de MacOS X permet de cr�er plusieurs configurations et de passer en un clic de l'une vers une autre (avec le sous-menu Configuration R�seau du menu pomme) ce qui est tr�s pratique pour les machines vou�es � �tre connect�es  plusieurs r�seaux successivement (les portables par exemple). On commencera donc par cr�er une nouvelle configuration r�seau. \\

\begin{center}
\includegraphics[width=4in]{partie1-config_reseau/mac_nouvelle_config} \\
\end{center}

Une fois la nouvelle configuration cr��e et s�lectionn�e, on va configurer l'interface r�seau ethernet. Tu trouveras les valeurs des groupes AAA et BBB dans le chapitre [1.1]. Pour avoir acc�s au web mondial, il faut ensuite rentrer les proxies � utiliser pour y acc�der. MacOS X 10.3 utilise un script automatique pour param�trer les proxies, il n'est donc pas utile de donner explicitement tous les proxies comme c'�tait le cas avec MacOS X 10.2 (dans quel cas il faut mettre kuzh.polytechnique.fr port 8080). \\

\begin{center}
\includegraphics[width=4in]{partie1-config_reseau/mac_config_ip} \\
\includegraphics[width=4in]{partie1-config_reseau/mac_config_proxy} \\
\end{center}

\subsubsection{Configuration Web}

\includegraphics[width=0.5in]{partie1-config_reseau/mac_safari_icone}
\app{Safari}, le navigateur web d'Apple est maintenant enti�rement op�rationnel.
Un conseil : pense � activer le blocage des fen�tres popup (dans le menu Safari) et la navigation par onglets (dans les Pr�f�rences -> Onglets).
Certains lui pr�f�rent OmniWeb, Camino ou FireFox,  toi de choisir.

\subsubsection{Configuration Mail}

\includegraphics[width=0.5in]{partie1-config_reseau/mac_mail_icone}
\app{Mail} : un client mail. Contient un filtre antispam bay�sien qui est capable d'apprendre quels types de mails sont du spam. Il sait aussi regrouper les mails correspondant  une m�me discussion. \\

Au premier lancement, Mail te demandera de remplir les informations concernant ton compte mail sur poly, il suffit de le remplir comme indiqu� sur la premi�re image. Si tu as d�j� cr�� un compte pr�c�demment, il faut aller dans les pr�f�rences, onglet Comptes (accessible depuis le menu Mail), cr�er un autre compte et remplir comme dans la deuxi�me image. \\

\begin{center}
\includegraphics[width=4in]{partie1-config_reseau/mac_mail_config} \\
\includegraphics[width=4in]{partie1-config_reseau/mac_mail_description} \\
\end{center}

Dans les deux cas il faut aussi activer le cryptage SSL (sinon, �a ne marchera pas) : \\

\begin{center}
\includegraphics[width=4in]{partie1-config_reseau/mac_mail_ssl} \\
\end{center}

Cette configuration marche pour acc�der � ses mails depuis l'int�rieur de l'X mais aussi de l'ext�rieur, sans rien changer.

\subsubsection{Configuration News}

\includegraphics[width=0.5in]{partie1-config_reseau/mac_mt_icone}
\app{MT-NewsWatcher} : un client news permettant d'acc�der aux forums de discussion de frankiz (mais aussi de usenet gr�ce au serveur polynews.polytechnique.fr). Dans la m�me cat�gorie, tu pr�f�reras peut-�tre utiliser Unison (attention c'est un shareware) ou Thunderbird qui est le client mail/news distribu� par mozilla. La configuration reste la m�me. \\

Au premier lancement, l'application affiche un message de bienvenue et t'invite � configurer ton client.\\

\begin{center}
\includegraphics[width=4in]{partie1-config_reseau/mac_mt_bienvenue} \\
\includegraphics[width=4in]{partie1-config_reseau/mac_mt_serveurs} \\
\includegraphics[width=4in]{partie1-config_reseau/mac_mt_identifiant} \\
\end{center}

Une fois ceci effectu� tu peux commencer � s�lectionner les forums de discussion (dans la fen�tre Liste de tous les groupes sur 'frankiz') auquel tu souhaite t'abonner en les glissant-d�posant dans la fen�tre "sans titre".\\

\begin{center}
\includegraphics[width=2in]{partie1-config_reseau/mac_mt_profil}
\includegraphics[width=2in]{partie1-config_reseau/mac_mt_liste} \\
\end{center}

Tu peux ensuite enregistrer le profil "sans titre" dans le menu \menu{Fichier->Enregistrer} et le sauvegarder sous le nom "News Frankiz" o� tu veux (dans un dossier MT-NewsWatcher dans \rep{~/Documents} par exemple, o� \rep{~} correspond � ton dossier utilisateur).\\ \\
Pour �crire un post sur un des forum, choisir Nouvel article dans le menu Nouvelles.\\
Pour passer d'un article � un autre dans une enfilade ou entre les enfilades, il suffit d'appuyer sur espace.\\ \\
Tu peux ajouter des serveurs en s�lectionnant Serveurs de nouvelles... dans le menu Sp�cial et ramnener la liste des forums de discussion de ce serveur le s�lectionnant dans le menu \menu{Fen�tres->Afficher la liste de tous les groupes}. \\ \\
Enfin, si tu veux �viter le profil que tu viens de sauvegarder � chaque d�marrage, il faut utiliser la configuration suivante dans \menu{MT-NewsWatcher->Pr�f�rences->Enreg. de fichiers} : \\

\begin{center}
\includegraphics[width=4in]{partie1-config_reseau/mac_mt_prefs} \\
\end{center}

 \subsubsection{Configuration Ftp}
 
 \includegraphics[width=0.5in]{partie1-config_reseau/mac_cyberduck_icone}
\app{Cyberduck} : un client ftp tr�s simple d'utilisant mais performant. Il te permettra d'aller t�l�charger des fichiers sur les serveurs ftp des autres �l�ves. Dans la m�me cat�gorie, il existe aussi Transmit (shareware) qui dispose de plus de fonctionnalit�s. \\ \\
La configuration ne pose aucun probl�me puisqu'il n'y rien � faire. Pour se connecter � un serveur, il suffit de taper son nom (exmple : gwennoz) dans le cadre "Connexion rapide" et appuyer sur entr�e. Tu pourras ensuite naviguer sur le serveur et t�l�charger ou transf�rer des fichiers. Attention, certains serveurs (intentionnellement) mal configur�s ne permettent qu'une connection � la fois. Or, le t�l�chargement d'un fichier demande l'ouverture d'une nouvelle connection. Il faut donc se d�connecter (bouton "D�connecter") puis lancer le t�l�chargement en double-cliquant sur le fichier : \\

\begin{center}
\includegraphics[width=4in]{partie1-config_reseau/mac_cyberduck_connection} \\
\end{center}

Les signets de permettent de sauvegarder les serveurs sur lesquels tu te connectes souvent. \\

\subsubsection{Autres logiciels}

Voici plusieurs logiciels que tu voudras s�rement t�l�charger pour profiter au maximum des possibilit�s du r�seau.\\

\includegraphics[width=0.5in]{partie1-config_reseau/mac_qrezix_icone}
\app{qR�zix} permet de chater avec les x connect�s, de savoir s'ils ont un serveur FTP ou Web auquel on peut se connecter. Il dispose aussi d'un moteur de recherche et de t�l�chargement pour r�cup�rer des fichiers sur les serveurs FTP activ�s. \\
\includegraphics[width=0.5in]{partie1-config_reseau/mac_cocoaxnet_icone}
\app{CocoaXNet} est un �quivalent de qR�zix moins complet (ne dipose pas de moteur de recherche). \\
\includegraphics[width=0.5in]{partie1-config_reseau/mac_pureftpd_icone}
\app{PureFTPd Manager} est une interface graphique qui te permet de configurer ais�ment un serveur FTP sur ta machine. Il en existe aussi un int�grer � Mac OS X mais celui-ci dispose de plus de foncionnalit�s. \\
\includegraphics[width=0.5in]{partie1-config_reseau/mac_conversation_icone}
\app{Conversation}, un client IRC dans le m�me esprit qu'iChat. Dispose d'une interface tr�s simple ne n�cessitant pas de conna�tre les commandes irc. \\



\pagebreak
\bghdr{fond-infobr}
%$Id$

\section{Pr�sentation du r�seau et des services du BR}

%$Id$

\subsection{La structure du r�seau et les serveurs}

\subsubsection{Sch�ma g�n�ral}

Voici tout ce qu'il y a autour de ton ordinateur sur le r�seau !

\imagepos{partie2-presentation_reseau/schema_reseau}{.92}{Sch�ma du r�seau de l'X}{h}

\subsubsection{R�seau DSI}

Tu dois l'avoir compris, deux entit�s se partagent la gestion du r�seau pour les �l�ves : le BR et la DSI.

La DSI contr�le tous les acc�s vers l'ext�rieur de l'X. Elle poss�de une connexion � 100Mb/s sur le r�seau Renater --- le r�seau scientifique fran�ais --- prot�g�e par de puissants firewalls qui emp�chent les intrusions de pirates sur le r�seau de l'X. Cela a des inconv�nients : il faut passer par des serveurs proxy pour acc�der � l'ext�rieur et certaines choses, comme le peer-to-peer ou les jeux sur internet, sont impossibles. Rappelle-toi n�anmoins que le r�seau n'est cens� te servir que pour ton travail :-), et surtout que c'est gr�ce � ces firewalls que le r�seau interne est � peu pr�s tranquille, qu'on peut y laisser des serveurs FTP ouverts, etc. Bref, c'est une protection et pas du tout une contrainte gratuite.

La DSI g�re le r�seau physique (switches, etc.) et les serveurs de l'�cole ; elle offre plusieurs services aux �l�ves : les salles info (comptes \server{moned}), le serveur de mails (comptes \server{poly}) o� tu as ton adresse \mail{prenom.nom@polytechnique.fr}, les proxies qui nous permettent de passer le firewall et de nous connecter � l'ext�rieur de l'�cole --- \server{kuzh} et \server{sil}, deux serveurs DNS --- \server{milou} et \server{rackham}, etc. Donc si tu n'arrives pas � te loguer en salle info, ou si tu as perdu ton mot de passe pour tes mails, c'est eux qu'il faut aller voir. Leurs bureaux sont dans le couloir qui relie l'arri�re du Grand Hall aux laboratoires.

Enfin, c'est la DSI qui t'a fait signer une charte de bon usage du mat�riel et de tous les services qu'elle met � ta disposition. C'est donc � elle que tu devras rendre des comptes en cas de violation de cette charte.

\subsubsection{R�seau �l�ves}

Le BR poss�de plusieurs serveurs et offre des services qui sont largements d�crits dans cette partie de l'InfoBR : \fkz et notamment ses FAQ et sa rubrique \menu{T�l�charger}, encore appel�e \menu{X-Share}, les newsgroups, \app{qRezix}, le domaine Windows, la t�l�\ldots et l'InfoBR !

Une grosse partie du travail du BR consiste � maintenir ces services en �tat de marche et � le am�liorer. Nous faisons aussi du support, c'est � dire qu'on peut te donner des conseils pour configurer ta machine --- mais \emph{grosso modo} presque tout est d�j� �crit dans cet InfoBR...

De plus, le BR organise et surveille le bon fonctionnement du r�seau dans les b�timents des �l�ves, � savoir les caserts et le bataclan, en collaboration avec la DSI. Nous restons d�pendants de la DSI car ils poss�dent tout le mat�riel physique, sauf nos serveurs --- pour l'instant.

\newcommand{\dscserver}[3]{\item \server{#1} (#2) : #3}

\subsubsection{Description des serveurs}

\paragraph{Serveurs du BR}

Voici quelques-uns des serveurs du BR --- ceux que tu as le droit de conna�tre ;) Les services h�berg�s sur chaque serveurs sont donn�s � titre informatif, et peuvent �tre chang�s � tout moment. Pour ne pas avoir de mauvaise surprise, utilise les alias qui te sont donn�s dans cet InfoBR.
\begin{itemize}
	\dscserver{frankiz}{129.104.201.51}{site web, newsgroups, DNS secondaire}
	\dscserver{gwennoz}{129.104.201.52}{CVS, miroirs, DNS secondaire}
	\dscserver{heol}{129.104.201.53}{xnetserver, DNS principal}
	\dscserver{skinwel}{129.104.201.54}{T�l�, DNS secondaire}
	\dscserver{enez}{129.104.201.61}{Domaine windows}
\end{itemize}

\paragraph{Serveurs de la DSI}

Comme les �l�ves vivent dans un sous-r�seau de celui de la DSI, tu risques un jour o� l'autre d'avoir affaire � l'un des serveurs de la DSI.
\begin{itemize}
	\dscserver{poly}{129.104.247.5}{mails, tu le connais certainement}
	\dscserver{kuzh}{129.104.247.2}{proxy http, pour internet :)}
	\dscserver{sil}{129.104.247.3}{proxy, ta seule passerelle ssh vers/depuis l'ext�rieur}
	\dscserver{milou}{129.104.30.41}{ntp, DNS de l'Ecole}
	\dscserver{rackham}{129.104.32.41}{DNS de l'Ecole}
\end{itemize}

La DNS du BR comprend la DNS de l'Ecole et des noms en .eleves.polytechnique.fr pour toutes les machines des �l�ves connect�es sur xNet, gr�ce � qRezix en g�n�ral.

\begin{center}
  \fbox{
    \begin{minipage}{.95\textwidth}
      \begin{center}
Les serveurs de la DSI sont l� � ta disposition pour des services bien d�finis. Ce ne sont en aucun cas des serveurs de stockage. Seul \server{sil} est pr�vu pour le transfert de fichiers... Dans tous les cas, tout abus pourra entra�ner la perte des tes comptes sur \server{poly} (plus de mails), \server{moned} (plus de possibilit� de connection en salle infal) et \server{sil}.
      \end{center}
    \end{minipage}
  }
\end{center}


%$Id$

\subsection{Frankiz}

\label{frankiz}

La page web \fkz est la page des �l�ves. Elle est visible de l'int�rieur et de l'ext�rieur de l'X, en int�gralit� si tu t'es identifi� ou partiellement pour les autres utilisateurs. Tu peux automatiser ta connexion gr�ce � un cookie d'authentification. Nous te conseillons de faire de \url{http://frankiz/} la page d'accueil de ton navigateur Internet.

Elle permet en particulier l'acc�s aux services suivants : les annonces et les activit�s du plat�l, l'annuaire des �l�ves (\menu{TOL} pour Trombi-On-Line), le t�l�chargement de logiciels gratuits (\menu{X-Share}), la foire aux questions (\menu{FAQ}), la question du jour (\menu{QDJ}) et m�me la m�t�o. Cette page est ais�ment personnalisable (lien \menu{Pr�f�rences}). � toi d'explorer tout ce que tu peux y trouver !

Les annonces, les sondages et les activit�s permettent d'informer les �l�ves de ce qui se passe � l'�cole. Les annonces sont tri�es dans un sommaire et tu peux �ventuellement faire disparaitre celles qui ne t'int�ressent pas (en fonction des skins). Les activit�s apparaissent sur la page principale le jour o� elles ont lieu. Les sondages apparaissent sur la page principale et lorsque le vote est termin�, tu peux voir les r�sultats. Tu peux proposer des annonces, des sondages et des activit�s en utilisant les liens \menu{Proposer\ldots}.

L'annuaire permet de trouver des renseignements utiles sur tous les �l�ves sur le pl�tal. Tu peux mettre ta fiche � jour (lien \menu{Pr�f�rences}) et si tu es le pr�sident d'un binet, g�rer la liste des membres.

La \menu{Foire Aux Questions} contient les r�ponses aux questions les plus courantes. C'est souvent plus rapide d'y faire un petit tour que d'appeler quelqu'un, en plus il y a un moteur de recherche. Et si tu vois une erreur, tu peux la corriger directement !

La rubrique \menu{X-Share} permet de t�l�charger des logiciels pour Windows, Mac et Linux s�lectionn�s par le BR et des documents importants comme cet InfoBR. En particulier, c'est l� que tu trouveras les logiciels d�velopp�s par le BR, dont \app{qRezix}. Si tu cherches un logiciel pour un usage particulier, commence par l� !

La QDJ est une question binaire, s�rieuse parfois mais le plus souvent bas�e sur un jeu de mots ou sur l'activit� sur le campus. Tu peux voter tous les jours et m�me proposer des questions au QDJMaster.

La rubrique \menu{Sites �l�ves} contient la liste des sites personnels des �l�ves h�b�rg�s sur \fkz. Tu peux toi aussi publier ton site web en utilisant le lien dans \menu{Pr�f�rences}. De m�me, la rubrique \menu{Binets} contient la liste des binets, une description de celui-ci et le lien vers le site de ce binet (�ventuellement h�berg� sur \fkz). Si tu es le webmestre de ton binet, tu peux mettre � jour le site web et demander de modifier l'ic�ne et la description du binet. 

M�me si \fkz est l'oeuvre de tous, les webmestres se r�servent le droit de ne pas publier une annonce ou d'interdire un site web si le contenu n'est pas jug� adapt� mais aussi, dans le cas des annonces si elle g�ne la lisibilit� g�n�rale, selon le bon principe : \og trop d'information tue l'information \fg.

Rappelons quelques r�gles �videntes : 
\begin{itemize}
\item Tout contenu pol�mique est banni des annonces (publie tes aigreurs sur dans l'IK ou sur \ngname{br.binet.polemix}).
\item Tout contenu portant atteinte � une tierce personne ou � un groupe est interdit dans les annonces et les sites web, ainsi que tout lien vers un site ou document de ce type.
\item Tout contenu ill�gal, en particulier tout document (quel que soit son type) non libre de droits ou ayant un caract�re pornographique, est interdit, ainsi que tout lien vers un site ou document de ce type.
\item Si un contenu d'un des deux types pr�c�dents �chappe toutefois � l'attention des webmestres, seuls leurs auteurs pourraient en �tre tenus responsables. Tout contenu de cette sorte qui appara�trait sur le site doit �tre imm�diatement signal�.
\end{itemize}

Les r�gles �l�mentaires pour pr�server la lisibilit� de \fkz :
\begin{itemize}
\item Les annonces ne doivent pas �tre trop longues (pas plus d'une 
quinzaine de lignes)
\item Les titres des annonces ne doivent pas �tre en majuscules ou pr�c�d�s de 
signes de ponctuation.
\item Un binet ne peut pas avoir deux annonces en m�me temps.
\end{itemize}

Alors s'il te pla�t, quand tu fais la com' de ton binet, r�fl�chis et �cris une belle annonce : plus elle est concise et pr�cise, plus elle sera lue !


%$Id : 2-utilisation\_des\_ng.tex,v 1.2 2004/11/19 10 :35 :41 myk Exp $

\newcommand{\itemng}[1]{\item \ngname{#1}}
\newcommand{\itemngpp}[1]{\item \ngname{#1} : }

\subsection{Les newsgroups}

\label{newsgroups}

\subsubsection{Les r�gles d'or des newsgroups}

\begin{enumerate}
  \item Pas d'attaques personnelles, pas d'insultes, pas de calomnie !
  \item Seules les questions concernant directement le BR ont leur place sur \ngname{br.binet.br} ; pour les question d'informatique il y a les \ngname{br.informatique.*}
  \item Les petites annonces vont sur \ngname{br.pa} et nulle part ailleurs !
  \item Faire des crossposts proprement (cf. page \pageref{crosspost})
\end{enumerate}

Et, comme sur tout lieu de discussion virtuel, il convient de \emph{garder son calme} !

\subsubsection{C'est quoi les newsgroups ?}

Pour se tenir au courant de ce qui se passe sur le plateau, en plus du site web \url{http\string://frankiz/}, les news, ou encore groupes de discussion, sont indispensables. Ces \emph{newsgroups}, parfois abusivement appel�s les \og br \fg, te permettent aussi de poster des annonces, des questions, etc. Leur fonctionnement est celui d'un forum, tu peux lire les messages des autres, poster des messages dans le fil d'une discussion (\emph{thread}) ou cr�er une nouvelle discussion.

\subsubsection{Quels sont les diff�rents newsgroups sur frankiz ?}

Le serveur de news \server{frankiz} h�berge plus de 200 newsgroups, mais seuls quelques uns vont t'int�resser. Les principaux newsgroups sont les suivants :

\begin{itemize}
  \itemngpp{br.binet.ton\_binet} chaque binet ou presque a son newsgroup, au besoin on peut demander sa cr�ation. Il sert le plus souvent de vecteur de communication interne au binet, mais peut aussi servir aux annonces du binet, aux questions address�es au binet, etc. C'est en tout cas le meilleur endroit pour prendre le premier contact avec le binet en question!
  \itemngpp{br.binet.br} le newsgroup du binet r�seau, pour sa communication interne et externe. Pour toute question qui a rapport avec le BR (probl�mes sur nos services, comptes ou sites webs ou newsgroups br.binet.* sur nos machines, etc.). \emph{Les questions g�n�rales informatiques (d�pannage, logiciels, achat, programmation, mat�riel) ont, quant � elles, leur place sur les newsgroups ci-dessous :}
  \itemngpp{br.informatique.*} pour les questions ou probl�mes informatiques li�s :
  \begin{itemize} 
    \itemngpp{br.informatique.reseau} au r�seau
    \itemngpp{br.informatique.windows, linux, mac} � ton syst�me d'exploitation.
    \itemngpp{br.informatique.virus} aux virus --- des solutions de d�sinfections s'y trouvent s�rement d�j�
    \itemngpp{br.informatique.divers} quand tu es vraiment perdu :)
    \itemngpp{br.informatique.*} \ngname{urls, media, langages}, etc. � toi de d�couvrir !
  \end{itemize}
  \itemngpp{br.binet.eleves} posts g�n�raux concernant un grand nombre d'�l�ves.
  \itemngpp{br.binet.pa} petites annonces --- et \emph{pas} sur le br.binet.eleves !)
  \itemngpp{br.promo.*} pour les posts concernant une seule promotion (Rouje, OranJe ou Jone)
  \itemngpp{br.enseignement} questions et annonces li�es � l'enseignement.
  \itemngpp{br.kes} le forum de la K�s, annonces, questions et r�ponses.
  \itemngpp{br.section.ta\_section} c'est vous qui voyez, mais c'est vraiment bien pour programmer des activit�s section!
  \itemngpp{br.communaute.*} rassemblent les discussions : par musique, par religion, etc.
  \itemngpp{br.binet.polemix} le seul newsgroup o� l'on peut faire un thread de 500 posts en 24 heures ! Enfin, pour ceux qui aiment �a, et qui ont le temps\ldots
  \itemngpp{br.test} pour les tests --- ca emb�te moins de gens que de le faire sur br.eleves ! Posts toujours accompagn�s de la blague qui va bien, sinon ce n'est pas marrant --- il faut penser � ceux qui y sont abonn�s � longueur d'ann�e !
  \itemngpp{br.trash} pour les insultes, les craquages !
  \itemngpp{br.binet.lose} pour raconter tes meilleures loses !
  \itemngpp{public.*} ce sont des newsgroups accessibles � tout le personnel de l'�cole... m�me si il y a peu d'activit� dessus, c'est un lien avec le personnel enseignant en particuliers.
\end{itemize}


\subsubsection{D'autres newsgroups\ldots}

Polytechnique.org (si ce n'est pas d�j� fait, inscris-toi !!!) fournit �galement un service de newsgroups avec plus de 12000 polytechniciens... et te permet d'envisager des newsgroups plus s�rieux (par exemple \ngname{xorg.pa.emploi}). Les sujet sont divers :

\begin{itemize}
	\itemng{xorg.promo.*}{permet de diffuser des infos dans ta promo}
	\itemng{xorg.pa.*}{pour les petites-annonces...}
	\itemng{xorg.informatique.*}{pour des informations concernantn l'informatique}
	\itemng{xorg.section.*}{pour ta section... mais sur plusieurs d�cennies}
	\itemng{xorg.discussions.*}{pour les discussions en vrac}
\end{itemize}
Pour te conecter aux news de polytechnique.org : le serveur est \server{ssl.polytechnique.org}, accessible en SSL avec un mot de passe � d�finir sur \url{https\string://www.polytechnique.org/acces\_smtp.php}.

Si tu veux avoir acc�s � certains newsgroups ext�rieurs � l'�cole, tu peux utiliser \server{polynews}. Ce server de la DSI est synchronis� avec l'ext�rieur. Si tu cherches un newsgroup qui ne s'y trouve pas n'h�site pas � demander aux newsmestres.

\subsubsection{En bref !}

Nos conseils pour tes abonnements aux news de \server{frankiz} :
\begin{itemize}
\itemng {br.binet.binets\_qui\_m'int�ressent}
\itemng {br.section.ma\_section}
\itemng {br.kes}
\itemng {br.eleves}
\itemng {br.promo.jone}
\itemng {br.pa}
\itemng {br.informatique.reseau, divers, (windows \textrm{ou} linux \textrm{ou} mac)}
\itemng {br.enseignement}
\end{itemize}

Mais attention ! Les newsgroups risquent de prendre beaucoup de ton temps, surtout cette ann�e !


%$Id$

\subsection{qRezix : connecte-toi}
\label{qrezix}

\app{qRezix} est un programme d�velopp� par le BR pour simplifier la vie sur le r�seau. Il permet en particulier de fixer le nom que ta machine portera sur le r�seau (nom DNS), qui permettra aux autres de se connecter � ton serveur FTP (et HTTP, news, etc.).

\subsubsection{Installation de qRezix}

\app{qRezix} est disponible sur \xshare. L'installation se fait en g�n�ral simplement, quelque soit ton OS\footnote{Sous Linux tu pourrais avoir � compiler le programme � la main ; en cas de probl�me, n'h�site pas � demander de l'aide aux d�veloppeurs.}.

Pour que le chat et l'Xplo fonctionnent, n'oublie pas d'ouvrir les ports 5050, 5053 et 5054 en TCP de ton firewall si tu en as un.

\app{qRezix} �volue r�guli�rement. Donc n'h�site pas � installer les derni�res mises � jour et les nouveaux plug-ins, qui sont en g�n�ral annonc�s sur les newsgroups.

\subsubsection{Que fait qRezix ?}

\app{qRezix} te permet de conna�tre la liste de toutes les personnes connect�es, de d�finir des favoris, de chatter avec tes amis avec une interface conviviale. Il y a plein d'options marrantes, alors n'h�site pas � explorer les \menu{Pr�f�rences}.

\image{partie2-presentation_reseau/qrezix.png}{0.8}{qRezix}

\app{qRezix} te permet �galement de conna�tre l'�tat des serveurs des personnes connect�es\ldots\ ainsi d'un simple coup d'\oe il tu sais si une personne a un serveur FTP, un site web ou un serveur de news, tu sais quelle est sa promo, etc.

En plus gr�ce � des plug-ins, on peut �tendre ses capacit�s. Il existe notamment un plug-in \app{Xplo} tr�s utile pour rechercher parmi les fichiers partag�s sur le r�seau.

En bref : \app{qRezix} c'est \emph{le} logiciel indispensable sur le r�seau !

\subsubsection{En savoir plus}

La gestion du domaine \url{.eleves.polytechnique.fr} est faite par le BR gr�ce � \app{xNet}, dont \app{qRezix} est le principal client. Donc c'est gr�ce � \app{qRezix} qu'on peut facilement donner un nom � chaque ordinateur, en r�alit� � son IP.

Si tu te sens l'�me d'un d�veloppeur et qu'un travail multi-plateforme (en Qt) t'int�resse, n'h�site pas � prendre contact avec les d�veloppeurs (par mail : \mail{qrezix@frankiz}, ou sur \ngname{br.binet.br.devel}) pour participer ! Si tu veux cr�er un th�me d'ic�ne, ou une traduction de \app{qRezix}, idem, n'h�site pas � nous contacter. On n'attend que �a ;-)



\subsection{La t�l� sur le r�seau}

\label{tele}

OS concern�s : \win \hspace{1em} \nux \hspace{1em} \mac

\image{partie2-presentation_reseau/vlc.png}{0.8}{VLC}

Gr�ce au lecteur multim�dia VLC d�velopp� par l'�quipe de VIA (�quivalent centralien du BR) , nous commen�ons � mettre un place un service de diffusion de la t�l� via le r�seau �l�ves. Cette fonctionnalit� , pour l'instant en cours de construction te permet de regarder une cha�ne nationale choisie par les administrateurs TV (El-Sio et Esope) � la demande des �l�ves (�a a l'air artisanal comme �a mais c'est d�j� une r�ussite, croyez-le!).

\subsubsection{Comment r�cup�rer VLC ?}

VLC est disponible dans \xshare.
L'installation est tr�s simple sous windows, Mac OS X et sur les distributions Linux utilisant des paquets (debian et mandrake) pour les autres distributions, il te faudra peut-�tre compiler le programme � la main, n'h�site pas alors � demander de l'aide aux devels.

\subsubsection{Configurer VLC pour regarder la t�l�}

\image{partie2-presentation_reseau/vlcconfig.png}{0.8}{Configurer VLC}

\begin{itemize}
\item ouvre VLC et clique sur \menu{Fichier}.
\item dans le menu qui appara�t choisis \menu{Ouvrir un flux r�seau}.
\item choisis l'option \menu{multidiffusion RDP/UDP} et rentre l'adresse suivante : \url{tv} .
\item laisse le port par d�faut sur \url{1234}.
\item cliques ensuite sur \menu{OK}... et voil� !
\end{itemize}

\subsubsection{Si rien ne se passe}

Pas de panique ! Quelques points � v�rifier avant d'appeler un responsable TV :

\begin{itemize}
\item v�rifie que ta prise r�seau est bien branch�e (�a serait b�te quand m�me :p).
\item v�rifie que ton firewall laisse passer les flux UDP sur le port \url{1234}.
\item essaye en rempla�ant l'adresse \url{tv} par \url{225.0.0.1}.
\item essaye de lire un autre fichier vid�o avec VLC (un film qui est sur ta machine).
\item si tu es sous linux , v�rifie que tu as bien activ� l'option multicast dans ton noyau.
\item enfin, �cris nous un mail donnant les r�sultats de ces quelques tests pour que l'on puisse t'aider � r�soudre ton probl�me.
\end{itemize}

\subsubsection{Avertissement}

Comme il a d�j� �t� dit dans cette description, le service t�l� est actuellement en test et en constante am�lioration. Il se peut donc qu'il soit coup� sans que vous soyez pr�venus ou que la qualit� diminue brutalement. Mais tous les efforts sont faits pour vous offrir au plus vite un service avec de nombreuses cha�nes diff�rentes en haute qualit�. Les remarques et suggestions sont donc les bienvenues.


\pagebreak
%$Id$

\section{R�f�rence rapide}

%$Id$

\subsection{les cross-posts}

Il se peut qu'un jour tu aies besoin de poster le m�me message sur plusieurs newsgroups � la fois !

Surtout \emph{ne fais pas un post dans chacun des newsgroups} en copiant-collant ton texte � chaque fois car :

\begin{itemize}
  \item Tu vas y passer 10 minutes, et il y a tellement de choses beaucoup plus passionnantes � faire sur le plateau !!
  \item Si tu postes sur 15 newsgroups, tu vas devoir lire les r�ponses � ton post sur chacun des 15 newsgroups concern�s, et tu vas y passer la nuit \ldots
  \item Si tu fais un cross-post, la plupart des clients news marqueront le message du cross-post comme lu sur tous les newsgroups d�s qu'il sera lu une fois, au lieu d'avoir � le lire 15 fois. Ainsi tout le monde gagnera du temps en ne lisant qu'une fois ton message --- lire plus de trois fois la m�me chose, ca �nerve\ldots
  \item C'est SAAAAALE et nous au BR on n'aime pas :)
\end{itemize}
 
Quelle est la solution alors ? Un seul mot, le mot magique : CROSS-POST !
T'inqui�tes pas, c'est tr�s simple � faire, tous les lecteurs de News sont pr�vus pour, voici pas-�-pas comment faire un cross-post dans Outlook Express 6
.0\footnote{Mais les newsmestres te conseillent KNode ou Mozilla quand m�me !}. Ca marche aussi chez les autres clients news : il faut juste suivre la m�me philosophie !
\begin{itemize}
  \item Clique sur 'Nouveau message'/'New Post'   (eh oui, au BR, on est presque bilingue :)
  \item Dans le menu 'Affichage', cocher 'Tous les en-t�tes' ('View'->'All Headers') A ce niveau-l� tu as une fen�tre vide pour taper ton texte avec plein de labels obscurs � remplir au-dessus.
  \item Ne t'en occupe pas pour l'instant et tape le corps de ton message.
  \item Dans le premier en-t�te en haut de la fen�tre d'�dition ('Groupe de discussion' / 'Newsgroup'), mets tous les brs o� tu veux que ton message apparaisse, s�par�s par des ';'. Par exemple, 'br.eleves; br.promo.rouje; br.promo.jone; br.binet.bob' permet d'�crire en une seule fois sur ces 4 newsgroups.
\end{itemize}

Mais l'int�r�t du crosspost ne r�side pas seulement dans le fait de t'�pargner un fastidieux copier-coller ! Il permet de rediriger toutes tes r�ponses vers un seul br parmi ceux que tu as entr�s � l'�tape pr�c�dente. Pour cela, remplis le deuxi�me en-t�te ('Transf�rer �' / 'Follow-up to') avec le nom du br o� tu veux que les r�ponses soient redirig�es (un seul !!! sinon ton message sera refus� par le serveur de news et tu auras le message d'erreur qui va bien...) Ca y est tu peux envoyer ! (tu n'as pas oubli� de mettre un sujet au moins ? :)) Tu viens de faire un beau crosspost bien propre. Tu remarqueras dans ton lecteur de news qu'une fois ton post marqu� comme lu dans un quelconque des newsgroups o� il est apparu, il est automatiquement marqu� comme lu dans tous les autres ! C'est un gros avantage du crosspost\ldots


%$Id$

\subsection{Comment \emph{bien} utiliser les newsgroups}

\input{partie3-reference_rapide/3-utilisation_des_autres_moyens}


\subsection{qRezix : connectes-toi}

OS concern�s : \win \nux \mac

qRezix est un programme d�velopp� par le BR dans le but de permettre de simplifier la vie sur le r�seau. Il permet en particulier de fixer le nom que ta machine portera sur le r�seau, et donc le nom qui permettra aux autres de se connecter sur ta machine (serveur ftp, http, news...).

\subsubsection{Installation de qRezix}

\includegraphics{partie3-reference_rapide/xshare-qrezix.png}

qRezix est disponible dans la section 'T�l�charger' sur http://frankiz. L'installation se fait simplement, quelque soit ton OS. Sous linux il se peut que tu aies � compiler le programme � la main, en cas de probl�me, n'h�sites pas � demander de l'aide aux devels.

\subsubsection{Que fais qRezix ?}

qRezix te permet de conna�tre la liste de toutes les personnes connect�es. De d�finir des favoris pour lesquels tu pourras avoir ou non une notification automatique de l'�tat de connexion.
qRezix te permet de chatter avec tes amis avec une interface conviviale.

\includegraphics{partie3-reference_rapide/qrezix.png}

qRezix te permet �galement de conna�tre l'�tat des serveurs des personnes connect�es... ainsi d'un simple coup d'oeil tu sait si une personne a un serveur ftp, un site web ou un serveur de news, tu sais qu'elle est sa promo...
qRezix est l'outil indispensable pour la vie sur le r�seau... et en plus gr�ce � la possibilit� de mettre des plug-ins pour qRezix, on peut imaginer des milliers d'application de qRezix.

En gros qRezi c'est LE logiciel indispensable sur le r�seau.

\subsubsection{En savoir plus}

La gestion du domaine .eleves.polytechnique.fr est faite par le BR gr�ce au xNet, dont qRezix est le principal client. Donc c'est grace � qRezix qu'on peut facilement donner un nom � chaque IP.
qRezix est d�velopp� par le BR et �volue r�guli�rement. Donc n'h�site pas � jeter r�guli�rement un oeil dans la section 'T�l�charger' de http://frankiz. Tu pourras y trouver les derni�res mises � jour, les nouveaux plug-ins...
qRezix permet de rajouter des plug-ins... donc si tu te sens l'�me d'un d�veloppeur, qu'un travail multiplateforme (en Qt) t'int�resse, n'h�site pas � prendre contact avec les devels (qrezix@frankiz.polytechnique.fr) pour avoir des informations sur les possibilit�s offertes par qRezix...
Si tu veux cr�er un th�me d'ic�ne, ou une traduction de qRezix, idem, n'h�site pas � nous contacter. On attend que �a :)



\input{partie3-reference_rapide/5-lexique}

\input{partie3-reference_rapide/6-membres_du_br}


\end{document} 
