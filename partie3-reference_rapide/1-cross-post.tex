%$Id$

\subsection{Les crossposts}

Il se peut comme tout le monde qu'un jour tu veuilles annoncer un grand ev�nement sur les newsgroups... et donc pour ceci il faut que tu postes ton message sur plusieurs newsgroups... Alors avant de commencer � poster sur un newsgroup, faire copier-coller, poster sur le newsgroup suivant... lis ce qui suit et apprend � crossposter proprement.

\subsubsection{Pourquoi faire un crosspost ?}

\begin{itemize}
  \item Parce que crossposter te permet de ne poster qu'une seule fois un message sur plusieurs newsgroups
  \item Parce que le crosspost permet de regrouper sur le m�me newsgroup toutes les r�ponses � ton post
  \item Parce que les newsmestres n'aime vraiment pas du tout les gens qui crosspost mal... et que donc tu risquerait de recevoir un rtfibrpg\pageref{crosspost} en r�ponse.
\end{itemize}

\subsubsection{Comment faire un crosspost correct ?}

\label{crosspost}

Un crosspost, consiste � d�finir pour ton client plusieurs newsgroups de destination et 1 newsgroups vers lesquel seront transf�r�es toutes les r�ponses � ton post. Et il n'y a pas besoin d'avoir fait Polytechnique pour y arriver ;-).

Tous les logiciels de consultation de news permettent de faire des cross-posts ; voici une description pas-�-pas de comment faire un cross-post dans Outlook Express 6
.0\footnote{Mais les newsmestres te conseillent Thunderbird (ou KNode sous linux) quand m�me !}, que beaucoup de personnes emploient � l'X.

\begin{itemize}
  \item Clique sur 'Nouveau message' ('New Post').
  \item Dans le menu 'Affichage', coche 'Tous les en-t�tes' ('View'->'All Headers'). � ce niveau-l� tu as une fen�tre vide pour taper ton texte avec plein de champs obscurs � remplir au-dessus. Ne t'en occupe pas pour l'instant et tape le texte de ton message.
  \item Dans le premier en-t�te en haut de la fen�tre d'�dition : '\texttt{Groupe de discussion}' ('\texttt{Newsgroup}'), mets tous les newgroups o� tu veux que ton message apparaisse, s�par�s par des ';'. Par exemple, 'br.eleves; br.promo.rouje; br.promo.jone; br.binet.bob' permet d'�crire en une seule fois sur ces 4 newsgroups.
  \item Remplis le deuxi�me en-t�te : '\texttt{Transf�rer �}' ('\texttt{Follow-up to}') avec le nom du newsgroup o� tu veux que toutes les r�ponses arrivent --- un et un seul newsgroup, sinon ton message sera refus� par le serveur de news et tu auras un message d'erreur.
  \item Ca y est, tu peux envoyer ! Tu n'as pas oubli� de mettre un sujet au moins ? :)
\end{itemize}

Sur un autre client news, comme KNode, Thunderbird ou MacSOUP, la philosophie est identique... apr�s il peut y avoir des d�tails qui change (par exemple sur KNode, le s�parateur entre les newsgroups c'est la virgule)...

\emph{En bref :} deux conditions sont requises pour qu'un cross-post soit propre :
\begin{itemize}
  \item mettre plusieurs newsgroups s�par�s par '\texttt{; }' dans '\texttt{To :}' --- sinon ca ne sert pas � grand chose :)
  \item mettre un unique newsgroup dans '\texttt{Transf�rer � :}' ('\texttt{Follow-up to :}')
\end{itemize}
Il n'ya alors plus qu'� remplir le reste comme d'habitude et envoyer ! Et �a marche � tous les coups !


