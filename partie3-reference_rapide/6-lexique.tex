%$Id$

\subsection{Lexique}

\begin{description}
  \item[BR] Binet R�seau : le binet qui s'occupe d'administrer le r�seau des �l�ves, de d�velopper et de maintenir le site \url{http://frankiz/} et le logiciel \app{qRezix}.
  \item[client] (voir serveur) : programme qui permet de se connecter � un serveur. Par exemple un client news (comme \app{Thunderbird}), un client FTP (comme \app{SmartFtp}), un client xNet (comme \app{qRezix})
  \item[crosspost] mot pr�f�r� des newsmestres qui entra�ne souvent une avalanche de mails d'insultes. D�crit l'art et la mani�re de ne pas pourrir les newsgroups.
  \item[firewall] Logiciel de protection de ton ordinateur contre les infiltrations de vers ou de pirates informatiques.
  \item[FTP] File Transfer Protocol : protocole r�seau qui permet de s'�changer des fichiers.
  \item[InfoBR] tu l'as entre les mains.
  \item[IP] Addresse de ton ordinateur sur le r�seau, compos�e de 4 nombres compris entre 0 et 255 (129.104.xxx.xxx). Elle identifie ta machine aupr�s des autres utilisateurs du r�seau.
  \item[proxy] machine qui autorise (et du m�me coup) restreint les communications avec l'ext�rieur. Le proxy prot�ge le r�seau et tous les ordinateurs qui sont dessus des attaques.
  \item[R�solution DNS] trouver l'IP associ�e � un nom DNS pour pouvoir se connecter � la machine concern�e. Par exemple, la r�solution DNS de 
  \item[RTF\ldots] Read The Fucking\ldots : allusion sympathique au fait que tu aurais pu trouver l'information autre part\ldots tout pr�s souvent (voir RTFIBR, RTFFAQ, RTFM, etc.)
  \item[RTFFAQ] tu devrais trouver la r�ponses � la question dans la FAQ \url{http://frankiz}
  \item[RTFIBR] il semblerait que la r�ponse soit dans l'InfoBR. Souvent suivi du num�ro de la page. L'usage de cette expression est souvent associ�e au mot crosspost.
  \item[RTFM] \ldots M, c'est le manuel, le reste on l'a d�j� expliqu� ;)
  \item[serveur] (voir client) : programme qui permet d'accueillir des services. Comme par exemple le partage de fichier, le voisinage r�seau, un site web\ldots
  \item[Serveur] (c pas le m�me qu'avant ;)) : machine qui accueille des serveurs (l� c'est celui d'au-dessus). Le BR poss�de des serveurs soigneusement cach�s dans une salle secr�te et blind�e.
  \item[STFW] variante : Search The Fucking Web. \url{http://www.google.fr} est ton ami.
\end{description}
