%$Id$

\subsection{Lexique}

\begin{description}
  \item[BR] Binet R�seau : le binet qui s'occupe d'administrer le r�seau des �l�ves, de d�velopper et de maintenir le site \texttt{http://frankiz/} et le logiciel \texttt{qRezix}.
  \item[client] (voir serveur) : programme qui permet de se connecter � un serveur. Par exemple un client news (comme Thunderbird), un client ftp (comme SmartFtp), un client xNet (comme \texttt{qRezix}
  \item[crosspost] mot pr�f�r� des newsmestres qui entraine souvent une avalanche de mails d'insultes. D�crit l'art et la mani�re de ne pas pourir les newsgroups.
  \item[firewall] Logiciel de protection de ton ordinateur contre les infiltrations de vers ou de pirates informatiques.
  \item[FTP] File Transfer Protocol : protocole r�seau qui permet de s'�changer des fichiers.
  \item[InfoBR] vous l'avez entre les mains.
  \item[IP] Addresse de ton ordinateur sur le r�seau, compos�e de 4 chiffres (129.104.xxx.xxx). Elle identifie ta machine aupr�s des autres utilisateurs du r�seau.
  \item[proxy] 
  \item[RTF...] Read The Fucking : allusion sympathique au fait que tu aurais pu trouver l'information autre part... tout pr�t souvent (voir RTFiBr, RTFFAQ, RTFM...)
  \item[RTFFAQ] ... tu devrais trouver la r�ponses � la question dans la FAQ \texttt{http://frankiz}
  \item[RTFiBr] ... il semblerait que la r�ponse soit dans l'InfoBR... souvent suit du num�ro de la page. L'usage de cette expression est souvent associ�e au mot crosspost.
  \item[RTFM] ... M, c'est le manuel, le reste on l'a d�j� expliqu� ;)
  \item[serveur] (voir client) : programme qui permet d'accueillir des services. Comme par exemple le partage de fichier, le voisinage r�seau, un site web...
  \item[Serveur] (c pas le m�me qu'avant ;) : machine qui accueil des serveurs (l� c'est celui d'au-dessus). Le BR poss�de ... serveurs (bah nan, on vous dira pas combien ;).
\end{description}
