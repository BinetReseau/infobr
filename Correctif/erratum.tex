%$Id$

\documentclass[12pt,a5paper]{article}
\usepackage[latin1]{inputenc}
\usepackage[T1]{fontenc}
\usepackage[francais]{babel}
\AddThinSpaceBeforeFootnotes
\FrenchFootnotes
\usepackage{xspace}
\usepackage{lmodern}
\usepackage{amsfonts}
\usepackage{amsmath}
\usepackage{amssymb}
\usepackage{eurosym}
\usepackage{ifthen}
\usepackage{alltt}
\usepackage{wallpaper}
\usepackage{wrapfig}
\usepackage{graphicx}
\usepackage{color}
\usepackage{epstopdf}
\DeclareGraphicsRule{.tif}{png}{.png}{`convert #1 `dirname #1`/`basename #1 .tif`.png}

\usepackage{fancyhdr}
\pagestyle{empty}


\title{InfoBR 2k4}
\author{BR 2k3}
\date{\today}

\begin{document}

\hspace{30pt} {\Large InfoBR 2k4 : erratum}

\vspace{30pt}

On aurait bien aim� vous donner un InfoBR sans erreur, mais la loose a port� ses coups aveugles contre lui\ldots\ Alors on vous �crit un gentil correctif !\\

L'imprimerie a �chang� deux pages, la 37 et la 11. Donc, il faut aller chercher une partie de la configuration Windows � la fin du livret.

Pourquoi ? Parce que pour faire une blague subtile que vous comprendrez quand vous serez grands, la page 11 portait le num�ro 37 et la page 37 le num�ro 11. Avec cette interversion, les num�ros de page sont dans l'ordre mais plus le texte\ldots\\

Bref, vous allez vous en sortir, mais faites attention � ne pas sauter une partie de la configuration !\\

Avec toutes nos excuses,

\vspace{20pt}

\begin{flushright}
mYk, respo InfoBR 2k3,\\
pour un binet toujours plus mythique ! 
\end{flushright}

\ThisCenterWallPaper{.9}{br-light-gray}


\end{document} 
