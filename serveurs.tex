\subsection{Descriptions des diff�rents serveurs}
{\bf Serveurs du BR : }Voici la liste des principaux serveurs du BR,
que tu vas principalement utiliser durant tes deux ann\'ees sur le
plateau, ainsi que leurs IPs et les services qu'ils h\'ebergent.
Note que ces services peuvent � tout moment migrer d'une machine � une
autre en cas de besoin.


\begin{description}
        \item[frankiz] (\server{129.104.201.51}) : DNS secondaire,
        news, site web, sites des binets
        \item[gwennoz] (\server{129.104.201.52}) : DNS secondaire,
        d\'eveloppement, miroirs Linux
        \item[heol] (\server{129.104.201.53}) : DNS principale,
        xnetserver, ircserver
        \item[skinwel] (\serveur{129.104.201.54}) : DNS secondaire,
        SVN, t\'el\'e
        \item[wifi] (\serveur{129.104.201.56}) : Wifi
	\item[enez] (\serveur{129.104.201.61}) : Domaine windows
\end {description}

{\bf Serveurs de la DSI : }Etant donn\'e que le r�seau �l�ves est un
sous-r\'eseau de celui de la DSI, nous utilisons �galement les serveurs
de celle-ci et les services qu'ils h\'ebergent.

\begin{description}
        \item[kuzh] (\server{129.104.247.2}) : proxy http, pour internet
        \item[sil] (\server{129.104.247.3}) : proxy ftp, acc\`es ssh
        vers et depuis l'ext\'erieur
        \item[poly] (\server{129.104.247.5}) : mails
        \item[moned] : serveur d'authentification, permettant de
        changer ton mot de passe moned. Ce mot de passe est celui qui
        te permet de te connecter et d'utiliser n'importe
        quelle machine de salle info. Ton travail n'�tant pas stock\'e
        en local, il t'est donc accessible, quelque soit le PC sur
        lequel tu te connectes.
\end {description}

{\bf ATTENTION : Les serveurs de la DSI sont \`a ta disposiion pour des usages bien pr\'ecis, et ne servent pas de serveurs de stockage. Seul sil est pr\'evu pour du transfert de fichiers. Tout abus sera sanctionn\'e et pourra entra\^iner la perte de tes comptes poly, moned ou sil}